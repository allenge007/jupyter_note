\documentclass[11pt]{article}

    \usepackage[breakable]{tcolorbox}
    \usepackage{parskip} % Stop auto-indenting (to mimic markdown behaviour)
    \usepackage{xeCJK}

    % Basic figure setup, for now with no caption control since it's done
    % automatically by Pandoc (which extracts ![](path) syntax from Markdown).
    \usepackage{graphicx}
    % Keep aspect ratio if custom image width or height is specified
    \setkeys{Gin}{keepaspectratio}
    % Maintain compatibility with old templates. Remove in nbconvert 6.0
    \let\Oldincludegraphics\includegraphics
    % Ensure that by default, figures have no caption (until we provide a
    % proper Figure object with a Caption API and a way to capture that
    % in the conversion process - todo).
    \usepackage{caption}
    \DeclareCaptionFormat{nocaption}{}
    \captionsetup{format=nocaption,aboveskip=0pt,belowskip=0pt}

    \usepackage{float}
    \floatplacement{figure}{H} % forces figures to be placed at the correct location
    \usepackage{xcolor} % Allow colors to be defined
    \usepackage{enumerate} % Needed for markdown enumerations to work
    \usepackage{geometry} % Used to adjust the document margins
    \usepackage{amsmath} % Equations
    \usepackage{amssymb} % Equations
    \usepackage{textcomp} % defines textquotesingle
    % Hack from http://tex.stackexchange.com/a/47451/13684:
    \AtBeginDocument{%
        \def\PYZsq{\textquotesingle}% Upright quotes in Pygmentized code
    }
    \usepackage{upquote} % Upright quotes for verbatim code
    \usepackage{eurosym} % defines \euro

    \usepackage{iftex}
    \ifPDFTeX
        \usepackage[T1]{fontenc}
        \IfFileExists{alphabeta.sty}{
              \usepackage{alphabeta}
          }{
              \usepackage[mathletters]{ucs}
              \usepackage[utf8x]{inputenc}
          }
    \else
        \usepackage{fontspec}
        \usepackage{unicode-math}
    \fi

    \usepackage{fancyvrb} % verbatim replacement that allows latex
    \usepackage{grffile} % extends the file name processing of package graphics
                         % to support a larger range
    \makeatletter % fix for old versions of grffile with XeLaTeX
    \@ifpackagelater{grffile}{2019/11/01}
    {
      % Do nothing on new versions
    }
    {
      \def\Gread@@xetex#1{%
        \IfFileExists{"\Gin@base".bb}%
        {\Gread@eps{\Gin@base.bb}}%
        {\Gread@@xetex@aux#1}%
      }
    }
    \makeatother
    \usepackage[Export]{adjustbox} % Used to constrain images to a maximum size
    \adjustboxset{max size={0.9\linewidth}{0.9\paperheight}}

    % The hyperref package gives us a pdf with properly built
    % internal navigation ('pdf bookmarks' for the table of contents,
    % internal cross-reference links, web links for URLs, etc.)
    \usepackage{hyperref}
    % The default LaTeX title has an obnoxious amount of whitespace. By default,
    % titling removes some of it. It also provides customization options.
    \usepackage{titling}
    \usepackage{longtable} % longtable support required by pandoc >1.10
    \usepackage{booktabs}  % table support for pandoc > 1.12.2
    \usepackage{array}     % table support for pandoc >= 2.11.3
    \usepackage{calc}      % table minipage width calculation for pandoc >= 2.11.1
    \usepackage[inline]{enumitem} % IRkernel/repr support (it uses the enumerate* environment)
    \usepackage[normalem]{ulem} % ulem is needed to support strikethroughs (\sout)
                                % normalem makes italics be italics, not underlines
    \usepackage{soul}      % strikethrough (\st) support for pandoc >= 3.0.0
    \usepackage{mathrsfs}
    

    
    % Colors for the hyperref package
    \definecolor{urlcolor}{rgb}{0,.145,.698}
    \definecolor{linkcolor}{rgb}{.71,0.21,0.01}
    \definecolor{citecolor}{rgb}{.12,.54,.11}

    % ANSI colors
    \definecolor{ansi-black}{HTML}{3E424D}
    \definecolor{ansi-black-intense}{HTML}{282C36}
    \definecolor{ansi-red}{HTML}{E75C58}
    \definecolor{ansi-red-intense}{HTML}{B22B31}
    \definecolor{ansi-green}{HTML}{00A250}
    \definecolor{ansi-green-intense}{HTML}{007427}
    \definecolor{ansi-yellow}{HTML}{DDB62B}
    \definecolor{ansi-yellow-intense}{HTML}{B27D12}
    \definecolor{ansi-blue}{HTML}{208FFB}
    \definecolor{ansi-blue-intense}{HTML}{0065CA}
    \definecolor{ansi-magenta}{HTML}{D160C4}
    \definecolor{ansi-magenta-intense}{HTML}{A03196}
    \definecolor{ansi-cyan}{HTML}{60C6C8}
    \definecolor{ansi-cyan-intense}{HTML}{258F8F}
    \definecolor{ansi-white}{HTML}{C5C1B4}
    \definecolor{ansi-white-intense}{HTML}{A1A6B2}
    \definecolor{ansi-default-inverse-fg}{HTML}{FFFFFF}
    \definecolor{ansi-default-inverse-bg}{HTML}{000000}

    % common color for the border for error outputs.
    \definecolor{outerrorbackground}{HTML}{FFDFDF}

    % commands and environments needed by pandoc snippets
    % extracted from the output of `pandoc -s`
    \providecommand{\tightlist}{%
      \setlength{\itemsep}{0pt}\setlength{\parskip}{0pt}}
    \DefineVerbatimEnvironment{Highlighting}{Verbatim}{commandchars=\\\{\}}
    % Add ',fontsize=\small' for more characters per line
    \newenvironment{Shaded}{}{}
    \newcommand{\KeywordTok}[1]{\textcolor[rgb]{0.00,0.44,0.13}{\textbf{{#1}}}}
    \newcommand{\DataTypeTok}[1]{\textcolor[rgb]{0.56,0.13,0.00}{{#1}}}
    \newcommand{\DecValTok}[1]{\textcolor[rgb]{0.25,0.63,0.44}{{#1}}}
    \newcommand{\BaseNTok}[1]{\textcolor[rgb]{0.25,0.63,0.44}{{#1}}}
    \newcommand{\FloatTok}[1]{\textcolor[rgb]{0.25,0.63,0.44}{{#1}}}
    \newcommand{\CharTok}[1]{\textcolor[rgb]{0.25,0.44,0.63}{{#1}}}
    \newcommand{\StringTok}[1]{\textcolor[rgb]{0.25,0.44,0.63}{{#1}}}
    \newcommand{\CommentTok}[1]{\textcolor[rgb]{0.38,0.63,0.69}{\textit{{#1}}}}
    \newcommand{\OtherTok}[1]{\textcolor[rgb]{0.00,0.44,0.13}{{#1}}}
    \newcommand{\AlertTok}[1]{\textcolor[rgb]{1.00,0.00,0.00}{\textbf{{#1}}}}
    \newcommand{\FunctionTok}[1]{\textcolor[rgb]{0.02,0.16,0.49}{{#1}}}
    \newcommand{\RegionMarkerTok}[1]{{#1}}
    \newcommand{\ErrorTok}[1]{\textcolor[rgb]{1.00,0.00,0.00}{\textbf{{#1}}}}
    \newcommand{\NormalTok}[1]{{#1}}

    % Additional commands for more recent versions of Pandoc
    \newcommand{\ConstantTok}[1]{\textcolor[rgb]{0.53,0.00,0.00}{{#1}}}
    \newcommand{\SpecialCharTok}[1]{\textcolor[rgb]{0.25,0.44,0.63}{{#1}}}
    \newcommand{\VerbatimStringTok}[1]{\textcolor[rgb]{0.25,0.44,0.63}{{#1}}}
    \newcommand{\SpecialStringTok}[1]{\textcolor[rgb]{0.73,0.40,0.53}{{#1}}}
    \newcommand{\ImportTok}[1]{{#1}}
    \newcommand{\DocumentationTok}[1]{\textcolor[rgb]{0.73,0.13,0.13}{\textit{{#1}}}}
    \newcommand{\AnnotationTok}[1]{\textcolor[rgb]{0.38,0.63,0.69}{\textbf{\textit{{#1}}}}}
    \newcommand{\CommentVarTok}[1]{\textcolor[rgb]{0.38,0.63,0.69}{\textbf{\textit{{#1}}}}}
    \newcommand{\VariableTok}[1]{\textcolor[rgb]{0.10,0.09,0.49}{{#1}}}
    \newcommand{\ControlFlowTok}[1]{\textcolor[rgb]{0.00,0.44,0.13}{\textbf{{#1}}}}
    \newcommand{\OperatorTok}[1]{\textcolor[rgb]{0.40,0.40,0.40}{{#1}}}
    \newcommand{\BuiltInTok}[1]{{#1}}
    \newcommand{\ExtensionTok}[1]{{#1}}
    \newcommand{\PreprocessorTok}[1]{\textcolor[rgb]{0.74,0.48,0.00}{{#1}}}
    \newcommand{\AttributeTok}[1]{\textcolor[rgb]{0.49,0.56,0.16}{{#1}}}
    \newcommand{\InformationTok}[1]{\textcolor[rgb]{0.38,0.63,0.69}{\textbf{\textit{{#1}}}}}
    \newcommand{\WarningTok}[1]{\textcolor[rgb]{0.38,0.63,0.69}{\textbf{\textit{{#1}}}}}


    % Define a nice break command that doesn't care if a line doesn't already
    % exist.
    \def\br{\hspace*{\fill} \\* }
    % Math Jax compatibility definitions
    \def\gt{>}
    \def\lt{<}
    \let\Oldtex\TeX
    \let\Oldlatex\LaTeX
    \renewcommand{\TeX}{\textrm{\Oldtex}}
    \renewcommand{\LaTeX}{\textrm{\Oldlatex}}
    % Document parameters
    % Document title
    \title{assn06}
    
    
    
    
    
    
    
% Pygments definitions
\makeatletter
\def\PY@reset{\let\PY@it=\relax \let\PY@bf=\relax%
    \let\PY@ul=\relax \let\PY@tc=\relax%
    \let\PY@bc=\relax \let\PY@ff=\relax}
\def\PY@tok#1{\csname PY@tok@#1\endcsname}
\def\PY@toks#1+{\ifx\relax#1\empty\else%
    \PY@tok{#1}\expandafter\PY@toks\fi}
\def\PY@do#1{\PY@bc{\PY@tc{\PY@ul{%
    \PY@it{\PY@bf{\PY@ff{#1}}}}}}}
\def\PY#1#2{\PY@reset\PY@toks#1+\relax+\PY@do{#2}}

\@namedef{PY@tok@w}{\def\PY@tc##1{\textcolor[rgb]{0.73,0.73,0.73}{##1}}}
\@namedef{PY@tok@c}{\let\PY@it=\textit\def\PY@tc##1{\textcolor[rgb]{0.24,0.48,0.48}{##1}}}
\@namedef{PY@tok@cp}{\def\PY@tc##1{\textcolor[rgb]{0.61,0.40,0.00}{##1}}}
\@namedef{PY@tok@k}{\let\PY@bf=\textbf\def\PY@tc##1{\textcolor[rgb]{0.00,0.50,0.00}{##1}}}
\@namedef{PY@tok@kp}{\def\PY@tc##1{\textcolor[rgb]{0.00,0.50,0.00}{##1}}}
\@namedef{PY@tok@kt}{\def\PY@tc##1{\textcolor[rgb]{0.69,0.00,0.25}{##1}}}
\@namedef{PY@tok@o}{\def\PY@tc##1{\textcolor[rgb]{0.40,0.40,0.40}{##1}}}
\@namedef{PY@tok@ow}{\let\PY@bf=\textbf\def\PY@tc##1{\textcolor[rgb]{0.67,0.13,1.00}{##1}}}
\@namedef{PY@tok@nb}{\def\PY@tc##1{\textcolor[rgb]{0.00,0.50,0.00}{##1}}}
\@namedef{PY@tok@nf}{\def\PY@tc##1{\textcolor[rgb]{0.00,0.00,1.00}{##1}}}
\@namedef{PY@tok@nc}{\let\PY@bf=\textbf\def\PY@tc##1{\textcolor[rgb]{0.00,0.00,1.00}{##1}}}
\@namedef{PY@tok@nn}{\let\PY@bf=\textbf\def\PY@tc##1{\textcolor[rgb]{0.00,0.00,1.00}{##1}}}
\@namedef{PY@tok@ne}{\let\PY@bf=\textbf\def\PY@tc##1{\textcolor[rgb]{0.80,0.25,0.22}{##1}}}
\@namedef{PY@tok@nv}{\def\PY@tc##1{\textcolor[rgb]{0.10,0.09,0.49}{##1}}}
\@namedef{PY@tok@no}{\def\PY@tc##1{\textcolor[rgb]{0.53,0.00,0.00}{##1}}}
\@namedef{PY@tok@nl}{\def\PY@tc##1{\textcolor[rgb]{0.46,0.46,0.00}{##1}}}
\@namedef{PY@tok@ni}{\let\PY@bf=\textbf\def\PY@tc##1{\textcolor[rgb]{0.44,0.44,0.44}{##1}}}
\@namedef{PY@tok@na}{\def\PY@tc##1{\textcolor[rgb]{0.41,0.47,0.13}{##1}}}
\@namedef{PY@tok@nt}{\let\PY@bf=\textbf\def\PY@tc##1{\textcolor[rgb]{0.00,0.50,0.00}{##1}}}
\@namedef{PY@tok@nd}{\def\PY@tc##1{\textcolor[rgb]{0.67,0.13,1.00}{##1}}}
\@namedef{PY@tok@s}{\def\PY@tc##1{\textcolor[rgb]{0.73,0.13,0.13}{##1}}}
\@namedef{PY@tok@sd}{\let\PY@it=\textit\def\PY@tc##1{\textcolor[rgb]{0.73,0.13,0.13}{##1}}}
\@namedef{PY@tok@si}{\let\PY@bf=\textbf\def\PY@tc##1{\textcolor[rgb]{0.64,0.35,0.47}{##1}}}
\@namedef{PY@tok@se}{\let\PY@bf=\textbf\def\PY@tc##1{\textcolor[rgb]{0.67,0.36,0.12}{##1}}}
\@namedef{PY@tok@sr}{\def\PY@tc##1{\textcolor[rgb]{0.64,0.35,0.47}{##1}}}
\@namedef{PY@tok@ss}{\def\PY@tc##1{\textcolor[rgb]{0.10,0.09,0.49}{##1}}}
\@namedef{PY@tok@sx}{\def\PY@tc##1{\textcolor[rgb]{0.00,0.50,0.00}{##1}}}
\@namedef{PY@tok@m}{\def\PY@tc##1{\textcolor[rgb]{0.40,0.40,0.40}{##1}}}
\@namedef{PY@tok@gh}{\let\PY@bf=\textbf\def\PY@tc##1{\textcolor[rgb]{0.00,0.00,0.50}{##1}}}
\@namedef{PY@tok@gu}{\let\PY@bf=\textbf\def\PY@tc##1{\textcolor[rgb]{0.50,0.00,0.50}{##1}}}
\@namedef{PY@tok@gd}{\def\PY@tc##1{\textcolor[rgb]{0.63,0.00,0.00}{##1}}}
\@namedef{PY@tok@gi}{\def\PY@tc##1{\textcolor[rgb]{0.00,0.52,0.00}{##1}}}
\@namedef{PY@tok@gr}{\def\PY@tc##1{\textcolor[rgb]{0.89,0.00,0.00}{##1}}}
\@namedef{PY@tok@ge}{\let\PY@it=\textit}
\@namedef{PY@tok@gs}{\let\PY@bf=\textbf}
\@namedef{PY@tok@gp}{\let\PY@bf=\textbf\def\PY@tc##1{\textcolor[rgb]{0.00,0.00,0.50}{##1}}}
\@namedef{PY@tok@go}{\def\PY@tc##1{\textcolor[rgb]{0.44,0.44,0.44}{##1}}}
\@namedef{PY@tok@gt}{\def\PY@tc##1{\textcolor[rgb]{0.00,0.27,0.87}{##1}}}
\@namedef{PY@tok@err}{\def\PY@bc##1{{\setlength{\fboxsep}{\string -\fboxrule}\fcolorbox[rgb]{1.00,0.00,0.00}{1,1,1}{\strut ##1}}}}
\@namedef{PY@tok@kc}{\let\PY@bf=\textbf\def\PY@tc##1{\textcolor[rgb]{0.00,0.50,0.00}{##1}}}
\@namedef{PY@tok@kd}{\let\PY@bf=\textbf\def\PY@tc##1{\textcolor[rgb]{0.00,0.50,0.00}{##1}}}
\@namedef{PY@tok@kn}{\let\PY@bf=\textbf\def\PY@tc##1{\textcolor[rgb]{0.00,0.50,0.00}{##1}}}
\@namedef{PY@tok@kr}{\let\PY@bf=\textbf\def\PY@tc##1{\textcolor[rgb]{0.00,0.50,0.00}{##1}}}
\@namedef{PY@tok@bp}{\def\PY@tc##1{\textcolor[rgb]{0.00,0.50,0.00}{##1}}}
\@namedef{PY@tok@fm}{\def\PY@tc##1{\textcolor[rgb]{0.00,0.00,1.00}{##1}}}
\@namedef{PY@tok@vc}{\def\PY@tc##1{\textcolor[rgb]{0.10,0.09,0.49}{##1}}}
\@namedef{PY@tok@vg}{\def\PY@tc##1{\textcolor[rgb]{0.10,0.09,0.49}{##1}}}
\@namedef{PY@tok@vi}{\def\PY@tc##1{\textcolor[rgb]{0.10,0.09,0.49}{##1}}}
\@namedef{PY@tok@vm}{\def\PY@tc##1{\textcolor[rgb]{0.10,0.09,0.49}{##1}}}
\@namedef{PY@tok@sa}{\def\PY@tc##1{\textcolor[rgb]{0.73,0.13,0.13}{##1}}}
\@namedef{PY@tok@sb}{\def\PY@tc##1{\textcolor[rgb]{0.73,0.13,0.13}{##1}}}
\@namedef{PY@tok@sc}{\def\PY@tc##1{\textcolor[rgb]{0.73,0.13,0.13}{##1}}}
\@namedef{PY@tok@dl}{\def\PY@tc##1{\textcolor[rgb]{0.73,0.13,0.13}{##1}}}
\@namedef{PY@tok@s2}{\def\PY@tc##1{\textcolor[rgb]{0.73,0.13,0.13}{##1}}}
\@namedef{PY@tok@sh}{\def\PY@tc##1{\textcolor[rgb]{0.73,0.13,0.13}{##1}}}
\@namedef{PY@tok@s1}{\def\PY@tc##1{\textcolor[rgb]{0.73,0.13,0.13}{##1}}}
\@namedef{PY@tok@mb}{\def\PY@tc##1{\textcolor[rgb]{0.40,0.40,0.40}{##1}}}
\@namedef{PY@tok@mf}{\def\PY@tc##1{\textcolor[rgb]{0.40,0.40,0.40}{##1}}}
\@namedef{PY@tok@mh}{\def\PY@tc##1{\textcolor[rgb]{0.40,0.40,0.40}{##1}}}
\@namedef{PY@tok@mi}{\def\PY@tc##1{\textcolor[rgb]{0.40,0.40,0.40}{##1}}}
\@namedef{PY@tok@il}{\def\PY@tc##1{\textcolor[rgb]{0.40,0.40,0.40}{##1}}}
\@namedef{PY@tok@mo}{\def\PY@tc##1{\textcolor[rgb]{0.40,0.40,0.40}{##1}}}
\@namedef{PY@tok@ch}{\let\PY@it=\textit\def\PY@tc##1{\textcolor[rgb]{0.24,0.48,0.48}{##1}}}
\@namedef{PY@tok@cm}{\let\PY@it=\textit\def\PY@tc##1{\textcolor[rgb]{0.24,0.48,0.48}{##1}}}
\@namedef{PY@tok@cpf}{\let\PY@it=\textit\def\PY@tc##1{\textcolor[rgb]{0.24,0.48,0.48}{##1}}}
\@namedef{PY@tok@c1}{\let\PY@it=\textit\def\PY@tc##1{\textcolor[rgb]{0.24,0.48,0.48}{##1}}}
\@namedef{PY@tok@cs}{\let\PY@it=\textit\def\PY@tc##1{\textcolor[rgb]{0.24,0.48,0.48}{##1}}}

\def\PYZbs{\char`\\}
\def\PYZus{\char`\_}
\def\PYZob{\char`\{}
\def\PYZcb{\char`\}}
\def\PYZca{\char`\^}
\def\PYZam{\char`\&}
\def\PYZlt{\char`\<}
\def\PYZgt{\char`\>}
\def\PYZsh{\char`\#}
\def\PYZpc{\char`\%}
\def\PYZdl{\char`\$}
\def\PYZhy{\char`\-}
\def\PYZsq{\char`\'}
\def\PYZdq{\char`\"}
\def\PYZti{\char`\~}
% for compatibility with earlier versions
\def\PYZat{@}
\def\PYZlb{[}
\def\PYZrb{]}
\makeatother


    % For linebreaks inside Verbatim environment from package fancyvrb.
    \makeatletter
        \newbox\Wrappedcontinuationbox
        \newbox\Wrappedvisiblespacebox
        \newcommand*\Wrappedvisiblespace {\textcolor{red}{\textvisiblespace}}
        \newcommand*\Wrappedcontinuationsymbol {\textcolor{red}{\llap{\tiny$\m@th\hookrightarrow$}}}
        \newcommand*\Wrappedcontinuationindent {3ex }
        \newcommand*\Wrappedafterbreak {\kern\Wrappedcontinuationindent\copy\Wrappedcontinuationbox}
        % Take advantage of the already applied Pygments mark-up to insert
        % potential linebreaks for TeX processing.
        %        {, <, #, %, $, ' and ": go to next line.
        %        _, }, ^, &, >, - and ~: stay at end of broken line.
        % Use of \textquotesingle for straight quote.
        \newcommand*\Wrappedbreaksatspecials {%
            \def\PYGZus{\discretionary{\char`\_}{\Wrappedafterbreak}{\char`\_}}%
            \def\PYGZob{\discretionary{}{\Wrappedafterbreak\char`\{}{\char`\{}}%
            \def\PYGZcb{\discretionary{\char`\}}{\Wrappedafterbreak}{\char`\}}}%
            \def\PYGZca{\discretionary{\char`\^}{\Wrappedafterbreak}{\char`\^}}%
            \def\PYGZam{\discretionary{\char`\&}{\Wrappedafterbreak}{\char`\&}}%
            \def\PYGZlt{\discretionary{}{\Wrappedafterbreak\char`\<}{\char`\<}}%
            \def\PYGZgt{\discretionary{\char`\>}{\Wrappedafterbreak}{\char`\>}}%
            \def\PYGZsh{\discretionary{}{\Wrappedafterbreak\char`\#}{\char`\#}}%
            \def\PYGZpc{\discretionary{}{\Wrappedafterbreak\char`\%}{\char`\%}}%
            \def\PYGZdl{\discretionary{}{\Wrappedafterbreak\char`\$}{\char`\$}}%
            \def\PYGZhy{\discretionary{\char`\-}{\Wrappedafterbreak}{\char`\-}}%
            \def\PYGZsq{\discretionary{}{\Wrappedafterbreak\textquotesingle}{\textquotesingle}}%
            \def\PYGZdq{\discretionary{}{\Wrappedafterbreak\char`\"}{\char`\"}}%
            \def\PYGZti{\discretionary{\char`\~}{\Wrappedafterbreak}{\char`\~}}%
        }
        % Some characters . , ; ? ! / are not pygmentized.
        % This macro makes them "active" and they will insert potential linebreaks
        \newcommand*\Wrappedbreaksatpunct {%
            \lccode`\~`\.\lowercase{\def~}{\discretionary{\hbox{\char`\.}}{\Wrappedafterbreak}{\hbox{\char`\.}}}%
            \lccode`\~`\,\lowercase{\def~}{\discretionary{\hbox{\char`\,}}{\Wrappedafterbreak}{\hbox{\char`\,}}}%
            \lccode`\~`\;\lowercase{\def~}{\discretionary{\hbox{\char`\;}}{\Wrappedafterbreak}{\hbox{\char`\;}}}%
            \lccode`\~`\:\lowercase{\def~}{\discretionary{\hbox{\char`\:}}{\Wrappedafterbreak}{\hbox{\char`\:}}}%
            \lccode`\~`\?\lowercase{\def~}{\discretionary{\hbox{\char`\?}}{\Wrappedafterbreak}{\hbox{\char`\?}}}%
            \lccode`\~`\!\lowercase{\def~}{\discretionary{\hbox{\char`\!}}{\Wrappedafterbreak}{\hbox{\char`\!}}}%
            \lccode`\~`\/\lowercase{\def~}{\discretionary{\hbox{\char`\/}}{\Wrappedafterbreak}{\hbox{\char`\/}}}%
            \catcode`\.\active
            \catcode`\,\active
            \catcode`\;\active
            \catcode`\:\active
            \catcode`\?\active
            \catcode`\!\active
            \catcode`\/\active
            \lccode`\~`\~
        }
    \makeatother

    \let\OriginalVerbatim=\Verbatim
    \makeatletter
    \renewcommand{\Verbatim}[1][1]{%
        %\parskip\z@skip
        \sbox\Wrappedcontinuationbox {\Wrappedcontinuationsymbol}%
        \sbox\Wrappedvisiblespacebox {\FV@SetupFont\Wrappedvisiblespace}%
        \def\FancyVerbFormatLine ##1{\hsize\linewidth
            \vtop{\raggedright\hyphenpenalty\z@\exhyphenpenalty\z@
                \doublehyphendemerits\z@\finalhyphendemerits\z@
                \strut ##1\strut}%
        }%
        % If the linebreak is at a space, the latter will be displayed as visible
        % space at end of first line, and a continuation symbol starts next line.
        % Stretch/shrink are however usually zero for typewriter font.
        \def\FV@Space {%
            \nobreak\hskip\z@ plus\fontdimen3\font minus\fontdimen4\font
            \discretionary{\copy\Wrappedvisiblespacebox}{\Wrappedafterbreak}
            {\kern\fontdimen2\font}%
        }%

        % Allow breaks at special characters using \PYG... macros.
        \Wrappedbreaksatspecials
        % Breaks at punctuation characters . , ; ? ! and / need catcode=\active
        \OriginalVerbatim[#1,codes*=\Wrappedbreaksatpunct]%
    }
    \makeatother

    % Exact colors from NB
    \definecolor{incolor}{HTML}{303F9F}
    \definecolor{outcolor}{HTML}{D84315}
    \definecolor{cellborder}{HTML}{CFCFCF}
    \definecolor{cellbackground}{HTML}{F7F7F7}

    % prompt
    \makeatletter
    \newcommand{\boxspacing}{\kern\kvtcb@left@rule\kern\kvtcb@boxsep}
    \makeatother
    \newcommand{\prompt}[4]{
        {\ttfamily\llap{{\color{#2}[#3]:\hspace{3pt}#4}}\vspace{-\baselineskip}}
    }
    

    
    % Prevent overflowing lines due to hard-to-break entities
    \sloppy
    % Setup hyperref package
    \hypersetup{
      breaklinks=true,  % so long urls are correctly broken across lines
      colorlinks=true,
      urlcolor=urlcolor,
      linkcolor=linkcolor,
      citecolor=citecolor,
      }
    % Slightly bigger margins than the latex defaults
    
    \geometry{verbose,tmargin=1in,bmargin=1in,lmargin=1in,rmargin=1in}
    
    

\begin{document}
    
    \maketitle
    
    

    
    \section{10}\label{section}

试构造高斯型求积公式

\[
\int_0^1 \frac{1}{\sqrt x} f(x) dx \approx A_0f(x_0) + A_1f(x_1)
\]

    \subsection{Solution}\label{solution}

令权函数 \(w(x)=x^{-1/2}\) ,其矩为 \[
\mu_k=\int_0^1 x^{k-1/2}\,dx=\frac{1}{\,k+\tfrac12},\quad k=0,1,2,3.
\] 于是: \[
\begin{aligned}
\mu_0&=\int_0^1 x^{-1/2}dx=\frac{1}{0+\tfrac12}=2,\\[1mm]
\mu_1&=\int_0^1 x^{1/2}dx=\frac{1}{1+\tfrac12}=\frac{2}{3},\\[1mm]
\mu_2&=\int_0^1 x^{3/2}dx=\frac{1}{2+\tfrac12}=\frac{2}{5},\\[1mm]
\mu_3&=\int_0^1 x^{5/2}dx=\frac{1}{3+\tfrac12}=\frac{2}{7}\,.
\end{aligned}
\]

节点为相对于权函数 \(w(x)=x^{-1/2}\) 在 \([0,1]\) 上的正交多项式
\(P_2(x)\) 的两个零点。设 \[
P_2(x)=x^2+ b\,x+c.
\] 要求 \(P_2(x)\) 与低次多项式正交,即 \[
\begin{aligned}
\int_0^1 x^{-1/2} P_2(x)\,dx&=\int_0^1 \left(x^{3/2}+b\,x^{1/2}+c\,x^{-1/2}\right)dx=0,\\[1mm]
\int_0^1 x^{1/2} P_2(x)\,dx&=\int_0^1 \left(x^{5/2}+b\,x^{3/2}+c\,x^{1/2}\right)dx=0.
\end{aligned}
\] 可得: \[
\begin{cases}
\mu_2 +b\,\mu_1+c\,\mu_0= \frac{2}{5}+\frac{2}{3}\,b+2c=0,\\[1mm]
\mu_3+b\,\mu_2+c\,\mu_1= \frac{2}{7}+\frac{2}{5}\,b+\frac{2}{3}c=0.
\end{cases}
\] 解得 \[
b=-\frac{6}{7},\quad c=\frac{3}{35}\,.
\] 因此, \[
P_2(x)=x^2-\frac{6}{7}x+\frac{3}{35}\,.
\]

令 \(P_2(x)=0\) 解方程得 \[
x_{0,1}=\frac{6/7\pm\sqrt{96/245}}{2}=\frac{6/7\pm \frac{4\sqrt6}{7\sqrt5}}{2}
=\frac{3\pm \frac{2\sqrt6}{\sqrt5}}{7}\,.
\]

数值估计: \[
\sqrt{\frac{6}{5}}\approx 1.0954,\quad x_0\approx\frac{3-2.1908}{7}\approx 0.116,\quad
x_1\approx\frac{3+2.1908}{7}\approx 0.741.
\]

令求积公式对 \(f(x)=1\) 和 \(f(x)=x\) 精确,则有 \[
\begin{cases}
A_0+A_1=\mu_0=2,\\[1mm]
A_0\,x_0+A_1\,x_1=\mu_1=\frac{2}{3}\,.
\end{cases}
\] 解得 \[
A_0=\frac{\mu_1-2\,x_1}{x_0-x_1},\quad A_1=2-A_0.
\] 得到近似结果: \[
A_0\approx 1.306,\quad A_1\approx 0.694.
\]

因此,两节点高斯求积公式为 \[\boxed{
\int_0^1 \frac{1}{\sqrt{x}}f(x)\,dx \approx 1.306\,f(0.116)+0.694\,f(0.741).}
\]

    \section{18}\label{section}

用三点公式求 \(f(x) = \frac{1}{1 + x^2}\) 在 \(x = 1.0, 1.1, 1.2\)
处的导数,并估计误差。

\begin{longtable}[]{@{}llll@{}}
\toprule\noalign{}
x & 1.0 & 1.1 & 1.2 \\
\midrule\noalign{}
\endhead
\bottomrule\noalign{}
\endlastfoot
f(x) & 0.2500 & 0.2268 & 0.2066 \\
\end{longtable}

    \subsection{Solution}\label{solution}

根据三点公式,有

\[
\begin{cases}
f'(1.0)\approx \frac{-3f(1.0)+4f(1.1)-f(1.2)}{2h} \\
f'(1.1)\approx \frac{f(1.2)-f(1.0)}{2h} \\
f'(1.2)\approx \frac{f(1.0)-4f(1.1)+fF(1.2)}{2h}
\end{cases}
\]

使用 python 计算结果

    \begin{tcolorbox}[breakable, size=fbox, boxrule=1pt, pad at break*=1mm,colback=cellbackground, colframe=cellborder]
\prompt{In}{incolor}{10}{\boxspacing}
\begin{Verbatim}[commandchars=\\\{\}]
\PY{k+kn}{import} \PY{n+nn}{numpy} \PY{k}{as} \PY{n+nn}{np}

\PY{c+c1}{\PYZsh{} 已知节点及函数值}
\PY{n}{x} \PY{o}{=} \PY{n}{np}\PY{o}{.}\PY{n}{array}\PY{p}{(}\PY{p}{[}\PY{l+m+mf}{1.0}\PY{p}{,} \PY{l+m+mf}{1.1}\PY{p}{,} \PY{l+m+mf}{1.2}\PY{p}{]}\PY{p}{)}
\PY{n}{F} \PY{o}{=} \PY{n}{np}\PY{o}{.}\PY{n}{array}\PY{p}{(}\PY{p}{[}\PY{l+m+mf}{0.2500}\PY{p}{,} \PY{l+m+mf}{0.2268}\PY{p}{,} \PY{l+m+mf}{0.2066}\PY{p}{]}\PY{p}{)}
\PY{n}{h} \PY{o}{=} \PY{n}{x}\PY{p}{[}\PY{l+m+mi}{1}\PY{p}{]} \PY{o}{\PYZhy{}} \PY{n}{x}\PY{p}{[}\PY{l+m+mi}{0}\PY{p}{]}  \PY{c+c1}{\PYZsh{} h = 0.1}

\PY{c+c1}{\PYZsh{} 三点公式计算}
\PY{c+c1}{\PYZsh{} 前向差分 at x=1.0}
\PY{n}{Fprime\PYZus{}forward} \PY{o}{=} \PY{p}{(}\PY{o}{\PYZhy{}}\PY{l+m+mi}{3}\PY{o}{*}\PY{n}{F}\PY{p}{[}\PY{l+m+mi}{0}\PY{p}{]} \PY{o}{+} \PY{l+m+mi}{4}\PY{o}{*}\PY{n}{F}\PY{p}{[}\PY{l+m+mi}{1}\PY{p}{]} \PY{o}{\PYZhy{}} \PY{n}{F}\PY{p}{[}\PY{l+m+mi}{2}\PY{p}{]}\PY{p}{)} \PY{o}{/} \PY{p}{(}\PY{l+m+mi}{2}\PY{o}{*}\PY{n}{h}\PY{p}{)}
\PY{c+c1}{\PYZsh{} 中心差分 at x=1.1}
\PY{n}{Fprime\PYZus{}central} \PY{o}{=} \PY{p}{(}\PY{n}{F}\PY{p}{[}\PY{l+m+mi}{2}\PY{p}{]} \PY{o}{\PYZhy{}} \PY{n}{F}\PY{p}{[}\PY{l+m+mi}{0}\PY{p}{]}\PY{p}{)} \PY{o}{/} \PY{p}{(}\PY{l+m+mi}{2}\PY{o}{*}\PY{n}{h}\PY{p}{)}
\PY{c+c1}{\PYZsh{} 后向差分 at x=1.2}
\PY{n}{Fprime\PYZus{}backward} \PY{o}{=} \PY{p}{(}\PY{n}{F}\PY{p}{[}\PY{l+m+mi}{0}\PY{p}{]} \PY{o}{\PYZhy{}} \PY{l+m+mi}{4}\PY{o}{*}\PY{n}{F}\PY{p}{[}\PY{l+m+mi}{1}\PY{p}{]} \PY{o}{+} \PY{l+m+mi}{3}\PY{o}{*}\PY{n}{F}\PY{p}{[}\PY{l+m+mi}{2}\PY{p}{]}\PY{p}{)} \PY{o}{/} \PY{p}{(}\PY{l+m+mi}{2}\PY{o}{*}\PY{n}{h}\PY{p}{)}

\PY{n+nb}{print}\PY{p}{(}\PY{l+s+s2}{\PYZdq{}}\PY{l+s+s2}{前向差分 F}\PY{l+s+s2}{\PYZsq{}}\PY{l+s+s2}{(1.0) ≈}\PY{l+s+s2}{\PYZdq{}}\PY{p}{,} \PY{n}{Fprime\PYZus{}forward}\PY{p}{)}
\PY{n+nb}{print}\PY{p}{(}\PY{l+s+s2}{\PYZdq{}}\PY{l+s+s2}{中心差分 F}\PY{l+s+s2}{\PYZsq{}}\PY{l+s+s2}{(1.1) ≈}\PY{l+s+s2}{\PYZdq{}}\PY{p}{,} \PY{n}{Fprime\PYZus{}central}\PY{p}{)}
\PY{n+nb}{print}\PY{p}{(}\PY{l+s+s2}{\PYZdq{}}\PY{l+s+s2}{后向差分 F}\PY{l+s+s2}{\PYZsq{}}\PY{l+s+s2}{(1.2) ≈}\PY{l+s+s2}{\PYZdq{}}\PY{p}{,} \PY{n}{Fprime\PYZus{}backward}\PY{p}{)}

\PY{c+c1}{\PYZsh{} 理论精确值(F\PYZsq{}(x) = \PYZhy{}2/(1+x)\PYZca{}3)}
\PY{k}{def} \PY{n+nf}{F\PYZus{}exact\PYZus{}prime}\PY{p}{(}\PY{n}{x}\PY{p}{)}\PY{p}{:}
    \PY{k}{return} \PY{o}{\PYZhy{}}\PY{l+m+mi}{2}\PY{o}{/}\PY{p}{(}\PY{n}{x} \PY{o}{+} \PY{l+m+mi}{1}\PY{p}{)}\PY{o}{*}\PY{o}{*}\PY{l+m+mi}{3}

\PY{n}{F\PYZus{}exact} \PY{o}{=} \PY{n}{np}\PY{o}{.}\PY{n}{array}\PY{p}{(}\PY{p}{[}\PY{n}{F\PYZus{}exact\PYZus{}prime}\PY{p}{(}\PY{n}{xi}\PY{p}{)} \PY{k}{for} \PY{n}{xi} \PY{o+ow}{in} \PY{n}{x}\PY{p}{]}\PY{p}{)}
\PY{n+nb}{print}\PY{p}{(}\PY{l+s+s2}{\PYZdq{}}\PY{l+s+s2}{理论精确值:}\PY{l+s+s2}{\PYZdq{}}\PY{p}{,} \PY{n}{F\PYZus{}exact}\PY{p}{)}

\PY{n}{error} \PY{o}{=} \PY{n}{np}\PY{o}{.}\PY{n}{abs}\PY{p}{(}\PY{n}{np}\PY{o}{.}\PY{n}{array}\PY{p}{(}\PY{p}{[}\PY{n}{Fprime\PYZus{}forward}\PY{p}{,} \PY{n}{Fprime\PYZus{}central}\PY{p}{,} \PY{n}{Fprime\PYZus{}backward}\PY{p}{]}\PY{p}{)} \PY{o}{\PYZhy{}} \PY{n}{F\PYZus{}exact}\PY{p}{)}
\PY{n+nb}{print}\PY{p}{(}\PY{l+s+s2}{\PYZdq{}}\PY{l+s+s2}{绝对误差:}\PY{l+s+s2}{\PYZdq{}}\PY{p}{,} \PY{n}{error}\PY{p}{)}
\end{Verbatim}
\end{tcolorbox}

    \begin{Verbatim}[commandchars=\\\{\}]
前向差分 F'(1.0) ≈ -0.24699999999999978
中心差分 F'(1.1) ≈ -0.21699999999999978
后向差分 F'(1.2) ≈ -0.18699999999999978
理论精确值: [-0.25      -0.2159594 -0.1878287]
绝对误差: [0.003     0.0010406 0.0008287]
    \end{Verbatim}

    \subsection{误差分析}\label{ux8befux5deeux5206ux6790}

带余项的三点公式如下:

\[
\begin{cases}
f'(x_0) = \frac{1}{2h}[-3f(x_0) + 4f(x_1) - f(x_2)] + \frac{h^2}{3}f'''(\xi_0) \\
f'(x_1) = \frac{1}{2h}[-f(x_0) + f(x_2)] - \frac{h^2}{6}f'''(\xi_1) \\
f'(x_2) = \frac{1}{2h}[f(x_0) - 4f(x_1) + 3f(x_2)] + \frac{h^2}{3}f'''(\xi_2)
\end{cases}
\]

假设在区间 \([x_0,x_2]\) 内,存在常数

\[
M_3 = \max_{x\in[x_0,x_2]} \lvert f'''(x)\rvert,
\]

对于 \(f(x)=\frac{1}{1+x^2}\) 在区间 \([1.0,\,1.2]\) 上, \[
M_3 = \max_{x\in[1.0,1.2]} \lvert f'''(x)\rvert \approx 0.75.
\]

则根据余项,可以给出各公式的误差上界:

\begin{itemize}
\item
  对于前向差分公式(在 \(x_0\) 处):

  \[
  \left\lvert E(x_0) \right\rvert = \left\lvert \frac{h^2}{3} f'''(\xi_0) \right\rvert \le \frac{h^2}{3}\, M_3 \approx 0.00250.
  \]
\item
  对于中央差分公式(在 \(x_1\) 处):

  \[
  \left\lvert E(x_1) \right\rvert = \left\lvert \frac{h^2}{6} f'''(\xi_1) \right\rvert \le \frac{h^2}{6}\, M_3 \approx 0.00125.
  \]
\item
  对于后向差分公式(在 \(x_2\) 处):

  \[
  \left\lvert E(x_2) \right\rvert = \left\lvert \frac{h^2}{3} f'''(\xi_2) \right\rvert \le \frac{h^2}{3}\, M_3 \approx 0.00250.
  \]
\end{itemize}

    \begin{tcolorbox}[breakable, size=fbox, boxrule=1pt, pad at break*=1mm,colback=cellbackground, colframe=cellborder]
\prompt{In}{incolor}{14}{\boxspacing}
\begin{Verbatim}[commandchars=\\\{\}]
\PY{k+kn}{import} \PY{n+nn}{sympy} \PY{k}{as} \PY{n+nn}{sp}
\PY{k+kn}{import} \PY{n+nn}{numpy} \PY{k}{as} \PY{n+nn}{np}

\PY{c+c1}{\PYZsh{} 定义符号变量和函数}
\PY{n}{x} \PY{o}{=} \PY{n}{sp}\PY{o}{.}\PY{n}{symbols}\PY{p}{(}\PY{l+s+s1}{\PYZsq{}}\PY{l+s+s1}{x}\PY{l+s+s1}{\PYZsq{}}\PY{p}{,} \PY{n}{real}\PY{o}{=}\PY{k+kc}{True}\PY{p}{)}
\PY{n}{f} \PY{o}{=} \PY{l+m+mi}{1}\PY{o}{/}\PY{p}{(}\PY{l+m+mi}{1} \PY{o}{+} \PY{n}{x}\PY{p}{)}\PY{o}{*}\PY{o}{*}\PY{l+m+mi}{2}

\PY{c+c1}{\PYZsh{} 求三阶导数}
\PY{n}{f3} \PY{o}{=} \PY{n}{sp}\PY{o}{.}\PY{n}{diff}\PY{p}{(}\PY{n}{f}\PY{p}{,} \PY{n}{x}\PY{p}{,} \PY{l+m+mi}{3}\PY{p}{)}
\PY{n+nb}{print}\PY{p}{(}\PY{l+s+s2}{\PYZdq{}}\PY{l+s+s2}{f}\PY{l+s+s2}{\PYZsq{}}\PY{l+s+s2}{\PYZsq{}}\PY{l+s+s2}{\PYZsq{}}\PY{l+s+s2}{(x) =}\PY{l+s+s2}{\PYZdq{}}\PY{p}{,} \PY{n}{sp}\PY{o}{.}\PY{n}{simplify}\PY{p}{(}\PY{n}{f3}\PY{p}{)}\PY{p}{)}

\PY{c+c1}{\PYZsh{} 取绝对值,并创建数值计算函数}
\PY{n}{f3\PYZus{}abs} \PY{o}{=} \PY{n}{sp}\PY{o}{.}\PY{n}{Abs}\PY{p}{(}\PY{n}{f3}\PY{p}{)}
\PY{n}{f3\PYZus{}abs\PYZus{}func} \PY{o}{=} \PY{n}{sp}\PY{o}{.}\PY{n}{lambdify}\PY{p}{(}\PY{n}{x}\PY{p}{,} \PY{n}{f3\PYZus{}abs}\PY{p}{,} \PY{l+s+s2}{\PYZdq{}}\PY{l+s+s2}{numpy}\PY{l+s+s2}{\PYZdq{}}\PY{p}{)}

\PY{c+c1}{\PYZsh{} 在区间 [1, 1.2] 上采样计算}
\PY{n}{xs} \PY{o}{=} \PY{n}{np}\PY{o}{.}\PY{n}{linspace}\PY{p}{(}\PY{l+m+mf}{1.0}\PY{p}{,} \PY{l+m+mf}{1.2}\PY{p}{,} \PY{l+m+mi}{10000}\PY{p}{)}
\PY{n}{f3\PYZus{}vals} \PY{o}{=} \PY{n}{f3\PYZus{}abs\PYZus{}func}\PY{p}{(}\PY{n}{xs}\PY{p}{)}
\PY{n}{M3} \PY{o}{=} \PY{n}{np}\PY{o}{.}\PY{n}{max}\PY{p}{(}\PY{n}{f3\PYZus{}vals}\PY{p}{)}
\PY{n+nb}{print}\PY{p}{(}\PY{l+s+s2}{\PYZdq{}}\PY{l+s+s2}{在 [1, 1.2] 上 |f}\PY{l+s+s2}{\PYZsq{}}\PY{l+s+s2}{\PYZsq{}}\PY{l+s+s2}{\PYZsq{}}\PY{l+s+s2}{(x)| 的最大值为:}\PY{l+s+s2}{\PYZdq{}}\PY{p}{,} \PY{n}{M3}\PY{p}{)}
\end{Verbatim}
\end{tcolorbox}

    \begin{Verbatim}[commandchars=\\\{\}]
f'''(x) = -24/(x + 1)**5
在 [1, 1.2] 上 |f'''(x)| 的最大值为: 0.75
    \end{Verbatim}

    \section{1}\label{section}

设 \(A\) 是对称矩阵且 \(a_{11}\not=0\),经过一步高斯消元后,\(A\) 约化为

\[
\begin{bmatrix}
a_{11} \quad \boldsymbol{a}_1^T \\
\boldsymbol{0} \quad \boldsymbol{A}_2
\end{bmatrix}
\]

证明:\(A_2\) 是对称矩阵。

    \subsection{Proof}\label{proof}

假设对称矩阵 \[
A = \begin{bmatrix}
a_{11} & \boldsymbol{a}_1^T \\
\boldsymbol{a}_1 & \tilde{A}
\end{bmatrix},
\] 其中 \(a_{11}\neq 0\) 且 \(\tilde{A}\) 对称(因为 \(A\) 对称)。

在第一步高斯消元中,我们用初等变换消去第一列 (除第一个分量外)
的元素。令初等矩阵 \[
E = \begin{bmatrix}
1 & \boldsymbol{0}^T \\
-\frac{1}{a_{11}}\boldsymbol{a}_1 & I
\end{bmatrix}.
\] 则消元后矩阵变为 \[
EA = 
\begin{bmatrix}
a_{11} & \boldsymbol{a}_1^T \\
\boldsymbol{0} & \tilde{A} - \frac{1}{a_{11}}\boldsymbol{a}_1 \boldsymbol{a}_1^T
\end{bmatrix}.
\] 记 \[
A_2 = \tilde{A} - \frac{1}{a_{11}}\boldsymbol{a}_1 \boldsymbol{a}_1^T.
\]

注意到: - \(\tilde{A}\) 是对称矩阵; - 外积
\(\boldsymbol{a}_1 \boldsymbol{a}_1^T\) 对称。

因此,它们的线性组合 \(A_2\) 也对称,即 \[
A_2^T = \tilde{A}^T - \frac{1}{a_{11}}(\boldsymbol{a}_1 \boldsymbol{a}_1^T)^T = \tilde{A} - \frac{1}{a_{11}} \boldsymbol{a}_1 \boldsymbol{a}_1^T = A_2.
\]

    \section{7}\label{section}

用列主元消去法解线性方程组

\[
\begin{cases}
12x_1 - 3x_2 + 3x_3 = 15 \\
-18x_1 + 3x_2 - x_3 = -15 \\
x_1 + x_2 + x_3 = 6
\end{cases}
\]

并求出系数矩阵 \(A\) 的行列式 \(\det(A)\)

    \subsection{Solution}\label{solution}

系数矩阵为

\[
A = \begin{pmatrix}
12 & -3 & 3 \\
-18 & 3 & -1 \\
1 & 1 & 1
\end{pmatrix}\,.
\]

通过消元得到上三角矩阵

\[
\begin{pmatrix}
12 & -3 & 3 & |\,15\\[1mm]
0 & -1.5 & 3.5 & |\,7.5\\[1mm]
0 & 0 & \tfrac{11}{3} & |\,11
\end{pmatrix}\,.
\]

解得 \[
(x_1,x_2,x_3)=(1,2,3).
\]

行列式等于各主元的乘积(在没有行交换的情况下): \[
\det(A)=12 \times (-1.5)\times \frac{11}{3} = -66\,.
\]


    % Add a bibliography block to the postdoc
    
    
    
\end{document}
