\documentclass[11pt]{article}

    \usepackage[breakable]{tcolorbox}
    \usepackage{parskip} % Stop auto-indenting (to mimic markdown behaviour)
    \usepackage{xeCJK}

    % Basic figure setup, for now with no caption control since it's done
    % automatically by Pandoc (which extracts ![](path) syntax from Markdown).
    \usepackage{graphicx}
    % Keep aspect ratio if custom image width or height is specified
    \setkeys{Gin}{keepaspectratio}
    % Maintain compatibility with old templates. Remove in nbconvert 6.0
    \let\Oldincludegraphics\includegraphics
    % Ensure that by default, figures have no caption (until we provide a
    % proper Figure object with a Caption API and a way to capture that
    % in the conversion process - todo).
    \usepackage{caption}
    \DeclareCaptionFormat{nocaption}{}
    \captionsetup{format=nocaption,aboveskip=0pt,belowskip=0pt}

    \usepackage{float}
    \floatplacement{figure}{H} % forces figures to be placed at the correct location
    \usepackage{xcolor} % Allow colors to be defined
    \usepackage{enumerate} % Needed for markdown enumerations to work
    \usepackage{geometry} % Used to adjust the document margins
    \usepackage{amsmath} % Equations
    \usepackage{amssymb} % Equations
    \usepackage{textcomp} % defines textquotesingle
    % Hack from http://tex.stackexchange.com/a/47451/13684:
    \AtBeginDocument{%
        \def\PYZsq{\textquotesingle}% Upright quotes in Pygmentized code
    }
    \usepackage{upquote} % Upright quotes for verbatim code
    \usepackage{eurosym} % defines \euro

    \usepackage{iftex}
    \ifPDFTeX
        \usepackage[T1]{fontenc}
        \IfFileExists{alphabeta.sty}{
              \usepackage{alphabeta}
          }{
              \usepackage[mathletters]{ucs}
              \usepackage[utf8x]{inputenc}
          }
    \else
        \usepackage{fontspec}
        \usepackage{unicode-math}
    \fi

    \usepackage{fancyvrb} % verbatim replacement that allows latex
    \usepackage{grffile} % extends the file name processing of package graphics
                         % to support a larger range
    \makeatletter % fix for old versions of grffile with XeLaTeX
    \@ifpackagelater{grffile}{2019/11/01}
    {
      % Do nothing on new versions
    }
    {
      \def\Gread@@xetex#1{%
        \IfFileExists{"\Gin@base".bb}%
        {\Gread@eps{\Gin@base.bb}}%
        {\Gread@@xetex@aux#1}%
      }
    }
    \makeatother
    \usepackage[Export]{adjustbox} % Used to constrain images to a maximum size
    \adjustboxset{max size={0.9\linewidth}{0.9\paperheight}}

    % The hyperref package gives us a pdf with properly built
    % internal navigation ('pdf bookmarks' for the table of contents,
    % internal cross-reference links, web links for URLs, etc.)
    \usepackage{hyperref}
    % The default LaTeX title has an obnoxious amount of whitespace. By default,
    % titling removes some of it. It also provides customization options.
    \usepackage{titling}
    \usepackage{longtable} % longtable support required by pandoc >1.10
    \usepackage{booktabs}  % table support for pandoc > 1.12.2
    \usepackage{array}     % table support for pandoc >= 2.11.3
    \usepackage{calc}      % table minipage width calculation for pandoc >= 2.11.1
    \usepackage[inline]{enumitem} % IRkernel/repr support (it uses the enumerate* environment)
    \usepackage[normalem]{ulem} % ulem is needed to support strikethroughs (\sout)
                                % normalem makes italics be italics, not underlines
    \usepackage{soul}      % strikethrough (\st) support for pandoc >= 3.0.0
    \usepackage{mathrsfs}
    

    
    % Colors for the hyperref package
    \definecolor{urlcolor}{rgb}{0,.145,.698}
    \definecolor{linkcolor}{rgb}{.71,0.21,0.01}
    \definecolor{citecolor}{rgb}{.12,.54,.11}

    % ANSI colors
    \definecolor{ansi-black}{HTML}{3E424D}
    \definecolor{ansi-black-intense}{HTML}{282C36}
    \definecolor{ansi-red}{HTML}{E75C58}
    \definecolor{ansi-red-intense}{HTML}{B22B31}
    \definecolor{ansi-green}{HTML}{00A250}
    \definecolor{ansi-green-intense}{HTML}{007427}
    \definecolor{ansi-yellow}{HTML}{DDB62B}
    \definecolor{ansi-yellow-intense}{HTML}{B27D12}
    \definecolor{ansi-blue}{HTML}{208FFB}
    \definecolor{ansi-blue-intense}{HTML}{0065CA}
    \definecolor{ansi-magenta}{HTML}{D160C4}
    \definecolor{ansi-magenta-intense}{HTML}{A03196}
    \definecolor{ansi-cyan}{HTML}{60C6C8}
    \definecolor{ansi-cyan-intense}{HTML}{258F8F}
    \definecolor{ansi-white}{HTML}{C5C1B4}
    \definecolor{ansi-white-intense}{HTML}{A1A6B2}
    \definecolor{ansi-default-inverse-fg}{HTML}{FFFFFF}
    \definecolor{ansi-default-inverse-bg}{HTML}{000000}

    % common color for the border for error outputs.
    \definecolor{outerrorbackground}{HTML}{FFDFDF}

    % commands and environments needed by pandoc snippets
    % extracted from the output of `pandoc -s`
    \providecommand{\tightlist}{%
      \setlength{\itemsep}{0pt}\setlength{\parskip}{0pt}}
    \DefineVerbatimEnvironment{Highlighting}{Verbatim}{commandchars=\\\{\}}
    % Add ',fontsize=\small' for more characters per line
    \newenvironment{Shaded}{}{}
    \newcommand{\KeywordTok}[1]{\textcolor[rgb]{0.00,0.44,0.13}{\textbf{{#1}}}}
    \newcommand{\DataTypeTok}[1]{\textcolor[rgb]{0.56,0.13,0.00}{{#1}}}
    \newcommand{\DecValTok}[1]{\textcolor[rgb]{0.25,0.63,0.44}{{#1}}}
    \newcommand{\BaseNTok}[1]{\textcolor[rgb]{0.25,0.63,0.44}{{#1}}}
    \newcommand{\FloatTok}[1]{\textcolor[rgb]{0.25,0.63,0.44}{{#1}}}
    \newcommand{\CharTok}[1]{\textcolor[rgb]{0.25,0.44,0.63}{{#1}}}
    \newcommand{\StringTok}[1]{\textcolor[rgb]{0.25,0.44,0.63}{{#1}}}
    \newcommand{\CommentTok}[1]{\textcolor[rgb]{0.38,0.63,0.69}{\textit{{#1}}}}
    \newcommand{\OtherTok}[1]{\textcolor[rgb]{0.00,0.44,0.13}{{#1}}}
    \newcommand{\AlertTok}[1]{\textcolor[rgb]{1.00,0.00,0.00}{\textbf{{#1}}}}
    \newcommand{\FunctionTok}[1]{\textcolor[rgb]{0.02,0.16,0.49}{{#1}}}
    \newcommand{\RegionMarkerTok}[1]{{#1}}
    \newcommand{\ErrorTok}[1]{\textcolor[rgb]{1.00,0.00,0.00}{\textbf{{#1}}}}
    \newcommand{\NormalTok}[1]{{#1}}

    % Additional commands for more recent versions of Pandoc
    \newcommand{\ConstantTok}[1]{\textcolor[rgb]{0.53,0.00,0.00}{{#1}}}
    \newcommand{\SpecialCharTok}[1]{\textcolor[rgb]{0.25,0.44,0.63}{{#1}}}
    \newcommand{\VerbatimStringTok}[1]{\textcolor[rgb]{0.25,0.44,0.63}{{#1}}}
    \newcommand{\SpecialStringTok}[1]{\textcolor[rgb]{0.73,0.40,0.53}{{#1}}}
    \newcommand{\ImportTok}[1]{{#1}}
    \newcommand{\DocumentationTok}[1]{\textcolor[rgb]{0.73,0.13,0.13}{\textit{{#1}}}}
    \newcommand{\AnnotationTok}[1]{\textcolor[rgb]{0.38,0.63,0.69}{\textbf{\textit{{#1}}}}}
    \newcommand{\CommentVarTok}[1]{\textcolor[rgb]{0.38,0.63,0.69}{\textbf{\textit{{#1}}}}}
    \newcommand{\VariableTok}[1]{\textcolor[rgb]{0.10,0.09,0.49}{{#1}}}
    \newcommand{\ControlFlowTok}[1]{\textcolor[rgb]{0.00,0.44,0.13}{\textbf{{#1}}}}
    \newcommand{\OperatorTok}[1]{\textcolor[rgb]{0.40,0.40,0.40}{{#1}}}
    \newcommand{\BuiltInTok}[1]{{#1}}
    \newcommand{\ExtensionTok}[1]{{#1}}
    \newcommand{\PreprocessorTok}[1]{\textcolor[rgb]{0.74,0.48,0.00}{{#1}}}
    \newcommand{\AttributeTok}[1]{\textcolor[rgb]{0.49,0.56,0.16}{{#1}}}
    \newcommand{\InformationTok}[1]{\textcolor[rgb]{0.38,0.63,0.69}{\textbf{\textit{{#1}}}}}
    \newcommand{\WarningTok}[1]{\textcolor[rgb]{0.38,0.63,0.69}{\textbf{\textit{{#1}}}}}
    \makeatletter
    \newsavebox\pandoc@box
    \newcommand*\pandocbounded[1]{%
      \sbox\pandoc@box{#1}%
      % scaling factors for width and height
      \Gscale@div\@tempa\textheight{\dimexpr\ht\pandoc@box+\dp\pandoc@box\relax}%
      \Gscale@div\@tempb\linewidth{\wd\pandoc@box}%
      % select the smaller of both
      \ifdim\@tempb\p@<\@tempa\p@
        \let\@tempa\@tempb
      \fi
      % scaling accordingly (\@tempa < 1)
      \ifdim\@tempa\p@<\p@
        \scalebox{\@tempa}{\usebox\pandoc@box}%
      % scaling not needed, use as it is
      \else
        \usebox{\pandoc@box}%
      \fi
    }
    \makeatother

    % Define a nice break command that doesn't care if a line doesn't already
    % exist.
    \def\br{\hspace*{\fill} \\* }
    % Math Jax compatibility definitions
    \def\gt{>}
    \def\lt{<}
    \let\Oldtex\TeX
    \let\Oldlatex\LaTeX
    \renewcommand{\TeX}{\textrm{\Oldtex}}
    \renewcommand{\LaTeX}{\textrm{\Oldlatex}}
    % Document parameters
    % Document title
    \title{assn01}
    
    
    
    
    
    
    
% Pygments definitions
\makeatletter
\def\PY@reset{\let\PY@it=\relax \let\PY@bf=\relax%
    \let\PY@ul=\relax \let\PY@tc=\relax%
    \let\PY@bc=\relax \let\PY@ff=\relax}
\def\PY@tok#1{\csname PY@tok@#1\endcsname}
\def\PY@toks#1+{\ifx\relax#1\empty\else%
    \PY@tok{#1}\expandafter\PY@toks\fi}
\def\PY@do#1{\PY@bc{\PY@tc{\PY@ul{%
    \PY@it{\PY@bf{\PY@ff{#1}}}}}}}
\def\PY#1#2{\PY@reset\PY@toks#1+\relax+\PY@do{#2}}

\@namedef{PY@tok@w}{\def\PY@tc##1{\textcolor[rgb]{0.73,0.73,0.73}{##1}}}
\@namedef{PY@tok@c}{\let\PY@it=\textit\def\PY@tc##1{\textcolor[rgb]{0.24,0.48,0.48}{##1}}}
\@namedef{PY@tok@cp}{\def\PY@tc##1{\textcolor[rgb]{0.61,0.40,0.00}{##1}}}
\@namedef{PY@tok@k}{\let\PY@bf=\textbf\def\PY@tc##1{\textcolor[rgb]{0.00,0.50,0.00}{##1}}}
\@namedef{PY@tok@kp}{\def\PY@tc##1{\textcolor[rgb]{0.00,0.50,0.00}{##1}}}
\@namedef{PY@tok@kt}{\def\PY@tc##1{\textcolor[rgb]{0.69,0.00,0.25}{##1}}}
\@namedef{PY@tok@o}{\def\PY@tc##1{\textcolor[rgb]{0.40,0.40,0.40}{##1}}}
\@namedef{PY@tok@ow}{\let\PY@bf=\textbf\def\PY@tc##1{\textcolor[rgb]{0.67,0.13,1.00}{##1}}}
\@namedef{PY@tok@nb}{\def\PY@tc##1{\textcolor[rgb]{0.00,0.50,0.00}{##1}}}
\@namedef{PY@tok@nf}{\def\PY@tc##1{\textcolor[rgb]{0.00,0.00,1.00}{##1}}}
\@namedef{PY@tok@nc}{\let\PY@bf=\textbf\def\PY@tc##1{\textcolor[rgb]{0.00,0.00,1.00}{##1}}}
\@namedef{PY@tok@nn}{\let\PY@bf=\textbf\def\PY@tc##1{\textcolor[rgb]{0.00,0.00,1.00}{##1}}}
\@namedef{PY@tok@ne}{\let\PY@bf=\textbf\def\PY@tc##1{\textcolor[rgb]{0.80,0.25,0.22}{##1}}}
\@namedef{PY@tok@nv}{\def\PY@tc##1{\textcolor[rgb]{0.10,0.09,0.49}{##1}}}
\@namedef{PY@tok@no}{\def\PY@tc##1{\textcolor[rgb]{0.53,0.00,0.00}{##1}}}
\@namedef{PY@tok@nl}{\def\PY@tc##1{\textcolor[rgb]{0.46,0.46,0.00}{##1}}}
\@namedef{PY@tok@ni}{\let\PY@bf=\textbf\def\PY@tc##1{\textcolor[rgb]{0.44,0.44,0.44}{##1}}}
\@namedef{PY@tok@na}{\def\PY@tc##1{\textcolor[rgb]{0.41,0.47,0.13}{##1}}}
\@namedef{PY@tok@nt}{\let\PY@bf=\textbf\def\PY@tc##1{\textcolor[rgb]{0.00,0.50,0.00}{##1}}}
\@namedef{PY@tok@nd}{\def\PY@tc##1{\textcolor[rgb]{0.67,0.13,1.00}{##1}}}
\@namedef{PY@tok@s}{\def\PY@tc##1{\textcolor[rgb]{0.73,0.13,0.13}{##1}}}
\@namedef{PY@tok@sd}{\let\PY@it=\textit\def\PY@tc##1{\textcolor[rgb]{0.73,0.13,0.13}{##1}}}
\@namedef{PY@tok@si}{\let\PY@bf=\textbf\def\PY@tc##1{\textcolor[rgb]{0.64,0.35,0.47}{##1}}}
\@namedef{PY@tok@se}{\let\PY@bf=\textbf\def\PY@tc##1{\textcolor[rgb]{0.67,0.36,0.12}{##1}}}
\@namedef{PY@tok@sr}{\def\PY@tc##1{\textcolor[rgb]{0.64,0.35,0.47}{##1}}}
\@namedef{PY@tok@ss}{\def\PY@tc##1{\textcolor[rgb]{0.10,0.09,0.49}{##1}}}
\@namedef{PY@tok@sx}{\def\PY@tc##1{\textcolor[rgb]{0.00,0.50,0.00}{##1}}}
\@namedef{PY@tok@m}{\def\PY@tc##1{\textcolor[rgb]{0.40,0.40,0.40}{##1}}}
\@namedef{PY@tok@gh}{\let\PY@bf=\textbf\def\PY@tc##1{\textcolor[rgb]{0.00,0.00,0.50}{##1}}}
\@namedef{PY@tok@gu}{\let\PY@bf=\textbf\def\PY@tc##1{\textcolor[rgb]{0.50,0.00,0.50}{##1}}}
\@namedef{PY@tok@gd}{\def\PY@tc##1{\textcolor[rgb]{0.63,0.00,0.00}{##1}}}
\@namedef{PY@tok@gi}{\def\PY@tc##1{\textcolor[rgb]{0.00,0.52,0.00}{##1}}}
\@namedef{PY@tok@gr}{\def\PY@tc##1{\textcolor[rgb]{0.89,0.00,0.00}{##1}}}
\@namedef{PY@tok@ge}{\let\PY@it=\textit}
\@namedef{PY@tok@gs}{\let\PY@bf=\textbf}
\@namedef{PY@tok@gp}{\let\PY@bf=\textbf\def\PY@tc##1{\textcolor[rgb]{0.00,0.00,0.50}{##1}}}
\@namedef{PY@tok@go}{\def\PY@tc##1{\textcolor[rgb]{0.44,0.44,0.44}{##1}}}
\@namedef{PY@tok@gt}{\def\PY@tc##1{\textcolor[rgb]{0.00,0.27,0.87}{##1}}}
\@namedef{PY@tok@err}{\def\PY@bc##1{{\setlength{\fboxsep}{\string -\fboxrule}\fcolorbox[rgb]{1.00,0.00,0.00}{1,1,1}{\strut ##1}}}}
\@namedef{PY@tok@kc}{\let\PY@bf=\textbf\def\PY@tc##1{\textcolor[rgb]{0.00,0.50,0.00}{##1}}}
\@namedef{PY@tok@kd}{\let\PY@bf=\textbf\def\PY@tc##1{\textcolor[rgb]{0.00,0.50,0.00}{##1}}}
\@namedef{PY@tok@kn}{\let\PY@bf=\textbf\def\PY@tc##1{\textcolor[rgb]{0.00,0.50,0.00}{##1}}}
\@namedef{PY@tok@kr}{\let\PY@bf=\textbf\def\PY@tc##1{\textcolor[rgb]{0.00,0.50,0.00}{##1}}}
\@namedef{PY@tok@bp}{\def\PY@tc##1{\textcolor[rgb]{0.00,0.50,0.00}{##1}}}
\@namedef{PY@tok@fm}{\def\PY@tc##1{\textcolor[rgb]{0.00,0.00,1.00}{##1}}}
\@namedef{PY@tok@vc}{\def\PY@tc##1{\textcolor[rgb]{0.10,0.09,0.49}{##1}}}
\@namedef{PY@tok@vg}{\def\PY@tc##1{\textcolor[rgb]{0.10,0.09,0.49}{##1}}}
\@namedef{PY@tok@vi}{\def\PY@tc##1{\textcolor[rgb]{0.10,0.09,0.49}{##1}}}
\@namedef{PY@tok@vm}{\def\PY@tc##1{\textcolor[rgb]{0.10,0.09,0.49}{##1}}}
\@namedef{PY@tok@sa}{\def\PY@tc##1{\textcolor[rgb]{0.73,0.13,0.13}{##1}}}
\@namedef{PY@tok@sb}{\def\PY@tc##1{\textcolor[rgb]{0.73,0.13,0.13}{##1}}}
\@namedef{PY@tok@sc}{\def\PY@tc##1{\textcolor[rgb]{0.73,0.13,0.13}{##1}}}
\@namedef{PY@tok@dl}{\def\PY@tc##1{\textcolor[rgb]{0.73,0.13,0.13}{##1}}}
\@namedef{PY@tok@s2}{\def\PY@tc##1{\textcolor[rgb]{0.73,0.13,0.13}{##1}}}
\@namedef{PY@tok@sh}{\def\PY@tc##1{\textcolor[rgb]{0.73,0.13,0.13}{##1}}}
\@namedef{PY@tok@s1}{\def\PY@tc##1{\textcolor[rgb]{0.73,0.13,0.13}{##1}}}
\@namedef{PY@tok@mb}{\def\PY@tc##1{\textcolor[rgb]{0.40,0.40,0.40}{##1}}}
\@namedef{PY@tok@mf}{\def\PY@tc##1{\textcolor[rgb]{0.40,0.40,0.40}{##1}}}
\@namedef{PY@tok@mh}{\def\PY@tc##1{\textcolor[rgb]{0.40,0.40,0.40}{##1}}}
\@namedef{PY@tok@mi}{\def\PY@tc##1{\textcolor[rgb]{0.40,0.40,0.40}{##1}}}
\@namedef{PY@tok@il}{\def\PY@tc##1{\textcolor[rgb]{0.40,0.40,0.40}{##1}}}
\@namedef{PY@tok@mo}{\def\PY@tc##1{\textcolor[rgb]{0.40,0.40,0.40}{##1}}}
\@namedef{PY@tok@ch}{\let\PY@it=\textit\def\PY@tc##1{\textcolor[rgb]{0.24,0.48,0.48}{##1}}}
\@namedef{PY@tok@cm}{\let\PY@it=\textit\def\PY@tc##1{\textcolor[rgb]{0.24,0.48,0.48}{##1}}}
\@namedef{PY@tok@cpf}{\let\PY@it=\textit\def\PY@tc##1{\textcolor[rgb]{0.24,0.48,0.48}{##1}}}
\@namedef{PY@tok@c1}{\let\PY@it=\textit\def\PY@tc##1{\textcolor[rgb]{0.24,0.48,0.48}{##1}}}
\@namedef{PY@tok@cs}{\let\PY@it=\textit\def\PY@tc##1{\textcolor[rgb]{0.24,0.48,0.48}{##1}}}

\def\PYZbs{\char`\\}
\def\PYZus{\char`\_}
\def\PYZob{\char`\{}
\def\PYZcb{\char`\}}
\def\PYZca{\char`\^}
\def\PYZam{\char`\&}
\def\PYZlt{\char`\<}
\def\PYZgt{\char`\>}
\def\PYZsh{\char`\#}
\def\PYZpc{\char`\%}
\def\PYZdl{\char`\$}
\def\PYZhy{\char`\-}
\def\PYZsq{\char`\'}
\def\PYZdq{\char`\"}
\def\PYZti{\char`\~}
% for compatibility with earlier versions
\def\PYZat{@}
\def\PYZlb{[}
\def\PYZrb{]}
\makeatother


    % For linebreaks inside Verbatim environment from package fancyvrb.
    \makeatletter
        \newbox\Wrappedcontinuationbox
        \newbox\Wrappedvisiblespacebox
        \newcommand*\Wrappedvisiblespace {\textcolor{red}{\textvisiblespace}}
        \newcommand*\Wrappedcontinuationsymbol {\textcolor{red}{\llap{\tiny$\m@th\hookrightarrow$}}}
        \newcommand*\Wrappedcontinuationindent {3ex }
        \newcommand*\Wrappedafterbreak {\kern\Wrappedcontinuationindent\copy\Wrappedcontinuationbox}
        % Take advantage of the already applied Pygments mark-up to insert
        % potential linebreaks for TeX processing.
        %        {, <, #, %, $, ' and ": go to next line.
        %        _, }, ^, &, >, - and ~: stay at end of broken line.
        % Use of \textquotesingle for straight quote.
        \newcommand*\Wrappedbreaksatspecials {%
            \def\PYGZus{\discretionary{\char`\_}{\Wrappedafterbreak}{\char`\_}}%
            \def\PYGZob{\discretionary{}{\Wrappedafterbreak\char`\{}{\char`\{}}%
            \def\PYGZcb{\discretionary{\char`\}}{\Wrappedafterbreak}{\char`\}}}%
            \def\PYGZca{\discretionary{\char`\^}{\Wrappedafterbreak}{\char`\^}}%
            \def\PYGZam{\discretionary{\char`\&}{\Wrappedafterbreak}{\char`\&}}%
            \def\PYGZlt{\discretionary{}{\Wrappedafterbreak\char`\<}{\char`\<}}%
            \def\PYGZgt{\discretionary{\char`\>}{\Wrappedafterbreak}{\char`\>}}%
            \def\PYGZsh{\discretionary{}{\Wrappedafterbreak\char`\#}{\char`\#}}%
            \def\PYGZpc{\discretionary{}{\Wrappedafterbreak\char`\%}{\char`\%}}%
            \def\PYGZdl{\discretionary{}{\Wrappedafterbreak\char`\$}{\char`\$}}%
            \def\PYGZhy{\discretionary{\char`\-}{\Wrappedafterbreak}{\char`\-}}%
            \def\PYGZsq{\discretionary{}{\Wrappedafterbreak\textquotesingle}{\textquotesingle}}%
            \def\PYGZdq{\discretionary{}{\Wrappedafterbreak\char`\"}{\char`\"}}%
            \def\PYGZti{\discretionary{\char`\~}{\Wrappedafterbreak}{\char`\~}}%
        }
        % Some characters . , ; ? ! / are not pygmentized.
        % This macro makes them "active" and they will insert potential linebreaks
        \newcommand*\Wrappedbreaksatpunct {%
            \lccode`\~`\.\lowercase{\def~}{\discretionary{\hbox{\char`\.}}{\Wrappedafterbreak}{\hbox{\char`\.}}}%
            \lccode`\~`\,\lowercase{\def~}{\discretionary{\hbox{\char`\,}}{\Wrappedafterbreak}{\hbox{\char`\,}}}%
            \lccode`\~`\;\lowercase{\def~}{\discretionary{\hbox{\char`\;}}{\Wrappedafterbreak}{\hbox{\char`\;}}}%
            \lccode`\~`\:\lowercase{\def~}{\discretionary{\hbox{\char`\:}}{\Wrappedafterbreak}{\hbox{\char`\:}}}%
            \lccode`\~`\?\lowercase{\def~}{\discretionary{\hbox{\char`\?}}{\Wrappedafterbreak}{\hbox{\char`\?}}}%
            \lccode`\~`\!\lowercase{\def~}{\discretionary{\hbox{\char`\!}}{\Wrappedafterbreak}{\hbox{\char`\!}}}%
            \lccode`\~`\/\lowercase{\def~}{\discretionary{\hbox{\char`\/}}{\Wrappedafterbreak}{\hbox{\char`\/}}}%
            \catcode`\.\active
            \catcode`\,\active
            \catcode`\;\active
            \catcode`\:\active
            \catcode`\?\active
            \catcode`\!\active
            \catcode`\/\active
            \lccode`\~`\~
        }
    \makeatother

    \let\OriginalVerbatim=\Verbatim
    \makeatletter
    \renewcommand{\Verbatim}[1][1]{%
        %\parskip\z@skip
        \sbox\Wrappedcontinuationbox {\Wrappedcontinuationsymbol}%
        \sbox\Wrappedvisiblespacebox {\FV@SetupFont\Wrappedvisiblespace}%
        \def\FancyVerbFormatLine ##1{\hsize\linewidth
            \vtop{\raggedright\hyphenpenalty\z@\exhyphenpenalty\z@
                \doublehyphendemerits\z@\finalhyphendemerits\z@
                \strut ##1\strut}%
        }%
        % If the linebreak is at a space, the latter will be displayed as visible
        % space at end of first line, and a continuation symbol starts next line.
        % Stretch/shrink are however usually zero for typewriter font.
        \def\FV@Space {%
            \nobreak\hskip\z@ plus\fontdimen3\font minus\fontdimen4\font
            \discretionary{\copy\Wrappedvisiblespacebox}{\Wrappedafterbreak}
            {\kern\fontdimen2\font}%
        }%

        % Allow breaks at special characters using \PYG... macros.
        \Wrappedbreaksatspecials
        % Breaks at punctuation characters . , ; ? ! and / need catcode=\active
        \OriginalVerbatim[#1,codes*=\Wrappedbreaksatpunct]%
    }
    \makeatother

    % Exact colors from NB
    \definecolor{incolor}{HTML}{303F9F}
    \definecolor{outcolor}{HTML}{D84315}
    \definecolor{cellborder}{HTML}{CFCFCF}
    \definecolor{cellbackground}{HTML}{F7F7F7}

    % prompt
    \makeatletter
    \newcommand{\boxspacing}{\kern\kvtcb@left@rule\kern\kvtcb@boxsep}
    \makeatother
    \newcommand{\prompt}[4]{
        {\ttfamily\llap{{\color{#2}[#3]:\hspace{3pt}#4}}\vspace{-\baselineskip}}
    }
    

    
    % Prevent overflowing lines due to hard-to-break entities
    \sloppy
    % Setup hyperref package
    \hypersetup{
      breaklinks=true,  % so long urls are correctly broken across lines
      colorlinks=true,
      urlcolor=urlcolor,
      linkcolor=linkcolor,
      citecolor=citecolor,
      }
    % Slightly bigger margins than the latex defaults
    
    \geometry{verbose,tmargin=1in,bmargin=1in,lmargin=1in,rmargin=1in}
    
    

\begin{document}
    
    \maketitle
    
    

    
    \section{2}\label{section}

设 \(x\) 的相对误差为 \(2\%\),求 \(x^n\) 的相对误差。

\subsection{Answer}\label{answer}

已知 \(x\) 的相对误差为 \[
\frac{|x^*-x|}{|x|}=0.02,
\]

考虑函数 \[
f(x)=x^n.
\] 利用泰勒展开,并忽略高阶项,\(x^*\) 对应的 \(f(x^*)\) 可近似写为 \[
f(x^*)\approx f(x)+f'(x)(x^*-x),
\] 其中 \[
f'(x)=n\,x^{n-1}.
\] 因此,\(f(x)=x^n\) 的绝对误差近似为 \[
f^* \approx |f'(x)||(x^*-x)| = n\,x^{n-1}\,|x^*-x|.
\] 将 \(f(x)=x^n\) 代入,可得相对误差 \[
\frac{f^*}{f(x)} \approx \frac{n\,x^{n-1}\,|x^*-x|}{x^n}
=n\,\frac{|x^*-x|}{x}.
\] 带入得, \[
\frac{f^*}{f(x)}\approx n\cdot 0.02.
\] 即,\(x^n\) 的相对误差为 \[
\boxed{2\% \times n.}
\]

    \section{5}\label{section}

计算球体的体积要使相对误差限为 1\%,那么测量半径 \(R\)
的允许相对误差是多少?

\subsection{Answer}\label{answer}

球体体积的公式为 \[
V=\frac{4}{3}\pi R^3.
\] 对 \(R\) 的误差进行泰勒展开近似,可得体积的相对误差为 \[
\frac{V*}{V}\approx 3\,\frac{R^*}{R}.
\] 体积的相对误差不超过 1\%,即 \[
\frac{V^*}{V}\le 0.01.
\] 因此有: \[
3\,\frac{R^*}{R}\le 0.01.
\] 解得 \[
\frac{R^*}{R}\le \frac{0.01}{3}\approx 0.00333,
\]

所以,测量半径 \(R\) 的允许相对误差为 \[
\boxed{0.33\%.}
\]

    \section{7}\label{section}

求方程的两个根,使它至少具有 4 位有效数字 \[
x^2 - 56x + 1 = 0
\] (\(\sqrt{783}\approx 27.982\))

\subsection{Answer}\label{answer}

使用求根公式,

\[
x = \frac{-b \pm \sqrt{b^2 - 4ac}}{2a}
\] 当 \(a=1\), \(b=-56\), \(c=1\) 时 \[
x = \frac{56 \pm \sqrt{56^2 - 4\cdot1\cdot1}}{2} = \frac{56 \pm \sqrt{3132}}{2} = 28 \pm \sqrt{783}.
\]

因此两根近似值为

\[
\boxed{
\begin{aligned}
x_0 &\approx 28 + 27.982 = 55.982,\\
x_1 &\approx 28 - 27.982=0.018
\end{aligned}
}
\]

    \section{9}\label{section}

正方形的边长大约为 \(100\,\mathrm{cm}\), 怎样测量才能使其面积误差不超过
\(1\,\mathrm{cm}^2\)?

\subsection{Answer}\label{answer}

设正方形边长为 \(L\) 绝对误差为 \(L^*\), 则面积 \(A = L²\)
的绝对误差为为: \[
A^* \approx 2L\,L^*.
\] 由题意知, \[
2L\,L^* \leq 1.
\] 令 \(L ≈ 100 cm\), 有 \[
2 \times 100\, L^* \leq 1 \quad \Longrightarrow \quad 200\,L^* \leq 1.
\] 所以, \[
\boxed{
L^* \leq \frac{1}{200} = 0.005\,\mathrm{cm}.
}
\]

    \section{11}\label{section}

给定序列 \(\{y_n\}\) 满足递推关系 \[
y_n = 10\,y_{n-1} - 1,\quad n = 1,2,\cdots,
\] 并且初值
\(y_0 = \sqrt{2} \approx 1.41\)(保留三位有效数字),求计算到 \(y_{10}\)
时误差有多大?这个计算过程是否稳定?

\subsection{Answer}\label{answer}

由递推关系知,每次迭代,误差变化为

\[
|y^*_n-y_n| = 10|y_{n-1}^* - y_{n-1}|
\]

即每次迭代会将绝对误差放大 10 倍。

初值的绝对误差为 \[
\left|y_0^* -y_0\right| = \left|1.41 - \sqrt{2}\right| \lesssim \frac{1}{2}\times 10^{-2}.
\]

经过 \(n\) 步迭代后,误差放大为 \[
|y_n^* - y_n| = 10^n\, |y_0^* - y_0|.
\]

当 \(n = 10\) 时,误差约为 \[
|y_{10}^* - y_{10}| \approx 10^{10}\,|y_0^* - y_0| \approx 5\times10^{7}.
\]

绝对误差的数量级大约为 \(5\times10^7\),远大于合理的计算值。

这种初始误差的指数级放大表明,该计算过程是不稳定的。

    \begin{tcolorbox}[breakable, size=fbox, boxrule=1pt, pad at break*=1mm,colback=cellbackground, colframe=cellborder]
\prompt{In}{incolor}{2}{\boxspacing}
\begin{Verbatim}[commandchars=\\\{\}]
\PY{k+kn}{import} \PY{n+nn}{numpy} \PY{k}{as} \PY{n+nn}{np}

\PY{n}{y\PYZus{}exact} \PY{o}{=} \PY{n}{np}\PY{o}{.}\PY{n}{sqrt}\PY{p}{(}\PY{l+m+mi}{2}\PY{p}{)}
\PY{n}{y\PYZus{}approx} \PY{o}{=} \PY{l+m+mf}{1.41}

\PY{n}{n} \PY{o}{=} \PY{l+m+mi}{10}

\PY{k}{for} \PY{n}{i} \PY{o+ow}{in} \PY{n+nb}{range}\PY{p}{(}\PY{l+m+mi}{1}\PY{p}{,} \PY{n}{n}\PY{o}{+}\PY{l+m+mi}{1}\PY{p}{)}\PY{p}{:}
    \PY{n}{y\PYZus{}exact} \PY{o}{=} \PY{l+m+mi}{10} \PY{o}{*} \PY{n}{y\PYZus{}exact} \PY{o}{\PYZhy{}} \PY{l+m+mi}{1}
    \PY{n}{y\PYZus{}approx} \PY{o}{=} \PY{l+m+mi}{10} \PY{o}{*} \PY{n}{y\PYZus{}approx} \PY{o}{\PYZhy{}} \PY{l+m+mi}{1}

\PY{n}{error} \PY{o}{=} \PY{n+nb}{abs}\PY{p}{(}\PY{n}{y\PYZus{}exact} \PY{o}{\PYZhy{}} \PY{n}{y\PYZus{}approx}\PY{p}{)}

\PY{n+nb}{print}\PY{p}{(}\PY{l+s+s2}{\PYZdq{}}\PY{l+s+s2}{Exact y10:   }\PY{l+s+s2}{\PYZdq{}}\PY{p}{,} \PY{n}{y\PYZus{}exact}\PY{p}{)}
\PY{n+nb}{print}\PY{p}{(}\PY{l+s+s2}{\PYZdq{}}\PY{l+s+s2}{Approx y10:  }\PY{l+s+s2}{\PYZdq{}}\PY{p}{,} \PY{n}{y\PYZus{}approx}\PY{p}{)}
\PY{n+nb}{print}\PY{p}{(}\PY{l+s+s2}{\PYZdq{}}\PY{l+s+s2}{Error in y10:}\PY{l+s+s2}{\PYZdq{}}\PY{p}{,} \PY{n}{error}\PY{p}{)}
\end{Verbatim}
\end{tcolorbox}

    \begin{Verbatim}[commandchars=\\\{\}]
Exact y10:    13031024512.73095
Approx y10:   12988888889.0
Error in y10: 42135623.7309494
    \end{Verbatim}

    程序运行结果表明,估计是准确的。

    \subsection{12}\label{section}

计算\\
\[
f = (\sqrt{2}-1)^6,
\] 取近似值 \(\sqrt{2}\approx 1.4\),并利用下列等式计算\\
\[
\frac{1}{(1+\sqrt{2})^6},\quad (3-2\sqrt{2})^3,\quad \frac{1}{(3+2\sqrt{2})^3},\quad 99-70\sqrt{2},
\] 哪一个得到的结果最好?

\section{Answer}\label{answer}

\[
\text{若 } f(x) \text{ 依赖于 } x,\quad f^* \approx |f'(x)||x^*-x|,\quad \text{则相对误差为 } \frac{f^*}{f(x)}\approx\left|\frac{d\ln f}{dx}\right||x^*-x|.
\]

令 \(x=\sqrt{2}\), \(x^* = 1.4\),则
\(|x^*-x|\lesssim 0.5\times10^{-1}\)

\begin{enumerate}
\def\labelenumi{\arabic{enumi}.}
\item
  \textbf{\(f=\frac{1}{(1+x)^6}\)}

  记 \(g(x)=(1+x)^{-6}\)。则 \[
  \ln g(x)=-6\ln(1+x),\quad \frac{d\ln g(x)}{dx}=-\frac{6}{1+x}.
  \] 用近似值 \(x\approx1.4\) 得 \[
  \left|\frac{d\ln g}{dx}\right|\approx \frac{6}{1+1.4}=\frac{6}{2.4}=2.5.
  \] 即近似 \(\sqrt{2}\) 的相对误差会被放大约 2.5 倍。
\item
  \textbf{\(f=(3-2x)^3\)}

  令 \(h(x)=(3-2x)^3\)。则 \[
  \ln h(x)=3\ln(3-2x),\quad \frac{d\ln h(x)}{dx}=\frac{-6}{3-2x}.
  \] 对 \(x\approx1.4\) 有 \[
  \left|\frac{d\ln h(x)}{dx}\right|\approx \frac{6}{0.2}=30.
  \] 这表示相对误差被放大 30 倍。
\item
  \textbf{\(f=\frac{1}{(3+2x)^3}\)}

  记 \(q(x)=(3+2x)^{-3}\)。则 \[
  \ln q(x)=-3\ln(3+2x),\quad \frac{d\ln q(x)}{dx}=-\frac{6}{3+2x}.
  \] 当 \(x\approx1.4\) 时, \[
  \left|\frac{d\ln q(x)}{dx}\right|\approx \frac{6}{5.8}\approx 1.03.
  \] 相对误差仅放大约 1.03 倍。
\item
  \textbf{\(f=99-70x\)} 记 \(p(x) = 99-70x\)。则 \[
  \ln p(x) = \ln(99 -70x), \quad \frac{d\ln p(x)}{dx} = -\frac{70}{99-70x}
  \] 当 \(x\approx 1.4\) 时, \[
  \left|\frac{d\ln p(x)}{dx}\right|\approx \frac{70}{1}=70
  \] 相对误差被放大 70 倍。
\end{enumerate}

\textbf{结论:}

在这四种表达中,使用\\
\[
\boxed{
f=\frac{1}{(3+2\sqrt{2})^3}
}
\] 的表达对 \(\sqrt{2}\) 的近似误差最不敏感,因此给出的计算结果最准确。

    \section{14}\label{section}

用秦九韶算法求多项式 \[
p(x)=3x^5-2x^3+x+7
\] 在 \(x=3\) 处的值。

\subsection{Answer}\label{answer}

\[
\begin{aligned}
p(3) &= 3\cdot 3^5 + 0\cdot 3^4 -2\cdot 3^3 + 0\cdot 3^2 + 1\cdot 3 + 7 \\\\
&= (((3\cdot 3 + 0)\cdot 3 -2)\cdot 3 + 0)\cdot 3 + 1)\cdot 3 + 7 \\\\
\text{故 } b_5 &= 3, \\\\
b_4 &= 3\cdot 3 + 0 = 9, \\\\
b_3 &= 9\cdot 3 - 2 = 27 - 2 = 25, \\\\
b_2 &= 25\cdot 3 + 0 = 75, \\\\
b_1 &= 75\cdot 3 + 1 = 225 + 1 = 226, \\\\
b_0 &= 226\cdot 3 + 7 = 678 + 7 = 685.
\end{aligned}
\]

因此, \[
\boxed{
p(3) = 685.
}
\]

    \section{1}\label{section}

当 \(x = 1, -1, 2\) 时 \[
f(1)=0,\quad f(-1)=-3,\quad f(2)=4,
\] 求 \(f(x)\) 的二次插值多项式.

\subsection{(1)
用单项式基底}\label{ux7528ux5355ux9879ux5f0fux57faux5e95}

\subsubsection{Answer}\label{answer}

\[
\begin{aligned}
\text{令 } p(x)&=ax^2+bx+c.\\[1mm]
\text{得 } x=1:&\quad a+b+c=0,\\[1mm]
x=-1:&\quad a-b+c=-3,\\[1mm]
x=2:&\quad 4a+2b+c=4.
\end{aligned}
\]

因此解得

\[
\begin{aligned}
a &= \frac{5}{6} \\
b &= \frac{3}{2} \\
c &= -\frac{7}{3}
\end{aligned}
\]

因此, \[
\boxed{
p(x)=\frac{5}{6}x^2+\frac{3}{2}x-\frac{7}{3}.
}
\]

\subsection{(2)
使用拉格朗日基底}\label{ux4f7fux7528ux62c9ux683cux6717ux65e5ux57faux5e95}

\subsubsection{Answer}\label{answer-1}

\[
(1,0),\quad (-1,-3),\quad (2,4).
\] 拉格朗日多项式表示如下 \[
p(x)=\sum_{j=1}^3 f(x_j)L_j(x),
\] 拉格朗日函数为 \[
L_j(x)=\prod_{\substack{i=1 \\ i\neq j}}^{3}\frac{x-x_i}{x_j-x_i}.
\]

设 \[
x_1=1,\quad x_2=-1,\quad x_3=2,
\]

\[
f(1)=0,\quad f(-1)=-3,\quad f(2)=4,
\]

计算:

\begin{enumerate}
\def\labelenumi{\arabic{enumi}.}
\item
  \(x_1=1\): \[
  L_1(x)=\frac{(x-x_2)(x-x_3)}{(1-(-1))(1-2)}=\frac{(x+1)(x-2)}{(2)(-1)}=-\frac{(x+1)(x-2)}{2}.
  \]
\item
  \(x_2=-1\): \[
  L_2(x)=\frac{(x-x_1)(x-x_3)}{(-1-1)(-1-2)}=\frac{(x-1)(x-2)}{(-2)(-3)}=\frac{(x-1)(x-2)}{6}.
  \]
\item
  \(x_3=2\): \[
  L_3(x)=\frac{(x-x_1)(x-x_2)}{(2-1)(2-(-1))}=\frac{(x-1)(x+1)}{(1)(3)}=\frac{(x-1)(x+1)}{3}.
  \]
\end{enumerate}

故多项式可表示为 \[
p(x)=0\cdot L_1(x)-3\cdot L_2(x)+4\cdot L_3(x).
\] 即 \[
\boxed{
p(x)=-3\cdot \frac{(x-1)(x-2)}{6}+4\cdot \frac{(x-1)(x+1)}{3}.
}
\]

    \section{3}\label{section}

给出 \(f(x)=\ln x\) 的数值表:

\[
\begin{array}{|c|c|c|c|c|c|}
\hline
x & 0.4 & 0.5 & 0.6 & 0.7 & 0.8 \\
\hline
\ln x & -0.916291 & -0.693147 & -0.510826 & -0.356675 & -0.223144 \\
\hline
\end{array}
\]

使用线性插值和二次插值计算 \(\ln(0.54)\) 的近似值.

\subsection{Answer}\label{answer}

\[\textbf{线性插值}\]

选取 \(x_0=0.5\),\(f(x_0)=\ln(0.5)=-0.693147\) 以及
\(x_1=0.6\),\(f(x_1)=\ln(0.6)=-0.510826\). 有

\[
\ln(0.54) \approx \ln(0.5) + \frac{\ln(0.6)-\ln(0.5)}{0.6-0.5}\,(0.54-0.5).
\]

即,

\[
\ln(0.54) \approx -0.693147 + \frac{-0.510826 + 0.693147}{0.1}\,(0.04)
\]

\[
\ln(0.54) \approx -0.693147 + \frac{0.182321}{0.1}\,(0.04)
\]

\[
\ln(0.54) \approx -0.693147 + 1.82321 \times 0.04 \approx -0.693147 + 0.072928 \approx -0.620219.
\]

\[
\textbf{二次插值}
\]

选取三个点 \(x_0=0.5\), \(x_1=0.6\), \(x_2=0.7\),

\[
f(0.5)=-0.693147,\quad f(0.6)=-0.510826,\quad f(0.7)=-0.356675.
\]

\(x=0.54\) 时,拉格朗日基底为:

\[
L_0(0.54)=\frac{(0.54-0.6)(0.54-0.7)}{(0.5-0.6)(0.5-0.7)}
=\frac{(-0.06)(-0.16)}{(-0.1)(-0.2)}
=\frac{0.0096}{0.02}=0.48,
\]

\[
L_1(0.54)=\frac{(0.54-0.5)(0.54-0.7)}{(0.6-0.5)(0.6-0.7)}
=\frac{(0.04)(-0.16)}{(0.1)(-0.1)}
=\frac{-0.0064}{-0.01}=0.64,
\]

\[
L_2(0.54)=\frac{(0.54-0.5)(0.54-0.6)}{(0.7-0.5)(0.7-0.6)}
=\frac{(0.04)(-0.06)}{(0.2)(0.1)}
=\frac{-0.0024}{0.02}=-0.12.
\]

故,二次插值估计为

\[
\ln(0.54) \approx f(0.5)L_0(0.54) + f(0.6)L_1(0.54) + f(0.7)L_2(0.54)
\]

\[
\ln(0.54) \approx (-0.693147)(0.48) + (-0.510826)(0.64) + (-0.356675)(-0.12).
\]

\[
\ln(0.54) \approx -0.332711 - 0.326928 + 0.042801 \approx -0.616838.
\]

因此,

\[
\boxed{\begin{array}{ll}
\text{线性插值:} & \ln(0.54)\approx -0.6202, \\
\text{二次插值:} & \ln(0.54)\approx -0.6168.
\end{array}}
\]


    % Add a bibliography block to the postdoc
    
    
    
\end{document}
