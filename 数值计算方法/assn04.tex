\documentclass[11pt]{article}

    \usepackage[breakable]{tcolorbox}
    \usepackage{parskip} % Stop auto-indenting (to mimic markdown behaviour)
    \usepackage{xeCJK}

    % Basic figure setup, for now with no caption control since it's done
    % automatically by Pandoc (which extracts ![](path) syntax from Markdown).
    \usepackage{graphicx}
    % Keep aspect ratio if custom image width or height is specified
    \setkeys{Gin}{keepaspectratio}
    % Maintain compatibility with old templates. Remove in nbconvert 6.0
    \let\Oldincludegraphics\includegraphics
    % Ensure that by default, figures have no caption (until we provide a
    % proper Figure object with a Caption API and a way to capture that
    % in the conversion process - todo).
    \usepackage{caption}
    \DeclareCaptionFormat{nocaption}{}
    \captionsetup{format=nocaption,aboveskip=0pt,belowskip=0pt}

    \usepackage{float}
    \floatplacement{figure}{H} % forces figures to be placed at the correct location
    \usepackage{xcolor} % Allow colors to be defined
    \usepackage{enumerate} % Needed for markdown enumerations to work
    \usepackage{geometry} % Used to adjust the document margins
    \usepackage{amsmath} % Equations
    \usepackage{amssymb} % Equations
    \usepackage{textcomp} % defines textquotesingle
    % Hack from http://tex.stackexchange.com/a/47451/13684:
    \AtBeginDocument{%
        \def\PYZsq{\textquotesingle}% Upright quotes in Pygmentized code
    }
    \usepackage{upquote} % Upright quotes for verbatim code
    \usepackage{eurosym} % defines \euro

    \usepackage{iftex}
    \ifPDFTeX
        \usepackage[T1]{fontenc}
        \IfFileExists{alphabeta.sty}{
              \usepackage{alphabeta}
          }{
              \usepackage[mathletters]{ucs}
              \usepackage[utf8x]{inputenc}
          }
    \else
        \usepackage{fontspec}
        \usepackage{unicode-math}
    \fi

    \usepackage{fancyvrb} % verbatim replacement that allows latex
    \usepackage{grffile} % extends the file name processing of package graphics
                         % to support a larger range
    \makeatletter % fix for old versions of grffile with XeLaTeX
    \@ifpackagelater{grffile}{2019/11/01}
    {
      % Do nothing on new versions
    }
    {
      \def\Gread@@xetex#1{%
        \IfFileExists{"\Gin@base".bb}%
        {\Gread@eps{\Gin@base.bb}}%
        {\Gread@@xetex@aux#1}%
      }
    }
    \makeatother
    \usepackage[Export]{adjustbox} % Used to constrain images to a maximum size
    \adjustboxset{max size={0.9\linewidth}{0.9\paperheight}}

    % The hyperref package gives us a pdf with properly built
    % internal navigation ('pdf bookmarks' for the table of contents,
    % internal cross-reference links, web links for URLs, etc.)
    \usepackage{hyperref}
    % The default LaTeX title has an obnoxious amount of whitespace. By default,
    % titling removes some of it. It also provides customization options.
    \usepackage{titling}
    \usepackage{longtable} % longtable support required by pandoc >1.10
    \usepackage{booktabs}  % table support for pandoc > 1.12.2
    \usepackage{array}     % table support for pandoc >= 2.11.3
    \usepackage{calc}      % table minipage width calculation for pandoc >= 2.11.1
    \usepackage[inline]{enumitem} % IRkernel/repr support (it uses the enumerate* environment)
    \usepackage[normalem]{ulem} % ulem is needed to support strikethroughs (\sout)
                                % normalem makes italics be italics, not underlines
    \usepackage{soul}      % strikethrough (\st) support for pandoc >= 3.0.0
    \usepackage{mathrsfs}
    

    
    % Colors for the hyperref package
    \definecolor{urlcolor}{rgb}{0,.145,.698}
    \definecolor{linkcolor}{rgb}{.71,0.21,0.01}
    \definecolor{citecolor}{rgb}{.12,.54,.11}

    % ANSI colors
    \definecolor{ansi-black}{HTML}{3E424D}
    \definecolor{ansi-black-intense}{HTML}{282C36}
    \definecolor{ansi-red}{HTML}{E75C58}
    \definecolor{ansi-red-intense}{HTML}{B22B31}
    \definecolor{ansi-green}{HTML}{00A250}
    \definecolor{ansi-green-intense}{HTML}{007427}
    \definecolor{ansi-yellow}{HTML}{DDB62B}
    \definecolor{ansi-yellow-intense}{HTML}{B27D12}
    \definecolor{ansi-blue}{HTML}{208FFB}
    \definecolor{ansi-blue-intense}{HTML}{0065CA}
    \definecolor{ansi-magenta}{HTML}{D160C4}
    \definecolor{ansi-magenta-intense}{HTML}{A03196}
    \definecolor{ansi-cyan}{HTML}{60C6C8}
    \definecolor{ansi-cyan-intense}{HTML}{258F8F}
    \definecolor{ansi-white}{HTML}{C5C1B4}
    \definecolor{ansi-white-intense}{HTML}{A1A6B2}
    \definecolor{ansi-default-inverse-fg}{HTML}{FFFFFF}
    \definecolor{ansi-default-inverse-bg}{HTML}{000000}

    % common color for the border for error outputs.
    \definecolor{outerrorbackground}{HTML}{FFDFDF}

    % commands and environments needed by pandoc snippets
    % extracted from the output of `pandoc -s`
    \providecommand{\tightlist}{%
      \setlength{\itemsep}{0pt}\setlength{\parskip}{0pt}}
    \DefineVerbatimEnvironment{Highlighting}{Verbatim}{commandchars=\\\{\}}
    % Add ',fontsize=\small' for more characters per line
    \newenvironment{Shaded}{}{}
    \newcommand{\KeywordTok}[1]{\textcolor[rgb]{0.00,0.44,0.13}{\textbf{{#1}}}}
    \newcommand{\DataTypeTok}[1]{\textcolor[rgb]{0.56,0.13,0.00}{{#1}}}
    \newcommand{\DecValTok}[1]{\textcolor[rgb]{0.25,0.63,0.44}{{#1}}}
    \newcommand{\BaseNTok}[1]{\textcolor[rgb]{0.25,0.63,0.44}{{#1}}}
    \newcommand{\FloatTok}[1]{\textcolor[rgb]{0.25,0.63,0.44}{{#1}}}
    \newcommand{\CharTok}[1]{\textcolor[rgb]{0.25,0.44,0.63}{{#1}}}
    \newcommand{\StringTok}[1]{\textcolor[rgb]{0.25,0.44,0.63}{{#1}}}
    \newcommand{\CommentTok}[1]{\textcolor[rgb]{0.38,0.63,0.69}{\textit{{#1}}}}
    \newcommand{\OtherTok}[1]{\textcolor[rgb]{0.00,0.44,0.13}{{#1}}}
    \newcommand{\AlertTok}[1]{\textcolor[rgb]{1.00,0.00,0.00}{\textbf{{#1}}}}
    \newcommand{\FunctionTok}[1]{\textcolor[rgb]{0.02,0.16,0.49}{{#1}}}
    \newcommand{\RegionMarkerTok}[1]{{#1}}
    \newcommand{\ErrorTok}[1]{\textcolor[rgb]{1.00,0.00,0.00}{\textbf{{#1}}}}
    \newcommand{\NormalTok}[1]{{#1}}

    % Additional commands for more recent versions of Pandoc
    \newcommand{\ConstantTok}[1]{\textcolor[rgb]{0.53,0.00,0.00}{{#1}}}
    \newcommand{\SpecialCharTok}[1]{\textcolor[rgb]{0.25,0.44,0.63}{{#1}}}
    \newcommand{\VerbatimStringTok}[1]{\textcolor[rgb]{0.25,0.44,0.63}{{#1}}}
    \newcommand{\SpecialStringTok}[1]{\textcolor[rgb]{0.73,0.40,0.53}{{#1}}}
    \newcommand{\ImportTok}[1]{{#1}}
    \newcommand{\DocumentationTok}[1]{\textcolor[rgb]{0.73,0.13,0.13}{\textit{{#1}}}}
    \newcommand{\AnnotationTok}[1]{\textcolor[rgb]{0.38,0.63,0.69}{\textbf{\textit{{#1}}}}}
    \newcommand{\CommentVarTok}[1]{\textcolor[rgb]{0.38,0.63,0.69}{\textbf{\textit{{#1}}}}}
    \newcommand{\VariableTok}[1]{\textcolor[rgb]{0.10,0.09,0.49}{{#1}}}
    \newcommand{\ControlFlowTok}[1]{\textcolor[rgb]{0.00,0.44,0.13}{\textbf{{#1}}}}
    \newcommand{\OperatorTok}[1]{\textcolor[rgb]{0.40,0.40,0.40}{{#1}}}
    \newcommand{\BuiltInTok}[1]{{#1}}
    \newcommand{\ExtensionTok}[1]{{#1}}
    \newcommand{\PreprocessorTok}[1]{\textcolor[rgb]{0.74,0.48,0.00}{{#1}}}
    \newcommand{\AttributeTok}[1]{\textcolor[rgb]{0.49,0.56,0.16}{{#1}}}
    \newcommand{\InformationTok}[1]{\textcolor[rgb]{0.38,0.63,0.69}{\textbf{\textit{{#1}}}}}
    \newcommand{\WarningTok}[1]{\textcolor[rgb]{0.38,0.63,0.69}{\textbf{\textit{{#1}}}}}


    % Define a nice break command that doesn't care if a line doesn't already
    % exist.
    \def\br{\hspace*{\fill} \\* }
    % Math Jax compatibility definitions
    \def\gt{>}
    \def\lt{<}
    \let\Oldtex\TeX
    \let\Oldlatex\LaTeX
    \renewcommand{\TeX}{\textrm{\Oldtex}}
    \renewcommand{\LaTeX}{\textrm{\Oldlatex}}
    % Document parameters
    % Document title
    \title{assn04}
    
    
    
    
    
    
    
% Pygments definitions
\makeatletter
\def\PY@reset{\let\PY@it=\relax \let\PY@bf=\relax%
    \let\PY@ul=\relax \let\PY@tc=\relax%
    \let\PY@bc=\relax \let\PY@ff=\relax}
\def\PY@tok#1{\csname PY@tok@#1\endcsname}
\def\PY@toks#1+{\ifx\relax#1\empty\else%
    \PY@tok{#1}\expandafter\PY@toks\fi}
\def\PY@do#1{\PY@bc{\PY@tc{\PY@ul{%
    \PY@it{\PY@bf{\PY@ff{#1}}}}}}}
\def\PY#1#2{\PY@reset\PY@toks#1+\relax+\PY@do{#2}}

\@namedef{PY@tok@w}{\def\PY@tc##1{\textcolor[rgb]{0.73,0.73,0.73}{##1}}}
\@namedef{PY@tok@c}{\let\PY@it=\textit\def\PY@tc##1{\textcolor[rgb]{0.24,0.48,0.48}{##1}}}
\@namedef{PY@tok@cp}{\def\PY@tc##1{\textcolor[rgb]{0.61,0.40,0.00}{##1}}}
\@namedef{PY@tok@k}{\let\PY@bf=\textbf\def\PY@tc##1{\textcolor[rgb]{0.00,0.50,0.00}{##1}}}
\@namedef{PY@tok@kp}{\def\PY@tc##1{\textcolor[rgb]{0.00,0.50,0.00}{##1}}}
\@namedef{PY@tok@kt}{\def\PY@tc##1{\textcolor[rgb]{0.69,0.00,0.25}{##1}}}
\@namedef{PY@tok@o}{\def\PY@tc##1{\textcolor[rgb]{0.40,0.40,0.40}{##1}}}
\@namedef{PY@tok@ow}{\let\PY@bf=\textbf\def\PY@tc##1{\textcolor[rgb]{0.67,0.13,1.00}{##1}}}
\@namedef{PY@tok@nb}{\def\PY@tc##1{\textcolor[rgb]{0.00,0.50,0.00}{##1}}}
\@namedef{PY@tok@nf}{\def\PY@tc##1{\textcolor[rgb]{0.00,0.00,1.00}{##1}}}
\@namedef{PY@tok@nc}{\let\PY@bf=\textbf\def\PY@tc##1{\textcolor[rgb]{0.00,0.00,1.00}{##1}}}
\@namedef{PY@tok@nn}{\let\PY@bf=\textbf\def\PY@tc##1{\textcolor[rgb]{0.00,0.00,1.00}{##1}}}
\@namedef{PY@tok@ne}{\let\PY@bf=\textbf\def\PY@tc##1{\textcolor[rgb]{0.80,0.25,0.22}{##1}}}
\@namedef{PY@tok@nv}{\def\PY@tc##1{\textcolor[rgb]{0.10,0.09,0.49}{##1}}}
\@namedef{PY@tok@no}{\def\PY@tc##1{\textcolor[rgb]{0.53,0.00,0.00}{##1}}}
\@namedef{PY@tok@nl}{\def\PY@tc##1{\textcolor[rgb]{0.46,0.46,0.00}{##1}}}
\@namedef{PY@tok@ni}{\let\PY@bf=\textbf\def\PY@tc##1{\textcolor[rgb]{0.44,0.44,0.44}{##1}}}
\@namedef{PY@tok@na}{\def\PY@tc##1{\textcolor[rgb]{0.41,0.47,0.13}{##1}}}
\@namedef{PY@tok@nt}{\let\PY@bf=\textbf\def\PY@tc##1{\textcolor[rgb]{0.00,0.50,0.00}{##1}}}
\@namedef{PY@tok@nd}{\def\PY@tc##1{\textcolor[rgb]{0.67,0.13,1.00}{##1}}}
\@namedef{PY@tok@s}{\def\PY@tc##1{\textcolor[rgb]{0.73,0.13,0.13}{##1}}}
\@namedef{PY@tok@sd}{\let\PY@it=\textit\def\PY@tc##1{\textcolor[rgb]{0.73,0.13,0.13}{##1}}}
\@namedef{PY@tok@si}{\let\PY@bf=\textbf\def\PY@tc##1{\textcolor[rgb]{0.64,0.35,0.47}{##1}}}
\@namedef{PY@tok@se}{\let\PY@bf=\textbf\def\PY@tc##1{\textcolor[rgb]{0.67,0.36,0.12}{##1}}}
\@namedef{PY@tok@sr}{\def\PY@tc##1{\textcolor[rgb]{0.64,0.35,0.47}{##1}}}
\@namedef{PY@tok@ss}{\def\PY@tc##1{\textcolor[rgb]{0.10,0.09,0.49}{##1}}}
\@namedef{PY@tok@sx}{\def\PY@tc##1{\textcolor[rgb]{0.00,0.50,0.00}{##1}}}
\@namedef{PY@tok@m}{\def\PY@tc##1{\textcolor[rgb]{0.40,0.40,0.40}{##1}}}
\@namedef{PY@tok@gh}{\let\PY@bf=\textbf\def\PY@tc##1{\textcolor[rgb]{0.00,0.00,0.50}{##1}}}
\@namedef{PY@tok@gu}{\let\PY@bf=\textbf\def\PY@tc##1{\textcolor[rgb]{0.50,0.00,0.50}{##1}}}
\@namedef{PY@tok@gd}{\def\PY@tc##1{\textcolor[rgb]{0.63,0.00,0.00}{##1}}}
\@namedef{PY@tok@gi}{\def\PY@tc##1{\textcolor[rgb]{0.00,0.52,0.00}{##1}}}
\@namedef{PY@tok@gr}{\def\PY@tc##1{\textcolor[rgb]{0.89,0.00,0.00}{##1}}}
\@namedef{PY@tok@ge}{\let\PY@it=\textit}
\@namedef{PY@tok@gs}{\let\PY@bf=\textbf}
\@namedef{PY@tok@gp}{\let\PY@bf=\textbf\def\PY@tc##1{\textcolor[rgb]{0.00,0.00,0.50}{##1}}}
\@namedef{PY@tok@go}{\def\PY@tc##1{\textcolor[rgb]{0.44,0.44,0.44}{##1}}}
\@namedef{PY@tok@gt}{\def\PY@tc##1{\textcolor[rgb]{0.00,0.27,0.87}{##1}}}
\@namedef{PY@tok@err}{\def\PY@bc##1{{\setlength{\fboxsep}{\string -\fboxrule}\fcolorbox[rgb]{1.00,0.00,0.00}{1,1,1}{\strut ##1}}}}
\@namedef{PY@tok@kc}{\let\PY@bf=\textbf\def\PY@tc##1{\textcolor[rgb]{0.00,0.50,0.00}{##1}}}
\@namedef{PY@tok@kd}{\let\PY@bf=\textbf\def\PY@tc##1{\textcolor[rgb]{0.00,0.50,0.00}{##1}}}
\@namedef{PY@tok@kn}{\let\PY@bf=\textbf\def\PY@tc##1{\textcolor[rgb]{0.00,0.50,0.00}{##1}}}
\@namedef{PY@tok@kr}{\let\PY@bf=\textbf\def\PY@tc##1{\textcolor[rgb]{0.00,0.50,0.00}{##1}}}
\@namedef{PY@tok@bp}{\def\PY@tc##1{\textcolor[rgb]{0.00,0.50,0.00}{##1}}}
\@namedef{PY@tok@fm}{\def\PY@tc##1{\textcolor[rgb]{0.00,0.00,1.00}{##1}}}
\@namedef{PY@tok@vc}{\def\PY@tc##1{\textcolor[rgb]{0.10,0.09,0.49}{##1}}}
\@namedef{PY@tok@vg}{\def\PY@tc##1{\textcolor[rgb]{0.10,0.09,0.49}{##1}}}
\@namedef{PY@tok@vi}{\def\PY@tc##1{\textcolor[rgb]{0.10,0.09,0.49}{##1}}}
\@namedef{PY@tok@vm}{\def\PY@tc##1{\textcolor[rgb]{0.10,0.09,0.49}{##1}}}
\@namedef{PY@tok@sa}{\def\PY@tc##1{\textcolor[rgb]{0.73,0.13,0.13}{##1}}}
\@namedef{PY@tok@sb}{\def\PY@tc##1{\textcolor[rgb]{0.73,0.13,0.13}{##1}}}
\@namedef{PY@tok@sc}{\def\PY@tc##1{\textcolor[rgb]{0.73,0.13,0.13}{##1}}}
\@namedef{PY@tok@dl}{\def\PY@tc##1{\textcolor[rgb]{0.73,0.13,0.13}{##1}}}
\@namedef{PY@tok@s2}{\def\PY@tc##1{\textcolor[rgb]{0.73,0.13,0.13}{##1}}}
\@namedef{PY@tok@sh}{\def\PY@tc##1{\textcolor[rgb]{0.73,0.13,0.13}{##1}}}
\@namedef{PY@tok@s1}{\def\PY@tc##1{\textcolor[rgb]{0.73,0.13,0.13}{##1}}}
\@namedef{PY@tok@mb}{\def\PY@tc##1{\textcolor[rgb]{0.40,0.40,0.40}{##1}}}
\@namedef{PY@tok@mf}{\def\PY@tc##1{\textcolor[rgb]{0.40,0.40,0.40}{##1}}}
\@namedef{PY@tok@mh}{\def\PY@tc##1{\textcolor[rgb]{0.40,0.40,0.40}{##1}}}
\@namedef{PY@tok@mi}{\def\PY@tc##1{\textcolor[rgb]{0.40,0.40,0.40}{##1}}}
\@namedef{PY@tok@il}{\def\PY@tc##1{\textcolor[rgb]{0.40,0.40,0.40}{##1}}}
\@namedef{PY@tok@mo}{\def\PY@tc##1{\textcolor[rgb]{0.40,0.40,0.40}{##1}}}
\@namedef{PY@tok@ch}{\let\PY@it=\textit\def\PY@tc##1{\textcolor[rgb]{0.24,0.48,0.48}{##1}}}
\@namedef{PY@tok@cm}{\let\PY@it=\textit\def\PY@tc##1{\textcolor[rgb]{0.24,0.48,0.48}{##1}}}
\@namedef{PY@tok@cpf}{\let\PY@it=\textit\def\PY@tc##1{\textcolor[rgb]{0.24,0.48,0.48}{##1}}}
\@namedef{PY@tok@c1}{\let\PY@it=\textit\def\PY@tc##1{\textcolor[rgb]{0.24,0.48,0.48}{##1}}}
\@namedef{PY@tok@cs}{\let\PY@it=\textit\def\PY@tc##1{\textcolor[rgb]{0.24,0.48,0.48}{##1}}}

\def\PYZbs{\char`\\}
\def\PYZus{\char`\_}
\def\PYZob{\char`\{}
\def\PYZcb{\char`\}}
\def\PYZca{\char`\^}
\def\PYZam{\char`\&}
\def\PYZlt{\char`\<}
\def\PYZgt{\char`\>}
\def\PYZsh{\char`\#}
\def\PYZpc{\char`\%}
\def\PYZdl{\char`\$}
\def\PYZhy{\char`\-}
\def\PYZsq{\char`\'}
\def\PYZdq{\char`\"}
\def\PYZti{\char`\~}
% for compatibility with earlier versions
\def\PYZat{@}
\def\PYZlb{[}
\def\PYZrb{]}
\makeatother


    % For linebreaks inside Verbatim environment from package fancyvrb.
    \makeatletter
        \newbox\Wrappedcontinuationbox
        \newbox\Wrappedvisiblespacebox
        \newcommand*\Wrappedvisiblespace {\textcolor{red}{\textvisiblespace}}
        \newcommand*\Wrappedcontinuationsymbol {\textcolor{red}{\llap{\tiny$\m@th\hookrightarrow$}}}
        \newcommand*\Wrappedcontinuationindent {3ex }
        \newcommand*\Wrappedafterbreak {\kern\Wrappedcontinuationindent\copy\Wrappedcontinuationbox}
        % Take advantage of the already applied Pygments mark-up to insert
        % potential linebreaks for TeX processing.
        %        {, <, #, %, $, ' and ": go to next line.
        %        _, }, ^, &, >, - and ~: stay at end of broken line.
        % Use of \textquotesingle for straight quote.
        \newcommand*\Wrappedbreaksatspecials {%
            \def\PYGZus{\discretionary{\char`\_}{\Wrappedafterbreak}{\char`\_}}%
            \def\PYGZob{\discretionary{}{\Wrappedafterbreak\char`\{}{\char`\{}}%
            \def\PYGZcb{\discretionary{\char`\}}{\Wrappedafterbreak}{\char`\}}}%
            \def\PYGZca{\discretionary{\char`\^}{\Wrappedafterbreak}{\char`\^}}%
            \def\PYGZam{\discretionary{\char`\&}{\Wrappedafterbreak}{\char`\&}}%
            \def\PYGZlt{\discretionary{}{\Wrappedafterbreak\char`\<}{\char`\<}}%
            \def\PYGZgt{\discretionary{\char`\>}{\Wrappedafterbreak}{\char`\>}}%
            \def\PYGZsh{\discretionary{}{\Wrappedafterbreak\char`\#}{\char`\#}}%
            \def\PYGZpc{\discretionary{}{\Wrappedafterbreak\char`\%}{\char`\%}}%
            \def\PYGZdl{\discretionary{}{\Wrappedafterbreak\char`\$}{\char`\$}}%
            \def\PYGZhy{\discretionary{\char`\-}{\Wrappedafterbreak}{\char`\-}}%
            \def\PYGZsq{\discretionary{}{\Wrappedafterbreak\textquotesingle}{\textquotesingle}}%
            \def\PYGZdq{\discretionary{}{\Wrappedafterbreak\char`\"}{\char`\"}}%
            \def\PYGZti{\discretionary{\char`\~}{\Wrappedafterbreak}{\char`\~}}%
        }
        % Some characters . , ; ? ! / are not pygmentized.
        % This macro makes them "active" and they will insert potential linebreaks
        \newcommand*\Wrappedbreaksatpunct {%
            \lccode`\~`\.\lowercase{\def~}{\discretionary{\hbox{\char`\.}}{\Wrappedafterbreak}{\hbox{\char`\.}}}%
            \lccode`\~`\,\lowercase{\def~}{\discretionary{\hbox{\char`\,}}{\Wrappedafterbreak}{\hbox{\char`\,}}}%
            \lccode`\~`\;\lowercase{\def~}{\discretionary{\hbox{\char`\;}}{\Wrappedafterbreak}{\hbox{\char`\;}}}%
            \lccode`\~`\:\lowercase{\def~}{\discretionary{\hbox{\char`\:}}{\Wrappedafterbreak}{\hbox{\char`\:}}}%
            \lccode`\~`\?\lowercase{\def~}{\discretionary{\hbox{\char`\?}}{\Wrappedafterbreak}{\hbox{\char`\?}}}%
            \lccode`\~`\!\lowercase{\def~}{\discretionary{\hbox{\char`\!}}{\Wrappedafterbreak}{\hbox{\char`\!}}}%
            \lccode`\~`\/\lowercase{\def~}{\discretionary{\hbox{\char`\/}}{\Wrappedafterbreak}{\hbox{\char`\/}}}%
            \catcode`\.\active
            \catcode`\,\active
            \catcode`\;\active
            \catcode`\:\active
            \catcode`\?\active
            \catcode`\!\active
            \catcode`\/\active
            \lccode`\~`\~
        }
    \makeatother

    \let\OriginalVerbatim=\Verbatim
    \makeatletter
    \renewcommand{\Verbatim}[1][1]{%
        %\parskip\z@skip
        \sbox\Wrappedcontinuationbox {\Wrappedcontinuationsymbol}%
        \sbox\Wrappedvisiblespacebox {\FV@SetupFont\Wrappedvisiblespace}%
        \def\FancyVerbFormatLine ##1{\hsize\linewidth
            \vtop{\raggedright\hyphenpenalty\z@\exhyphenpenalty\z@
                \doublehyphendemerits\z@\finalhyphendemerits\z@
                \strut ##1\strut}%
        }%
        % If the linebreak is at a space, the latter will be displayed as visible
        % space at end of first line, and a continuation symbol starts next line.
        % Stretch/shrink are however usually zero for typewriter font.
        \def\FV@Space {%
            \nobreak\hskip\z@ plus\fontdimen3\font minus\fontdimen4\font
            \discretionary{\copy\Wrappedvisiblespacebox}{\Wrappedafterbreak}
            {\kern\fontdimen2\font}%
        }%

        % Allow breaks at special characters using \PYG... macros.
        \Wrappedbreaksatspecials
        % Breaks at punctuation characters . , ; ? ! and / need catcode=\active
        \OriginalVerbatim[#1,codes*=\Wrappedbreaksatpunct]%
    }
    \makeatother

    % Exact colors from NB
    \definecolor{incolor}{HTML}{303F9F}
    \definecolor{outcolor}{HTML}{D84315}
    \definecolor{cellborder}{HTML}{CFCFCF}
    \definecolor{cellbackground}{HTML}{F7F7F7}

    % prompt
    \makeatletter
    \newcommand{\boxspacing}{\kern\kvtcb@left@rule\kern\kvtcb@boxsep}
    \makeatother
    \newcommand{\prompt}[4]{
        {\ttfamily\llap{{\color{#2}[#3]:\hspace{3pt}#4}}\vspace{-\baselineskip}}
    }
    

    
    % Prevent overflowing lines due to hard-to-break entities
    \sloppy
    % Setup hyperref package
    \hypersetup{
      breaklinks=true,  % so long urls are correctly broken across lines
      colorlinks=true,
      urlcolor=urlcolor,
      linkcolor=linkcolor,
      citecolor=citecolor,
      }
    % Slightly bigger margins than the latex defaults
    
    \geometry{verbose,tmargin=1in,bmargin=1in,lmargin=1in,rmargin=1in}
    
    

\begin{document}
    
    \maketitle
    
    

    
    \section{3}\label{section}

证明函数

\[
1,x,x^2,\cdots, x^n
\]

线性无关。

    \subsection{Proof}\label{proof}

我们取实数范围内~\(n+1\)~个互异点~\(x_0, x_1, \dots, x_n\),考虑下列齐次线性方程组:

\[
\begin{pmatrix}
1 & x_0 & x_0^2 & \cdots & x_0^n \\
1 & x_1 & x_1^2 & \cdots & x_1^n \\
\vdots & \vdots & \vdots & \ddots & \vdots \\
1 & x_n & x_n^2 & \cdots & x_n^n
\end{pmatrix}
\begin{pmatrix}
a_0\\
a_1\\
\vdots\\
a_n
\end{pmatrix}
=\begin{pmatrix}
0\\
0\\
\vdots\\
0
\end{pmatrix}
\]

该矩阵为 Vandermonde 矩阵,故其行列式为

\[
\det(V) \;=\; \prod_{0 \le i < j \le n} (x_j - x_i).
\]

由于选取的点两两不同,故 \(\det(V)\neq 0\),该齐次方程组仅有平凡解

\[
a_0 = a_1 = \cdots = a_n = 0.
\]

因此,集合 \[
\{\,1,\,x,\,x^2,\,\dots,\,x^n\}
\] 线性无关。

    \section{4}\label{section}

计算下列函数 \(f(x)\) 关于 \(C[0, 1]\)
的\(\left\lVert f\right\rVert_\infty,\left\lVert f\right\rVert_1,\left\lVert f\right\rVert_2
\)

\begin{enumerate}
\def\labelenumi{(\arabic{enumi})}
\item
  \(f(x) = (x - 1) ^ 2\)
\item
  \(f(x) = \left| x - \frac{1}{2} \right|\)
\item
  \(f(x) = x^m(1 - x)^n\),m 与 n 为正整数
\end{enumerate}

    \subsection{Solution}\label{solution}

\subsubsection{(1)}\label{section}

\[
\|f\|_{\infty} = 1.
\]

\[
\|f\|_{1} = \int_0^1 (1-x)^2\,dx=\frac{1}{3}.
\]

\[
\|f\|_{2} = \left(\int_0^1 \left[(x-1)^2\right]^2\,dx\right)^{\frac{1}{2}} = \left(\int_0^1 (x-1)^4\,dx\right)^{\frac{1}{2}} = \sqrt{\frac{1}{5}} = \frac{1}{\sqrt{5}}.
\]

\begin{center}\rule{0.5\linewidth}{0.5pt}\end{center}

\subsubsection{(2)}\label{section-1}

在 \([0,1]\) 上,\(\left|x-\frac{1}{2}\right|\) 的最大值出现在 \(x=0\)
和 \(x=1\),均有 \[
\left|0-\frac{1}{2}\right| = \left|1-\frac{1}{2}\right| = \frac{1}{2}.
\] 因此, \[
\|f\|_{\infty} = \frac{1}{2}.
\]

\[
\|f\|_{1} = \int_0^1 \left|x-\frac{1}{2}\right|\,dx = 2\int_0^{1/2} \left(\frac{1}{2}-x\right)\,dx = \frac{1}{8}.
\]

\[
\|f\|_{2} = \left(\int_0^1 \left(x-\frac{1}{2}\right)^2 dx\right)^{\frac{1}{2}}=\sqrt{\frac{1}{12}} = \frac{1}{2\sqrt{3}}.
\]

\begin{center}\rule{0.5\linewidth}{0.5pt}\end{center}

\subsection{(3)}\label{section-2}

对于 \(m,n>0\),在 \([0,1]\) 上端点处
\(f(0)=f(1)=0\),函数在内部取得最大值。令
\(g(x)=\ln f(x)= m\ln x+n\ln(1-x)\),求导得到: \[
g'(x)=\frac{m}{x}-\frac{n}{1-x}=0 \quad \Longrightarrow \quad x=\frac{m}{m+n}.
\] 代入得最大值: \[
\|f\|_{\infty} = \left(\frac{m}{m+n}\right)^m \left(\frac{n}{m+n}\right)^n.
\]

根据 Beta 函数,有

\[
\|f\|_{1} = \int_0^1 x^m(1-x)^n\,dx = B(m+1,n+1) = \frac{m! \, n!}{(m+n+1)!}.
\]

\[
\|f\|_{2} = \left(\int_0^1 x^{2m}(1-x)^{2n}\,dx\right)^{\frac{1}{2}} = \sqrt{B(2m+1,2n+1)},
\]

其中, \[
B(2m+1,2n+1) = \frac{(2m)! \, (2n)!}{(2m+2n+1)!}.
\]

    \section{5}\label{section}

证明

\[
\left\lVert f - g\right\rVert \geqslant \left\lVert f\right\rVert  - \left\lVert g\right\rVert 
\]

    \subsection{Proof}\label{proof}

根据三角不等式,对于任意 \(f, g\) 有 \[
   \|f\| = \|(f-g) + g\| \leqslant \|f-g\| + \|g\|.
   \]

由此可以得到 \[
   \|f\| - \|g\| \leqslant \|f-g\|.
   \]

整理后即得: \[
   \|f-g\| \geqslant \|f\| - \|g\|.
   \] 证毕。

    \section{6}\label{section}

对 \(f(x),g(x) \in C^1[a,b]\),定义

\begin{enumerate}
\def\labelenumi{(\arabic{enumi})}
\tightlist
\item
\end{enumerate}

\[
(f,g) = \int_a^bf'(x)g'(x)\,dx
\]

\begin{enumerate}
\def\labelenumi{(\arabic{enumi})}
\setcounter{enumi}{1}
\tightlist
\item
\end{enumerate}

\[
(f,g) = \int_a^bf'(x)g'(x)\,dx + f(a)g(a)
\]

问它们是否构成内积。

    \subsection{Solution}\label{solution}

\subsubsection{(1)}\label{section}

\textbf{对称性} 和 \textbf{线性性} 容易验证。检查正定性:\\
要求对于任意 \(f\) 有 \[
(f,f)=\int_a^b \left[f'(x)\right]^2\,dx\ge0,
\] 且当且仅当 \(f\equiv0\)时有 \((f,f)=0\)。

注意到如果 \(f\)为常数函数(例如 \(f(x)=C\),其中 \(C\neq0\)),则有\\
\[
f'(x)=0,\quad \forall x\in[a,b],
\] 从而 \[
(f,f)=\int_a^b 0^2\,dx=0,
\] 但 \(f\not\equiv0\)。不满足内积的正定性条件。

故定义 (1) 不能构成内积。

\begin{center}\rule{0.5\linewidth}{0.5pt}\end{center}

\subsubsection{(2)}\label{section-1}

同样,显然该式满足\textbf{对称性}和\textbf{线性性}。接下来验证\textbf{正定性}:\\
对任意 \(f\in C^1[a,b]\),有 \[
(f,f)=\int_a^b \left[f'(x)\right]^2\,dx + f(a)^2.
\] 显然两个部分均非负,因此 \((f,f)\ge0\)。若 \((f,f)=0\),则必有 \[
\int_a^b \left[f'(x)\right]^2\,dx = 0 \quad \text{且} \quad f(a)^2 = 0.
\] 由积分为零可知 \(f'(x)=0\) 几乎处处成立,从而 \(f\) 为常数函数;又
\(f(a)=0\) 故 \(f\equiv0\)。

故定义 (2) 构成内积。

    \section{7}\label{section}

令\\
\[
T^*_n(x)=T_n(2x-1),\quad x\in [0,1],
\] 其中 \(T_n(x)\) 为第一类 Chebyshev 多项式,即\\
\[
T_n(x)=\cos\bigl(n\arccos x\bigr).
\] 证明集合 \(\{T^*_n(x)\}_{n\ge0}\) 在区间 \([0,1]\) 上关于权函数\\
\[
\rho(x)=\frac{1}{\sqrt{x-x^2}}
\] 是正交多项式序列,并求出
\(T^*_0(x),\,T^*_1(x),\,T^*_2(x),\,T^*_3(x)\) 的显式表达式。

    \subsection{Proof}\label{proof}

对任意连续函数 \(f(x)\),作变量代换\\
\[
t=2x-1,\quad \Rightarrow\quad x=\frac{t+1}{2},\quad dx=\frac{dt}{2}.
\] 注意到\\
\[
x-x^2=\frac{t+1}{2}\Bigl(1-\frac{t+1}{2}\Bigr)
=\frac{1-t^2}{4},
\] 故有\\
\[
\sqrt{x-x^2}=\frac{1}{2}\sqrt{1-t^2},\quad \rho(x)=\frac{1}{\sqrt{x-x^2}}=\frac{2}{\sqrt{1-t^2}}.
\]

因此,积分 \[
\int_0^1 T^*_n(x)T^*_m(x)\,\rho(x)\,dx
\] 经过代换变为 \[
\int_{-1}^{1} T_n(t)T_m(t)\,\frac{2}{\sqrt{1-t^2}}\cdot \frac{dt}{2}
=\int_{-1}^{1} \frac{T_n(t)T_m(t)}{\sqrt{1-t^2}}\,dt.
\]

Chebyshev 多项式具有正交性质\\
\[
\int_{-1}^{1} \frac{T_n(t)T_m(t)}{\sqrt{1-t^2}}\,dt=0,\quad n\neq m.
\] 因此可知\\
\[
\int_0^1 T^*_n(x)T^*_m(x)\,\rho(x)\,dx=0,\quad n\neq m,
\]

正交性得证。

\begin{center}\rule{0.5\linewidth}{0.5pt}\end{center}

\subsection{Solution}\label{solution}

已知 Chebyshev 多项式的标准表达式为 \[
\begin{aligned}
T_0(x)&=1,\\[1mm]
T_1(x)&=x,\\[1mm]
T_2(x)&=2x^2-1,\\[1mm]
T_3(x)&=4x^3-3x.
\end{aligned}
\]

由定义 \(T^*_n(x)=T_n(2x-1)\),直接计算可得:

\begin{enumerate}
\def\labelenumi{\arabic{enumi}.}
\item
  当 \(n=0\) 时, \[
  T^*_0(x)=T_0(2x-1)=1.
  \]
\item
  当 \(n=1\) 时, \[
  T^*_1(x)=T_1(2x-1)=2x-1.
  \]
\item
  当 \(n=2\) 时, \[
  \begin{aligned}
  T^*_2(x)&=T_2(2x-1)=2(2x-1)^2-1\\[1mm]
  &=2\Bigl[4x^2-4x+1\Bigr]-1=8x^2-8x+2-1\\[1mm]
  &=8x^2-8x+1.
  \end{aligned}
  \]
\item
  当 \(n=3\) 时, \[
  \begin{aligned}
  T^*_3(x)&=T_3(2x-1)=4(2x-1)^3-3(2x-1).\\[1mm]
  &=32x^3-48x^2+18x-1.
  \end{aligned}
  \]
\end{enumerate}

    \section{8}\label{section}

对权函数 \(\rho (x) = 1 + x^2\),区间 \([-1, 1]\),求首项系数为 1
的正交多项式 \(\varphi_n(x), n = 0,1,2,3\)

    \subsection{Solution}\label{solution}

采用施密特(Gram--Schmidt)正交化方法

\begin{center}\rule{0.5\linewidth}{0.5pt}\end{center}

取基中最低次项: \[
\varphi_0(x)=1.
\] 其首项系数为1。

\begin{center}\rule{0.5\linewidth}{0.5pt}\end{center}

取 \(f_1(x)=x\)。正交化公式为 \[
\varphi_1(x)=f_1(x)-\frac{\langle f_1, \varphi_0\rangle}{\langle \varphi_0,\varphi_0\rangle}\varphi_0(x).
\] 计算 \[
\langle x,1\rangle=\int_{-1}^{1}x(1+x^2)\,dx.
\] 注意积分区间下 \(x(1+x^2)\) 为奇函数,其积分为0,因此 \[
\varphi_1(x)=x.
\]

\begin{center}\rule{0.5\linewidth}{0.5pt}\end{center}

取 \(f_2(x)=x^2\),先写出 \[
\varphi_2(x)=x^2-\frac{\langle x^2,\varphi_0\rangle}{\langle \varphi_0,\varphi_0\rangle}\varphi_0(x) - \frac{\langle x^2,\varphi_1\rangle}{\langle \varphi_1,\varphi_1\rangle}\varphi_1(x).
\] 计算各内积:

\begin{enumerate}
\def\labelenumi{\arabic{enumi}.}
\item
  与 \(\varphi_0(x)=1\): \[
  \langle x^2,1\rangle=\int_{-1}^{1}(x^2+x^4)\,dx =\frac{16}{15},
  \] 而 \[
  \langle 1,1\rangle=\int_{-1}^{1}(1+x^2)\,dx = \frac{8}{3}.
  \] 因此第一投影系数为 \[
  a_{20}=\frac{16/15}{8/3}=\frac{2}{5}.
  \]
\item
  与 \(\varphi_1(x)=x\): \[
  \langle x^2,x\rangle=\int_{-1}^{1}x^3(1+x^2)\,dx,
  \] 因为 (x\textsuperscript{3(1+x}2))
  奇函数,积分为0,因此第二投影系数为0。
\end{enumerate}

综上, \[
\varphi_2(x)=x^2-\frac{2}{5}.
\]

\begin{center}\rule{0.5\linewidth}{0.5pt}\end{center}

取 \(f_3(x)=x^3\)。由于 \(x^3\) 为奇函数,我们可设形式为 \[
\varphi_3(x)=x^3+ \beta\,x,
\] (此时不含偶次项,保证不受前面偶函数的影响)。

要求 \(\varphi_3(x)\) 与 \(\varphi_1(x)=x\) 正交,即 \[
\langle x^3+\beta x,\, x\rangle=0.
\] 计算: \[
\langle x^3, x\rangle=\int_{-1}^{1}x^4(1+x^2)\,dx=\int_{-1}^{1}(x^4+x^6)\,dx =\frac{24}{35},
\] 且 \[
\langle x, x\rangle=\int_{-1}^{1}x^2(1+x^2)\,dx=\frac{16}{15}.
\] 故有 \[
\frac{24}{35}+\beta\frac{16}{15}=0\quad\Longrightarrow\quad \beta=-\frac{9}{14}.
\] 因此, \[
\varphi_3(x)= x^3-\frac{9}{14}\,x.
\]

其他正交条件(与 \(\varphi_0\) 和
\(\varphi_2\))可通过奇偶性验证(因为奇函数与偶函数正交)。

\begin{center}\rule{0.5\linewidth}{0.5pt}\end{center}

综上所述,

\[
\begin{aligned}
\varphi_0(x)&=1,\\[1mm]
\varphi_1(x)&=x,\\[1mm]
\varphi_2(x)&=x^2-\frac{2}{5},\\[1mm]
\varphi_3(x)&=x^3-\frac{9}{14}\,x.
\end{aligned}
\]

    \section{11}\label{section}

用 \(T_3(x)\) 的零点做插值点,求 \(f(x) = e^x\) 在区间 \([-1, 1]\)
上的二次插值多项式,并估计最大误差界。

    \subsection{Solution}\label{solution}

已知第一类 Chebyshev 多项式 \[
T_3(x)=4x^3-3x,
\] 其零点为\\
\[
x_0=-\cos\frac{\pi}{6}=-\frac{\sqrt{3}}{2},\quad x_1=-\cos\frac{3\pi}{6}=0,\quad x_2=-\cos\frac{5\pi}{6}=\frac{\sqrt{3}}{2}.
\]

因此选取插值节点为 \[
x_0=-\frac{\sqrt{3}}{2},\quad x_1=0,\quad x_2=\frac{\sqrt{3}}{2}.
\]

\begin{center}\rule{0.5\linewidth}{0.5pt}\end{center}

构造 Lagrange 基函数

\begin{itemize}
\item
  对于 \(L_0(x)\): \[
  L_0(x)=\frac{(x-x_1)(x-x_2)}{(x_0-x_1)(x_0-x_2)}
  =\frac{2}{3}x\left(x-\frac{\sqrt{3}}{2}\right).
  \]
\item
  对于 \(L_1(x)\): \[
  L_1(x)=\frac{(x-x_0)(x-x_2)}{(x_1-x_0)(x_1-x_2)}
  =\frac{3-4x^2}{3}.
  \]
\item
  对于 \(L_2(x)\): \[
  L_2(x)=\frac{(x-x_0)(x-x_1)}{(x_2-x_0)(x_2-x_1)}
  =\frac{2}{3}x\left(x+\frac{\sqrt{3}}{2}\right).
  \]
\end{itemize}

\begin{center}\rule{0.5\linewidth}{0.5pt}\end{center}

利用 Lagrange 插值公式得: \[
P_2(x)=f(x_0)L_0(x)+f(x_1)L_1(x)+f(x_2)L_2(x).
\] 由于 \[
f(x)=e^x,\quad f\left(-\frac{\sqrt{3}}{2}\right)=e^{-\frac{\sqrt{3}}{2}},\quad f(0)=1,\quad f\left(\frac{\sqrt{3}}{2}\right)=e^{\frac{\sqrt{3}}{2}},
\] 故 \[
P_2(x)=e^{-\frac{\sqrt{3}}{2}}\cdot\frac{2}{3}x\left(x-\frac{\sqrt{3}}{2}\right)
+\frac{3-4x^2}{3}
+e^{\frac{\sqrt{3}}{2}}\cdot\frac{2}{3}x\left(x+\frac{\sqrt{3}}{2}\right).
\]

\begin{center}\rule{0.5\linewidth}{0.5pt}\end{center}

根据定理 7 有

\[
\max\limits_{-1\le x\le 1} |f(x) - L_n(x)| \le \frac{1}{2^n(n+1)!}\left\lVert f^{(n+1)}(x)\right\rVert _\infty 
\]

故二次插值(\(n=2\))的最大误差满足

\[
|f(x) – P_2(x)| \le \frac{1}{2^2\cdot 3!} \max\limits_{-1\le x\le 1} |f⁽³⁾(x)|.
\]

对于 \(f(x)=eˣ\),其三阶导数仍为 \(eˣ\),在区间 \([–1,1]\) 上有最大值
\(e\)。因此,

\[
\max\limits_{-1\le x\le 1} |eˣ – P₂(x)| \le \frac{e}{24}.
\]

因此,该二次插值多项式在 \([–1,1]\) 上的最大误差界为 \(e/24\)。


    % Add a bibliography block to the postdoc
    
    
    
\end{document}
