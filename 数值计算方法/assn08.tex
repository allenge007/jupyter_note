\documentclass[11pt]{article}

    \usepackage[breakable]{tcolorbox}
    \usepackage{parskip} % Stop auto-indenting (to mimic markdown behaviour)
    \usepackage{xeCJK}

    % Basic figure setup, for now with no caption control since it's done
    % automatically by Pandoc (which extracts ![](path) syntax from Markdown).
    \usepackage{graphicx}
    % Keep aspect ratio if custom image width or height is specified
    \setkeys{Gin}{keepaspectratio}
    % Maintain compatibility with old templates. Remove in nbconvert 6.0
    \let\Oldincludegraphics\includegraphics
    % Ensure that by default, figures have no caption (until we provide a
    % proper Figure object with a Caption API and a way to capture that
    % in the conversion process - todo).
    \usepackage{caption}
    \DeclareCaptionFormat{nocaption}{}
    \captionsetup{format=nocaption,aboveskip=0pt,belowskip=0pt}

    \usepackage{float}
    \floatplacement{figure}{H} % forces figures to be placed at the correct location
    \usepackage{xcolor} % Allow colors to be defined
    \usepackage{enumerate} % Needed for markdown enumerations to work
    \usepackage{geometry} % Used to adjust the document margins
    \usepackage{amsmath} % Equations
    \usepackage{amssymb} % Equations
    \usepackage{textcomp} % defines textquotesingle
    % Hack from http://tex.stackexchange.com/a/47451/13684:
    \AtBeginDocument{%
        \def\PYZsq{\textquotesingle}% Upright quotes in Pygmentized code
    }
    \usepackage{upquote} % Upright quotes for verbatim code
    \usepackage{eurosym} % defines \euro

    \usepackage{iftex}
    \ifPDFTeX
        \usepackage[T1]{fontenc}
        \IfFileExists{alphabeta.sty}{
              \usepackage{alphabeta}
          }{
              \usepackage[mathletters]{ucs}
              \usepackage[utf8x]{inputenc}
          }
    \else
        \usepackage{fontspec}
        \usepackage{unicode-math}
    \fi

    \usepackage{fancyvrb} % verbatim replacement that allows latex
    \usepackage{grffile} % extends the file name processing of package graphics
                         % to support a larger range
    \makeatletter % fix for old versions of grffile with XeLaTeX
    \@ifpackagelater{grffile}{2019/11/01}
    {
      % Do nothing on new versions
    }
    {
      \def\Gread@@xetex#1{%
        \IfFileExists{"\Gin@base".bb}%
        {\Gread@eps{\Gin@base.bb}}%
        {\Gread@@xetex@aux#1}%
      }
    }
    \makeatother
    \usepackage[Export]{adjustbox} % Used to constrain images to a maximum size
    \adjustboxset{max size={0.9\linewidth}{0.9\paperheight}}

    % The hyperref package gives us a pdf with properly built
    % internal navigation ('pdf bookmarks' for the table of contents,
    % internal cross-reference links, web links for URLs, etc.)
    \usepackage{hyperref}
    % The default LaTeX title has an obnoxious amount of whitespace. By default,
    % titling removes some of it. It also provides customization options.
    \usepackage{titling}
    \usepackage{longtable} % longtable support required by pandoc >1.10
    \usepackage{booktabs}  % table support for pandoc > 1.12.2
    \usepackage{array}     % table support for pandoc >= 2.11.3
    \usepackage{calc}      % table minipage width calculation for pandoc >= 2.11.1
    \usepackage[inline]{enumitem} % IRkernel/repr support (it uses the enumerate* environment)
    \usepackage[normalem]{ulem} % ulem is needed to support strikethroughs (\sout)
                                % normalem makes italics be italics, not underlines
    \usepackage{soul}      % strikethrough (\st) support for pandoc >= 3.0.0
    \usepackage{mathrsfs}
    

    
    % Colors for the hyperref package
    \definecolor{urlcolor}{rgb}{0,.145,.698}
    \definecolor{linkcolor}{rgb}{.71,0.21,0.01}
    \definecolor{citecolor}{rgb}{.12,.54,.11}

    % ANSI colors
    \definecolor{ansi-black}{HTML}{3E424D}
    \definecolor{ansi-black-intense}{HTML}{282C36}
    \definecolor{ansi-red}{HTML}{E75C58}
    \definecolor{ansi-red-intense}{HTML}{B22B31}
    \definecolor{ansi-green}{HTML}{00A250}
    \definecolor{ansi-green-intense}{HTML}{007427}
    \definecolor{ansi-yellow}{HTML}{DDB62B}
    \definecolor{ansi-yellow-intense}{HTML}{B27D12}
    \definecolor{ansi-blue}{HTML}{208FFB}
    \definecolor{ansi-blue-intense}{HTML}{0065CA}
    \definecolor{ansi-magenta}{HTML}{D160C4}
    \definecolor{ansi-magenta-intense}{HTML}{A03196}
    \definecolor{ansi-cyan}{HTML}{60C6C8}
    \definecolor{ansi-cyan-intense}{HTML}{258F8F}
    \definecolor{ansi-white}{HTML}{C5C1B4}
    \definecolor{ansi-white-intense}{HTML}{A1A6B2}
    \definecolor{ansi-default-inverse-fg}{HTML}{FFFFFF}
    \definecolor{ansi-default-inverse-bg}{HTML}{000000}

    % common color for the border for error outputs.
    \definecolor{outerrorbackground}{HTML}{FFDFDF}

    % commands and environments needed by pandoc snippets
    % extracted from the output of `pandoc -s`
    \providecommand{\tightlist}{%
      \setlength{\itemsep}{0pt}\setlength{\parskip}{0pt}}
    \DefineVerbatimEnvironment{Highlighting}{Verbatim}{commandchars=\\\{\}}
    % Add ',fontsize=\small' for more characters per line
    \newenvironment{Shaded}{}{}
    \newcommand{\KeywordTok}[1]{\textcolor[rgb]{0.00,0.44,0.13}{\textbf{{#1}}}}
    \newcommand{\DataTypeTok}[1]{\textcolor[rgb]{0.56,0.13,0.00}{{#1}}}
    \newcommand{\DecValTok}[1]{\textcolor[rgb]{0.25,0.63,0.44}{{#1}}}
    \newcommand{\BaseNTok}[1]{\textcolor[rgb]{0.25,0.63,0.44}{{#1}}}
    \newcommand{\FloatTok}[1]{\textcolor[rgb]{0.25,0.63,0.44}{{#1}}}
    \newcommand{\CharTok}[1]{\textcolor[rgb]{0.25,0.44,0.63}{{#1}}}
    \newcommand{\StringTok}[1]{\textcolor[rgb]{0.25,0.44,0.63}{{#1}}}
    \newcommand{\CommentTok}[1]{\textcolor[rgb]{0.38,0.63,0.69}{\textit{{#1}}}}
    \newcommand{\OtherTok}[1]{\textcolor[rgb]{0.00,0.44,0.13}{{#1}}}
    \newcommand{\AlertTok}[1]{\textcolor[rgb]{1.00,0.00,0.00}{\textbf{{#1}}}}
    \newcommand{\FunctionTok}[1]{\textcolor[rgb]{0.02,0.16,0.49}{{#1}}}
    \newcommand{\RegionMarkerTok}[1]{{#1}}
    \newcommand{\ErrorTok}[1]{\textcolor[rgb]{1.00,0.00,0.00}{\textbf{{#1}}}}
    \newcommand{\NormalTok}[1]{{#1}}

    % Additional commands for more recent versions of Pandoc
    \newcommand{\ConstantTok}[1]{\textcolor[rgb]{0.53,0.00,0.00}{{#1}}}
    \newcommand{\SpecialCharTok}[1]{\textcolor[rgb]{0.25,0.44,0.63}{{#1}}}
    \newcommand{\VerbatimStringTok}[1]{\textcolor[rgb]{0.25,0.44,0.63}{{#1}}}
    \newcommand{\SpecialStringTok}[1]{\textcolor[rgb]{0.73,0.40,0.53}{{#1}}}
    \newcommand{\ImportTok}[1]{{#1}}
    \newcommand{\DocumentationTok}[1]{\textcolor[rgb]{0.73,0.13,0.13}{\textit{{#1}}}}
    \newcommand{\AnnotationTok}[1]{\textcolor[rgb]{0.38,0.63,0.69}{\textbf{\textit{{#1}}}}}
    \newcommand{\CommentVarTok}[1]{\textcolor[rgb]{0.38,0.63,0.69}{\textbf{\textit{{#1}}}}}
    \newcommand{\VariableTok}[1]{\textcolor[rgb]{0.10,0.09,0.49}{{#1}}}
    \newcommand{\ControlFlowTok}[1]{\textcolor[rgb]{0.00,0.44,0.13}{\textbf{{#1}}}}
    \newcommand{\OperatorTok}[1]{\textcolor[rgb]{0.40,0.40,0.40}{{#1}}}
    \newcommand{\BuiltInTok}[1]{{#1}}
    \newcommand{\ExtensionTok}[1]{{#1}}
    \newcommand{\PreprocessorTok}[1]{\textcolor[rgb]{0.74,0.48,0.00}{{#1}}}
    \newcommand{\AttributeTok}[1]{\textcolor[rgb]{0.49,0.56,0.16}{{#1}}}
    \newcommand{\InformationTok}[1]{\textcolor[rgb]{0.38,0.63,0.69}{\textbf{\textit{{#1}}}}}
    \newcommand{\WarningTok}[1]{\textcolor[rgb]{0.38,0.63,0.69}{\textbf{\textit{{#1}}}}}


    % Define a nice break command that doesn't care if a line doesn't already
    % exist.
    \def\br{\hspace*{\fill} \\* }
    % Math Jax compatibility definitions
    \def\gt{>}
    \def\lt{<}
    \let\Oldtex\TeX
    \let\Oldlatex\LaTeX
    \renewcommand{\TeX}{\textrm{\Oldtex}}
    \renewcommand{\LaTeX}{\textrm{\Oldlatex}}
    % Document parameters
    % Document title
    \title{assn08}
    
    
    
    
    
    
    
% Pygments definitions
\makeatletter
\def\PY@reset{\let\PY@it=\relax \let\PY@bf=\relax%
    \let\PY@ul=\relax \let\PY@tc=\relax%
    \let\PY@bc=\relax \let\PY@ff=\relax}
\def\PY@tok#1{\csname PY@tok@#1\endcsname}
\def\PY@toks#1+{\ifx\relax#1\empty\else%
    \PY@tok{#1}\expandafter\PY@toks\fi}
\def\PY@do#1{\PY@bc{\PY@tc{\PY@ul{%
    \PY@it{\PY@bf{\PY@ff{#1}}}}}}}
\def\PY#1#2{\PY@reset\PY@toks#1+\relax+\PY@do{#2}}

\@namedef{PY@tok@w}{\def\PY@tc##1{\textcolor[rgb]{0.73,0.73,0.73}{##1}}}
\@namedef{PY@tok@c}{\let\PY@it=\textit\def\PY@tc##1{\textcolor[rgb]{0.24,0.48,0.48}{##1}}}
\@namedef{PY@tok@cp}{\def\PY@tc##1{\textcolor[rgb]{0.61,0.40,0.00}{##1}}}
\@namedef{PY@tok@k}{\let\PY@bf=\textbf\def\PY@tc##1{\textcolor[rgb]{0.00,0.50,0.00}{##1}}}
\@namedef{PY@tok@kp}{\def\PY@tc##1{\textcolor[rgb]{0.00,0.50,0.00}{##1}}}
\@namedef{PY@tok@kt}{\def\PY@tc##1{\textcolor[rgb]{0.69,0.00,0.25}{##1}}}
\@namedef{PY@tok@o}{\def\PY@tc##1{\textcolor[rgb]{0.40,0.40,0.40}{##1}}}
\@namedef{PY@tok@ow}{\let\PY@bf=\textbf\def\PY@tc##1{\textcolor[rgb]{0.67,0.13,1.00}{##1}}}
\@namedef{PY@tok@nb}{\def\PY@tc##1{\textcolor[rgb]{0.00,0.50,0.00}{##1}}}
\@namedef{PY@tok@nf}{\def\PY@tc##1{\textcolor[rgb]{0.00,0.00,1.00}{##1}}}
\@namedef{PY@tok@nc}{\let\PY@bf=\textbf\def\PY@tc##1{\textcolor[rgb]{0.00,0.00,1.00}{##1}}}
\@namedef{PY@tok@nn}{\let\PY@bf=\textbf\def\PY@tc##1{\textcolor[rgb]{0.00,0.00,1.00}{##1}}}
\@namedef{PY@tok@ne}{\let\PY@bf=\textbf\def\PY@tc##1{\textcolor[rgb]{0.80,0.25,0.22}{##1}}}
\@namedef{PY@tok@nv}{\def\PY@tc##1{\textcolor[rgb]{0.10,0.09,0.49}{##1}}}
\@namedef{PY@tok@no}{\def\PY@tc##1{\textcolor[rgb]{0.53,0.00,0.00}{##1}}}
\@namedef{PY@tok@nl}{\def\PY@tc##1{\textcolor[rgb]{0.46,0.46,0.00}{##1}}}
\@namedef{PY@tok@ni}{\let\PY@bf=\textbf\def\PY@tc##1{\textcolor[rgb]{0.44,0.44,0.44}{##1}}}
\@namedef{PY@tok@na}{\def\PY@tc##1{\textcolor[rgb]{0.41,0.47,0.13}{##1}}}
\@namedef{PY@tok@nt}{\let\PY@bf=\textbf\def\PY@tc##1{\textcolor[rgb]{0.00,0.50,0.00}{##1}}}
\@namedef{PY@tok@nd}{\def\PY@tc##1{\textcolor[rgb]{0.67,0.13,1.00}{##1}}}
\@namedef{PY@tok@s}{\def\PY@tc##1{\textcolor[rgb]{0.73,0.13,0.13}{##1}}}
\@namedef{PY@tok@sd}{\let\PY@it=\textit\def\PY@tc##1{\textcolor[rgb]{0.73,0.13,0.13}{##1}}}
\@namedef{PY@tok@si}{\let\PY@bf=\textbf\def\PY@tc##1{\textcolor[rgb]{0.64,0.35,0.47}{##1}}}
\@namedef{PY@tok@se}{\let\PY@bf=\textbf\def\PY@tc##1{\textcolor[rgb]{0.67,0.36,0.12}{##1}}}
\@namedef{PY@tok@sr}{\def\PY@tc##1{\textcolor[rgb]{0.64,0.35,0.47}{##1}}}
\@namedef{PY@tok@ss}{\def\PY@tc##1{\textcolor[rgb]{0.10,0.09,0.49}{##1}}}
\@namedef{PY@tok@sx}{\def\PY@tc##1{\textcolor[rgb]{0.00,0.50,0.00}{##1}}}
\@namedef{PY@tok@m}{\def\PY@tc##1{\textcolor[rgb]{0.40,0.40,0.40}{##1}}}
\@namedef{PY@tok@gh}{\let\PY@bf=\textbf\def\PY@tc##1{\textcolor[rgb]{0.00,0.00,0.50}{##1}}}
\@namedef{PY@tok@gu}{\let\PY@bf=\textbf\def\PY@tc##1{\textcolor[rgb]{0.50,0.00,0.50}{##1}}}
\@namedef{PY@tok@gd}{\def\PY@tc##1{\textcolor[rgb]{0.63,0.00,0.00}{##1}}}
\@namedef{PY@tok@gi}{\def\PY@tc##1{\textcolor[rgb]{0.00,0.52,0.00}{##1}}}
\@namedef{PY@tok@gr}{\def\PY@tc##1{\textcolor[rgb]{0.89,0.00,0.00}{##1}}}
\@namedef{PY@tok@ge}{\let\PY@it=\textit}
\@namedef{PY@tok@gs}{\let\PY@bf=\textbf}
\@namedef{PY@tok@gp}{\let\PY@bf=\textbf\def\PY@tc##1{\textcolor[rgb]{0.00,0.00,0.50}{##1}}}
\@namedef{PY@tok@go}{\def\PY@tc##1{\textcolor[rgb]{0.44,0.44,0.44}{##1}}}
\@namedef{PY@tok@gt}{\def\PY@tc##1{\textcolor[rgb]{0.00,0.27,0.87}{##1}}}
\@namedef{PY@tok@err}{\def\PY@bc##1{{\setlength{\fboxsep}{\string -\fboxrule}\fcolorbox[rgb]{1.00,0.00,0.00}{1,1,1}{\strut ##1}}}}
\@namedef{PY@tok@kc}{\let\PY@bf=\textbf\def\PY@tc##1{\textcolor[rgb]{0.00,0.50,0.00}{##1}}}
\@namedef{PY@tok@kd}{\let\PY@bf=\textbf\def\PY@tc##1{\textcolor[rgb]{0.00,0.50,0.00}{##1}}}
\@namedef{PY@tok@kn}{\let\PY@bf=\textbf\def\PY@tc##1{\textcolor[rgb]{0.00,0.50,0.00}{##1}}}
\@namedef{PY@tok@kr}{\let\PY@bf=\textbf\def\PY@tc##1{\textcolor[rgb]{0.00,0.50,0.00}{##1}}}
\@namedef{PY@tok@bp}{\def\PY@tc##1{\textcolor[rgb]{0.00,0.50,0.00}{##1}}}
\@namedef{PY@tok@fm}{\def\PY@tc##1{\textcolor[rgb]{0.00,0.00,1.00}{##1}}}
\@namedef{PY@tok@vc}{\def\PY@tc##1{\textcolor[rgb]{0.10,0.09,0.49}{##1}}}
\@namedef{PY@tok@vg}{\def\PY@tc##1{\textcolor[rgb]{0.10,0.09,0.49}{##1}}}
\@namedef{PY@tok@vi}{\def\PY@tc##1{\textcolor[rgb]{0.10,0.09,0.49}{##1}}}
\@namedef{PY@tok@vm}{\def\PY@tc##1{\textcolor[rgb]{0.10,0.09,0.49}{##1}}}
\@namedef{PY@tok@sa}{\def\PY@tc##1{\textcolor[rgb]{0.73,0.13,0.13}{##1}}}
\@namedef{PY@tok@sb}{\def\PY@tc##1{\textcolor[rgb]{0.73,0.13,0.13}{##1}}}
\@namedef{PY@tok@sc}{\def\PY@tc##1{\textcolor[rgb]{0.73,0.13,0.13}{##1}}}
\@namedef{PY@tok@dl}{\def\PY@tc##1{\textcolor[rgb]{0.73,0.13,0.13}{##1}}}
\@namedef{PY@tok@s2}{\def\PY@tc##1{\textcolor[rgb]{0.73,0.13,0.13}{##1}}}
\@namedef{PY@tok@sh}{\def\PY@tc##1{\textcolor[rgb]{0.73,0.13,0.13}{##1}}}
\@namedef{PY@tok@s1}{\def\PY@tc##1{\textcolor[rgb]{0.73,0.13,0.13}{##1}}}
\@namedef{PY@tok@mb}{\def\PY@tc##1{\textcolor[rgb]{0.40,0.40,0.40}{##1}}}
\@namedef{PY@tok@mf}{\def\PY@tc##1{\textcolor[rgb]{0.40,0.40,0.40}{##1}}}
\@namedef{PY@tok@mh}{\def\PY@tc##1{\textcolor[rgb]{0.40,0.40,0.40}{##1}}}
\@namedef{PY@tok@mi}{\def\PY@tc##1{\textcolor[rgb]{0.40,0.40,0.40}{##1}}}
\@namedef{PY@tok@il}{\def\PY@tc##1{\textcolor[rgb]{0.40,0.40,0.40}{##1}}}
\@namedef{PY@tok@mo}{\def\PY@tc##1{\textcolor[rgb]{0.40,0.40,0.40}{##1}}}
\@namedef{PY@tok@ch}{\let\PY@it=\textit\def\PY@tc##1{\textcolor[rgb]{0.24,0.48,0.48}{##1}}}
\@namedef{PY@tok@cm}{\let\PY@it=\textit\def\PY@tc##1{\textcolor[rgb]{0.24,0.48,0.48}{##1}}}
\@namedef{PY@tok@cpf}{\let\PY@it=\textit\def\PY@tc##1{\textcolor[rgb]{0.24,0.48,0.48}{##1}}}
\@namedef{PY@tok@c1}{\let\PY@it=\textit\def\PY@tc##1{\textcolor[rgb]{0.24,0.48,0.48}{##1}}}
\@namedef{PY@tok@cs}{\let\PY@it=\textit\def\PY@tc##1{\textcolor[rgb]{0.24,0.48,0.48}{##1}}}

\def\PYZbs{\char`\\}
\def\PYZus{\char`\_}
\def\PYZob{\char`\{}
\def\PYZcb{\char`\}}
\def\PYZca{\char`\^}
\def\PYZam{\char`\&}
\def\PYZlt{\char`\<}
\def\PYZgt{\char`\>}
\def\PYZsh{\char`\#}
\def\PYZpc{\char`\%}
\def\PYZdl{\char`\$}
\def\PYZhy{\char`\-}
\def\PYZsq{\char`\'}
\def\PYZdq{\char`\"}
\def\PYZti{\char`\~}
% for compatibility with earlier versions
\def\PYZat{@}
\def\PYZlb{[}
\def\PYZrb{]}
\makeatother


    % For linebreaks inside Verbatim environment from package fancyvrb.
    \makeatletter
        \newbox\Wrappedcontinuationbox
        \newbox\Wrappedvisiblespacebox
        \newcommand*\Wrappedvisiblespace {\textcolor{red}{\textvisiblespace}}
        \newcommand*\Wrappedcontinuationsymbol {\textcolor{red}{\llap{\tiny$\m@th\hookrightarrow$}}}
        \newcommand*\Wrappedcontinuationindent {3ex }
        \newcommand*\Wrappedafterbreak {\kern\Wrappedcontinuationindent\copy\Wrappedcontinuationbox}
        % Take advantage of the already applied Pygments mark-up to insert
        % potential linebreaks for TeX processing.
        %        {, <, #, %, $, ' and ": go to next line.
        %        _, }, ^, &, >, - and ~: stay at end of broken line.
        % Use of \textquotesingle for straight quote.
        \newcommand*\Wrappedbreaksatspecials {%
            \def\PYGZus{\discretionary{\char`\_}{\Wrappedafterbreak}{\char`\_}}%
            \def\PYGZob{\discretionary{}{\Wrappedafterbreak\char`\{}{\char`\{}}%
            \def\PYGZcb{\discretionary{\char`\}}{\Wrappedafterbreak}{\char`\}}}%
            \def\PYGZca{\discretionary{\char`\^}{\Wrappedafterbreak}{\char`\^}}%
            \def\PYGZam{\discretionary{\char`\&}{\Wrappedafterbreak}{\char`\&}}%
            \def\PYGZlt{\discretionary{}{\Wrappedafterbreak\char`\<}{\char`\<}}%
            \def\PYGZgt{\discretionary{\char`\>}{\Wrappedafterbreak}{\char`\>}}%
            \def\PYGZsh{\discretionary{}{\Wrappedafterbreak\char`\#}{\char`\#}}%
            \def\PYGZpc{\discretionary{}{\Wrappedafterbreak\char`\%}{\char`\%}}%
            \def\PYGZdl{\discretionary{}{\Wrappedafterbreak\char`\$}{\char`\$}}%
            \def\PYGZhy{\discretionary{\char`\-}{\Wrappedafterbreak}{\char`\-}}%
            \def\PYGZsq{\discretionary{}{\Wrappedafterbreak\textquotesingle}{\textquotesingle}}%
            \def\PYGZdq{\discretionary{}{\Wrappedafterbreak\char`\"}{\char`\"}}%
            \def\PYGZti{\discretionary{\char`\~}{\Wrappedafterbreak}{\char`\~}}%
        }
        % Some characters . , ; ? ! / are not pygmentized.
        % This macro makes them "active" and they will insert potential linebreaks
        \newcommand*\Wrappedbreaksatpunct {%
            \lccode`\~`\.\lowercase{\def~}{\discretionary{\hbox{\char`\.}}{\Wrappedafterbreak}{\hbox{\char`\.}}}%
            \lccode`\~`\,\lowercase{\def~}{\discretionary{\hbox{\char`\,}}{\Wrappedafterbreak}{\hbox{\char`\,}}}%
            \lccode`\~`\;\lowercase{\def~}{\discretionary{\hbox{\char`\;}}{\Wrappedafterbreak}{\hbox{\char`\;}}}%
            \lccode`\~`\:\lowercase{\def~}{\discretionary{\hbox{\char`\:}}{\Wrappedafterbreak}{\hbox{\char`\:}}}%
            \lccode`\~`\?\lowercase{\def~}{\discretionary{\hbox{\char`\?}}{\Wrappedafterbreak}{\hbox{\char`\?}}}%
            \lccode`\~`\!\lowercase{\def~}{\discretionary{\hbox{\char`\!}}{\Wrappedafterbreak}{\hbox{\char`\!}}}%
            \lccode`\~`\/\lowercase{\def~}{\discretionary{\hbox{\char`\/}}{\Wrappedafterbreak}{\hbox{\char`\/}}}%
            \catcode`\.\active
            \catcode`\,\active
            \catcode`\;\active
            \catcode`\:\active
            \catcode`\?\active
            \catcode`\!\active
            \catcode`\/\active
            \lccode`\~`\~
        }
    \makeatother

    \let\OriginalVerbatim=\Verbatim
    \makeatletter
    \renewcommand{\Verbatim}[1][1]{%
        %\parskip\z@skip
        \sbox\Wrappedcontinuationbox {\Wrappedcontinuationsymbol}%
        \sbox\Wrappedvisiblespacebox {\FV@SetupFont\Wrappedvisiblespace}%
        \def\FancyVerbFormatLine ##1{\hsize\linewidth
            \vtop{\raggedright\hyphenpenalty\z@\exhyphenpenalty\z@
                \doublehyphendemerits\z@\finalhyphendemerits\z@
                \strut ##1\strut}%
        }%
        % If the linebreak is at a space, the latter will be displayed as visible
        % space at end of first line, and a continuation symbol starts next line.
        % Stretch/shrink are however usually zero for typewriter font.
        \def\FV@Space {%
            \nobreak\hskip\z@ plus\fontdimen3\font minus\fontdimen4\font
            \discretionary{\copy\Wrappedvisiblespacebox}{\Wrappedafterbreak}
            {\kern\fontdimen2\font}%
        }%

        % Allow breaks at special characters using \PYG... macros.
        \Wrappedbreaksatspecials
        % Breaks at punctuation characters . , ; ? ! and / need catcode=\active
        \OriginalVerbatim[#1,codes*=\Wrappedbreaksatpunct]%
    }
    \makeatother

    % Exact colors from NB
    \definecolor{incolor}{HTML}{303F9F}
    \definecolor{outcolor}{HTML}{D84315}
    \definecolor{cellborder}{HTML}{CFCFCF}
    \definecolor{cellbackground}{HTML}{F7F7F7}

    % prompt
    \makeatletter
    \newcommand{\boxspacing}{\kern\kvtcb@left@rule\kern\kvtcb@boxsep}
    \makeatother
    \newcommand{\prompt}[4]{
        {\ttfamily\llap{{\color{#2}[#3]:\hspace{3pt}#4}}\vspace{-\baselineskip}}
    }
    

    
    % Prevent overflowing lines due to hard-to-break entities
    \sloppy
    % Setup hyperref package
    \hypersetup{
      breaklinks=true,  % so long urls are correctly broken across lines
      colorlinks=true,
      urlcolor=urlcolor,
      linkcolor=linkcolor,
      citecolor=citecolor,
      }
    % Slightly bigger margins than the latex defaults
    
    \geometry{verbose,tmargin=1in,bmargin=1in,lmargin=1in,rmargin=1in}
    
    

\begin{document}
    
    \maketitle
    
    

    
    \section{1}\label{section}

设线性方程组

\[
\begin{cases}
5x_1 + 2x_2 + x_3 = -12\\
-x_1 + 4x_2 + 2x_3 = 20\\
2x_1 - 3x_2 + 10x_3 = 3
\end{cases}
\]

\begin{enumerate}
\def\labelenumi{(\arabic{enumi})}
\item
  考察用雅可比迭代法、高斯-塞德尔迭代法解此方程组的收敛性。
\item
  用雅可比迭代法及高斯-塞德尔迭代法解此方程组,要求当
  \(||\bold{x}^{k + 1} - \bold{x}^k||_{\infty} < 10^{-4}\) 时迭代终止。
\end{enumerate}

    \subsection{Solution}\label{solution}

\textbf{(1) 收敛性}

检查矩阵 A 是否为严格对角占优矩阵。 *
对于第一行:\(|5| > |2| + |1| \implies 5 > 3\) (成立) *
对于第二行:\(|4| > |-1| + |2| \implies 4 > 3\) (成立) *
对于第三行:\(|10| > |2| + |-3| \implies 10 > 5\) (成立)

由于矩阵 A 是严格对角占优的,因此雅可比迭代法和高斯-塞德尔迭代法都收敛。

\textbf{(2) 迭代求解}

使用 Python 计算雅可比迭代和高斯-塞德尔迭代:

    \begin{tcolorbox}[breakable, size=fbox, boxrule=1pt, pad at break*=1mm,colback=cellbackground, colframe=cellborder]
\prompt{In}{incolor}{ }{\boxspacing}
\begin{Verbatim}[commandchars=\\\{\}]
\PY{k+kn}{import} \PY{n+nn}{numpy} \PY{k}{as} \PY{n+nn}{np}

\PY{k}{def} \PY{n+nf}{solve\PYZus{}linear\PYZus{}system}\PY{p}{(}\PY{p}{)}\PY{p}{:}
    \PY{n}{A} \PY{o}{=} \PY{n}{np}\PY{o}{.}\PY{n}{array}\PY{p}{(}\PY{p}{[}\PY{p}{[}\PY{l+m+mi}{5}\PY{p}{,} \PY{l+m+mi}{2}\PY{p}{,} \PY{l+m+mi}{1}\PY{p}{]}\PY{p}{,}
                  \PY{p}{[}\PY{o}{\PYZhy{}}\PY{l+m+mi}{1}\PY{p}{,} \PY{l+m+mi}{4}\PY{p}{,} \PY{l+m+mi}{2}\PY{p}{]}\PY{p}{,}
                  \PY{p}{[}\PY{l+m+mi}{2}\PY{p}{,} \PY{o}{\PYZhy{}}\PY{l+m+mi}{3}\PY{p}{,} \PY{l+m+mi}{10}\PY{p}{]}\PY{p}{]}\PY{p}{,} \PY{n}{dtype}\PY{o}{=}\PY{n+nb}{float}\PY{p}{)}
    \PY{n}{b} \PY{o}{=} \PY{n}{np}\PY{o}{.}\PY{n}{array}\PY{p}{(}\PY{p}{[}\PY{o}{\PYZhy{}}\PY{l+m+mi}{12}\PY{p}{,} \PY{l+m+mi}{20}\PY{p}{,} \PY{l+m+mi}{3}\PY{p}{]}\PY{p}{,} \PY{n}{dtype}\PY{o}{=}\PY{n+nb}{float}\PY{p}{)}
    
    \PY{n}{n} \PY{o}{=} \PY{n+nb}{len}\PY{p}{(}\PY{n}{b}\PY{p}{)}
    \PY{n}{tol} \PY{o}{=} \PY{l+m+mf}{1e\PYZhy{}4}
    \PY{n}{max\PYZus{}iter} \PY{o}{=} \PY{l+m+mi}{1000}

    \PY{c+c1}{\PYZsh{} \PYZhy{}\PYZhy{}\PYZhy{} 雅可比迭代 \PYZhy{}\PYZhy{}\PYZhy{}}
    \PY{n+nb}{print}\PY{p}{(}\PY{l+s+s2}{\PYZdq{}}\PY{l+s+s2}{雅可比迭代:}\PY{l+s+s2}{\PYZdq{}}\PY{p}{)}
    \PY{n}{x\PYZus{}jacobi} \PY{o}{=} \PY{n}{np}\PY{o}{.}\PY{n}{zeros}\PY{p}{(}\PY{n}{n}\PY{p}{)}
    \PY{k}{for} \PY{n}{k\PYZus{}jacobi} \PY{o+ow}{in} \PY{n+nb}{range}\PY{p}{(}\PY{n}{max\PYZus{}iter}\PY{p}{)}\PY{p}{:}
        \PY{n}{x\PYZus{}new\PYZus{}jacobi} \PY{o}{=} \PY{n}{np}\PY{o}{.}\PY{n}{zeros}\PY{p}{(}\PY{n}{n}\PY{p}{)}
        \PY{k}{for} \PY{n}{i} \PY{o+ow}{in} \PY{n+nb}{range}\PY{p}{(}\PY{n}{n}\PY{p}{)}\PY{p}{:}
            \PY{n}{s} \PY{o}{=} \PY{n+nb}{sum}\PY{p}{(}\PY{n}{A}\PY{p}{[}\PY{n}{i}\PY{p}{,} \PY{n}{j}\PY{p}{]} \PY{o}{*} \PY{n}{x\PYZus{}jacobi}\PY{p}{[}\PY{n}{j}\PY{p}{]} \PY{k}{for} \PY{n}{j} \PY{o+ow}{in} \PY{n+nb}{range}\PY{p}{(}\PY{n}{n}\PY{p}{)} \PY{k}{if} \PY{n}{j} \PY{o}{!=} \PY{n}{i}\PY{p}{)}
            \PY{n}{x\PYZus{}new\PYZus{}jacobi}\PY{p}{[}\PY{n}{i}\PY{p}{]} \PY{o}{=} \PY{p}{(}\PY{n}{b}\PY{p}{[}\PY{n}{i}\PY{p}{]} \PY{o}{\PYZhy{}} \PY{n}{s}\PY{p}{)} \PY{o}{/} \PY{n}{A}\PY{p}{[}\PY{n}{i}\PY{p}{,} \PY{n}{i}\PY{p}{]}
        
        \PY{k}{if} \PY{n}{np}\PY{o}{.}\PY{n}{linalg}\PY{o}{.}\PY{n}{norm}\PY{p}{(}\PY{n}{x\PYZus{}new\PYZus{}jacobi} \PY{o}{\PYZhy{}} \PY{n}{x\PYZus{}jacobi}\PY{p}{,} \PY{n}{np}\PY{o}{.}\PY{n}{inf}\PY{p}{)} \PY{o}{\PYZlt{}} \PY{n}{tol}\PY{p}{:}
            \PY{n}{x\PYZus{}jacobi} \PY{o}{=} \PY{n}{x\PYZus{}new\PYZus{}jacobi}
            \PY{k}{break}
        \PY{n}{x\PYZus{}jacobi} \PY{o}{=} \PY{n}{x\PYZus{}new\PYZus{}jacobi}

    \PY{n+nb}{print}\PY{p}{(}\PY{l+s+sa}{f}\PY{l+s+s2}{\PYZdq{}}\PY{l+s+s2}{解: }\PY{l+s+si}{\PYZob{}}\PY{n}{x\PYZus{}jacobi}\PY{l+s+si}{\PYZcb{}}\PY{l+s+s2}{\PYZdq{}}\PY{p}{)}
    \PY{n+nb}{print}\PY{p}{(}\PY{l+s+sa}{f}\PY{l+s+s2}{\PYZdq{}}\PY{l+s+s2}{迭代次数: }\PY{l+s+si}{\PYZob{}}\PY{n}{k\PYZus{}jacobi}\PY{+w}{ }\PY{o}{+}\PY{+w}{ }\PY{l+m+mi}{1}\PY{l+s+si}{\PYZcb{}}\PY{l+s+s2}{\PYZdq{}}\PY{p}{)}
    \PY{n+nb}{print}\PY{p}{(}\PY{l+s+s2}{\PYZdq{}}\PY{l+s+s2}{\PYZhy{}}\PY{l+s+s2}{\PYZdq{}} \PY{o}{*} \PY{l+m+mi}{30}\PY{p}{)}

    \PY{c+c1}{\PYZsh{} \PYZhy{}\PYZhy{}\PYZhy{} 高斯\PYZhy{}塞德尔迭代 \PYZhy{}\PYZhy{}\PYZhy{}}
    \PY{n+nb}{print}\PY{p}{(}\PY{l+s+s2}{\PYZdq{}}\PY{l+s+s2}{高斯\PYZhy{}塞德尔迭代:}\PY{l+s+s2}{\PYZdq{}}\PY{p}{)}
    \PY{n}{x\PYZus{}gs} \PY{o}{=} \PY{n}{np}\PY{o}{.}\PY{n}{zeros}\PY{p}{(}\PY{n}{n}\PY{p}{)}
    \PY{k}{for} \PY{n}{k\PYZus{}gs} \PY{o+ow}{in} \PY{n+nb}{range}\PY{p}{(}\PY{n}{max\PYZus{}iter}\PY{p}{)}\PY{p}{:}
        \PY{n}{x\PYZus{}old\PYZus{}gs} \PY{o}{=} \PY{n}{x\PYZus{}gs}\PY{o}{.}\PY{n}{copy}\PY{p}{(}\PY{p}{)}
        \PY{k}{for} \PY{n}{i} \PY{o+ow}{in} \PY{n+nb}{range}\PY{p}{(}\PY{n}{n}\PY{p}{)}\PY{p}{:}
            \PY{n}{s1} \PY{o}{=} \PY{n+nb}{sum}\PY{p}{(}\PY{n}{A}\PY{p}{[}\PY{n}{i}\PY{p}{,} \PY{n}{j}\PY{p}{]} \PY{o}{*} \PY{n}{x\PYZus{}gs}\PY{p}{[}\PY{n}{j}\PY{p}{]} \PY{k}{for} \PY{n}{j} \PY{o+ow}{in} \PY{n+nb}{range}\PY{p}{(}\PY{n}{i}\PY{p}{)}\PY{p}{)}
            \PY{n}{s2} \PY{o}{=} \PY{n+nb}{sum}\PY{p}{(}\PY{n}{A}\PY{p}{[}\PY{n}{i}\PY{p}{,} \PY{n}{j}\PY{p}{]} \PY{o}{*} \PY{n}{x\PYZus{}old\PYZus{}gs}\PY{p}{[}\PY{n}{j}\PY{p}{]} \PY{k}{for} \PY{n}{j} \PY{o+ow}{in} \PY{n+nb}{range}\PY{p}{(}\PY{n}{i} \PY{o}{+} \PY{l+m+mi}{1}\PY{p}{,} \PY{n}{n}\PY{p}{)}\PY{p}{)}
            \PY{n}{x\PYZus{}gs}\PY{p}{[}\PY{n}{i}\PY{p}{]} \PY{o}{=} \PY{p}{(}\PY{n}{b}\PY{p}{[}\PY{n}{i}\PY{p}{]} \PY{o}{\PYZhy{}} \PY{n}{s1} \PY{o}{\PYZhy{}} \PY{n}{s2}\PY{p}{)} \PY{o}{/} \PY{n}{A}\PY{p}{[}\PY{n}{i}\PY{p}{,} \PY{n}{i}\PY{p}{]}
        
        \PY{k}{if} \PY{n}{np}\PY{o}{.}\PY{n}{linalg}\PY{o}{.}\PY{n}{norm}\PY{p}{(}\PY{n}{x\PYZus{}gs} \PY{o}{\PYZhy{}} \PY{n}{x\PYZus{}old\PYZus{}gs}\PY{p}{,} \PY{n}{np}\PY{o}{.}\PY{n}{inf}\PY{p}{)} \PY{o}{\PYZlt{}} \PY{n}{tol}\PY{p}{:}
            \PY{k}{break}

    \PY{n+nb}{print}\PY{p}{(}\PY{l+s+sa}{f}\PY{l+s+s2}{\PYZdq{}}\PY{l+s+s2}{解: }\PY{l+s+si}{\PYZob{}}\PY{n}{x\PYZus{}gs}\PY{l+s+si}{\PYZcb{}}\PY{l+s+s2}{\PYZdq{}}\PY{p}{)}
    \PY{n+nb}{print}\PY{p}{(}\PY{l+s+sa}{f}\PY{l+s+s2}{\PYZdq{}}\PY{l+s+s2}{迭代次数: }\PY{l+s+si}{\PYZob{}}\PY{n}{k\PYZus{}gs}\PY{+w}{ }\PY{o}{+}\PY{+w}{ }\PY{l+m+mi}{1}\PY{l+s+si}{\PYZcb{}}\PY{l+s+s2}{\PYZdq{}}\PY{p}{)}
    \PY{n+nb}{print}\PY{p}{(}\PY{l+s+s2}{\PYZdq{}}\PY{l+s+s2}{\PYZhy{}}\PY{l+s+s2}{\PYZdq{}} \PY{o}{*} \PY{l+m+mi}{30}\PY{p}{)}

\PY{n}{solve\PYZus{}linear\PYZus{}system}\PY{p}{(}\PY{p}{)}
\end{Verbatim}
\end{tcolorbox}

    \begin{Verbatim}[commandchars=\\\{\}]
雅可比迭代:
解: [-3.99999642  2.99997389  1.99999989]
迭代次数: 18
------------------------------
高斯-塞德尔迭代:
解: [-4.00003333  2.99998307  2.00000159]
迭代次数: 8
------------------------------
    \end{Verbatim}

    \section{4}\label{section}

设

\[
A = \begin{pmatrix}
10 \quad a \quad 0 \\
b \quad 10 \quad b \\
0 \quad a \quad 5
\end{pmatrix}, \quad \det(A) \neq 0
\]

用 \(a, b\) 表示解线性方程组 \(A\bold{x} = f\)
的雅可比迭代与高斯-塞德尔迭代收敛的充分必要条件。

    \subsection{Solution}\label{solution}

\textbf{1. 雅可比迭代法}

雅可比迭代矩阵 \(B_J = D^{-1}(L+U)\):

\[B_J = \begin{pmatrix}
0 & -a/10 & 0 \\
-b/10 & 0 & -b/10 \\
0 & -a/5 & 0
\end{pmatrix}\]

计算特征多项式:
\[\det(B_J - \lambda I) = -\lambda^3 + \frac{3ab}{100}\lambda = -\lambda(\lambda^2 - \frac{3ab}{100})\]

特征值:\(\lambda_1 = 0\),\(\lambda_{2,3} = \pm\sqrt{\frac{3ab}{100}}\)

谱半径:\(\rho(B_J) = \sqrt{\frac{|3ab|}{100}}\)

收敛条件:\(\rho(B_J) < 1 \Rightarrow |ab| < \frac{100}{3}\)

\textbf{2. 高斯-塞德尔迭代法}

对于具有一致排序性质的三对角矩阵,有性质:
\[\rho(B_{GS}) = [\rho(B_J)]^2\]

因此:\(\rho(B_{GS}) = \frac{|3ab|}{100}\)

收敛条件:\(\rho(B_{GS}) < 1 \Rightarrow |ab| < \frac{100}{3}\)

\textbf{结论}

雅可比迭代与高斯-塞德尔迭代收敛的充分必要条件都是:

\[|ab| < \frac{100}{3}\]

即:\[-\frac{100}{3} < ab < \frac{100}{3}\]

    \section{5}\label{section}

对线性方程组

\[
\begin{pmatrix}
3 \quad 2 \\
1 \quad 2
\end{pmatrix}
\begin{pmatrix}
x_1 \\
x_2
\end{pmatrix}
=\begin{pmatrix}
3 \\
-1
\end{pmatrix}
\]

若用迭代法

\[
\bold{x}^{(k + 1)} = \bold{x}^{(k)} + \alpha (A\bold{x}^{(k)} - \bold{b}), \quad k = 0, 1, \cdots
\]

求解,问 \(\alpha\) 在什么范围内取值可使迭代收敛; \(\alpha\)
取什么值可使迭代收敛最快。

    \subsection{Solution}\label{solution}

整理得:\(\bold{x}^{(k + 1)} = (I + \alpha A)\bold{x}^{(k)} - \alpha \bold{b}\)

迭代矩阵为:\(G = I + \alpha A\)

其中 \(A = \begin{pmatrix} 3 & 2 \\ 1 & 2 \end{pmatrix}\)

特征多项式:\(\det(A - \lambda I) = (3-\lambda)(2-\lambda) - 2 = \lambda^2 - 5\lambda + 4 = (\lambda-1)(\lambda-4)\)

特征值:\(\lambda_1 = 1, \lambda_2 = 4\)

\(G = I + \alpha A\)
的特征值为:\(\mu_1 = 1 + \alpha, \mu_2 = 1 + 4\alpha\)

迭代收敛的充分必要条件是 \(\rho(G) < 1\),即:
\[|1 + \alpha| < 1 \quad \text{且} \quad |1 + 4\alpha| < 1\]

解不等式: - \(|1 + \alpha| < 1 \Rightarrow -2 < \alpha < 0\) -
\(|1 + 4\alpha| < 1 \Rightarrow -\frac{1}{2} < \alpha < 0\)

取交集得收敛范围:\[-\frac{1}{2} < \alpha < 0\]

收敛速度由谱半径决定,\(\rho(G) = \max\{|1 + \alpha|, |1 + 4\alpha|\}\)

在区间 \((-\frac{1}{2}, 0)\) 内: - 当
\(\alpha \in (-\frac{1}{2}, -\frac{1}{3})\)
时,\(|1 + 4\alpha| > |1 + \alpha|\) - 当
\(\alpha \in (-\frac{1}{3}, 0)\) 时,\(|1 + \alpha| > |1 + 4\alpha|\)

最优值在 \(|1 + \alpha| = |1 + 4\alpha|\) 处取得:
\[1 + \alpha = -(1 + 4\alpha) \Rightarrow \alpha = -\frac{2}{5}\]

此时 \(\rho(G) = |1 - \frac{2}{5}| = \frac{3}{5} = 0.6\)

    \section{6}\label{section}

用雅可比迭代与高斯-塞德尔迭代解线性方程组
\(A\bold{x} = \bold{b}\),证明若取

\[
A = \begin{pmatrix}
3 \quad 0 \quad -2 \\
0 \quad 2 \quad 1 \\
-2 \quad 1 \quad 2
\end{pmatrix}
\]

则两种方法均收敛。试比较哪种方法收敛更快。

    \textbf{1. 收敛性证明}

矩阵
\(A = \begin{pmatrix} 3 & 0 & -2 \\ 0 & 2 & 1 \\ -2 & 1 & 2 \end{pmatrix}\)
是对称矩阵。

计算特征值: -
特征多项式:\(\det(A - \lambda I) = (3-\lambda)(2-\lambda)^2 - 4(2-\lambda) = -(λ-1)(λ-2)(λ-4)\)
- 特征值:\(\lambda_1 = 1, \lambda_2 = 2, \lambda_3 = 4\) (均为正数)

因为A是对称正定矩阵,所以: - 雅可比迭代法收敛 - 高斯-塞德尔迭代法收敛

\textbf{2. 收敛速度比较}

对于对称正定矩阵,有结论: \[\rho(B_{GS}) = [\rho(B_J)]^2\]

这意味着高斯-塞德尔迭代的谱半径是雅可比迭代谱半径的平方,因此: -
\(\rho(B_{GS}) < \rho(B_J)\)(当 \(\rho(B_J) < 1\) 时) -
\textbf{高斯-塞德尔迭代收敛更快}

    \section{8}\label{section}

用 SOR 方法解线性方程组(取松弛因子
\(\omega = 1.03, \omega = 1, \omega = 1.1\))

\[
\begin{cases}
4x_1 - x_2 = 1 \\
-x_1 + 4x_2 - x_3 = 4 \\
-x_2 + 4x_3 = -3
\end{cases}
\]

精确解 \(x^* = (\frac{1}{2}, 1, -\frac{1}{2})^T\),要求当
\(||x^{*} - x^{(k)}||_{\infty} < 5\times 10^{-6}\)
时迭代终止,并且对于每一个 \(\omega\) 值确定迭代次数。

    \subsection{Solution}\label{solution}

SOR 方法的迭代公式为:

\[x_i^{(k+1)} = (1-\omega)x_i^{(k)} + \frac{\omega}{a_{ii}}\left(b_i - \sum_{j=1}^{i-1}a_{ij}x_j^{(k+1)} - \sum_{j=i+1}^{n}a_{ij}x_j^{(k)}\right)\]

其中: - \(\omega = 1\):高斯-塞德尔迭代 - \(0 < \omega < 1\):低松弛 -
\(1 < \omega < 2\):超松弛

使用 python 求解,

    \begin{tcolorbox}[breakable, size=fbox, boxrule=1pt, pad at break*=1mm,colback=cellbackground, colframe=cellborder]
\prompt{In}{incolor}{6}{\boxspacing}
\begin{Verbatim}[commandchars=\\\{\}]
\PY{k+kn}{import} \PY{n+nn}{numpy} \PY{k}{as} \PY{n+nn}{np}

\PY{k}{def} \PY{n+nf}{sor\PYZus{}method}\PY{p}{(}\PY{p}{)}\PY{p}{:}
    \PY{c+c1}{\PYZsh{} 系数矩阵和右端向量}
    \PY{n}{A} \PY{o}{=} \PY{n}{np}\PY{o}{.}\PY{n}{array}\PY{p}{(}\PY{p}{[}\PY{p}{[}\PY{l+m+mi}{4}\PY{p}{,} \PY{o}{\PYZhy{}}\PY{l+m+mi}{1}\PY{p}{,} \PY{l+m+mi}{0}\PY{p}{]}\PY{p}{,}
                  \PY{p}{[}\PY{o}{\PYZhy{}}\PY{l+m+mi}{1}\PY{p}{,} \PY{l+m+mi}{4}\PY{p}{,} \PY{o}{\PYZhy{}}\PY{l+m+mi}{1}\PY{p}{]}\PY{p}{,}
                  \PY{p}{[}\PY{l+m+mi}{0}\PY{p}{,} \PY{o}{\PYZhy{}}\PY{l+m+mi}{1}\PY{p}{,} \PY{l+m+mi}{4}\PY{p}{]}\PY{p}{]}\PY{p}{,} \PY{n}{dtype}\PY{o}{=}\PY{n+nb}{float}\PY{p}{)}
    \PY{n}{b} \PY{o}{=} \PY{n}{np}\PY{o}{.}\PY{n}{array}\PY{p}{(}\PY{p}{[}\PY{l+m+mi}{1}\PY{p}{,} \PY{l+m+mi}{4}\PY{p}{,} \PY{o}{\PYZhy{}}\PY{l+m+mi}{3}\PY{p}{]}\PY{p}{,} \PY{n}{dtype}\PY{o}{=}\PY{n+nb}{float}\PY{p}{)}
    
    \PY{c+c1}{\PYZsh{} 精确解}
    \PY{n}{x\PYZus{}exact} \PY{o}{=} \PY{n}{np}\PY{o}{.}\PY{n}{array}\PY{p}{(}\PY{p}{[}\PY{l+m+mf}{0.5}\PY{p}{,} \PY{l+m+mf}{1.0}\PY{p}{,} \PY{o}{\PYZhy{}}\PY{l+m+mf}{0.5}\PY{p}{]}\PY{p}{)}
    
    \PY{c+c1}{\PYZsh{} 松弛因子}
    \PY{n}{omega\PYZus{}values} \PY{o}{=} \PY{p}{[}\PY{l+m+mf}{1.03}\PY{p}{,} \PY{l+m+mf}{1.0}\PY{p}{,} \PY{l+m+mf}{1.1}\PY{p}{]}
    
    \PY{c+c1}{\PYZsh{} 收敛容差}
    \PY{n}{tol} \PY{o}{=} \PY{l+m+mf}{5e\PYZhy{}6}
    \PY{n}{max\PYZus{}iter} \PY{o}{=} \PY{l+m+mi}{1000}
    
    \PY{n}{results} \PY{o}{=} \PY{p}{[}\PY{p}{]}
    
    \PY{k}{for} \PY{n}{omega} \PY{o+ow}{in} \PY{n}{omega\PYZus{}values}\PY{p}{:}
        \PY{n+nb}{print}\PY{p}{(}\PY{l+s+sa}{f}\PY{l+s+s2}{\PYZdq{}}\PY{l+s+se}{\PYZbs{}n}\PY{l+s+s2}{SOR方法 (ω = }\PY{l+s+si}{\PYZob{}}\PY{n}{omega}\PY{l+s+si}{\PYZcb{}}\PY{l+s+s2}{):}\PY{l+s+s2}{\PYZdq{}}\PY{p}{)}
        \PY{n+nb}{print}\PY{p}{(}\PY{l+s+s2}{\PYZdq{}}\PY{l+s+s2}{\PYZhy{}}\PY{l+s+s2}{\PYZdq{}} \PY{o}{*} \PY{l+m+mi}{40}\PY{p}{)}
        
        \PY{c+c1}{\PYZsh{} 初始值}
        \PY{n}{x} \PY{o}{=} \PY{n}{np}\PY{o}{.}\PY{n}{zeros}\PY{p}{(}\PY{l+m+mi}{3}\PY{p}{)}
        \PY{n}{n} \PY{o}{=} \PY{n+nb}{len}\PY{p}{(}\PY{n}{b}\PY{p}{)}
        
        \PY{c+c1}{\PYZsh{} SOR迭代}
        \PY{k}{for} \PY{n}{k} \PY{o+ow}{in} \PY{n+nb}{range}\PY{p}{(}\PY{n}{max\PYZus{}iter}\PY{p}{)}\PY{p}{:}
            \PY{n}{x\PYZus{}old} \PY{o}{=} \PY{n}{x}\PY{o}{.}\PY{n}{copy}\PY{p}{(}\PY{p}{)}
            
            \PY{c+c1}{\PYZsh{} SOR迭代公式}
            \PY{k}{for} \PY{n}{i} \PY{o+ow}{in} \PY{n+nb}{range}\PY{p}{(}\PY{n}{n}\PY{p}{)}\PY{p}{:}
                \PY{c+c1}{\PYZsh{} 计算前向替换部分 (已更新的分量)}
                \PY{n}{s1} \PY{o}{=} \PY{n+nb}{sum}\PY{p}{(}\PY{n}{A}\PY{p}{[}\PY{n}{i}\PY{p}{,} \PY{n}{j}\PY{p}{]} \PY{o}{*} \PY{n}{x}\PY{p}{[}\PY{n}{j}\PY{p}{]} \PY{k}{for} \PY{n}{j} \PY{o+ow}{in} \PY{n+nb}{range}\PY{p}{(}\PY{n}{i}\PY{p}{)}\PY{p}{)}
                \PY{c+c1}{\PYZsh{} 计算后向替换部分 (未更新的分量)}
                \PY{n}{s2} \PY{o}{=} \PY{n+nb}{sum}\PY{p}{(}\PY{n}{A}\PY{p}{[}\PY{n}{i}\PY{p}{,} \PY{n}{j}\PY{p}{]} \PY{o}{*} \PY{n}{x\PYZus{}old}\PY{p}{[}\PY{n}{j}\PY{p}{]} \PY{k}{for} \PY{n}{j} \PY{o+ow}{in} \PY{n+nb}{range}\PY{p}{(}\PY{n}{i} \PY{o}{+} \PY{l+m+mi}{1}\PY{p}{,} \PY{n}{n}\PY{p}{)}\PY{p}{)}
                
                \PY{c+c1}{\PYZsh{} SOR更新公式}
                \PY{n}{x\PYZus{}new\PYZus{}i} \PY{o}{=} \PY{p}{(}\PY{n}{b}\PY{p}{[}\PY{n}{i}\PY{p}{]} \PY{o}{\PYZhy{}} \PY{n}{s1} \PY{o}{\PYZhy{}} \PY{n}{s2}\PY{p}{)} \PY{o}{/} \PY{n}{A}\PY{p}{[}\PY{n}{i}\PY{p}{,} \PY{n}{i}\PY{p}{]}
                \PY{n}{x}\PY{p}{[}\PY{n}{i}\PY{p}{]} \PY{o}{=} \PY{p}{(}\PY{l+m+mi}{1} \PY{o}{\PYZhy{}} \PY{n}{omega}\PY{p}{)} \PY{o}{*} \PY{n}{x\PYZus{}old}\PY{p}{[}\PY{n}{i}\PY{p}{]} \PY{o}{+} \PY{n}{omega} \PY{o}{*} \PY{n}{x\PYZus{}new\PYZus{}i}
            
            \PY{c+c1}{\PYZsh{} 计算误差}
            \PY{n}{error} \PY{o}{=} \PY{n}{np}\PY{o}{.}\PY{n}{linalg}\PY{o}{.}\PY{n}{norm}\PY{p}{(}\PY{n}{x\PYZus{}exact} \PY{o}{\PYZhy{}} \PY{n}{x}\PY{p}{,} \PY{n}{np}\PY{o}{.}\PY{n}{inf}\PY{p}{)}
            
            \PY{c+c1}{\PYZsh{} 输出迭代过程 (前几步和最后几步)}
            \PY{k}{if} \PY{n}{k} \PY{o}{\PYZlt{}} \PY{l+m+mi}{5} \PY{o+ow}{or} \PY{n}{k} \PY{o}{\PYZpc{}} \PY{l+m+mi}{20} \PY{o}{==} \PY{l+m+mi}{0} \PY{o+ow}{or} \PY{n}{error} \PY{o}{\PYZlt{}} \PY{n}{tol}\PY{p}{:}
                \PY{n+nb}{print}\PY{p}{(}\PY{l+s+sa}{f}\PY{l+s+s2}{\PYZdq{}}\PY{l+s+s2}{k=}\PY{l+s+si}{\PYZob{}}\PY{n}{k}\PY{l+s+si}{:}\PY{l+s+s2}{3d}\PY{l+s+si}{\PYZcb{}}\PY{l+s+s2}{: x = [}\PY{l+s+si}{\PYZob{}}\PY{n}{x}\PY{p}{[}\PY{l+m+mi}{0}\PY{p}{]}\PY{l+s+si}{:}\PY{l+s+s2}{8.6f}\PY{l+s+si}{\PYZcb{}}\PY{l+s+s2}{, }\PY{l+s+si}{\PYZob{}}\PY{n}{x}\PY{p}{[}\PY{l+m+mi}{1}\PY{p}{]}\PY{l+s+si}{:}\PY{l+s+s2}{8.6f}\PY{l+s+si}{\PYZcb{}}\PY{l+s+s2}{, }\PY{l+s+si}{\PYZob{}}\PY{n}{x}\PY{p}{[}\PY{l+m+mi}{2}\PY{p}{]}\PY{l+s+si}{:}\PY{l+s+s2}{8.6f}\PY{l+s+si}{\PYZcb{}}\PY{l+s+s2}{], }\PY{l+s+s2}{\PYZdq{}}
                      \PY{l+s+sa}{f}\PY{l+s+s2}{\PYZdq{}}\PY{l+s+s2}{||x*\PYZhy{}x||∞ = }\PY{l+s+si}{\PYZob{}}\PY{n}{error}\PY{l+s+si}{:}\PY{l+s+s2}{.2e}\PY{l+s+si}{\PYZcb{}}\PY{l+s+s2}{\PYZdq{}}\PY{p}{)}
            
            \PY{c+c1}{\PYZsh{} 检查收敛}
            \PY{k}{if} \PY{n}{error} \PY{o}{\PYZlt{}} \PY{n}{tol}\PY{p}{:}
                \PY{n}{iterations} \PY{o}{=} \PY{n}{k} \PY{o}{+} \PY{l+m+mi}{1}
                \PY{n}{final\PYZus{}error} \PY{o}{=} \PY{n}{error}
                \PY{n}{final\PYZus{}solution} \PY{o}{=} \PY{n}{x}\PY{o}{.}\PY{n}{copy}\PY{p}{(}\PY{p}{)}
                \PY{k}{break}
        \PY{k}{else}\PY{p}{:}
            \PY{n}{iterations} \PY{o}{=} \PY{n}{max\PYZus{}iter}
            \PY{n}{final\PYZus{}error} \PY{o}{=} \PY{n}{error}
            \PY{n}{final\PYZus{}solution} \PY{o}{=} \PY{n}{x}\PY{o}{.}\PY{n}{copy}\PY{p}{(}\PY{p}{)}
            \PY{n+nb}{print}\PY{p}{(}\PY{l+s+s2}{\PYZdq{}}\PY{l+s+s2}{达到最大迭代次数,未收敛}\PY{l+s+s2}{\PYZdq{}}\PY{p}{)}
        
        \PY{n+nb}{print}\PY{p}{(}\PY{l+s+sa}{f}\PY{l+s+s2}{\PYZdq{}}\PY{l+s+se}{\PYZbs{}n}\PY{l+s+s2}{最终结果:}\PY{l+s+s2}{\PYZdq{}}\PY{p}{)}
        \PY{n+nb}{print}\PY{p}{(}\PY{l+s+sa}{f}\PY{l+s+s2}{\PYZdq{}}\PY{l+s+s2}{迭代次数: }\PY{l+s+si}{\PYZob{}}\PY{n}{iterations}\PY{l+s+si}{\PYZcb{}}\PY{l+s+s2}{\PYZdq{}}\PY{p}{)}
        \PY{n+nb}{print}\PY{p}{(}\PY{l+s+sa}{f}\PY{l+s+s2}{\PYZdq{}}\PY{l+s+s2}{最终解: x = [}\PY{l+s+si}{\PYZob{}}\PY{n}{final\PYZus{}solution}\PY{p}{[}\PY{l+m+mi}{0}\PY{p}{]}\PY{l+s+si}{:}\PY{l+s+s2}{.8f}\PY{l+s+si}{\PYZcb{}}\PY{l+s+s2}{, }\PY{l+s+si}{\PYZob{}}\PY{n}{final\PYZus{}solution}\PY{p}{[}\PY{l+m+mi}{1}\PY{p}{]}\PY{l+s+si}{:}\PY{l+s+s2}{.8f}\PY{l+s+si}{\PYZcb{}}\PY{l+s+s2}{, }\PY{l+s+si}{\PYZob{}}\PY{n}{final\PYZus{}solution}\PY{p}{[}\PY{l+m+mi}{2}\PY{p}{]}\PY{l+s+si}{:}\PY{l+s+s2}{.8f}\PY{l+s+si}{\PYZcb{}}\PY{l+s+s2}{]}\PY{l+s+s2}{\PYZdq{}}\PY{p}{)}
        \PY{n+nb}{print}\PY{p}{(}\PY{l+s+sa}{f}\PY{l+s+s2}{\PYZdq{}}\PY{l+s+s2}{最终误差: ||x* \PYZhy{} x||∞ = }\PY{l+s+si}{\PYZob{}}\PY{n}{final\PYZus{}error}\PY{l+s+si}{:}\PY{l+s+s2}{.2e}\PY{l+s+si}{\PYZcb{}}\PY{l+s+s2}{\PYZdq{}}\PY{p}{)}
        
        \PY{n}{results}\PY{o}{.}\PY{n}{append}\PY{p}{(}\PY{p}{\PYZob{}}
            \PY{l+s+s1}{\PYZsq{}}\PY{l+s+s1}{omega}\PY{l+s+s1}{\PYZsq{}}\PY{p}{:} \PY{n}{omega}\PY{p}{,}
            \PY{l+s+s1}{\PYZsq{}}\PY{l+s+s1}{iterations}\PY{l+s+s1}{\PYZsq{}}\PY{p}{:} \PY{n}{iterations}\PY{p}{,}
            \PY{l+s+s1}{\PYZsq{}}\PY{l+s+s1}{final\PYZus{}error}\PY{l+s+s1}{\PYZsq{}}\PY{p}{:} \PY{n}{final\PYZus{}error}\PY{p}{,}
            \PY{l+s+s1}{\PYZsq{}}\PY{l+s+s1}{solution}\PY{l+s+s1}{\PYZsq{}}\PY{p}{:} \PY{n}{final\PYZus{}solution}
        \PY{p}{\PYZcb{}}\PY{p}{)}
    
    \PY{c+c1}{\PYZsh{} 比较结果}
    \PY{n+nb}{print}\PY{p}{(}\PY{l+s+s2}{\PYZdq{}}\PY{l+s+se}{\PYZbs{}n}\PY{l+s+s2}{\PYZdq{}} \PY{o}{+} \PY{l+s+s2}{\PYZdq{}}\PY{l+s+s2}{=}\PY{l+s+s2}{\PYZdq{}}\PY{o}{*}\PY{l+m+mi}{60}\PY{p}{)}
    \PY{n+nb}{print}\PY{p}{(}\PY{l+s+s2}{\PYZdq{}}\PY{l+s+s2}{结果比较:}\PY{l+s+s2}{\PYZdq{}}\PY{p}{)}
    \PY{n+nb}{print}\PY{p}{(}\PY{l+s+s2}{\PYZdq{}}\PY{l+s+s2}{\PYZhy{}}\PY{l+s+s2}{\PYZdq{}} \PY{o}{*} \PY{l+m+mi}{60}\PY{p}{)}
    \PY{n+nb}{print}\PY{p}{(}\PY{l+s+sa}{f}\PY{l+s+s2}{\PYZdq{}}\PY{l+s+si}{\PYZob{}}\PY{l+s+s1}{\PYZsq{}}\PY{l+s+s1}{ω值}\PY{l+s+s1}{\PYZsq{}}\PY{l+s+si}{:}\PY{l+s+s2}{\PYZlt{}8}\PY{l+s+si}{\PYZcb{}}\PY{l+s+s2}{ }\PY{l+s+si}{\PYZob{}}\PY{l+s+s1}{\PYZsq{}}\PY{l+s+s1}{迭代次数}\PY{l+s+s1}{\PYZsq{}}\PY{l+s+si}{:}\PY{l+s+s2}{\PYZlt{}10}\PY{l+s+si}{\PYZcb{}}\PY{l+s+s2}{ }\PY{l+s+si}{\PYZob{}}\PY{l+s+s1}{\PYZsq{}}\PY{l+s+s1}{最终误差}\PY{l+s+s1}{\PYZsq{}}\PY{l+s+si}{:}\PY{l+s+s2}{\PYZlt{}15}\PY{l+s+si}{\PYZcb{}}\PY{l+s+s2}{ }\PY{l+s+si}{\PYZob{}}\PY{l+s+s1}{\PYZsq{}}\PY{l+s+s1}{收敛状态}\PY{l+s+s1}{\PYZsq{}}\PY{l+s+si}{\PYZcb{}}\PY{l+s+s2}{\PYZdq{}}\PY{p}{)}
    \PY{n+nb}{print}\PY{p}{(}\PY{l+s+s2}{\PYZdq{}}\PY{l+s+s2}{\PYZhy{}}\PY{l+s+s2}{\PYZdq{}} \PY{o}{*} \PY{l+m+mi}{60}\PY{p}{)}
    
    \PY{k}{for} \PY{n}{result} \PY{o+ow}{in} \PY{n}{results}\PY{p}{:}
        \PY{n}{status} \PY{o}{=} \PY{l+s+s2}{\PYZdq{}}\PY{l+s+s2}{收敛}\PY{l+s+s2}{\PYZdq{}} \PY{k}{if} \PY{n}{result}\PY{p}{[}\PY{l+s+s1}{\PYZsq{}}\PY{l+s+s1}{final\PYZus{}error}\PY{l+s+s1}{\PYZsq{}}\PY{p}{]} \PY{o}{\PYZlt{}} \PY{n}{tol} \PY{k}{else} \PY{l+s+s2}{\PYZdq{}}\PY{l+s+s2}{未收敛}\PY{l+s+s2}{\PYZdq{}}
        \PY{n+nb}{print}\PY{p}{(}\PY{l+s+sa}{f}\PY{l+s+s2}{\PYZdq{}}\PY{l+s+si}{\PYZob{}}\PY{n}{result}\PY{p}{[}\PY{l+s+s1}{\PYZsq{}}\PY{l+s+s1}{omega}\PY{l+s+s1}{\PYZsq{}}\PY{p}{]}\PY{l+s+si}{:}\PY{l+s+s2}{\PYZlt{}8}\PY{l+s+si}{\PYZcb{}}\PY{l+s+s2}{ }\PY{l+s+si}{\PYZob{}}\PY{n}{result}\PY{p}{[}\PY{l+s+s1}{\PYZsq{}}\PY{l+s+s1}{iterations}\PY{l+s+s1}{\PYZsq{}}\PY{p}{]}\PY{l+s+si}{:}\PY{l+s+s2}{\PYZlt{}10}\PY{l+s+si}{\PYZcb{}}\PY{l+s+s2}{ }\PY{l+s+si}{\PYZob{}}\PY{n}{result}\PY{p}{[}\PY{l+s+s1}{\PYZsq{}}\PY{l+s+s1}{final\PYZus{}error}\PY{l+s+s1}{\PYZsq{}}\PY{p}{]}\PY{l+s+si}{:}\PY{l+s+s2}{.2e}\PY{l+s+si}{\PYZcb{}}\PY{l+s+s2}{      }\PY{l+s+si}{\PYZob{}}\PY{n}{status}\PY{l+s+si}{\PYZcb{}}\PY{l+s+s2}{\PYZdq{}}\PY{p}{)}
    
    \PY{c+c1}{\PYZsh{} 找出最优松弛因子}
    \PY{n}{converged\PYZus{}results} \PY{o}{=} \PY{p}{[}\PY{n}{r} \PY{k}{for} \PY{n}{r} \PY{o+ow}{in} \PY{n}{results} \PY{k}{if} \PY{n}{r}\PY{p}{[}\PY{l+s+s1}{\PYZsq{}}\PY{l+s+s1}{final\PYZus{}error}\PY{l+s+s1}{\PYZsq{}}\PY{p}{]} \PY{o}{\PYZlt{}} \PY{n}{tol}\PY{p}{]}
    \PY{k}{if} \PY{n}{converged\PYZus{}results}\PY{p}{:}
        \PY{n}{best\PYZus{}result} \PY{o}{=} \PY{n+nb}{min}\PY{p}{(}\PY{n}{converged\PYZus{}results}\PY{p}{,} \PY{n}{key}\PY{o}{=}\PY{k}{lambda} \PY{n}{x}\PY{p}{:} \PY{n}{x}\PY{p}{[}\PY{l+s+s1}{\PYZsq{}}\PY{l+s+s1}{iterations}\PY{l+s+s1}{\PYZsq{}}\PY{p}{]}\PY{p}{)}
        \PY{n+nb}{print}\PY{p}{(}\PY{l+s+sa}{f}\PY{l+s+s2}{\PYZdq{}}\PY{l+s+se}{\PYZbs{}n}\PY{l+s+s2}{最优松弛因子: ω = }\PY{l+s+si}{\PYZob{}}\PY{n}{best\PYZus{}result}\PY{p}{[}\PY{l+s+s1}{\PYZsq{}}\PY{l+s+s1}{omega}\PY{l+s+s1}{\PYZsq{}}\PY{p}{]}\PY{l+s+si}{\PYZcb{}}\PY{l+s+s2}{\PYZdq{}}\PY{p}{)}
        \PY{n+nb}{print}\PY{p}{(}\PY{l+s+sa}{f}\PY{l+s+s2}{\PYZdq{}}\PY{l+s+s2}{最少迭代次数: }\PY{l+s+si}{\PYZob{}}\PY{n}{best\PYZus{}result}\PY{p}{[}\PY{l+s+s1}{\PYZsq{}}\PY{l+s+s1}{iterations}\PY{l+s+s1}{\PYZsq{}}\PY{p}{]}\PY{l+s+si}{\PYZcb{}}\PY{l+s+s2}{\PYZdq{}}\PY{p}{)}

\PY{n}{sor\PYZus{}method}\PY{p}{(}\PY{p}{)}
\end{Verbatim}
\end{tcolorbox}

    \begin{Verbatim}[commandchars=\\\{\}]

SOR方法 (ω = 1.03):
----------------------------------------
k=  0: x = [0.257500, 1.096306, -0.490201], ||x*-x||∞ = 2.42e-01
k=  1: x = [0.532074, 1.007893, -0.498262], ||x*-x||∞ = 3.21e-02
k=  2: x = [0.501070, 1.000486, -0.499927], ||x*-x||∞ = 1.07e-03
k=  3: x = [0.500093, 1.000028, -0.499995], ||x*-x||∞ = 9.32e-05
k=  4: x = [0.500004, 1.000002, -0.500000], ||x*-x||∞ = 4.47e-06

最终结果:
迭代次数: 5
最终解: x = [0.50000447, 1.00000161, -0.49999974]
最终误差: ||x* - x||∞ = 4.47e-06

SOR方法 (ω = 1.0):
----------------------------------------
k=  0: x = [0.250000, 1.062500, -0.484375], ||x*-x||∞ = 2.50e-01
k=  1: x = [0.515625, 1.007812, -0.498047], ||x*-x||∞ = 1.56e-02
k=  2: x = [0.501953, 1.000977, -0.499756], ||x*-x||∞ = 1.95e-03
k=  3: x = [0.500244, 1.000122, -0.499969], ||x*-x||∞ = 2.44e-04
k=  4: x = [0.500031, 1.000015, -0.499996], ||x*-x||∞ = 3.05e-05
k=  5: x = [0.500004, 1.000002, -0.500000], ||x*-x||∞ = 3.81e-06

最终结果:
迭代次数: 6
最终解: x = [0.50000381, 1.00000191, -0.49999952]
最终误差: ||x* - x||∞ = 3.81e-06

SOR方法 (ω = 1.1):
----------------------------------------
k=  0: x = [0.275000, 1.175625, -0.501703], ||x*-x||∞ = 2.25e-01
k=  1: x = [0.570797, 1.001438, -0.499434], ||x*-x||∞ = 7.08e-02
k=  2: x = [0.493316, 0.998174, -0.500559], ||x*-x||∞ = 6.68e-03
k=  3: x = [0.500166, 1.000075, -0.499924], ||x*-x||∞ = 1.66e-04
k=  4: x = [0.500004, 1.000015, -0.500004], ||x*-x||∞ = 1.46e-05
k=  5: x = [0.500004, 0.999999, -0.500000], ||x*-x||∞ = 3.63e-06

最终结果:
迭代次数: 6
最终解: x = [0.50000363, 0.99999854, -0.50000004]
最终误差: ||x* - x||∞ = 3.63e-06

============================================================
结果比较:
------------------------------------------------------------
ω值       迭代次数       最终误差            收敛状态
------------------------------------------------------------
1.03     5          4.47e-06      收敛
1.0      6          3.81e-06      收敛
1.1      6          3.63e-06      收敛

最优松弛因子: ω = 1.03
最少迭代次数: 5
    \end{Verbatim}

    \section{9}\label{section}

设有线性方程组 \(A\bold{x} = \bold{b}\),其中 \(A\)
为对称正定矩阵,迭代公式

\[
x^{(k + 1)} = x^{(k)} + \omega (\bold{b} - Ax^{(k)}), \quad k = 0, 1, 2, \cdots
\]

试证明当 \(0 < \omega < \frac{2}{\beta}\)
时上述迭代法收敛,其中,\(0 < \alpha \le \lambda(A) \le \beta\)

    \subsection{Proof}\label{proof}

整理得: \[x^{(k + 1)} = (I - \omega A)x^{(k)} + \omega \bold{b}\]

迭代矩阵为:\(G = I - \omega A\)

设 \(A\) 的特征值为 \(\lambda_i\),对应的特征向量为 \(v_i\),则:
\[Av_i = \lambda_i v_i\]

对于迭代矩阵 \(G = I - \omega A\):
\[Gv_i = (I - \omega A)v_i = v_i - \omega Av_i = v_i - \omega \lambda_i v_i = (1 - \omega \lambda_i)v_i\]

因此,\(G\) 的特征值为:\(\mu_i = 1 - \omega \lambda_i\)

迭代收敛的充分必要条件是谱半径 \(\rho(G) < 1\),即:
\[|1 - \omega \lambda_i| < 1, \quad \forall i\]

对于每个特征值 \(\lambda_i\),需要: \[|1 - \omega \lambda_i| < 1\]

这等价于: \[-1 < 1 - \omega \lambda_i < 1\]

即: \[-2 < -\omega \lambda_i < 0\] \[0 < \omega \lambda_i < 2\]

由于 \(A\) 是对称正定矩阵,所有特征值 \(\lambda_i > 0\),因此:
\[0 < \omega < \frac{2}{\lambda_i}, \quad \forall i\]

已知 \(0 < \alpha \le \lambda(A) \le \beta\),即所有特征值都满足:
\[\alpha \le \lambda_i \le \beta\]

为了保证对所有特征值都有 \(\omega < \frac{2}{\lambda_i}\),需要:
\[\omega < \min_i \frac{2}{\lambda_i} = \frac{2}{\max_i \lambda_i} = \frac{2}{\beta}\]

结合 \(\omega > 0\) 的要求,得到: \[0 < \omega < \frac{2}{\beta}\]

当 \(0 < \omega < \frac{2}{\beta}\) 时,对于任意特征值
\(\lambda_i \le \beta\):
\[\omega \lambda_i < \frac{2}{\beta} \cdot \beta = 2\]

因此: \[|1 - \omega \lambda_i| = |1 - \omega \lambda_i| < 1\]

(因为 \(0 < \omega \lambda_i < 2\),所以
\(-1 < 1 - \omega \lambda_i < 1\))

当 \(0 < \omega < \frac{2}{\beta}\) 时,迭代矩阵 \(G = I - \omega A\)
的谱半径 \(\rho(G) < 1\),因此迭代法收敛。

证毕。


    % Add a bibliography block to the postdoc
    
    
    
\end{document}
