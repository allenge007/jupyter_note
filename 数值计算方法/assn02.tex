\documentclass[11pt]{article}

    \usepackage[breakable]{tcolorbox}
    \usepackage{parskip} % Stop auto-indenting (to mimic markdown behaviour)
    \usepackage{xeCJK}
    \usepackage{gensymb}
    % Basic figure setup, for now with no caption control since it's done
    % automatically by Pandoc (which extracts ![](path) syntax from Markdown).
    \usepackage{graphicx}
    % Keep aspect ratio if custom image width or height is specified
    \setkeys{Gin}{keepaspectratio}
    % Maintain compatibility with old templates. Remove in nbconvert 6.0
    \let\Oldincludegraphics\includegraphics
    % Ensure that by default, figures have no caption (until we provide a
    % proper Figure object with a Caption API and a way to capture that
    % in the conversion process - todo).
    \usepackage{caption}
    \DeclareCaptionFormat{nocaption}{}
    \captionsetup{format=nocaption,aboveskip=0pt,belowskip=0pt}

    \usepackage{float}
    \floatplacement{figure}{H} % forces figures to be placed at the correct location
    \usepackage{xcolor} % Allow colors to be defined
    \usepackage{enumerate} % Needed for markdown enumerations to work
    \usepackage{geometry} % Used to adjust the document margins
    \usepackage{amsmath} % Equations
    \usepackage{amssymb} % Equations
    \usepackage{textcomp} % defines textquotesingle
    % Hack from http://tex.stackexchange.com/a/47451/13684:
    \AtBeginDocument{%
        \def\PYZsq{\textquotesingle}% Upright quotes in Pygmentized code
    }
    \usepackage{upquote} % Upright quotes for verbatim code
    \usepackage{eurosym} % defines \euro

    \usepackage{iftex}
    \ifPDFTeX
        \usepackage[T1]{fontenc}
        \IfFileExists{alphabeta.sty}{
              \usepackage{alphabeta}
          }{
              \usepackage[mathletters]{ucs}
              \usepackage[utf8x]{inputenc}
          }
    \else
        \usepackage{fontspec}
        \usepackage{unicode-math}
    \fi

    \usepackage{fancyvrb} % verbatim replacement that allows latex
    \usepackage{grffile} % extends the file name processing of package graphics
                         % to support a larger range
    \makeatletter % fix for old versions of grffile with XeLaTeX
    \@ifpackagelater{grffile}{2019/11/01}
    {
      % Do nothing on new versions
    }
    {
      \def\Gread@@xetex#1{%
        \IfFileExists{"\Gin@base".bb}%
        {\Gread@eps{\Gin@base.bb}}%
        {\Gread@@xetex@aux#1}%
      }
    }
    \makeatother
    \usepackage[Export]{adjustbox} % Used to constrain images to a maximum size
    \adjustboxset{max size={0.9\linewidth}{0.9\paperheight}}

    % The hyperref package gives us a pdf with properly built
    % internal navigation ('pdf bookmarks' for the table of contents,
    % internal cross-reference links, web links for URLs, etc.)
    \usepackage{hyperref}
    % The default LaTeX title has an obnoxious amount of whitespace. By default,
    % titling removes some of it. It also provides customization options.
    \usepackage{titling}
    \usepackage{longtable} % longtable support required by pandoc >1.10
    \usepackage{booktabs}  % table support for pandoc > 1.12.2
    \usepackage{array}     % table support for pandoc >= 2.11.3
    \usepackage{calc}      % table minipage width calculation for pandoc >= 2.11.1
    \usepackage[inline]{enumitem} % IRkernel/repr support (it uses the enumerate* environment)
    \usepackage[normalem]{ulem} % ulem is needed to support strikethroughs (\sout)
                                % normalem makes italics be italics, not underlines
    \usepackage{soul}      % strikethrough (\st) support for pandoc >= 3.0.0
    \usepackage{mathrsfs}
    

    
    % Colors for the hyperref package
    \definecolor{urlcolor}{rgb}{0,.145,.698}
    \definecolor{linkcolor}{rgb}{.71,0.21,0.01}
    \definecolor{citecolor}{rgb}{.12,.54,.11}

    % ANSI colors
    \definecolor{ansi-black}{HTML}{3E424D}
    \definecolor{ansi-black-intense}{HTML}{282C36}
    \definecolor{ansi-red}{HTML}{E75C58}
    \definecolor{ansi-red-intense}{HTML}{B22B31}
    \definecolor{ansi-green}{HTML}{00A250}
    \definecolor{ansi-green-intense}{HTML}{007427}
    \definecolor{ansi-yellow}{HTML}{DDB62B}
    \definecolor{ansi-yellow-intense}{HTML}{B27D12}
    \definecolor{ansi-blue}{HTML}{208FFB}
    \definecolor{ansi-blue-intense}{HTML}{0065CA}
    \definecolor{ansi-magenta}{HTML}{D160C4}
    \definecolor{ansi-magenta-intense}{HTML}{A03196}
    \definecolor{ansi-cyan}{HTML}{60C6C8}
    \definecolor{ansi-cyan-intense}{HTML}{258F8F}
    \definecolor{ansi-white}{HTML}{C5C1B4}
    \definecolor{ansi-white-intense}{HTML}{A1A6B2}
    \definecolor{ansi-default-inverse-fg}{HTML}{FFFFFF}
    \definecolor{ansi-default-inverse-bg}{HTML}{000000}

    % common color for the border for error outputs.
    \definecolor{outerrorbackground}{HTML}{FFDFDF}

    % commands and environments needed by pandoc snippets
    % extracted from the output of `pandoc -s`
    \providecommand{\tightlist}{%
      \setlength{\itemsep}{0pt}\setlength{\parskip}{0pt}}
    \DefineVerbatimEnvironment{Highlighting}{Verbatim}{commandchars=\\\{\}}
    % Add ',fontsize=\small' for more characters per line
    \newenvironment{Shaded}{}{}
    \newcommand{\KeywordTok}[1]{\textcolor[rgb]{0.00,0.44,0.13}{\textbf{{#1}}}}
    \newcommand{\DataTypeTok}[1]{\textcolor[rgb]{0.56,0.13,0.00}{{#1}}}
    \newcommand{\DecValTok}[1]{\textcolor[rgb]{0.25,0.63,0.44}{{#1}}}
    \newcommand{\BaseNTok}[1]{\textcolor[rgb]{0.25,0.63,0.44}{{#1}}}
    \newcommand{\FloatTok}[1]{\textcolor[rgb]{0.25,0.63,0.44}{{#1}}}
    \newcommand{\CharTok}[1]{\textcolor[rgb]{0.25,0.44,0.63}{{#1}}}
    \newcommand{\StringTok}[1]{\textcolor[rgb]{0.25,0.44,0.63}{{#1}}}
    \newcommand{\CommentTok}[1]{\textcolor[rgb]{0.38,0.63,0.69}{\textit{{#1}}}}
    \newcommand{\OtherTok}[1]{\textcolor[rgb]{0.00,0.44,0.13}{{#1}}}
    \newcommand{\AlertTok}[1]{\textcolor[rgb]{1.00,0.00,0.00}{\textbf{{#1}}}}
    \newcommand{\FunctionTok}[1]{\textcolor[rgb]{0.02,0.16,0.49}{{#1}}}
    \newcommand{\RegionMarkerTok}[1]{{#1}}
    \newcommand{\ErrorTok}[1]{\textcolor[rgb]{1.00,0.00,0.00}{\textbf{{#1}}}}
    \newcommand{\NormalTok}[1]{{#1}}

    % Additional commands for more recent versions of Pandoc
    \newcommand{\ConstantTok}[1]{\textcolor[rgb]{0.53,0.00,0.00}{{#1}}}
    \newcommand{\SpecialCharTok}[1]{\textcolor[rgb]{0.25,0.44,0.63}{{#1}}}
    \newcommand{\VerbatimStringTok}[1]{\textcolor[rgb]{0.25,0.44,0.63}{{#1}}}
    \newcommand{\SpecialStringTok}[1]{\textcolor[rgb]{0.73,0.40,0.53}{{#1}}}
    \newcommand{\ImportTok}[1]{{#1}}
    \newcommand{\DocumentationTok}[1]{\textcolor[rgb]{0.73,0.13,0.13}{\textit{{#1}}}}
    \newcommand{\AnnotationTok}[1]{\textcolor[rgb]{0.38,0.63,0.69}{\textbf{\textit{{#1}}}}}
    \newcommand{\CommentVarTok}[1]{\textcolor[rgb]{0.38,0.63,0.69}{\textbf{\textit{{#1}}}}}
    \newcommand{\VariableTok}[1]{\textcolor[rgb]{0.10,0.09,0.49}{{#1}}}
    \newcommand{\ControlFlowTok}[1]{\textcolor[rgb]{0.00,0.44,0.13}{\textbf{{#1}}}}
    \newcommand{\OperatorTok}[1]{\textcolor[rgb]{0.40,0.40,0.40}{{#1}}}
    \newcommand{\BuiltInTok}[1]{{#1}}
    \newcommand{\ExtensionTok}[1]{{#1}}
    \newcommand{\PreprocessorTok}[1]{\textcolor[rgb]{0.74,0.48,0.00}{{#1}}}
    \newcommand{\AttributeTok}[1]{\textcolor[rgb]{0.49,0.56,0.16}{{#1}}}
    \newcommand{\InformationTok}[1]{\textcolor[rgb]{0.38,0.63,0.69}{\textbf{\textit{{#1}}}}}
    \newcommand{\WarningTok}[1]{\textcolor[rgb]{0.38,0.63,0.69}{\textbf{\textit{{#1}}}}}
    \makeatletter
    \newsavebox\pandoc@box
    \newcommand*\pandocbounded[1]{%
      \sbox\pandoc@box{#1}%
      % scaling factors for width and height
      \Gscale@div\@tempa\textheight{\dimexpr\ht\pandoc@box+\dp\pandoc@box\relax}%
      \Gscale@div\@tempb\linewidth{\wd\pandoc@box}%
      % select the smaller of both
      \ifdim\@tempb\p@<\@tempa\p@
        \let\@tempa\@tempb
      \fi
      % scaling accordingly (\@tempa < 1)
      \ifdim\@tempa\p@<\p@
        \scalebox{\@tempa}{\usebox\pandoc@box}%
      % scaling not needed, use as it is
      \else
        \usebox{\pandoc@box}%
      \fi
    }
    \makeatother

    % Define a nice break command that doesn't care if a line doesn't already
    % exist.
    \def\br{\hspace*{\fill} \\* }
    % Math Jax compatibility definitions
    \def\gt{>}
    \def\lt{<}
    \let\Oldtex\TeX
    \let\Oldlatex\LaTeX
    \renewcommand{\TeX}{\textrm{\Oldtex}}
    \renewcommand{\LaTeX}{\textrm{\Oldlatex}}
    % Document parameters
    % Document title
    \title{assn02}
    
    
    
    
    
    
    
% Pygments definitions
\makeatletter
\def\PY@reset{\let\PY@it=\relax \let\PY@bf=\relax%
    \let\PY@ul=\relax \let\PY@tc=\relax%
    \let\PY@bc=\relax \let\PY@ff=\relax}
\def\PY@tok#1{\csname PY@tok@#1\endcsname}
\def\PY@toks#1+{\ifx\relax#1\empty\else%
    \PY@tok{#1}\expandafter\PY@toks\fi}
\def\PY@do#1{\PY@bc{\PY@tc{\PY@ul{%
    \PY@it{\PY@bf{\PY@ff{#1}}}}}}}
\def\PY#1#2{\PY@reset\PY@toks#1+\relax+\PY@do{#2}}

\@namedef{PY@tok@w}{\def\PY@tc##1{\textcolor[rgb]{0.73,0.73,0.73}{##1}}}
\@namedef{PY@tok@c}{\let\PY@it=\textit\def\PY@tc##1{\textcolor[rgb]{0.24,0.48,0.48}{##1}}}
\@namedef{PY@tok@cp}{\def\PY@tc##1{\textcolor[rgb]{0.61,0.40,0.00}{##1}}}
\@namedef{PY@tok@k}{\let\PY@bf=\textbf\def\PY@tc##1{\textcolor[rgb]{0.00,0.50,0.00}{##1}}}
\@namedef{PY@tok@kp}{\def\PY@tc##1{\textcolor[rgb]{0.00,0.50,0.00}{##1}}}
\@namedef{PY@tok@kt}{\def\PY@tc##1{\textcolor[rgb]{0.69,0.00,0.25}{##1}}}
\@namedef{PY@tok@o}{\def\PY@tc##1{\textcolor[rgb]{0.40,0.40,0.40}{##1}}}
\@namedef{PY@tok@ow}{\let\PY@bf=\textbf\def\PY@tc##1{\textcolor[rgb]{0.67,0.13,1.00}{##1}}}
\@namedef{PY@tok@nb}{\def\PY@tc##1{\textcolor[rgb]{0.00,0.50,0.00}{##1}}}
\@namedef{PY@tok@nf}{\def\PY@tc##1{\textcolor[rgb]{0.00,0.00,1.00}{##1}}}
\@namedef{PY@tok@nc}{\let\PY@bf=\textbf\def\PY@tc##1{\textcolor[rgb]{0.00,0.00,1.00}{##1}}}
\@namedef{PY@tok@nn}{\let\PY@bf=\textbf\def\PY@tc##1{\textcolor[rgb]{0.00,0.00,1.00}{##1}}}
\@namedef{PY@tok@ne}{\let\PY@bf=\textbf\def\PY@tc##1{\textcolor[rgb]{0.80,0.25,0.22}{##1}}}
\@namedef{PY@tok@nv}{\def\PY@tc##1{\textcolor[rgb]{0.10,0.09,0.49}{##1}}}
\@namedef{PY@tok@no}{\def\PY@tc##1{\textcolor[rgb]{0.53,0.00,0.00}{##1}}}
\@namedef{PY@tok@nl}{\def\PY@tc##1{\textcolor[rgb]{0.46,0.46,0.00}{##1}}}
\@namedef{PY@tok@ni}{\let\PY@bf=\textbf\def\PY@tc##1{\textcolor[rgb]{0.44,0.44,0.44}{##1}}}
\@namedef{PY@tok@na}{\def\PY@tc##1{\textcolor[rgb]{0.41,0.47,0.13}{##1}}}
\@namedef{PY@tok@nt}{\let\PY@bf=\textbf\def\PY@tc##1{\textcolor[rgb]{0.00,0.50,0.00}{##1}}}
\@namedef{PY@tok@nd}{\def\PY@tc##1{\textcolor[rgb]{0.67,0.13,1.00}{##1}}}
\@namedef{PY@tok@s}{\def\PY@tc##1{\textcolor[rgb]{0.73,0.13,0.13}{##1}}}
\@namedef{PY@tok@sd}{\let\PY@it=\textit\def\PY@tc##1{\textcolor[rgb]{0.73,0.13,0.13}{##1}}}
\@namedef{PY@tok@si}{\let\PY@bf=\textbf\def\PY@tc##1{\textcolor[rgb]{0.64,0.35,0.47}{##1}}}
\@namedef{PY@tok@se}{\let\PY@bf=\textbf\def\PY@tc##1{\textcolor[rgb]{0.67,0.36,0.12}{##1}}}
\@namedef{PY@tok@sr}{\def\PY@tc##1{\textcolor[rgb]{0.64,0.35,0.47}{##1}}}
\@namedef{PY@tok@ss}{\def\PY@tc##1{\textcolor[rgb]{0.10,0.09,0.49}{##1}}}
\@namedef{PY@tok@sx}{\def\PY@tc##1{\textcolor[rgb]{0.00,0.50,0.00}{##1}}}
\@namedef{PY@tok@m}{\def\PY@tc##1{\textcolor[rgb]{0.40,0.40,0.40}{##1}}}
\@namedef{PY@tok@gh}{\let\PY@bf=\textbf\def\PY@tc##1{\textcolor[rgb]{0.00,0.00,0.50}{##1}}}
\@namedef{PY@tok@gu}{\let\PY@bf=\textbf\def\PY@tc##1{\textcolor[rgb]{0.50,0.00,0.50}{##1}}}
\@namedef{PY@tok@gd}{\def\PY@tc##1{\textcolor[rgb]{0.63,0.00,0.00}{##1}}}
\@namedef{PY@tok@gi}{\def\PY@tc##1{\textcolor[rgb]{0.00,0.52,0.00}{##1}}}
\@namedef{PY@tok@gr}{\def\PY@tc##1{\textcolor[rgb]{0.89,0.00,0.00}{##1}}}
\@namedef{PY@tok@ge}{\let\PY@it=\textit}
\@namedef{PY@tok@gs}{\let\PY@bf=\textbf}
\@namedef{PY@tok@gp}{\let\PY@bf=\textbf\def\PY@tc##1{\textcolor[rgb]{0.00,0.00,0.50}{##1}}}
\@namedef{PY@tok@go}{\def\PY@tc##1{\textcolor[rgb]{0.44,0.44,0.44}{##1}}}
\@namedef{PY@tok@gt}{\def\PY@tc##1{\textcolor[rgb]{0.00,0.27,0.87}{##1}}}
\@namedef{PY@tok@err}{\def\PY@bc##1{{\setlength{\fboxsep}{\string -\fboxrule}\fcolorbox[rgb]{1.00,0.00,0.00}{1,1,1}{\strut ##1}}}}
\@namedef{PY@tok@kc}{\let\PY@bf=\textbf\def\PY@tc##1{\textcolor[rgb]{0.00,0.50,0.00}{##1}}}
\@namedef{PY@tok@kd}{\let\PY@bf=\textbf\def\PY@tc##1{\textcolor[rgb]{0.00,0.50,0.00}{##1}}}
\@namedef{PY@tok@kn}{\let\PY@bf=\textbf\def\PY@tc##1{\textcolor[rgb]{0.00,0.50,0.00}{##1}}}
\@namedef{PY@tok@kr}{\let\PY@bf=\textbf\def\PY@tc##1{\textcolor[rgb]{0.00,0.50,0.00}{##1}}}
\@namedef{PY@tok@bp}{\def\PY@tc##1{\textcolor[rgb]{0.00,0.50,0.00}{##1}}}
\@namedef{PY@tok@fm}{\def\PY@tc##1{\textcolor[rgb]{0.00,0.00,1.00}{##1}}}
\@namedef{PY@tok@vc}{\def\PY@tc##1{\textcolor[rgb]{0.10,0.09,0.49}{##1}}}
\@namedef{PY@tok@vg}{\def\PY@tc##1{\textcolor[rgb]{0.10,0.09,0.49}{##1}}}
\@namedef{PY@tok@vi}{\def\PY@tc##1{\textcolor[rgb]{0.10,0.09,0.49}{##1}}}
\@namedef{PY@tok@vm}{\def\PY@tc##1{\textcolor[rgb]{0.10,0.09,0.49}{##1}}}
\@namedef{PY@tok@sa}{\def\PY@tc##1{\textcolor[rgb]{0.73,0.13,0.13}{##1}}}
\@namedef{PY@tok@sb}{\def\PY@tc##1{\textcolor[rgb]{0.73,0.13,0.13}{##1}}}
\@namedef{PY@tok@sc}{\def\PY@tc##1{\textcolor[rgb]{0.73,0.13,0.13}{##1}}}
\@namedef{PY@tok@dl}{\def\PY@tc##1{\textcolor[rgb]{0.73,0.13,0.13}{##1}}}
\@namedef{PY@tok@s2}{\def\PY@tc##1{\textcolor[rgb]{0.73,0.13,0.13}{##1}}}
\@namedef{PY@tok@sh}{\def\PY@tc##1{\textcolor[rgb]{0.73,0.13,0.13}{##1}}}
\@namedef{PY@tok@s1}{\def\PY@tc##1{\textcolor[rgb]{0.73,0.13,0.13}{##1}}}
\@namedef{PY@tok@mb}{\def\PY@tc##1{\textcolor[rgb]{0.40,0.40,0.40}{##1}}}
\@namedef{PY@tok@mf}{\def\PY@tc##1{\textcolor[rgb]{0.40,0.40,0.40}{##1}}}
\@namedef{PY@tok@mh}{\def\PY@tc##1{\textcolor[rgb]{0.40,0.40,0.40}{##1}}}
\@namedef{PY@tok@mi}{\def\PY@tc##1{\textcolor[rgb]{0.40,0.40,0.40}{##1}}}
\@namedef{PY@tok@il}{\def\PY@tc##1{\textcolor[rgb]{0.40,0.40,0.40}{##1}}}
\@namedef{PY@tok@mo}{\def\PY@tc##1{\textcolor[rgb]{0.40,0.40,0.40}{##1}}}
\@namedef{PY@tok@ch}{\let\PY@it=\textit\def\PY@tc##1{\textcolor[rgb]{0.24,0.48,0.48}{##1}}}
\@namedef{PY@tok@cm}{\let\PY@it=\textit\def\PY@tc##1{\textcolor[rgb]{0.24,0.48,0.48}{##1}}}
\@namedef{PY@tok@cpf}{\let\PY@it=\textit\def\PY@tc##1{\textcolor[rgb]{0.24,0.48,0.48}{##1}}}
\@namedef{PY@tok@c1}{\let\PY@it=\textit\def\PY@tc##1{\textcolor[rgb]{0.24,0.48,0.48}{##1}}}
\@namedef{PY@tok@cs}{\let\PY@it=\textit\def\PY@tc##1{\textcolor[rgb]{0.24,0.48,0.48}{##1}}}

\def\PYZbs{\char`\\}
\def\PYZus{\char`\_}
\def\PYZob{\char`\{}
\def\PYZcb{\char`\}}
\def\PYZca{\char`\^}
\def\PYZam{\char`\&}
\def\PYZlt{\char`\<}
\def\PYZgt{\char`\>}
\def\PYZsh{\char`\#}
\def\PYZpc{\char`\%}
\def\PYZdl{\char`\$}
\def\PYZhy{\char`\-}
\def\PYZsq{\char`\'}
\def\PYZdq{\char`\"}
\def\PYZti{\char`\~}
% for compatibility with earlier versions
\def\PYZat{@}
\def\PYZlb{[}
\def\PYZrb{]}
\makeatother


    % For linebreaks inside Verbatim environment from package fancyvrb.
    \makeatletter
        \newbox\Wrappedcontinuationbox
        \newbox\Wrappedvisiblespacebox
        \newcommand*\Wrappedvisiblespace {\textcolor{red}{\textvisiblespace}}
        \newcommand*\Wrappedcontinuationsymbol {\textcolor{red}{\llap{\tiny$\m@th\hookrightarrow$}}}
        \newcommand*\Wrappedcontinuationindent {3ex }
        \newcommand*\Wrappedafterbreak {\kern\Wrappedcontinuationindent\copy\Wrappedcontinuationbox}
        % Take advantage of the already applied Pygments mark-up to insert
        % potential linebreaks for TeX processing.
        %        {, <, #, %, $, ' and ": go to next line.
        %        _, }, ^, &, >, - and ~: stay at end of broken line.
        % Use of \textquotesingle for straight quote.
        \newcommand*\Wrappedbreaksatspecials {%
            \def\PYGZus{\discretionary{\char`\_}{\Wrappedafterbreak}{\char`\_}}%
            \def\PYGZob{\discretionary{}{\Wrappedafterbreak\char`\{}{\char`\{}}%
            \def\PYGZcb{\discretionary{\char`\}}{\Wrappedafterbreak}{\char`\}}}%
            \def\PYGZca{\discretionary{\char`\^}{\Wrappedafterbreak}{\char`\^}}%
            \def\PYGZam{\discretionary{\char`\&}{\Wrappedafterbreak}{\char`\&}}%
            \def\PYGZlt{\discretionary{}{\Wrappedafterbreak\char`\<}{\char`\<}}%
            \def\PYGZgt{\discretionary{\char`\>}{\Wrappedafterbreak}{\char`\>}}%
            \def\PYGZsh{\discretionary{}{\Wrappedafterbreak\char`\#}{\char`\#}}%
            \def\PYGZpc{\discretionary{}{\Wrappedafterbreak\char`\%}{\char`\%}}%
            \def\PYGZdl{\discretionary{}{\Wrappedafterbreak\char`\$}{\char`\$}}%
            \def\PYGZhy{\discretionary{\char`\-}{\Wrappedafterbreak}{\char`\-}}%
            \def\PYGZsq{\discretionary{}{\Wrappedafterbreak\textquotesingle}{\textquotesingle}}%
            \def\PYGZdq{\discretionary{}{\Wrappedafterbreak\char`\"}{\char`\"}}%
            \def\PYGZti{\discretionary{\char`\~}{\Wrappedafterbreak}{\char`\~}}%
        }
        % Some characters . , ; ? ! / are not pygmentized.
        % This macro makes them "active" and they will insert potential linebreaks
        \newcommand*\Wrappedbreaksatpunct {%
            \lccode`\~`\.\lowercase{\def~}{\discretionary{\hbox{\char`\.}}{\Wrappedafterbreak}{\hbox{\char`\.}}}%
            \lccode`\~`\,\lowercase{\def~}{\discretionary{\hbox{\char`\,}}{\Wrappedafterbreak}{\hbox{\char`\,}}}%
            \lccode`\~`\;\lowercase{\def~}{\discretionary{\hbox{\char`\;}}{\Wrappedafterbreak}{\hbox{\char`\;}}}%
            \lccode`\~`\:\lowercase{\def~}{\discretionary{\hbox{\char`\:}}{\Wrappedafterbreak}{\hbox{\char`\:}}}%
            \lccode`\~`\?\lowercase{\def~}{\discretionary{\hbox{\char`\?}}{\Wrappedafterbreak}{\hbox{\char`\?}}}%
            \lccode`\~`\!\lowercase{\def~}{\discretionary{\hbox{\char`\!}}{\Wrappedafterbreak}{\hbox{\char`\!}}}%
            \lccode`\~`\/\lowercase{\def~}{\discretionary{\hbox{\char`\/}}{\Wrappedafterbreak}{\hbox{\char`\/}}}%
            \catcode`\.\active
            \catcode`\,\active
            \catcode`\;\active
            \catcode`\:\active
            \catcode`\?\active
            \catcode`\!\active
            \catcode`\/\active
            \lccode`\~`\~
        }
    \makeatother

    \let\OriginalVerbatim=\Verbatim
    \makeatletter
    \renewcommand{\Verbatim}[1][1]{%
        %\parskip\z@skip
        \sbox\Wrappedcontinuationbox {\Wrappedcontinuationsymbol}%
        \sbox\Wrappedvisiblespacebox {\FV@SetupFont\Wrappedvisiblespace}%
        \def\FancyVerbFormatLine ##1{\hsize\linewidth
            \vtop{\raggedright\hyphenpenalty\z@\exhyphenpenalty\z@
                \doublehyphendemerits\z@\finalhyphendemerits\z@
                \strut ##1\strut}%
        }%
        % If the linebreak is at a space, the latter will be displayed as visible
        % space at end of first line, and a continuation symbol starts next line.
        % Stretch/shrink are however usually zero for typewriter font.
        \def\FV@Space {%
            \nobreak\hskip\z@ plus\fontdimen3\font minus\fontdimen4\font
            \discretionary{\copy\Wrappedvisiblespacebox}{\Wrappedafterbreak}
            {\kern\fontdimen2\font}%
        }%

        % Allow breaks at special characters using \PYG... macros.
        \Wrappedbreaksatspecials
        % Breaks at punctuation characters . , ; ? ! and / need catcode=\active
        \OriginalVerbatim[#1,codes*=\Wrappedbreaksatpunct]%
    }
    \makeatother

    % Exact colors from NB
    \definecolor{incolor}{HTML}{303F9F}
    \definecolor{outcolor}{HTML}{D84315}
    \definecolor{cellborder}{HTML}{CFCFCF}
    \definecolor{cellbackground}{HTML}{F7F7F7}

    % prompt
    \makeatletter
    \newcommand{\boxspacing}{\kern\kvtcb@left@rule\kern\kvtcb@boxsep}
    \makeatother
    \newcommand{\prompt}[4]{
        {\ttfamily\llap{{\color{#2}[#3]:\hspace{3pt}#4}}\vspace{-\baselineskip}}
    }
    

    
    % Prevent overflowing lines due to hard-to-break entities
    \sloppy
    % Setup hyperref package
    \hypersetup{
      breaklinks=true,  % so long urls are correctly broken across lines
      colorlinks=true,
      urlcolor=urlcolor,
      linkcolor=linkcolor,
      citecolor=citecolor,
      }
    % Slightly bigger margins than the latex defaults
    
    \geometry{verbose,tmargin=1in,bmargin=1in,lmargin=1in,rmargin=1in}
    
    

\begin{document}
    
    \maketitle
    
    

    
    \section{1(3)}\label{section}

\[
f(1)=0,\quad f(-1)=-3,\quad f(2)=4.
\]

\[
x_0=1,\quad x_1=-1,\quad x_2=2
\]

使用牛顿插值法。

\subsection{Answer}\label{answer}

牛顿插值多项式可表示为

\[
P(x) = f(x_0) + f[x_0,x_1](x-x_0) + f[x_0,x_1,x_2](x-x_0)(x-x_1).
\]

\[
\begin{aligned}
f[x_0,x_1] &= \frac{-3-0}{-1-1} = \frac{3}{2} \\
f[x_1,x_2] &= \frac{4-(-3)}{2-(-1)} = \frac{7}{3} \\
f[x_0,x_1,x_2] &= \frac{\frac{7}{3}-\frac{3}{2}}{2-1} = \frac{5}{6}
\end{aligned}
\]

因此有

\[
\boxed{
P(x) = \frac{3}{2}(x-1) + \frac{5}{6}(x-1)(x+1)
}
\]

    \section{3}\label{section}

给出 \(\cos x, 0\le x\le 90\degree\) 的函数表,步长
\(h = 1' = (\frac{1}{60})\degree\)。若函数表具有 5
位有效数字,研究用线性插值求 \(\cos x\) 近似值时的总误差界。

\subsection{Answer}\label{answer}

\begin{enumerate}
\def\labelenumi{\arabic{enumi}.}
\item
  \textbf{舍入误差}

  由于表格中的数字保留 5 位有效数字,舍入误差界限为 \[
  \delta \le 0.5 \times 10^{-5},
  \]
\item
  \textbf{线性插值误差}

  当在相邻的表格点 \(z_i\) 与 \(z_{i+1}\) 之间(间距
  \(h = 1'\))使用线性插值时,对于一个 \(C^2\) 函数
  \(f(z)=\cos z\),插值误差满足 \[
  E_{\text{interp}} \le \frac{M}{8}(z_{i+1}-z_i)^2,
  \] 其中 \[
  M = \max_{z\in[z_i,\,z_{i+1}]} |f''(z)|.
  \]

  \(f(z) = \cos z\) 时,其二阶导数为 \(f'\,'(z) = -\cos z,\) 因此有
  \(|f''(z)| \le 1\)。

  \(h = 1' = \pi/10800 \, \rm rad\),则 \[
  E_{\text{interp}} \le \frac{1}{8}\left(\frac{\pi}{10800}\right)^2.
  \]

  计算得 \[
  \left(\frac{\pi}{10800}\right)^2 = \frac{\pi^2}{10800^2} \approx \frac{9.87}{116640000} \approx 8.46\times 10^{-8},
  \] 所以 \[
  E_{\text{interp}} \le \frac{8.46\times 10^{-8}}{8} \approx 1.06\times 10^{-8}.
  \]
\item
  \textbf{总误差界}

  \[
  E_{\text{total}} \lesssim \delta + E_{\text{interp}} \approx 5\times 10^{-6} + 1.06\times 10^{-8},
  \] 因此,总误差界主要由舍入误差主导。

  最终我们有 \[
  \boxed{E_{\text{total}} \lesssim 5\times 10^{-6}.}
  \]
\end{enumerate}

    \section{4(2)}\label{section}

设 \(x_0,x_1,\dots,x_n\) 是互异节点,求证对于
\(k = 1, 2, \cdots, n\),有 \[
\sum_{j=0}^{n}\bigl(x_j-x\bigr)^k\ell_j(x) \equiv 0.
\]

\subsection{Proof}\label{proof}

\[
\begin{aligned}
\sum_{j = 0}^n(x_j - x)^k\ell_j(x)&\equiv
\sum_{j = 0}^n\ell_j(x)\sum_{m = 0}^k(-1)^mx^mx_j^{k - m}
\end{aligned}
\]

由公式 2.17 知, \[
\sum_{i = 0}^nx_i^k\ell_i(x)\equiv x^k
\]

因此,

\[
\begin{aligned}
\sum_{j = 0}^n(x_j - x)^k\ell_j(x)&\equiv
\sum_{m = 0}^k\sum_{j = 0}^n(-1)^mx^mx_j^{k - m}\ell_j(x) \\ &\equiv
\sum_{m = 0}^k(-1)^mx^m\sum_{j = 0}^nx_j^{k - m}\ell_j(x)\\ &\equiv
\sum_{m = 0}^k (-1)^mx^mx^{k-m} \\ &\equiv 
(x - x)^k \\ &\equiv 
0
\end{aligned}
\]

    \section{6}\label{section}

给出区间 \[
-4 \le x \le 4
\] 上的等距函数表,用二次插值来近似 \(f(x)=e^x\),并要求截断误差不超过
\(10^{-6}.\) 问函数表的步长 \(h\) 应为多少?

\subsection{Answer}\label{answer}

对于在三个节点 \(x_0,\, x_1,\, x_2\)进行二次插值,在区间 \([x_0,x_2]\)
内任一点 \(x\) 的插值误差为 \[
R(x)=\frac{f^{(3)}(\xi)}{3!}(x-x_0)(x-x_1)(x-x_2),
\] 其中 \(\xi\) 取自 \([x_0,x_2]\) 内的某一点,节点是等距的,且步长为
\(h\)。

设三个连续节点为 \[
x_0,\quad x_1=x_0+h,\quad x_2=x_0+2h.
\] 设 \[
t=x-x_0,\quad \text{其中 } t\in[0,2h].
\] 则误差可以写成 \[
R(t)=\frac{f^{(3)}(\xi)}{6}\, t\,(t-h)\,(t-2h).
\]

当 \(t\in[0,2h]\) \[
|P(t)|=|t\,(t-h)\,(t-2h)|
\] 的最大值。经过求导可以发现,绝对值最大处出现在 \[
t = h\Bigl(1\pm \frac{1}{\sqrt{3}}\Bigr)
\] 处,因此有 \[
\max_{t\in[0,2h]}|t\,(t-h)\,(t-2h)| = \frac{2h^3}{3\sqrt{3}}.
\]

于是,任意子区间上的最大插值误差限为 \[
|R(x)| \le \frac{M_3}{6} \cdot \frac{2h^3}{3\sqrt{3}} = \frac{M_3\, h^3}{9\sqrt{3}},
\] 其中 \[
M_3 = \max_{x\in[-4,4]}|f^{(3)}(x)|.
\]

由于 \[
f(x)=e^x,\quad f^{(3)}(x)=e^x,
\] 故 \[
|f^{(3)}(x)|= e^x.
\] 在区间 \([-4,4]\) 上,最大值出现在 \(x=4\) 处,因此 \[
M_3 = e^4.
\]

故截断误差满足 \[
\frac{e^4\, h^3}{9\sqrt{3}} \le 10^{-6}.
\] 解得 \[
h^3 \le \frac{9\sqrt{3}\,10^{-6}}{e^4}.
\] 取立方根可得 \[
\boxed{
h \le \left(\frac{9\sqrt{3}\,10^{-6}}{e^4}\right)^{\frac{1}{3}}.
}
\]

    \section{7}\label{section}

证明 n 阶均差有下列性质

\subsection{性质(1)}\label{ux6027ux8d281}

\textbf{描述:}\\
若 \[
F(x)= c\, f(x),
\] 则 \[
F[x_0,x_1,\dots,x_n] = c\,f[x_0,x_1,\dots,x_n].
\]

\textbf{证明:}\\
使用数学归纳法。

\begin{itemize}
\item
  \textbf{归纳基 (\(n=0\))}:\\
  \[
  F[x_0] = F(x_0) = c\, f(x_0) = c\, f[x_0].
  \] 因此性质(1)对 \(n=0\) 成立.
\item
  \textbf{归纳步:}\\
  假设性质(1)对任意 \(0\le k \le n-1\) 成立,即证明 \[
  F[x_0,x_1,\dots,x_n] = \frac{F[x_1,x_2,\dots,x_n] - F[x_0,x_1,\dots,x_{n-1}]}{x_n - x_0}.
  \] 根据归纳假设,我们有 \[
  F[x_1,x_2,\dots,x_n] = c \, f[x_1,x_2,\dots,x_n] \quad\text{和}\quad F[x_0,x_1,\dots,x_{n-1}] = c\, f[x_0,x_1,\dots,x_{n-1}].
  \] 因此, \[
  F[x_0,x_1,\dots,x_n] = \frac{ c\, f[x_1,x_2,\dots,x_n] - c\, f[x_0,x_1,\dots,x_{n-1}]}{x_n-x_0} = c\, \frac{f[x_1,x_2,\dots,x_n]- f[x_0,x_1,\dots,x_{n-1}]}{x_n-x_0}.
  \] 根据 \(f[x_0,x_1,\dots,x_n]\) 的定义,有 \[
  F[x_0,x_1,\dots,x_n] = c\,f[x_0,x_1,\dots,x_n].
  \]
\end{itemize}

性质(1)得证。

\begin{center}\rule{0.5\linewidth}{0.5pt}\end{center}

\subsection{性质 (2)}\label{ux6027ux8d28-2}

\textbf{描述:}\\
若 \[
F(x) = f(x)+g(x),
\] 则 \[
F[x_0,x_1,\dots,x_n]= f[x_0,x_1,\dots,x_n]+g[x_0,x_1,\dots,x_n].
\]

\textbf{证明:}\\
再次运用数学归纳法。

\begin{itemize}
\item
  \textbf{归纳基 (\(n=0\))}:\\
  \[
  F[x_0] = F(x_0)= f(x_0)+g(x_0)= f[x_0]+g[x_0].
  \] 因此性质(2)对 \(n=0\) 成立.
\item
  \textbf{归纳步:}\\
  假设性质(2)对 \(0\le k \le n-1\) 成立,根据 \[
  F[x_0,x_1,\dots,x_n] = \frac{F[x_1,x_2,\dots,x_n]-F[x_0,x_1,\dots,x_{n-1}]}{x_n - x_0}.
  \] 使用归纳假设 \[
  F[x_1,x_2,\dots,x_n] = f[x_1,x_2,\dots,x_n] + g[x_1,x_2,\dots,x_n],
  \] 以及 \[
  F[x_0,x_1,\dots,x_{n-1}] = f[x_0,x_1,\dots,x_{n-1}] + g[x_0,x_1,\dots,x_{n-1}].
  \] 带入得到, \[
  \begin{aligned}
  F[x_0,x_1,\dots,x_n] &= \frac{\Bigl(f[x_1,x_2,\dots,x_n] + g[x_1,x_2,\dots,x_n]\Bigr) - \Bigl(f[x_0,x_1,\dots,x_{n-1}] + g[x_0,x_1,\dots,x_{n-1}]\Bigr)}{x_n-x_0} \\
  &= \frac{f[x_1,x_2,\dots,x_n] - f[x_0,x_1,\dots,x_{n-1}]}{x_n-x_0} + \frac{g[x_1,x_2,\dots,x_n] - g[x_0,x_1,\dots,x_{n-1}]}{x_n-x_0} \\
  &= f[x_0,x_1,\dots,x_n] + g[x_0,x_1,\dots,x_n].
  \end{aligned}
  \]
\end{itemize}

性质(2)得证。

    \section{8}\label{section}

已知 \[
f(x)=x^7+x^4+3x+1,
\]

求

\[
f[2^0,2^1,\dots,2^7] \quad \text{and} \quad f[2^0,2^1,\dots,2^8].
\]

\subsection{Answer}\label{answer}

根据 n 阶分差定义我们有

\[
f[x_0, x_1, \cdots, x_n]=\frac{f^{(n)}(\xi)}{n!}
\]

其中 \(\xi \in [x_0, x_n]\)

对于 \(f(x)=x^7+x^4+3x+1\)

\begin{enumerate}
\def\labelenumi{\arabic{enumi}.}
\item
  \textbf{\(2^0,2^1,\dots,2^7\):} 7 阶导为 \(7!\) 故 \[
  f[2^0,2^1,\dots,2^7] = 1.
    \]
\item
  \textbf{\(2^0,2^1,\dots,2^8\):} 8 阶导为 \(0\) 故 \[
  f[2^0,2^1,\dots,2^8] = 0.
    \]
\end{enumerate}

因此

\[
\boxed{f[2^0,2^1,\dots,2^7] = 1 \quad\text{and}\quad f[2^0,2^1,\dots,2^8] = 0.}
\]

    \section{9}\label{section}

证明 \[
\Delta\bigl(f_k g_k\bigr) = f_k\,\Delta g_k + g_{k+1}\,\Delta f_k.
\]

\subsection{Proof}\label{proof}

根据差分的定义我们有 \[
   \Delta\bigl(f_k g_k\bigr) = f_{k+1}g_{k+1} - f_k g_k.
   \]

故 \[
   f_{k+1}g_{k+1} - f_kg_k = \Bigl[f_{k+1}g_{k+1} - f_k g_{k+1}\Bigr] + \Bigl[f_k g_{k+1} - f_k g_k\Bigr].
   \]

\[
   f_{k+1}g_{k+1} - f_kg_{k+1} = g_{k+1}\bigl(f_{k+1} - f_k\bigr) = g_{k+1}\,\Delta f_k,
   \]

\[
   f_k g_{k+1} - f_k g_k = f_k\bigl(g_{k+1} - g_k\bigr) = f_k\,\Delta g_k.
   \]

\[
   \Delta\bigl(f_k g_k\bigr) = g_{k+1}\,\Delta f_k + f_k\,\Delta g_k,
   \] 因此 \[
   \boxed{\Delta\bigl(f_k g_k\bigr) = f_k\,\Delta g_k + g_{k+1}\,\Delta f_k.}
   \]

原式得证。

    \section{10}\label{section}

证明 \[
\sum_{k = 0}^{n - 1}f_k\Delta g_k = f_n g_n - f_0 g_0 - \sum_{k = 0}^{n - 1}g_{k + 1}\Delta f_k,
\]

\subsection{Proof}\label{proof}

裂项得到 \[
   f_n g_n - f_0 g_0 = \sum_{k=0}^{n-1}\Bigl( f_{k+1}g_{k+1} - f_k g_k \Bigr).
   \]

又因为 \[
   \Delta(f_k g_k) = f_k\,\Delta g_k + g_{k+1}\,\Delta f_k.
   \] 所以 \[
   f_{k+1}g_{k+1} - f_k g_k = f_k\Delta g_k + g_{k+1}\Delta f_k.
   \]

代入得到 \[
   f_n g_n - f_0 g_0 = \sum_{k=0}^{n-1}\Bigl( f_k\Delta g_k + g_{k+1}\Delta f_k\Bigr)
   = \sum_{k=0}^{n-1}f_k\Delta g_k + \sum_{k=0}^{n-1} g_{k+1}\Delta f_k.
   \]

因此 \[
   \boxed{
   \sum_{k=0}^{n-1}f_k\Delta g_k = f_n g_n - f_0 g_0 - \sum_{k=0}^{n-1} g_{k+1}\Delta f_k.}
   \]

原式得证。

    \section{11}\label{section}

证明

\[
\sum_{j = 0}^{n - 1} \Delta^2 y_j = \Delta y_n - \Delta y_0
\]

\subsection{Proof}\label{proof}

裂项得到

\[
\Delta y_n - \Delta y_0 = \sum_{j = 0}^{n - 1}\Delta y_{j + 1} - \Delta y_j = \sum_{j = 0}^{n - 1}\Delta^2y_j
\]

原式得证。

    \section{12}\label{section}

给定多项式 \[
f(x)=\sum_{i=0}^n a_i x^i = a_n\,(x-x_1)(x-x_2)\cdots (x-x_n)
\] 有 n 个互不相同的根。证明

\[
\sum \frac{x_j^k}{f'(x_j)} = 
\begin{cases}
0, & 0≤k≤n-2 \\
a_n^{-1}, & k=n-1
\end{cases}
\]

\subsection{Proof}\label{proof}

设

\[
\omega_n (x) = (x - x_1)(x - x_2)\cdots(x - x_n)
\]

原式可化为

\[
\sum_{j = 1}^n\frac{x_j^k}{f'(x_j)} = \sum_{j = 1}^n\frac{x_j^k}{a_n\omega_n'(x_j)}
\]

\[
\omega_n'(x_j)=\prod_{k\neq j}(x_j - x_k)
\]

令 \(g(x) = x^k\),则
\(g[\dots]=\sum_{j=1}^n\frac{x_j^k}{\omega'_n(x_j)}\)

又因为

\[
g[\cdots]=\frac{g^{(n-1)}(\xi)}{(n-1)!}
\]

故

\[
原式=\frac{1}{a_n}g[\dots]=\begin{cases}
0, &k \le n-2 \\
a_n^{-1}, &\rm otherwise
\end{cases}
\]

原式得证。

    \section{13}\label{section}

构造一个 3 次的多项式 \(P(x)\),使得它满足 \[
\begin{aligned}
P(x_0) &= f(x_0), \\
P'(x_0) &= f'(x_0), \\
P''(x_0) &= f''(x_0), \\
P(x_1) &= f(x_1).
\end{aligned}
\]

\subsection{Answer}\label{answer}

令 \[
P(x) = a_0 + a_1 (x-x_0) + a_2 (x-x_0)^2 + a_3 (x-x_0)^3.
\] 那么有 \[
\begin{aligned}
P(x_0) &= a_0,\\[1mm]
P'(x) &= a_1 + 2a_2 (x-x_0) + 3a_3 (x-x_0)^2,\quad \text{所以} \quad P'(x_0)=a_1,\\[1mm]
P''(x) &= 2a_2 + 6a_3 (x-x_0),\quad \text{所以} \quad P''(x_0)=2a_2.
\end{aligned}
\]

由给定条件可得 \[
\begin{aligned}
a_0 &= f(x_0),\\[1mm]
a_1 &= f'(x_0),\\[1mm]
2a_2 &= f''(x_0) \quad \Longrightarrow \quad a_2 = \frac{f''(x_0)}{2}.
\end{aligned}
\]

另外, \[
P(x_1)= f(x_1).
\] 由于 \[
P(x_1) = a_0 + a_1 (x_1-x_0) + a_2 (x_1-x_0)^2 + a_3 (x_1-x_0)^3,
\] 将 \(a_0, a_1, a_2\) 的表达式代入,得到 \[
f(x_0) + f'(x_0)(x_1-x_0) + \frac{f''(x_0)}{2}(x_1-x_0)^2 + a_3 (x_1-x_0)^3 = f(x_1).
\] 求解 \(a_3\) 得 \[
a_3 = \frac{f(x_1)- f(x_0) - f'(x_0)(x_1-x_0) - \frac{f''(x_0)}{2}(x_1-x_0)^2}{(x_1-x_0)^3}.
\]

因此,所求的多项式为 \[
\boxed{
f(x_0) + f'(x_0)(x-x_0) + \frac{f''(x_0)}{2}(x-x_0)^2 + \frac{f(x_1)- f(x_0) - f'(x_0)(x_1-x_0) - \frac{f''(x_0)}{2}(x_1-x_0)^2}{(x_1-x_0)^3}(x-x_0)^3.
}
\]


    % Add a bibliography block to the postdoc
    
    
    
\end{document}
