\documentclass[11pt]{article}

    \usepackage[breakable]{tcolorbox}
    \usepackage{parskip} % Stop auto-indenting (to mimic markdown behaviour)
    \usepackage{xeCJK}
    

    % Basic figure setup, for now with no caption control since it's done
    % automatically by Pandoc (which extracts ![](path) syntax from Markdown).
    \usepackage{graphicx}
    % Keep aspect ratio if custom image width or height is specified
    \setkeys{Gin}{keepaspectratio}
    % Maintain compatibility with old templates. Remove in nbconvert 6.0
    \let\Oldincludegraphics\includegraphics
    % Ensure that by default, figures have no caption (until we provide a
    % proper Figure object with a Caption API and a way to capture that
    % in the conversion process - todo).
    \usepackage{caption}
    \DeclareCaptionFormat{nocaption}{}
    \captionsetup{format=nocaption,aboveskip=0pt,belowskip=0pt}

    \usepackage{float}
    \floatplacement{figure}{H} % forces figures to be placed at the correct location
    \usepackage{xcolor} % Allow colors to be defined
    \usepackage{enumerate} % Needed for markdown enumerations to work
    \usepackage{geometry} % Used to adjust the document margins
    \usepackage{amsmath} % Equations
    \usepackage{amssymb} % Equations
    \usepackage{textcomp} % defines textquotesingle
    % Hack from http://tex.stackexchange.com/a/47451/13684:
    \AtBeginDocument{%
        \def\PYZsq{\textquotesingle}% Upright quotes in Pygmentized code
    }
    \usepackage{upquote} % Upright quotes for verbatim code
    \usepackage{eurosym} % defines \euro

    \usepackage{iftex}
    \ifPDFTeX
        \usepackage[T1]{fontenc}
        \IfFileExists{alphabeta.sty}{
              \usepackage{alphabeta}
          }{
              \usepackage[mathletters]{ucs}
              \usepackage[utf8x]{inputenc}
          }
    \else
        \usepackage{fontspec}
        \usepackage{unicode-math}
    \fi

    \usepackage{fancyvrb} % verbatim replacement that allows latex
    \usepackage{grffile} % extends the file name processing of package graphics
                         % to support a larger range
    \makeatletter % fix for old versions of grffile with XeLaTeX
    \@ifpackagelater{grffile}{2019/11/01}
    {
      % Do nothing on new versions
    }
    {
      \def\Gread@@xetex#1{%
        \IfFileExists{"\Gin@base".bb}%
        {\Gread@eps{\Gin@base.bb}}%
        {\Gread@@xetex@aux#1}%
      }
    }
    \makeatother
    \usepackage[Export]{adjustbox} % Used to constrain images to a maximum size
    \adjustboxset{max size={0.9\linewidth}{0.9\paperheight}}

    % The hyperref package gives us a pdf with properly built
    % internal navigation ('pdf bookmarks' for the table of contents,
    % internal cross-reference links, web links for URLs, etc.)
    \usepackage{hyperref}
    % The default LaTeX title has an obnoxious amount of whitespace. By default,
    % titling removes some of it. It also provides customization options.
    \usepackage{titling}
    \usepackage{longtable} % longtable support required by pandoc >1.10
    \usepackage{booktabs}  % table support for pandoc > 1.12.2
    \usepackage{array}     % table support for pandoc >= 2.11.3
    \usepackage{calc}      % table minipage width calculation for pandoc >= 2.11.1
    \usepackage[inline]{enumitem} % IRkernel/repr support (it uses the enumerate* environment)
    \usepackage[normalem]{ulem} % ulem is needed to support strikethroughs (\sout)
                                % normalem makes italics be italics, not underlines
    \usepackage{soul}      % strikethrough (\st) support for pandoc >= 3.0.0
    \usepackage{mathrsfs}
    

    
    % Colors for the hyperref package
    \definecolor{urlcolor}{rgb}{0,.145,.698}
    \definecolor{linkcolor}{rgb}{.71,0.21,0.01}
    \definecolor{citecolor}{rgb}{.12,.54,.11}

    % ANSI colors
    \definecolor{ansi-black}{HTML}{3E424D}
    \definecolor{ansi-black-intense}{HTML}{282C36}
    \definecolor{ansi-red}{HTML}{E75C58}
    \definecolor{ansi-red-intense}{HTML}{B22B31}
    \definecolor{ansi-green}{HTML}{00A250}
    \definecolor{ansi-green-intense}{HTML}{007427}
    \definecolor{ansi-yellow}{HTML}{DDB62B}
    \definecolor{ansi-yellow-intense}{HTML}{B27D12}
    \definecolor{ansi-blue}{HTML}{208FFB}
    \definecolor{ansi-blue-intense}{HTML}{0065CA}
    \definecolor{ansi-magenta}{HTML}{D160C4}
    \definecolor{ansi-magenta-intense}{HTML}{A03196}
    \definecolor{ansi-cyan}{HTML}{60C6C8}
    \definecolor{ansi-cyan-intense}{HTML}{258F8F}
    \definecolor{ansi-white}{HTML}{C5C1B4}
    \definecolor{ansi-white-intense}{HTML}{A1A6B2}
    \definecolor{ansi-default-inverse-fg}{HTML}{FFFFFF}
    \definecolor{ansi-default-inverse-bg}{HTML}{000000}

    % common color for the border for error outputs.
    \definecolor{outerrorbackground}{HTML}{FFDFDF}

    % commands and environments needed by pandoc snippets
    % extracted from the output of `pandoc -s`
    \providecommand{\tightlist}{%
      \setlength{\itemsep}{0pt}\setlength{\parskip}{0pt}}
    \DefineVerbatimEnvironment{Highlighting}{Verbatim}{commandchars=\\\{\}}
    % Add ',fontsize=\small' for more characters per line
    \newenvironment{Shaded}{}{}
    \newcommand{\KeywordTok}[1]{\textcolor[rgb]{0.00,0.44,0.13}{\textbf{{#1}}}}
    \newcommand{\DataTypeTok}[1]{\textcolor[rgb]{0.56,0.13,0.00}{{#1}}}
    \newcommand{\DecValTok}[1]{\textcolor[rgb]{0.25,0.63,0.44}{{#1}}}
    \newcommand{\BaseNTok}[1]{\textcolor[rgb]{0.25,0.63,0.44}{{#1}}}
    \newcommand{\FloatTok}[1]{\textcolor[rgb]{0.25,0.63,0.44}{{#1}}}
    \newcommand{\CharTok}[1]{\textcolor[rgb]{0.25,0.44,0.63}{{#1}}}
    \newcommand{\StringTok}[1]{\textcolor[rgb]{0.25,0.44,0.63}{{#1}}}
    \newcommand{\CommentTok}[1]{\textcolor[rgb]{0.38,0.63,0.69}{\textit{{#1}}}}
    \newcommand{\OtherTok}[1]{\textcolor[rgb]{0.00,0.44,0.13}{{#1}}}
    \newcommand{\AlertTok}[1]{\textcolor[rgb]{1.00,0.00,0.00}{\textbf{{#1}}}}
    \newcommand{\FunctionTok}[1]{\textcolor[rgb]{0.02,0.16,0.49}{{#1}}}
    \newcommand{\RegionMarkerTok}[1]{{#1}}
    \newcommand{\ErrorTok}[1]{\textcolor[rgb]{1.00,0.00,0.00}{\textbf{{#1}}}}
    \newcommand{\NormalTok}[1]{{#1}}

    % Additional commands for more recent versions of Pandoc
    \newcommand{\ConstantTok}[1]{\textcolor[rgb]{0.53,0.00,0.00}{{#1}}}
    \newcommand{\SpecialCharTok}[1]{\textcolor[rgb]{0.25,0.44,0.63}{{#1}}}
    \newcommand{\VerbatimStringTok}[1]{\textcolor[rgb]{0.25,0.44,0.63}{{#1}}}
    \newcommand{\SpecialStringTok}[1]{\textcolor[rgb]{0.73,0.40,0.53}{{#1}}}
    \newcommand{\ImportTok}[1]{{#1}}
    \newcommand{\DocumentationTok}[1]{\textcolor[rgb]{0.73,0.13,0.13}{\textit{{#1}}}}
    \newcommand{\AnnotationTok}[1]{\textcolor[rgb]{0.38,0.63,0.69}{\textbf{\textit{{#1}}}}}
    \newcommand{\CommentVarTok}[1]{\textcolor[rgb]{0.38,0.63,0.69}{\textbf{\textit{{#1}}}}}
    \newcommand{\VariableTok}[1]{\textcolor[rgb]{0.10,0.09,0.49}{{#1}}}
    \newcommand{\ControlFlowTok}[1]{\textcolor[rgb]{0.00,0.44,0.13}{\textbf{{#1}}}}
    \newcommand{\OperatorTok}[1]{\textcolor[rgb]{0.40,0.40,0.40}{{#1}}}
    \newcommand{\BuiltInTok}[1]{{#1}}
    \newcommand{\ExtensionTok}[1]{{#1}}
    \newcommand{\PreprocessorTok}[1]{\textcolor[rgb]{0.74,0.48,0.00}{{#1}}}
    \newcommand{\AttributeTok}[1]{\textcolor[rgb]{0.49,0.56,0.16}{{#1}}}
    \newcommand{\InformationTok}[1]{\textcolor[rgb]{0.38,0.63,0.69}{\textbf{\textit{{#1}}}}}
    \newcommand{\WarningTok}[1]{\textcolor[rgb]{0.38,0.63,0.69}{\textbf{\textit{{#1}}}}}


    % Define a nice break command that doesn't care if a line doesn't already
    % exist.
    \def\br{\hspace*{\fill} \\* }
    % Math Jax compatibility definitions
    \def\gt{>}
    \def\lt{<}
    \let\Oldtex\TeX
    \let\Oldlatex\LaTeX
    \renewcommand{\TeX}{\textrm{\Oldtex}}
    \renewcommand{\LaTeX}{\textrm{\Oldlatex}}
    % Document parameters
    % Document title
    \title{数值计算课程大作业}
    \author{陈政宇\ 23336003\ 计科2班}
    
    
    
    
    
    
    
% Pygments definitions
\makeatletter
\def\PY@reset{\let\PY@it=\relax \let\PY@bf=\relax%
    \let\PY@ul=\relax \let\PY@tc=\relax%
    \let\PY@bc=\relax \let\PY@ff=\relax}
\def\PY@tok#1{\csname PY@tok@#1\endcsname}
\def\PY@toks#1+{\ifx\relax#1\empty\else%
    \PY@tok{#1}\expandafter\PY@toks\fi}
\def\PY@do#1{\PY@bc{\PY@tc{\PY@ul{%
    \PY@it{\PY@bf{\PY@ff{#1}}}}}}}
\def\PY#1#2{\PY@reset\PY@toks#1+\relax+\PY@do{#2}}

\@namedef{PY@tok@w}{\def\PY@tc##1{\textcolor[rgb]{0.73,0.73,0.73}{##1}}}
\@namedef{PY@tok@c}{\let\PY@it=\textit\def\PY@tc##1{\textcolor[rgb]{0.24,0.48,0.48}{##1}}}
\@namedef{PY@tok@cp}{\def\PY@tc##1{\textcolor[rgb]{0.61,0.40,0.00}{##1}}}
\@namedef{PY@tok@k}{\let\PY@bf=\textbf\def\PY@tc##1{\textcolor[rgb]{0.00,0.50,0.00}{##1}}}
\@namedef{PY@tok@kp}{\def\PY@tc##1{\textcolor[rgb]{0.00,0.50,0.00}{##1}}}
\@namedef{PY@tok@kt}{\def\PY@tc##1{\textcolor[rgb]{0.69,0.00,0.25}{##1}}}
\@namedef{PY@tok@o}{\def\PY@tc##1{\textcolor[rgb]{0.40,0.40,0.40}{##1}}}
\@namedef{PY@tok@ow}{\let\PY@bf=\textbf\def\PY@tc##1{\textcolor[rgb]{0.67,0.13,1.00}{##1}}}
\@namedef{PY@tok@nb}{\def\PY@tc##1{\textcolor[rgb]{0.00,0.50,0.00}{##1}}}
\@namedef{PY@tok@nf}{\def\PY@tc##1{\textcolor[rgb]{0.00,0.00,1.00}{##1}}}
\@namedef{PY@tok@nc}{\let\PY@bf=\textbf\def\PY@tc##1{\textcolor[rgb]{0.00,0.00,1.00}{##1}}}
\@namedef{PY@tok@nn}{\let\PY@bf=\textbf\def\PY@tc##1{\textcolor[rgb]{0.00,0.00,1.00}{##1}}}
\@namedef{PY@tok@ne}{\let\PY@bf=\textbf\def\PY@tc##1{\textcolor[rgb]{0.80,0.25,0.22}{##1}}}
\@namedef{PY@tok@nv}{\def\PY@tc##1{\textcolor[rgb]{0.10,0.09,0.49}{##1}}}
\@namedef{PY@tok@no}{\def\PY@tc##1{\textcolor[rgb]{0.53,0.00,0.00}{##1}}}
\@namedef{PY@tok@nl}{\def\PY@tc##1{\textcolor[rgb]{0.46,0.46,0.00}{##1}}}
\@namedef{PY@tok@ni}{\let\PY@bf=\textbf\def\PY@tc##1{\textcolor[rgb]{0.44,0.44,0.44}{##1}}}
\@namedef{PY@tok@na}{\def\PY@tc##1{\textcolor[rgb]{0.41,0.47,0.13}{##1}}}
\@namedef{PY@tok@nt}{\let\PY@bf=\textbf\def\PY@tc##1{\textcolor[rgb]{0.00,0.50,0.00}{##1}}}
\@namedef{PY@tok@nd}{\def\PY@tc##1{\textcolor[rgb]{0.67,0.13,1.00}{##1}}}
\@namedef{PY@tok@s}{\def\PY@tc##1{\textcolor[rgb]{0.73,0.13,0.13}{##1}}}
\@namedef{PY@tok@sd}{\let\PY@it=\textit\def\PY@tc##1{\textcolor[rgb]{0.73,0.13,0.13}{##1}}}
\@namedef{PY@tok@si}{\let\PY@bf=\textbf\def\PY@tc##1{\textcolor[rgb]{0.64,0.35,0.47}{##1}}}
\@namedef{PY@tok@se}{\let\PY@bf=\textbf\def\PY@tc##1{\textcolor[rgb]{0.67,0.36,0.12}{##1}}}
\@namedef{PY@tok@sr}{\def\PY@tc##1{\textcolor[rgb]{0.64,0.35,0.47}{##1}}}
\@namedef{PY@tok@ss}{\def\PY@tc##1{\textcolor[rgb]{0.10,0.09,0.49}{##1}}}
\@namedef{PY@tok@sx}{\def\PY@tc##1{\textcolor[rgb]{0.00,0.50,0.00}{##1}}}
\@namedef{PY@tok@m}{\def\PY@tc##1{\textcolor[rgb]{0.40,0.40,0.40}{##1}}}
\@namedef{PY@tok@gh}{\let\PY@bf=\textbf\def\PY@tc##1{\textcolor[rgb]{0.00,0.00,0.50}{##1}}}
\@namedef{PY@tok@gu}{\let\PY@bf=\textbf\def\PY@tc##1{\textcolor[rgb]{0.50,0.00,0.50}{##1}}}
\@namedef{PY@tok@gd}{\def\PY@tc##1{\textcolor[rgb]{0.63,0.00,0.00}{##1}}}
\@namedef{PY@tok@gi}{\def\PY@tc##1{\textcolor[rgb]{0.00,0.52,0.00}{##1}}}
\@namedef{PY@tok@gr}{\def\PY@tc##1{\textcolor[rgb]{0.89,0.00,0.00}{##1}}}
\@namedef{PY@tok@ge}{\let\PY@it=\textit}
\@namedef{PY@tok@gs}{\let\PY@bf=\textbf}
\@namedef{PY@tok@gp}{\let\PY@bf=\textbf\def\PY@tc##1{\textcolor[rgb]{0.00,0.00,0.50}{##1}}}
\@namedef{PY@tok@go}{\def\PY@tc##1{\textcolor[rgb]{0.44,0.44,0.44}{##1}}}
\@namedef{PY@tok@gt}{\def\PY@tc##1{\textcolor[rgb]{0.00,0.27,0.87}{##1}}}
\@namedef{PY@tok@err}{\def\PY@bc##1{{\setlength{\fboxsep}{\string -\fboxrule}\fcolorbox[rgb]{1.00,0.00,0.00}{1,1,1}{\strut ##1}}}}
\@namedef{PY@tok@kc}{\let\PY@bf=\textbf\def\PY@tc##1{\textcolor[rgb]{0.00,0.50,0.00}{##1}}}
\@namedef{PY@tok@kd}{\let\PY@bf=\textbf\def\PY@tc##1{\textcolor[rgb]{0.00,0.50,0.00}{##1}}}
\@namedef{PY@tok@kn}{\let\PY@bf=\textbf\def\PY@tc##1{\textcolor[rgb]{0.00,0.50,0.00}{##1}}}
\@namedef{PY@tok@kr}{\let\PY@bf=\textbf\def\PY@tc##1{\textcolor[rgb]{0.00,0.50,0.00}{##1}}}
\@namedef{PY@tok@bp}{\def\PY@tc##1{\textcolor[rgb]{0.00,0.50,0.00}{##1}}}
\@namedef{PY@tok@fm}{\def\PY@tc##1{\textcolor[rgb]{0.00,0.00,1.00}{##1}}}
\@namedef{PY@tok@vc}{\def\PY@tc##1{\textcolor[rgb]{0.10,0.09,0.49}{##1}}}
\@namedef{PY@tok@vg}{\def\PY@tc##1{\textcolor[rgb]{0.10,0.09,0.49}{##1}}}
\@namedef{PY@tok@vi}{\def\PY@tc##1{\textcolor[rgb]{0.10,0.09,0.49}{##1}}}
\@namedef{PY@tok@vm}{\def\PY@tc##1{\textcolor[rgb]{0.10,0.09,0.49}{##1}}}
\@namedef{PY@tok@sa}{\def\PY@tc##1{\textcolor[rgb]{0.73,0.13,0.13}{##1}}}
\@namedef{PY@tok@sb}{\def\PY@tc##1{\textcolor[rgb]{0.73,0.13,0.13}{##1}}}
\@namedef{PY@tok@sc}{\def\PY@tc##1{\textcolor[rgb]{0.73,0.13,0.13}{##1}}}
\@namedef{PY@tok@dl}{\def\PY@tc##1{\textcolor[rgb]{0.73,0.13,0.13}{##1}}}
\@namedef{PY@tok@s2}{\def\PY@tc##1{\textcolor[rgb]{0.73,0.13,0.13}{##1}}}
\@namedef{PY@tok@sh}{\def\PY@tc##1{\textcolor[rgb]{0.73,0.13,0.13}{##1}}}
\@namedef{PY@tok@s1}{\def\PY@tc##1{\textcolor[rgb]{0.73,0.13,0.13}{##1}}}
\@namedef{PY@tok@mb}{\def\PY@tc##1{\textcolor[rgb]{0.40,0.40,0.40}{##1}}}
\@namedef{PY@tok@mf}{\def\PY@tc##1{\textcolor[rgb]{0.40,0.40,0.40}{##1}}}
\@namedef{PY@tok@mh}{\def\PY@tc##1{\textcolor[rgb]{0.40,0.40,0.40}{##1}}}
\@namedef{PY@tok@mi}{\def\PY@tc##1{\textcolor[rgb]{0.40,0.40,0.40}{##1}}}
\@namedef{PY@tok@il}{\def\PY@tc##1{\textcolor[rgb]{0.40,0.40,0.40}{##1}}}
\@namedef{PY@tok@mo}{\def\PY@tc##1{\textcolor[rgb]{0.40,0.40,0.40}{##1}}}
\@namedef{PY@tok@ch}{\let\PY@it=\textit\def\PY@tc##1{\textcolor[rgb]{0.24,0.48,0.48}{##1}}}
\@namedef{PY@tok@cm}{\let\PY@it=\textit\def\PY@tc##1{\textcolor[rgb]{0.24,0.48,0.48}{##1}}}
\@namedef{PY@tok@cpf}{\let\PY@it=\textit\def\PY@tc##1{\textcolor[rgb]{0.24,0.48,0.48}{##1}}}
\@namedef{PY@tok@c1}{\let\PY@it=\textit\def\PY@tc##1{\textcolor[rgb]{0.24,0.48,0.48}{##1}}}
\@namedef{PY@tok@cs}{\let\PY@it=\textit\def\PY@tc##1{\textcolor[rgb]{0.24,0.48,0.48}{##1}}}

\def\PYZbs{\char`\\}
\def\PYZus{\char`\_}
\def\PYZob{\char`\{}
\def\PYZcb{\char`\}}
\def\PYZca{\char`\^}
\def\PYZam{\char`\&}
\def\PYZlt{\char`\<}
\def\PYZgt{\char`\>}
\def\PYZsh{\char`\#}
\def\PYZpc{\char`\%}
\def\PYZdl{\char`\$}
\def\PYZhy{\char`\-}
\def\PYZsq{\char`\'}
\def\PYZdq{\char`\"}
\def\PYZti{\char`\~}
% for compatibility with earlier versions
\def\PYZat{@}
\def\PYZlb{[}
\def\PYZrb{]}
\makeatother


    % For linebreaks inside Verbatim environment from package fancyvrb.
    \makeatletter
        \newbox\Wrappedcontinuationbox
        \newbox\Wrappedvisiblespacebox
        \newcommand*\Wrappedvisiblespace {\textcolor{red}{\textvisiblespace}}
        \newcommand*\Wrappedcontinuationsymbol {\textcolor{red}{\llap{\tiny$\m@th\hookrightarrow$}}}
        \newcommand*\Wrappedcontinuationindent {3ex }
        \newcommand*\Wrappedafterbreak {\kern\Wrappedcontinuationindent\copy\Wrappedcontinuationbox}
        % Take advantage of the already applied Pygments mark-up to insert
        % potential linebreaks for TeX processing.
        %        {, <, #, %, $, ' and ": go to next line.
        %        _, }, ^, &, >, - and ~: stay at end of broken line.
        % Use of \textquotesingle for straight quote.
        \newcommand*\Wrappedbreaksatspecials {%
            \def\PYGZus{\discretionary{\char`\_}{\Wrappedafterbreak}{\char`\_}}%
            \def\PYGZob{\discretionary{}{\Wrappedafterbreak\char`\{}{\char`\{}}%
            \def\PYGZcb{\discretionary{\char`\}}{\Wrappedafterbreak}{\char`\}}}%
            \def\PYGZca{\discretionary{\char`\^}{\Wrappedafterbreak}{\char`\^}}%
            \def\PYGZam{\discretionary{\char`\&}{\Wrappedafterbreak}{\char`\&}}%
            \def\PYGZlt{\discretionary{}{\Wrappedafterbreak\char`\<}{\char`\<}}%
            \def\PYGZgt{\discretionary{\char`\>}{\Wrappedafterbreak}{\char`\>}}%
            \def\PYGZsh{\discretionary{}{\Wrappedafterbreak\char`\#}{\char`\#}}%
            \def\PYGZpc{\discretionary{}{\Wrappedafterbreak\char`\%}{\char`\%}}%
            \def\PYGZdl{\discretionary{}{\Wrappedafterbreak\char`\$}{\char`\$}}%
            \def\PYGZhy{\discretionary{\char`\-}{\Wrappedafterbreak}{\char`\-}}%
            \def\PYGZsq{\discretionary{}{\Wrappedafterbreak\textquotesingle}{\textquotesingle}}%
            \def\PYGZdq{\discretionary{}{\Wrappedafterbreak\char`\"}{\char`\"}}%
            \def\PYGZti{\discretionary{\char`\~}{\Wrappedafterbreak}{\char`\~}}%
        }
        % Some characters . , ; ? ! / are not pygmentized.
        % This macro makes them "active" and they will insert potential linebreaks
        \newcommand*\Wrappedbreaksatpunct {%
            \lccode`\~`\.\lowercase{\def~}{\discretionary{\hbox{\char`\.}}{\Wrappedafterbreak}{\hbox{\char`\.}}}%
            \lccode`\~`\,\lowercase{\def~}{\discretionary{\hbox{\char`\,}}{\Wrappedafterbreak}{\hbox{\char`\,}}}%
            \lccode`\~`\;\lowercase{\def~}{\discretionary{\hbox{\char`\;}}{\Wrappedafterbreak}{\hbox{\char`\;}}}%
            \lccode`\~`\:\lowercase{\def~}{\discretionary{\hbox{\char`\:}}{\Wrappedafterbreak}{\hbox{\char`\:}}}%
            \lccode`\~`\?\lowercase{\def~}{\discretionary{\hbox{\char`\?}}{\Wrappedafterbreak}{\hbox{\char`\?}}}%
            \lccode`\~`\!\lowercase{\def~}{\discretionary{\hbox{\char`\!}}{\Wrappedafterbreak}{\hbox{\char`\!}}}%
            \lccode`\~`\/\lowercase{\def~}{\discretionary{\hbox{\char`\/}}{\Wrappedafterbreak}{\hbox{\char`\/}}}%
            \catcode`\.\active
            \catcode`\,\active
            \catcode`\;\active
            \catcode`\:\active
            \catcode`\?\active
            \catcode`\!\active
            \catcode`\/\active
            \lccode`\~`\~
        }
    \makeatother

    \let\OriginalVerbatim=\Verbatim
    \makeatletter
    \renewcommand{\Verbatim}[1][1]{%
        %\parskip\z@skip
        \sbox\Wrappedcontinuationbox {\Wrappedcontinuationsymbol}%
        \sbox\Wrappedvisiblespacebox {\FV@SetupFont\Wrappedvisiblespace}%
        \def\FancyVerbFormatLine ##1{\hsize\linewidth
            \vtop{\raggedright\hyphenpenalty\z@\exhyphenpenalty\z@
                \doublehyphendemerits\z@\finalhyphendemerits\z@
                \strut ##1\strut}%
        }%
        % If the linebreak is at a space, the latter will be displayed as visible
        % space at end of first line, and a continuation symbol starts next line.
        % Stretch/shrink are however usually zero for typewriter font.
        \def\FV@Space {%
            \nobreak\hskip\z@ plus\fontdimen3\font minus\fontdimen4\font
            \discretionary{\copy\Wrappedvisiblespacebox}{\Wrappedafterbreak}
            {\kern\fontdimen2\font}%
        }%

        % Allow breaks at special characters using \PYG... macros.
        \Wrappedbreaksatspecials
        % Breaks at punctuation characters . , ; ? ! and / need catcode=\active
        \OriginalVerbatim[#1,codes*=\Wrappedbreaksatpunct]%
    }
    \makeatother

    % Exact colors from NB
    \definecolor{incolor}{HTML}{303F9F}
    \definecolor{outcolor}{HTML}{D84315}
    \definecolor{cellborder}{HTML}{CFCFCF}
    \definecolor{cellbackground}{HTML}{F7F7F7}

    % prompt
    \makeatletter
    \newcommand{\boxspacing}{\kern\kvtcb@left@rule\kern\kvtcb@boxsep}
    \makeatother
    \newcommand{\prompt}[4]{
        {\ttfamily\llap{{\color{#2}[#3]:\hspace{3pt}#4}}\vspace{-\baselineskip}}
    }
    

    
    % Prevent overflowing lines due to hard-to-break entities
    \sloppy
    % Setup hyperref package
    \hypersetup{
      breaklinks=true,  % so long urls are correctly broken across lines
      colorlinks=true,
      urlcolor=urlcolor,
      linkcolor=linkcolor,
      citecolor=citecolor,
      }
    % Slightly bigger margins than the latex defaults
    
    \geometry{verbose,tmargin=1in,bmargin=1in,lmargin=1in,rmargin=1in}
    
    

\begin{document}
    
    \maketitle
    \tableofcontents
    \newpage % Add a new page after the table of contents for better layout
    

    
    \section{问题一}\label{ux95eeux9898ux4e00}

用不同数值方法计算积分

\[
\int_0^1 \sqrt x\ln x dx = -\frac{4}{9}
\]

\begin{enumerate}
\def\labelenumi{(\arabic{enumi})}
\tightlist
\item
\end{enumerate}

取不同步长 \(h\),分别用复合梯形及复合辛普森求积计算积分,给出误差中关于
\(h\) 的函数,并于积分精确值比较两个公式的精度。是否存在一个最小的
\(h\),使得精度不能再被改善?

\begin{enumerate}
\def\labelenumi{(\arabic{enumi})}
\setcounter{enumi}{1}
\tightlist
\item
\end{enumerate}

用龙贝格求积计算完成问题 (1)

    \subsection{Solution}\label{solution}

\[ I = \int_0^1 \sqrt x\ln x dx \]

已知的精确值为 \(I = -\frac{4}{9}\)。

被积函数为 \(f(x) = \sqrt x \ln x\)。我们注意到,当 \(x \to 0^+\)
时,根据洛必达法则或已知极限 \(\lim_{x\to 0^+} x^\alpha \ln x = 0\)
(对于 \(\alpha > 0\)),可知
\(\lim_{x\to 0^+} \sqrt x \ln x = 0\)。因此,我们可以定义 \(f(0)=0\)
以使得函数在积分区间 \([0,1]\) 上有定义。

然而,函数 \(f(x)\) 的导数在 \(x=0\) 处是奇异的。

\(f'(x) = \frac{d}{dx}(x^{1/2} \ln x) = \frac{1}{2}x^{-1/2}\ln x + x^{1/2} \cdot \frac{1}{x} = \frac{\ln x}{2\sqrt{x}} + \frac{1}{\sqrt{x}} = \frac{\ln x + 2}{2\sqrt{x}}\)。

当 \(x \to 0^+\) 时,\(f'(x) \to -\infty\)。
\(f''(x) = \frac{d}{dx}\left(\frac{\ln x + 2}{2\sqrt{x}}\right) = \frac{\frac{1}{x}(2\sqrt{x}) - (\ln x + 2)(x^{-1/2})}{4x} = \frac{2x^{-1/2} - (\ln x + 2)x^{-1/2}}{4x} = \frac{(2 - \ln x - 2)x^{-1/2}}{4x} = \frac{-\ln x}{4x^{3/2}}\)。

当 \(x \to 0^+\) 时,\(f''(x) \to \infty\)。

这种在积分端点 \(x=0\)
处的奇异性(导数无界)将会影响数值积分方法的收敛速度和精度。

\subsubsection{复合梯形公式及复合辛普森公式}\label{ux590dux5408ux68afux5f62ux516cux5f0fux53caux590dux5408ux8f9bux666eux68eeux516cux5f0f}

\paragraph{数学公式}\label{ux6570ux5b66ux516cux5f0f}

\textbf{复合梯形公式:}

将积分区间 \([a, b]\) 分成 \(n\) 个等长的子区间,每个子区间的长度为
\(h = \frac{b-a}{n}\)。节点为 \(x_i = a + ih\),\(i=0, 1, \dots, n\)。

复合梯形公式为:

\[ T_n(f) = h \left[ \frac{1}{2}f(x_0) + \sum_{i=1}^{n-1} f(x_i) + \frac{1}{2}f(x_n) \right] \]

对于具有二阶连续导数的函数,其截断误差为:

\[ E_T(f) = I - T_n(f) = -\frac{b-a}{12}h^2 f''(\eta), \quad \text{其中 } \eta \in (a,b) \]

理论上,误差是 \(O(h^2)\) 阶的。

\textbf{复合辛普森公式:}

将积分区间 \([a, b]\) 分成 \(n\) 个等长的子区间(要求 \(n\)
必须为偶数),步长 \(h = \frac{b-a}{n}\)。 复合辛普森公式为:

\[ S_n(f) = \frac{h}{3} \left[ f(x_0) + 4\sum_{k=1}^{n/2} f(x_{2k-1}) + 2\sum_{k=1}^{n/2-1} f(x_{2k}) + f(x_n) \right] \]

对于具有四阶连续导数的函数,其截断误差为:

\[ E_S(f) = I - S_n(f) = -\frac{b-a}{180}h^4 f^{(4)}(\eta), \quad \text{其中 } \eta \in (a,b) \]

理论上,误差是 \(O(h^4)\) 阶的。

我们将通过数值实验来观察误差 \(|E(h)|\) 随 \(h\) 变化的具体关系。假设
\(|E(h)| \approx C \cdot h^p\)。在双对数坐标系中绘制 \(\log|E(h)|\) 对
\(\log h\) 的图像,其斜率近似为实际的收敛阶数 \(p\)。

\subsubsection{龙贝格求积}\label{ux9f99ux8d1dux683cux6c42ux79ef}

算法步骤如下: 1. 初始化:计算 \(R_{0,0}\) (即
\(T_0\)),这是使用整个区间 \([a,b]\) 的梯形公式(\(n=1\),
\(h_0 = b-a\)): \[ R_{0,0} = \frac{b-a}{2} [f(a) + f(b)] \] 2.
递推计算梯形值:对于 \(k = 1, 2, \dots, M\) (最大迭代次数),计算
\(R_{k,0}\) (即 \(T_{2^k}\)),这是使用 \(n_k = 2^k\)
个子区间的复合梯形公式,步长 \(h_k = (b-a)/2^k\):
\[ R_{k,0} = \frac{1}{2} R_{k-1,0} + h_k \sum_{i=1}^{2^{k-1}} f(a + (2i-1)h_k) \]
这个公式利用了 \(R_{k-1,0}\) 的结果来高效计算 \(R_{k,0}\)。 3.
进行外推:对于 \(m = 1, 2, \dots, k\):
\[ R_{k,m} = R_{k,m-1} + \frac{R_{k,m-1} - R_{k-1,m-1}}{4^m - 1} \]
这样会形成一个下三角的龙贝格表: \[
    \begin{array}{ccccc}
    R_{0,0} & & & & \\
    R_{1,0} & R_{1,1} & & & \\
    R_{2,0} & R_{2,1} & R_{2,2} & & \\
    \vdots & \vdots & \vdots & \ddots & \\
    R_{M,0} & R_{M,1} & R_{M,2} & \dots & R_{M,M}
    \end{array}
    \] 对角线上的元素 \(R_{k,k}\) 通常是具有最高精度的近似值。

\subsubsection{Python 实现}\label{python-ux5b9eux73b0}

    \begin{tcolorbox}[breakable, size=fbox, boxrule=1pt, pad at break*=1mm,colback=cellbackground, colframe=cellborder]
\prompt{In}{incolor}{13}{\boxspacing}
\begin{Verbatim}[commandchars=\\\{\}]
\PY{k+kn}{import} \PY{n+nn}{numpy} \PY{k}{as} \PY{n+nn}{np}
\PY{k+kn}{import} \PY{n+nn}{matplotlib}\PY{n+nn}{.}\PY{n+nn}{pyplot} \PY{k}{as} \PY{n+nn}{plt}
\PY{k+kn}{import} \PY{n+nn}{matplotlib}
\PY{n}{plt}\PY{o}{.}\PY{n}{rcParams}\PY{p}{[}\PY{l+s+s1}{\PYZsq{}}\PY{l+s+s1}{font.sans\PYZhy{}serif}\PY{l+s+s1}{\PYZsq{}}\PY{p}{]} \PY{o}{=} \PY{p}{[}\PY{l+s+s1}{\PYZsq{}}\PY{l+s+s1}{Arial Unicode MS}\PY{l+s+s1}{\PYZsq{}}\PY{p}{,} \PY{l+s+s1}{\PYZsq{}}\PY{l+s+s1}{sans\PYZhy{}serif}\PY{l+s+s1}{\PYZsq{}}\PY{p}{,} \PY{l+s+s1}{\PYZsq{}}\PY{l+s+s1}{SimSong}\PY{l+s+s1}{\PYZsq{}}\PY{p}{]}
\PY{n}{plt}\PY{o}{.}\PY{n}{rcParams}\PY{p}{[}\PY{l+s+s1}{\PYZsq{}}\PY{l+s+s1}{axes.unicode\PYZus{}minus}\PY{l+s+s1}{\PYZsq{}}\PY{p}{]} \PY{o}{=} \PY{k+kc}{False}  \PY{c+c1}{\PYZsh{} 解决负号\PYZsq{}\PYZhy{}\PYZsq{}显示为方块的问题}

\PY{c+c1}{\PYZsh{} Define the integrand}
\PY{k}{def} \PY{n+nf}{f}\PY{p}{(}\PY{n}{x}\PY{p}{)}\PY{p}{:}
    \PY{k}{if} \PY{n}{x} \PY{o}{==} \PY{l+m+mi}{0}\PY{p}{:}
        \PY{k}{return} \PY{l+m+mf}{0.0}
    \PY{k}{if} \PY{n}{x} \PY{o}{\PYZlt{}} \PY{l+m+mf}{1e\PYZhy{}100}\PY{p}{:} \PY{c+c1}{\PYZsh{} Avoid log(0) issues if x is extremely small positive}
        \PY{k}{return} \PY{l+m+mf}{0.0}
    \PY{k}{return} \PY{n}{np}\PY{o}{.}\PY{n}{sqrt}\PY{p}{(}\PY{n}{x}\PY{p}{)} \PY{o}{*} \PY{n}{np}\PY{o}{.}\PY{n}{log}\PY{p}{(}\PY{n}{x}\PY{p}{)}

\PY{n}{exact\PYZus{}value} \PY{o}{=} \PY{o}{\PYZhy{}}\PY{l+m+mi}{4}\PY{o}{/}\PY{l+m+mi}{9}

\PY{c+c1}{\PYZsh{} (1) Composite Trapezoidal and Simpson\PYZsq{}s Rules}

\PY{c+c1}{\PYZsh{} Composite Trapezoidal Rule}
\PY{k}{def} \PY{n+nf}{composite\PYZus{}trapezoidal}\PY{p}{(}\PY{n}{func}\PY{p}{,} \PY{n}{a}\PY{p}{,} \PY{n}{b}\PY{p}{,} \PY{n}{n}\PY{p}{)}\PY{p}{:}
    \PY{n}{h} \PY{o}{=} \PY{p}{(}\PY{n}{b} \PY{o}{\PYZhy{}} \PY{n}{a}\PY{p}{)} \PY{o}{/} \PY{n}{n}
    \PY{n}{integral} \PY{o}{=} \PY{l+m+mf}{0.5} \PY{o}{*} \PY{p}{(}\PY{n}{func}\PY{p}{(}\PY{n}{a}\PY{p}{)} \PY{o}{+} \PY{n}{func}\PY{p}{(}\PY{n}{b}\PY{p}{)}\PY{p}{)}
    \PY{k}{for} \PY{n}{i} \PY{o+ow}{in} \PY{n+nb}{range}\PY{p}{(}\PY{l+m+mi}{1}\PY{p}{,} \PY{n}{n}\PY{p}{)}\PY{p}{:}
        \PY{n}{integral} \PY{o}{+}\PY{o}{=} \PY{n}{func}\PY{p}{(}\PY{n}{a} \PY{o}{+} \PY{n}{i} \PY{o}{*} \PY{n}{h}\PY{p}{)}
    \PY{n}{integral} \PY{o}{*}\PY{o}{=} \PY{n}{h}
    \PY{k}{return} \PY{n}{integral}

\PY{c+c1}{\PYZsh{} Composite Simpson\PYZsq{}s Rule}
\PY{k}{def} \PY{n+nf}{composite\PYZus{}simpson}\PY{p}{(}\PY{n}{func}\PY{p}{,} \PY{n}{a}\PY{p}{,} \PY{n}{b}\PY{p}{,} \PY{n}{n}\PY{p}{)}\PY{p}{:}
    \PY{k}{if} \PY{n}{n} \PY{o}{\PYZpc{}} \PY{l+m+mi}{2} \PY{o}{!=} \PY{l+m+mi}{0}\PY{p}{:}
        \PY{k}{raise} \PY{n+ne}{ValueError}\PY{p}{(}\PY{l+s+s2}{\PYZdq{}}\PY{l+s+s2}{n must be an even number for Composite Simpson}\PY{l+s+s2}{\PYZsq{}}\PY{l+s+s2}{s Rule.}\PY{l+s+s2}{\PYZdq{}}\PY{p}{)}
    \PY{n}{h} \PY{o}{=} \PY{p}{(}\PY{n}{b} \PY{o}{\PYZhy{}} \PY{n}{a}\PY{p}{)} \PY{o}{/} \PY{n}{n}
    \PY{n}{integral} \PY{o}{=} \PY{n}{func}\PY{p}{(}\PY{n}{a}\PY{p}{)} \PY{o}{+} \PY{n}{func}\PY{p}{(}\PY{n}{b}\PY{p}{)}
    \PY{k}{for} \PY{n}{i} \PY{o+ow}{in} \PY{n+nb}{range}\PY{p}{(}\PY{l+m+mi}{1}\PY{p}{,} \PY{n}{n}\PY{p}{,} \PY{l+m+mi}{2}\PY{p}{)}\PY{p}{:} \PY{c+c1}{\PYZsh{} Odd indices}
        \PY{n}{integral} \PY{o}{+}\PY{o}{=} \PY{l+m+mi}{4} \PY{o}{*} \PY{n}{func}\PY{p}{(}\PY{n}{a} \PY{o}{+} \PY{n}{i} \PY{o}{*} \PY{n}{h}\PY{p}{)}
    \PY{k}{for} \PY{n}{i} \PY{o+ow}{in} \PY{n+nb}{range}\PY{p}{(}\PY{l+m+mi}{2}\PY{p}{,} \PY{n}{n}\PY{p}{,} \PY{l+m+mi}{2}\PY{p}{)}\PY{p}{:} \PY{c+c1}{\PYZsh{} Even indices (up to n\PYZhy{}2)}
        \PY{n}{integral} \PY{o}{+}\PY{o}{=} \PY{l+m+mi}{2} \PY{o}{*} \PY{n}{func}\PY{p}{(}\PY{n}{a} \PY{o}{+} \PY{n}{i} \PY{o}{*} \PY{n}{h}\PY{p}{)}
    \PY{n}{integral} \PY{o}{*}\PY{o}{=} \PY{n}{h} \PY{o}{/} \PY{l+m+mi}{3}
    \PY{k}{return} \PY{n}{integral}

\PY{c+c1}{\PYZsh{} (1) Composite Trapezoidal and Simpson\PYZsq{}s Rules}
\PY{c+c1}{\PYZsh{} ... (composite\PYZus{}trapezoidal and composite\PYZus{}simpson functions remain the same) ...}

\PY{c+c1}{\PYZsh{} Analysis for Part (1)}
\PY{n}{a}\PY{p}{,} \PY{n}{b} \PY{o}{=} \PY{l+m+mf}{0.0}\PY{p}{,} \PY{l+m+mf}{1.0}
\PY{c+c1}{\PYZsh{} n\PYZus{}values for Trapezoidal and Simpson, h\PYZus{}values, errors will be calculated as before}
\PY{n}{n\PYZus{}values} \PY{o}{=} \PY{p}{[}\PY{l+m+mi}{2}\PY{o}{*}\PY{o}{*}\PY{n}{k} \PY{k}{for} \PY{n}{k} \PY{o+ow}{in} \PY{n+nb}{range}\PY{p}{(}\PY{l+m+mi}{2}\PY{p}{,} \PY{l+m+mi}{21}\PY{p}{)}\PY{p}{]} 
\PY{n}{h\PYZus{}values} \PY{o}{=} \PY{p}{[}\PY{p}{]}
\PY{n}{trapezoidal\PYZus{}errors} \PY{o}{=} \PY{p}{[}\PY{p}{]}
\PY{n}{simpson\PYZus{}errors} \PY{o}{=} \PY{p}{[}\PY{p}{]}

\PY{n+nb}{print}\PY{p}{(}\PY{l+s+sa}{f}\PY{l+s+s2}{\PYZdq{}}\PY{l+s+s2}{Exact value: }\PY{l+s+si}{\PYZob{}}\PY{n}{exact\PYZus{}value}\PY{l+s+si}{\PYZcb{}}\PY{l+s+se}{\PYZbs{}n}\PY{l+s+s2}{\PYZdq{}}\PY{p}{)}
\PY{n+nb}{print}\PY{p}{(}\PY{l+s+s2}{\PYZdq{}}\PY{l+s+s2}{Part (1): Composite Trapezoidal and Simpson}\PY{l+s+s2}{\PYZsq{}}\PY{l+s+s2}{s Rule Analysis}\PY{l+s+s2}{\PYZdq{}}\PY{p}{)}
\PY{n+nb}{print}\PY{p}{(}\PY{l+s+s2}{\PYZdq{}}\PY{l+s+s2}{\PYZhy{}\PYZhy{}\PYZhy{}\PYZhy{}\PYZhy{}\PYZhy{}\PYZhy{}\PYZhy{}\PYZhy{}\PYZhy{}\PYZhy{}\PYZhy{}\PYZhy{}\PYZhy{}\PYZhy{}\PYZhy{}\PYZhy{}\PYZhy{}\PYZhy{}\PYZhy{}\PYZhy{}\PYZhy{}\PYZhy{}\PYZhy{}\PYZhy{}\PYZhy{}\PYZhy{}\PYZhy{}\PYZhy{}\PYZhy{}\PYZhy{}\PYZhy{}\PYZhy{}\PYZhy{}\PYZhy{}\PYZhy{}\PYZhy{}\PYZhy{}\PYZhy{}\PYZhy{}\PYZhy{}\PYZhy{}\PYZhy{}\PYZhy{}\PYZhy{}\PYZhy{}\PYZhy{}\PYZhy{}\PYZhy{}\PYZhy{}\PYZhy{}\PYZhy{}\PYZhy{}\PYZhy{}\PYZhy{}\PYZhy{}\PYZhy{}}\PY{l+s+s2}{\PYZdq{}}\PY{p}{)}
\PY{n+nb}{print}\PY{p}{(}\PY{l+s+sa}{f}\PY{l+s+s2}{\PYZdq{}}\PY{l+s+si}{\PYZob{}}\PY{l+s+s1}{\PYZsq{}}\PY{l+s+s1}{n}\PY{l+s+s1}{\PYZsq{}}\PY{l+s+si}{:}\PY{l+s+s2}{\PYZgt{}8s}\PY{l+s+si}{\PYZcb{}}\PY{l+s+s2}{ }\PY{l+s+si}{\PYZob{}}\PY{l+s+s1}{\PYZsq{}}\PY{l+s+s1}{h}\PY{l+s+s1}{\PYZsq{}}\PY{l+s+si}{:}\PY{l+s+s2}{\PYZgt{}12s}\PY{l+s+si}{\PYZcb{}}\PY{l+s+s2}{ }\PY{l+s+si}{\PYZob{}}\PY{l+s+s1}{\PYZsq{}}\PY{l+s+s1}{Trapezoidal}\PY{l+s+s1}{\PYZsq{}}\PY{l+s+si}{:}\PY{l+s+s2}{\PYZgt{}18s}\PY{l+s+si}{\PYZcb{}}\PY{l+s+s2}{ }\PY{l+s+si}{\PYZob{}}\PY{l+s+s1}{\PYZsq{}}\PY{l+s+s1}{Error\PYZus{}T}\PY{l+s+s1}{\PYZsq{}}\PY{l+s+si}{:}\PY{l+s+s2}{\PYZgt{}12s}\PY{l+s+si}{\PYZcb{}}\PY{l+s+s2}{ }\PY{l+s+si}{\PYZob{}}\PY{l+s+s1}{\PYZsq{}}\PY{l+s+s1}{Simpson}\PY{l+s+s1}{\PYZsq{}}\PY{l+s+si}{:}\PY{l+s+s2}{\PYZgt{}18s}\PY{l+s+si}{\PYZcb{}}\PY{l+s+s2}{ }\PY{l+s+si}{\PYZob{}}\PY{l+s+s1}{\PYZsq{}}\PY{l+s+s1}{Error\PYZus{}S}\PY{l+s+s1}{\PYZsq{}}\PY{l+s+si}{:}\PY{l+s+s2}{\PYZgt{}12s}\PY{l+s+si}{\PYZcb{}}\PY{l+s+s2}{\PYZdq{}}\PY{p}{)}

\PY{k}{for} \PY{n}{n\PYZus{}val} \PY{o+ow}{in} \PY{n}{n\PYZus{}values}\PY{p}{:}
    \PY{n}{h\PYZus{}val} \PY{o}{=} \PY{p}{(}\PY{n}{b} \PY{o}{\PYZhy{}} \PY{n}{a}\PY{p}{)} \PY{o}{/} \PY{n}{n\PYZus{}val}
    \PY{n}{h\PYZus{}values}\PY{o}{.}\PY{n}{append}\PY{p}{(}\PY{n}{h\PYZus{}val}\PY{p}{)}

    \PY{n}{approx\PYZus{}t} \PY{o}{=} \PY{n}{composite\PYZus{}trapezoidal}\PY{p}{(}\PY{n}{f}\PY{p}{,} \PY{n}{a}\PY{p}{,} \PY{n}{b}\PY{p}{,} \PY{n}{n\PYZus{}val}\PY{p}{)}
    \PY{n}{error\PYZus{}t} \PY{o}{=} \PY{n+nb}{abs}\PY{p}{(}\PY{n}{approx\PYZus{}t} \PY{o}{\PYZhy{}} \PY{n}{exact\PYZus{}value}\PY{p}{)}
    \PY{n}{trapezoidal\PYZus{}errors}\PY{o}{.}\PY{n}{append}\PY{p}{(}\PY{n}{error\PYZus{}t}\PY{p}{)}

    \PY{n}{approx\PYZus{}s} \PY{o}{=} \PY{n}{composite\PYZus{}simpson}\PY{p}{(}\PY{n}{f}\PY{p}{,} \PY{n}{a}\PY{p}{,} \PY{n}{b}\PY{p}{,} \PY{n}{n\PYZus{}val}\PY{p}{)}
    \PY{n}{error\PYZus{}s} \PY{o}{=} \PY{n+nb}{abs}\PY{p}{(}\PY{n}{approx\PYZus{}s} \PY{o}{\PYZhy{}} \PY{n}{exact\PYZus{}value}\PY{p}{)}
    \PY{n}{simpson\PYZus{}errors}\PY{o}{.}\PY{n}{append}\PY{p}{(}\PY{n}{error\PYZus{}s}\PY{p}{)}
    
    \PY{k}{if} \PY{n}{n\PYZus{}val} \PY{o}{\PYZlt{}}\PY{o}{=} \PY{l+m+mi}{2}\PY{o}{*}\PY{o}{*}\PY{l+m+mi}{5} \PY{o+ow}{or} \PY{n}{n\PYZus{}val} \PY{o}{==} \PY{n}{n\PYZus{}values}\PY{p}{[}\PY{o}{\PYZhy{}}\PY{l+m+mi}{1}\PY{p}{]}\PY{p}{:}
        \PY{n+nb}{print}\PY{p}{(}\PY{l+s+sa}{f}\PY{l+s+s2}{\PYZdq{}}\PY{l+s+si}{\PYZob{}}\PY{n}{n\PYZus{}val}\PY{l+s+si}{:}\PY{l+s+s2}{8d}\PY{l+s+si}{\PYZcb{}}\PY{l+s+s2}{ }\PY{l+s+si}{\PYZob{}}\PY{n}{h\PYZus{}val}\PY{l+s+si}{:}\PY{l+s+s2}{12.2e}\PY{l+s+si}{\PYZcb{}}\PY{l+s+s2}{ }\PY{l+s+si}{\PYZob{}}\PY{n}{approx\PYZus{}t}\PY{l+s+si}{:}\PY{l+s+s2}{18.10f}\PY{l+s+si}{\PYZcb{}}\PY{l+s+s2}{ }\PY{l+s+si}{\PYZob{}}\PY{n}{error\PYZus{}t}\PY{l+s+si}{:}\PY{l+s+s2}{12.2e}\PY{l+s+si}{\PYZcb{}}\PY{l+s+s2}{ }\PY{l+s+si}{\PYZob{}}\PY{n}{approx\PYZus{}s}\PY{l+s+si}{:}\PY{l+s+s2}{18.10f}\PY{l+s+si}{\PYZcb{}}\PY{l+s+s2}{ }\PY{l+s+si}{\PYZob{}}\PY{n}{error\PYZus{}s}\PY{l+s+si}{:}\PY{l+s+s2}{12.2e}\PY{l+s+si}{\PYZcb{}}\PY{l+s+s2}{\PYZdq{}}\PY{p}{)}

\PY{c+c1}{\PYZsh{} (2) Romberg Integration \PYZhy{} MODIFIED FUNCTION}
\PY{k}{def} \PY{n+nf}{romberg\PYZus{}integration}\PY{p}{(}\PY{n}{func}\PY{p}{,} \PY{n}{a}\PY{p}{,} \PY{n}{b}\PY{p}{,} \PY{n}{max\PYZus{}iter}\PY{o}{=}\PY{l+m+mi}{10}\PY{p}{,} \PY{n}{tol}\PY{o}{=}\PY{l+m+mf}{1e\PYZhy{}12}\PY{p}{,} \PY{n}{exact\PYZus{}val\PYZus{}for\PYZus{}err\PYZus{}calc}\PY{o}{=}\PY{k+kc}{None}\PY{p}{)}\PY{p}{:}
    \PY{n}{R} \PY{o}{=} \PY{n}{np}\PY{o}{.}\PY{n}{zeros}\PY{p}{(}\PY{p}{(}\PY{n}{max\PYZus{}iter}\PY{p}{,} \PY{n}{max\PYZus{}iter}\PY{p}{)}\PY{p}{,} \PY{n}{dtype}\PY{o}{=}\PY{n+nb}{float}\PY{p}{)}
    
    \PY{n}{romberg\PYZus{}h\PYZus{}for\PYZus{}plot} \PY{o}{=} \PY{p}{[}\PY{p}{]}
    \PY{n}{romberg\PYZus{}diag\PYZus{}errors\PYZus{}for\PYZus{}plot} \PY{o}{=} \PY{p}{[}\PY{p}{]}

    \PY{n}{current\PYZus{}h} \PY{o}{=} \PY{n}{b} \PY{o}{\PYZhy{}} \PY{n}{a} 
    \PY{n}{R}\PY{p}{[}\PY{l+m+mi}{0}\PY{p}{,} \PY{l+m+mi}{0}\PY{p}{]} \PY{o}{=} \PY{p}{(}\PY{n}{func}\PY{p}{(}\PY{n}{a}\PY{p}{)} \PY{o}{+} \PY{n}{func}\PY{p}{(}\PY{n}{b}\PY{p}{)}\PY{p}{)} \PY{o}{*} \PY{n}{current\PYZus{}h} \PY{o}{/} \PY{l+m+mf}{2.0}
    
    \PY{k}{if} \PY{n}{exact\PYZus{}val\PYZus{}for\PYZus{}err\PYZus{}calc} \PY{o+ow}{is} \PY{o+ow}{not} \PY{k+kc}{None}\PY{p}{:}
        \PY{n}{romberg\PYZus{}h\PYZus{}for\PYZus{}plot}\PY{o}{.}\PY{n}{append}\PY{p}{(}\PY{n}{current\PYZus{}h}\PY{p}{)}
        \PY{c+c1}{\PYZsh{} Handle potential zero error for R[0,0] if it\PYZsq{}s exactly the exact\PYZus{}value (unlikely here)}
        \PY{n}{err\PYZus{}r00} \PY{o}{=} \PY{n+nb}{abs}\PY{p}{(}\PY{n}{R}\PY{p}{[}\PY{l+m+mi}{0}\PY{p}{,}\PY{l+m+mi}{0}\PY{p}{]} \PY{o}{\PYZhy{}} \PY{n}{exact\PYZus{}val\PYZus{}for\PYZus{}err\PYZus{}calc}\PY{p}{)}
        \PY{n}{romberg\PYZus{}diag\PYZus{}errors\PYZus{}for\PYZus{}plot}\PY{o}{.}\PY{n}{append}\PY{p}{(}\PY{n}{err\PYZus{}r00} \PY{k}{if} \PY{n}{err\PYZus{}r00} \PY{o}{\PYZgt{}} \PY{l+m+mi}{0} \PY{k}{else} \PY{l+m+mf}{1e\PYZhy{}16}\PY{p}{)} \PY{c+c1}{\PYZsh{} Avoid log(0)}

    \PY{n+nb}{print}\PY{p}{(}\PY{l+s+s2}{\PYZdq{}}\PY{l+s+se}{\PYZbs{}n}\PY{l+s+s2}{Part (2): Romberg Integration}\PY{l+s+s2}{\PYZdq{}}\PY{p}{)}
    \PY{n+nb}{print}\PY{p}{(}\PY{l+s+s2}{\PYZdq{}}\PY{l+s+s2}{\PYZhy{}\PYZhy{}\PYZhy{}\PYZhy{}\PYZhy{}\PYZhy{}\PYZhy{}\PYZhy{}\PYZhy{}\PYZhy{}\PYZhy{}\PYZhy{}\PYZhy{}\PYZhy{}\PYZhy{}\PYZhy{}\PYZhy{}\PYZhy{}\PYZhy{}\PYZhy{}\PYZhy{}\PYZhy{}\PYZhy{}\PYZhy{}\PYZhy{}\PYZhy{}\PYZhy{}\PYZhy{}\PYZhy{}}\PY{l+s+s2}{\PYZdq{}}\PY{p}{)}
    \PY{n+nb}{print}\PY{p}{(}\PY{l+s+s2}{\PYZdq{}}\PY{l+s+s2}{Romberg Table (R[i,j]):}\PY{l+s+s2}{\PYZdq{}}\PY{p}{)}
    \PY{n+nb}{print}\PY{p}{(}\PY{l+s+sa}{f}\PY{l+s+s2}{\PYZdq{}}\PY{l+s+s2}{R[0,0] = }\PY{l+s+si}{\PYZob{}}\PY{n}{R}\PY{p}{[}\PY{l+m+mi}{0}\PY{p}{,}\PY{l+m+mi}{0}\PY{p}{]}\PY{l+s+si}{:}\PY{l+s+s2}{.10f}\PY{l+s+si}{\PYZcb{}}\PY{l+s+s2}{\PYZdq{}}\PY{p}{)}

    \PY{k}{for} \PY{n}{i} \PY{o+ow}{in} \PY{n+nb}{range}\PY{p}{(}\PY{l+m+mi}{1}\PY{p}{,} \PY{n}{max\PYZus{}iter}\PY{p}{)}\PY{p}{:}
        \PY{n}{current\PYZus{}h} \PY{o}{/}\PY{o}{=} \PY{l+m+mf}{2.0} 
        \PY{n}{sum\PYZus{}f} \PY{o}{=} \PY{l+m+mi}{0}
        \PY{k}{for} \PY{n}{k\PYZus{}loop} \PY{o+ow}{in} \PY{n+nb}{range}\PY{p}{(}\PY{l+m+mi}{1}\PY{p}{,} \PY{l+m+mi}{2}\PY{o}{*}\PY{o}{*}\PY{n}{i}\PY{p}{,} \PY{l+m+mi}{2}\PY{p}{)}\PY{p}{:} 
            \PY{n}{sum\PYZus{}f} \PY{o}{+}\PY{o}{=} \PY{n}{func}\PY{p}{(}\PY{n}{a} \PY{o}{+} \PY{n}{k\PYZus{}loop} \PY{o}{*} \PY{n}{current\PYZus{}h}\PY{p}{)}
        \PY{n}{R}\PY{p}{[}\PY{n}{i}\PY{p}{,} \PY{l+m+mi}{0}\PY{p}{]} \PY{o}{=} \PY{l+m+mf}{0.5} \PY{o}{*} \PY{n}{R}\PY{p}{[}\PY{n}{i}\PY{o}{\PYZhy{}}\PY{l+m+mi}{1}\PY{p}{,} \PY{l+m+mi}{0}\PY{p}{]} \PY{o}{+} \PY{n}{sum\PYZus{}f} \PY{o}{*} \PY{n}{current\PYZus{}h}

        \PY{n}{row\PYZus{}str} \PY{o}{=} \PY{l+s+sa}{f}\PY{l+s+s2}{\PYZdq{}}\PY{l+s+s2}{R[}\PY{l+s+si}{\PYZob{}}\PY{n}{i}\PY{l+s+si}{\PYZcb{}}\PY{l+s+s2}{,0] = }\PY{l+s+si}{\PYZob{}}\PY{n}{R}\PY{p}{[}\PY{n}{i}\PY{p}{,}\PY{l+m+mi}{0}\PY{p}{]}\PY{l+s+si}{:}\PY{l+s+s2}{.10f}\PY{l+s+si}{\PYZcb{}}\PY{l+s+s2}{\PYZdq{}}
        \PY{k}{for} \PY{n}{j} \PY{o+ow}{in} \PY{n+nb}{range}\PY{p}{(}\PY{l+m+mi}{1}\PY{p}{,} \PY{n}{i} \PY{o}{+} \PY{l+m+mi}{1}\PY{p}{)}\PY{p}{:}
            \PY{n}{R}\PY{p}{[}\PY{n}{i}\PY{p}{,} \PY{n}{j}\PY{p}{]} \PY{o}{=} \PY{n}{R}\PY{p}{[}\PY{n}{i}\PY{p}{,} \PY{n}{j}\PY{o}{\PYZhy{}}\PY{l+m+mi}{1}\PY{p}{]} \PY{o}{+} \PY{p}{(}\PY{n}{R}\PY{p}{[}\PY{n}{i}\PY{p}{,} \PY{n}{j}\PY{o}{\PYZhy{}}\PY{l+m+mi}{1}\PY{p}{]} \PY{o}{\PYZhy{}} \PY{n}{R}\PY{p}{[}\PY{n}{i}\PY{o}{\PYZhy{}}\PY{l+m+mi}{1}\PY{p}{,} \PY{n}{j}\PY{o}{\PYZhy{}}\PY{l+m+mi}{1}\PY{p}{]}\PY{p}{)} \PY{o}{/} \PY{p}{(}\PY{l+m+mi}{4}\PY{o}{*}\PY{o}{*}\PY{n}{j} \PY{o}{\PYZhy{}} \PY{l+m+mi}{1}\PY{p}{)}
            \PY{n}{row\PYZus{}str} \PY{o}{+}\PY{o}{=} \PY{l+s+sa}{f}\PY{l+s+s2}{\PYZdq{}}\PY{l+s+s2}{  R[}\PY{l+s+si}{\PYZob{}}\PY{n}{i}\PY{l+s+si}{\PYZcb{}}\PY{l+s+s2}{,}\PY{l+s+si}{\PYZob{}}\PY{n}{j}\PY{l+s+si}{\PYZcb{}}\PY{l+s+s2}{] = }\PY{l+s+si}{\PYZob{}}\PY{n}{R}\PY{p}{[}\PY{n}{i}\PY{p}{,}\PY{n}{j}\PY{p}{]}\PY{l+s+si}{:}\PY{l+s+s2}{.10f}\PY{l+s+si}{\PYZcb{}}\PY{l+s+s2}{\PYZdq{}}
        \PY{n+nb}{print}\PY{p}{(}\PY{n}{row\PYZus{}str}\PY{p}{)}
        
        \PY{k}{if} \PY{n}{exact\PYZus{}val\PYZus{}for\PYZus{}err\PYZus{}calc} \PY{o+ow}{is} \PY{o+ow}{not} \PY{k+kc}{None}\PY{p}{:}
            \PY{n}{romberg\PYZus{}h\PYZus{}for\PYZus{}plot}\PY{o}{.}\PY{n}{append}\PY{p}{(}\PY{n}{current\PYZus{}h}\PY{p}{)}
            \PY{n}{err\PYZus{}rii} \PY{o}{=} \PY{n+nb}{abs}\PY{p}{(}\PY{n}{R}\PY{p}{[}\PY{n}{i}\PY{p}{,}\PY{n}{i}\PY{p}{]} \PY{o}{\PYZhy{}} \PY{n}{exact\PYZus{}val\PYZus{}for\PYZus{}err\PYZus{}calc}\PY{p}{)}
            \PY{n}{romberg\PYZus{}diag\PYZus{}errors\PYZus{}for\PYZus{}plot}\PY{o}{.}\PY{n}{append}\PY{p}{(}\PY{n}{err\PYZus{}rii} \PY{k}{if} \PY{n}{err\PYZus{}rii} \PY{o}{\PYZgt{}} \PY{l+m+mi}{0} \PY{k}{else} \PY{l+m+mf}{1e\PYZhy{}16}\PY{p}{)} \PY{c+c1}{\PYZsh{} Avoid log(0)}

        \PY{k}{if} \PY{n}{i} \PY{o}{\PYZgt{}} \PY{l+m+mi}{0} \PY{o+ow}{and} \PY{n+nb}{abs}\PY{p}{(}\PY{n}{R}\PY{p}{[}\PY{n}{i}\PY{p}{,} \PY{n}{i}\PY{p}{]} \PY{o}{\PYZhy{}} \PY{n}{R}\PY{p}{[}\PY{n}{i}\PY{o}{\PYZhy{}}\PY{l+m+mi}{1}\PY{p}{,} \PY{n}{i}\PY{o}{\PYZhy{}}\PY{l+m+mi}{1}\PY{p}{]}\PY{p}{)} \PY{o}{\PYZlt{}} \PY{n}{tol}\PY{p}{:}
            \PY{n+nb}{print}\PY{p}{(}\PY{l+s+sa}{f}\PY{l+s+s2}{\PYZdq{}}\PY{l+s+se}{\PYZbs{}n}\PY{l+s+s2}{Convergence reached at iteration }\PY{l+s+si}{\PYZob{}}\PY{n}{i}\PY{l+s+si}{\PYZcb{}}\PY{l+s+s2}{.}\PY{l+s+s2}{\PYZdq{}}\PY{p}{)}
            \PY{k}{return} \PY{n}{R}\PY{p}{[}\PY{n}{i}\PY{p}{,} \PY{n}{i}\PY{p}{]}\PY{p}{,} \PY{n}{R}\PY{p}{[}\PY{p}{:}\PY{n}{i}\PY{o}{+}\PY{l+m+mi}{1}\PY{p}{,} \PY{p}{:}\PY{n}{i}\PY{o}{+}\PY{l+m+mi}{1}\PY{p}{]}\PY{p}{,} \PY{n}{romberg\PYZus{}h\PYZus{}for\PYZus{}plot}\PY{p}{,} \PY{n}{romberg\PYZus{}diag\PYZus{}errors\PYZus{}for\PYZus{}plot}

    \PY{n+nb}{print}\PY{p}{(}\PY{l+s+sa}{f}\PY{l+s+s2}{\PYZdq{}}\PY{l+s+se}{\PYZbs{}n}\PY{l+s+s2}{Max iterations (}\PY{l+s+si}{\PYZob{}}\PY{n}{max\PYZus{}iter}\PY{l+s+si}{\PYZcb{}}\PY{l+s+s2}{) reached.}\PY{l+s+s2}{\PYZdq{}}\PY{p}{)}
    \PY{k}{return} \PY{n}{R}\PY{p}{[}\PY{n}{max\PYZus{}iter}\PY{o}{\PYZhy{}}\PY{l+m+mi}{1}\PY{p}{,} \PY{n}{max\PYZus{}iter}\PY{o}{\PYZhy{}}\PY{l+m+mi}{1}\PY{p}{]}\PY{p}{,} \PY{n}{R}\PY{p}{,} \PY{n}{romberg\PYZus{}h\PYZus{}for\PYZus{}plot}\PY{p}{,} \PY{n}{romberg\PYZus{}diag\PYZus{}errors\PYZus{}for\PYZus{}plot}

\PY{c+c1}{\PYZsh{} Call Romberg integration and get values for plotting}
\PY{n}{romberg\PYZus{}max\PYZus{}iter} \PY{o}{=} \PY{l+m+mi}{21}
\PY{n}{romberg\PYZus{}result}\PY{p}{,} \PY{n}{romberg\PYZus{}table}\PY{p}{,} \PY{n}{romberg\PYZus{}h\PYZus{}plot\PYZus{}vals}\PY{p}{,} \PY{n}{romberg\PYZus{}diag\PYZus{}err\PYZus{}vals} \PY{o}{=} \PY{n}{romberg\PYZus{}integration}\PY{p}{(}
    \PY{n}{f}\PY{p}{,} \PY{n}{a}\PY{p}{,} \PY{n}{b}\PY{p}{,} \PY{n}{max\PYZus{}iter}\PY{o}{=}\PY{n}{romberg\PYZus{}max\PYZus{}iter}\PY{p}{,} \PY{n}{tol}\PY{o}{=}\PY{l+m+mf}{1e\PYZhy{}10}\PY{p}{,} \PY{n}{exact\PYZus{}val\PYZus{}for\PYZus{}err\PYZus{}calc}\PY{o}{=}\PY{n}{exact\PYZus{}value}
\PY{p}{)}
\PY{n}{romberg\PYZus{}error} \PY{o}{=} \PY{n+nb}{abs}\PY{p}{(}\PY{n}{romberg\PYZus{}result} \PY{o}{\PYZhy{}} \PY{n}{exact\PYZus{}value}\PY{p}{)}


\PY{c+c1}{\PYZsh{} Plotting errors for ALL methods}
\PY{n}{plt}\PY{o}{.}\PY{n}{figure}\PY{p}{(}\PY{n}{figsize}\PY{o}{=}\PY{p}{(}\PY{l+m+mi}{12}\PY{p}{,} \PY{l+m+mi}{7}\PY{p}{)}\PY{p}{)} \PY{c+c1}{\PYZsh{} Adjusted figure size for better readability}
\PY{n}{plt}\PY{o}{.}\PY{n}{loglog}\PY{p}{(}\PY{n}{h\PYZus{}values}\PY{p}{,} \PY{n}{trapezoidal\PYZus{}errors}\PY{p}{,} \PY{l+s+s1}{\PYZsq{}}\PY{l+s+s1}{o\PYZhy{}}\PY{l+s+s1}{\PYZsq{}}\PY{p}{,} \PY{n}{label}\PY{o}{=}\PY{l+s+s1}{\PYZsq{}}\PY{l+s+s1}{复合梯形法误差}\PY{l+s+s1}{\PYZsq{}}\PY{p}{)}
\PY{n}{plt}\PY{o}{.}\PY{n}{loglog}\PY{p}{(}\PY{n}{h\PYZus{}values}\PY{p}{,} \PY{n}{simpson\PYZus{}errors}\PY{p}{,} \PY{l+s+s1}{\PYZsq{}}\PY{l+s+s1}{s\PYZhy{}}\PY{l+s+s1}{\PYZsq{}}\PY{p}{,} \PY{n}{label}\PY{o}{=}\PY{l+s+s1}{\PYZsq{}}\PY{l+s+s1}{复合辛普森法误差}\PY{l+s+s1}{\PYZsq{}}\PY{p}{)}

\PY{k}{if} \PY{n}{romberg\PYZus{}h\PYZus{}plot\PYZus{}vals} \PY{o+ow}{and} \PY{n}{romberg\PYZus{}diag\PYZus{}err\PYZus{}vals}\PY{p}{:}
    \PY{n}{plt}\PY{o}{.}\PY{n}{loglog}\PY{p}{(}\PY{n}{romberg\PYZus{}h\PYZus{}plot\PYZus{}vals}\PY{p}{,} \PY{n}{romberg\PYZus{}diag\PYZus{}err\PYZus{}vals}\PY{p}{,} \PY{l+s+s1}{\PYZsq{}}\PY{l+s+s1}{\PYZca{}\PYZhy{}}\PY{l+s+s1}{\PYZsq{}}\PY{p}{,} \PY{n}{label}\PY{o}{=}\PY{l+s+s1}{\PYZsq{}}\PY{l+s+s1}{龙贝格法误差 (R[i,i])}\PY{l+s+s1}{\PYZsq{}}\PY{p}{)}

\PY{c+c1}{\PYZsh{} Estimate order of convergence p from log\PYZhy{}log plot (slope)}
\PY{k}{if} \PY{n+nb}{len}\PY{p}{(}\PY{n}{h\PYZus{}values}\PY{p}{)} \PY{o}{\PYZgt{}}\PY{o}{=} \PY{l+m+mi}{5}\PY{p}{:}
    \PY{c+c1}{\PYZsh{} Trapezoidal}
    \PY{n}{valid\PYZus{}t\PYZus{}indices} \PY{o}{=} \PY{p}{[}\PY{n}{k} \PY{k}{for} \PY{n}{k}\PY{p}{,} \PY{n}{err} \PY{o+ow}{in} \PY{n+nb}{enumerate}\PY{p}{(}\PY{n}{trapezoidal\PYZus{}errors}\PY{p}{[}\PY{o}{\PYZhy{}}\PY{l+m+mi}{5}\PY{p}{:}\PY{p}{]}\PY{p}{)} \PY{k}{if} \PY{n}{err} \PY{o}{\PYZgt{}} \PY{l+m+mi}{0}\PY{p}{]}
    \PY{k}{if} \PY{n+nb}{len}\PY{p}{(}\PY{n}{valid\PYZus{}t\PYZus{}indices}\PY{p}{)} \PY{o}{\PYZgt{}}\PY{o}{=} \PY{l+m+mi}{2}\PY{p}{:}
        \PY{n}{log\PYZus{}h\PYZus{}t} \PY{o}{=} \PY{n}{np}\PY{o}{.}\PY{n}{log}\PY{p}{(}\PY{n}{np}\PY{o}{.}\PY{n}{array}\PY{p}{(}\PY{n}{h\PYZus{}values}\PY{p}{[}\PY{o}{\PYZhy{}}\PY{l+m+mi}{5}\PY{p}{:}\PY{p}{]}\PY{p}{)}\PY{p}{[}\PY{n}{valid\PYZus{}t\PYZus{}indices}\PY{p}{]}\PY{p}{)}
        \PY{n}{log\PYZus{}err\PYZus{}t} \PY{o}{=} \PY{n}{np}\PY{o}{.}\PY{n}{log}\PY{p}{(}\PY{n}{np}\PY{o}{.}\PY{n}{array}\PY{p}{(}\PY{n}{trapezoidal\PYZus{}errors}\PY{p}{[}\PY{o}{\PYZhy{}}\PY{l+m+mi}{5}\PY{p}{:}\PY{p}{]}\PY{p}{)}\PY{p}{[}\PY{n}{valid\PYZus{}t\PYZus{}indices}\PY{p}{]}\PY{p}{)}
        \PY{k}{if} \PY{n+nb}{len}\PY{p}{(}\PY{n}{log\PYZus{}h\PYZus{}t}\PY{p}{)} \PY{o}{\PYZgt{}}\PY{o}{=}\PY{l+m+mi}{2}\PY{p}{:} \PY{c+c1}{\PYZsh{} Need at least 2 points for polyfit}
            \PY{n}{slope\PYZus{}t}\PY{p}{,} \PY{n}{\PYZus{}} \PY{o}{=} \PY{n}{np}\PY{o}{.}\PY{n}{polyfit}\PY{p}{(}\PY{n}{log\PYZus{}h\PYZus{}t}\PY{p}{,} \PY{n}{log\PYZus{}err\PYZus{}t}\PY{p}{,} \PY{l+m+mi}{1}\PY{p}{)}
            \PY{n}{plt}\PY{o}{.}\PY{n}{text}\PY{p}{(}\PY{n}{h\PYZus{}values}\PY{p}{[}\PY{l+m+mi}{2}\PY{p}{]}\PY{p}{,} \PY{n}{trapezoidal\PYZus{}errors}\PY{p}{[}\PY{l+m+mi}{2}\PY{p}{]}\PY{o}{*}\PY{l+m+mf}{1.5}\PY{p}{,} \PY{l+s+sa}{f}\PY{l+s+s1}{\PYZsq{}}\PY{l+s+s1}{梯形法: p ≈ }\PY{l+s+si}{\PYZob{}}\PY{n}{slope\PYZus{}t}\PY{l+s+si}{:}\PY{l+s+s1}{.2f}\PY{l+s+si}{\PYZcb{}}\PY{l+s+s1}{\PYZsq{}}\PY{p}{,} \PY{n}{va}\PY{o}{=}\PY{l+s+s1}{\PYZsq{}}\PY{l+s+s1}{bottom}\PY{l+s+s1}{\PYZsq{}}\PY{p}{)}

    \PY{c+c1}{\PYZsh{} Simpson}
    \PY{n}{valid\PYZus{}s\PYZus{}indices} \PY{o}{=} \PY{p}{[}\PY{n}{k} \PY{k}{for} \PY{n}{k}\PY{p}{,} \PY{n}{err} \PY{o+ow}{in} \PY{n+nb}{enumerate}\PY{p}{(}\PY{n}{simpson\PYZus{}errors}\PY{p}{[}\PY{o}{\PYZhy{}}\PY{l+m+mi}{5}\PY{p}{:}\PY{p}{]}\PY{p}{)} \PY{k}{if} \PY{n}{err} \PY{o}{\PYZgt{}} \PY{l+m+mi}{0}\PY{p}{]}
    \PY{k}{if} \PY{n+nb}{len}\PY{p}{(}\PY{n}{valid\PYZus{}s\PYZus{}indices}\PY{p}{)} \PY{o}{\PYZgt{}}\PY{o}{=} \PY{l+m+mi}{2}\PY{p}{:}
        \PY{n}{log\PYZus{}h\PYZus{}s} \PY{o}{=} \PY{n}{np}\PY{o}{.}\PY{n}{log}\PY{p}{(}\PY{n}{np}\PY{o}{.}\PY{n}{array}\PY{p}{(}\PY{n}{h\PYZus{}values}\PY{p}{[}\PY{o}{\PYZhy{}}\PY{l+m+mi}{5}\PY{p}{:}\PY{p}{]}\PY{p}{)}\PY{p}{[}\PY{n}{valid\PYZus{}s\PYZus{}indices}\PY{p}{]}\PY{p}{)}
        \PY{n}{log\PYZus{}err\PYZus{}s} \PY{o}{=} \PY{n}{np}\PY{o}{.}\PY{n}{log}\PY{p}{(}\PY{n}{np}\PY{o}{.}\PY{n}{array}\PY{p}{(}\PY{n}{simpson\PYZus{}errors}\PY{p}{[}\PY{o}{\PYZhy{}}\PY{l+m+mi}{5}\PY{p}{:}\PY{p}{]}\PY{p}{)}\PY{p}{[}\PY{n}{valid\PYZus{}s\PYZus{}indices}\PY{p}{]}\PY{p}{)}
        \PY{k}{if} \PY{n+nb}{len}\PY{p}{(}\PY{n}{log\PYZus{}h\PYZus{}s}\PY{p}{)} \PY{o}{\PYZgt{}}\PY{o}{=} \PY{l+m+mi}{2}\PY{p}{:}
            \PY{n}{slope\PYZus{}s}\PY{p}{,} \PY{n}{\PYZus{}} \PY{o}{=} \PY{n}{np}\PY{o}{.}\PY{n}{polyfit}\PY{p}{(}\PY{n}{log\PYZus{}h\PYZus{}s}\PY{p}{,} \PY{n}{log\PYZus{}err\PYZus{}s}\PY{p}{,} \PY{l+m+mi}{1}\PY{p}{)}
            \PY{n}{plt}\PY{o}{.}\PY{n}{text}\PY{p}{(}\PY{n}{h\PYZus{}values}\PY{p}{[}\PY{l+m+mi}{2}\PY{p}{]}\PY{p}{,} \PY{n}{simpson\PYZus{}errors}\PY{p}{[}\PY{l+m+mi}{2}\PY{p}{]}\PY{o}{*}\PY{l+m+mf}{0.5}\PY{p}{,} \PY{l+s+sa}{f}\PY{l+s+s1}{\PYZsq{}}\PY{l+s+s1}{辛普森法: p ≈ }\PY{l+s+si}{\PYZob{}}\PY{n}{slope\PYZus{}s}\PY{l+s+si}{:}\PY{l+s+s1}{.2f}\PY{l+s+si}{\PYZcb{}}\PY{l+s+s1}{\PYZsq{}}\PY{p}{,} \PY{n}{va}\PY{o}{=}\PY{l+s+s1}{\PYZsq{}}\PY{l+s+s1}{top}\PY{l+s+s1}{\PYZsq{}}\PY{p}{)}

\PY{c+c1}{\PYZsh{} Slope for Romberg}
\PY{k}{if} \PY{n+nb}{len}\PY{p}{(}\PY{n}{romberg\PYZus{}h\PYZus{}plot\PYZus{}vals}\PY{p}{)} \PY{o}{\PYZgt{}}\PY{o}{=} \PY{l+m+mi}{2}\PY{p}{:} \PY{c+c1}{\PYZsh{} Check if there are enough points}
    \PY{c+c1}{\PYZsh{} Use all available points for Romberg if less than 5, or last 5 if more}
    \PY{n}{points\PYZus{}to\PYZus{}use\PYZus{}r} \PY{o}{=} \PY{n+nb}{min}\PY{p}{(}\PY{n+nb}{len}\PY{p}{(}\PY{n}{romberg\PYZus{}h\PYZus{}plot\PYZus{}vals}\PY{p}{)}\PY{p}{,} \PY{l+m+mi}{5}\PY{p}{)}
    \PY{k}{if} \PY{n}{points\PYZus{}to\PYZus{}use\PYZus{}r} \PY{o}{\PYZgt{}}\PY{o}{=}\PY{l+m+mi}{2}\PY{p}{:}
        \PY{n}{h\PYZus{}r\PYZus{}slope} \PY{o}{=} \PY{n}{np}\PY{o}{.}\PY{n}{array}\PY{p}{(}\PY{n}{romberg\PYZus{}h\PYZus{}plot\PYZus{}vals}\PY{p}{[}\PY{o}{\PYZhy{}}\PY{n}{points\PYZus{}to\PYZus{}use\PYZus{}r}\PY{p}{:}\PY{p}{]}\PY{p}{)}
        \PY{n}{err\PYZus{}r\PYZus{}slope} \PY{o}{=} \PY{n}{np}\PY{o}{.}\PY{n}{array}\PY{p}{(}\PY{n}{romberg\PYZus{}diag\PYZus{}err\PYZus{}vals}\PY{p}{[}\PY{o}{\PYZhy{}}\PY{n}{points\PYZus{}to\PYZus{}use\PYZus{}r}\PY{p}{:}\PY{p}{]}\PY{p}{)}
        
        \PY{n}{valid\PYZus{}r\PYZus{}indices\PYZus{}slope} \PY{o}{=} \PY{p}{[}\PY{n}{k} \PY{k}{for} \PY{n}{k}\PY{p}{,} \PY{n}{err} \PY{o+ow}{in} \PY{n+nb}{enumerate}\PY{p}{(}\PY{n}{err\PYZus{}r\PYZus{}slope}\PY{p}{)} \PY{k}{if} \PY{n}{err} \PY{o}{\PYZgt{}} \PY{l+m+mf}{1e\PYZhy{}15}\PY{p}{]} \PY{c+c1}{\PYZsh{} Check against small number}
        \PY{k}{if} \PY{n+nb}{len}\PY{p}{(}\PY{n}{valid\PYZus{}r\PYZus{}indices\PYZus{}slope}\PY{p}{)} \PY{o}{\PYZgt{}}\PY{o}{=} \PY{l+m+mi}{2}\PY{p}{:}
            \PY{n}{log\PYZus{}h\PYZus{}r} \PY{o}{=} \PY{n}{np}\PY{o}{.}\PY{n}{log}\PY{p}{(}\PY{n}{h\PYZus{}r\PYZus{}slope}\PY{p}{[}\PY{n}{valid\PYZus{}r\PYZus{}indices\PYZus{}slope}\PY{p}{]}\PY{p}{)}
            \PY{n}{log\PYZus{}err\PYZus{}r} \PY{o}{=} \PY{n}{np}\PY{o}{.}\PY{n}{log}\PY{p}{(}\PY{n}{err\PYZus{}r\PYZus{}slope}\PY{p}{[}\PY{n}{valid\PYZus{}r\PYZus{}indices\PYZus{}slope}\PY{p}{]}\PY{p}{)}
            \PY{n}{slope\PYZus{}r}\PY{p}{,} \PY{n}{\PYZus{}} \PY{o}{=} \PY{n}{np}\PY{o}{.}\PY{n}{polyfit}\PY{p}{(}\PY{n}{log\PYZus{}h\PYZus{}r}\PY{p}{,} \PY{n}{log\PYZus{}err\PYZus{}r}\PY{p}{,} \PY{l+m+mi}{1}\PY{p}{)}
            \PY{c+c1}{\PYZsh{} Adjust text position for Romberg slope}
            \PY{n}{text\PYZus{}idx\PYZus{}r} \PY{o}{=} \PY{n+nb}{len}\PY{p}{(}\PY{n}{romberg\PYZus{}h\PYZus{}plot\PYZus{}vals}\PY{p}{)} \PY{o}{/}\PY{o}{/} \PY{l+m+mi}{3}
            \PY{k}{if} \PY{n}{text\PYZus{}idx\PYZus{}r} \PY{o}{\PYZlt{}} \PY{n+nb}{len}\PY{p}{(}\PY{n}{romberg\PYZus{}h\PYZus{}plot\PYZus{}vals}\PY{p}{)} \PY{o+ow}{and} \PY{n}{romberg\PYZus{}diag\PYZus{}err\PYZus{}vals}\PY{p}{[}\PY{n}{text\PYZus{}idx\PYZus{}r}\PY{p}{]} \PY{o}{\PYZgt{}} \PY{l+m+mi}{0} \PY{p}{:}
                 \PY{n}{plt}\PY{o}{.}\PY{n}{text}\PY{p}{(}\PY{n}{romberg\PYZus{}h\PYZus{}plot\PYZus{}vals}\PY{p}{[}\PY{n}{text\PYZus{}idx\PYZus{}r}\PY{p}{]}\PY{p}{,} \PY{n}{romberg\PYZus{}diag\PYZus{}err\PYZus{}vals}\PY{p}{[}\PY{n}{text\PYZus{}idx\PYZus{}r}\PY{p}{]}\PY{o}{*}\PY{l+m+mf}{0.2} \PY{p}{,} \PY{l+s+sa}{f}\PY{l+s+s1}{\PYZsq{}}\PY{l+s+s1}{龙贝格法: p ≈ }\PY{l+s+si}{\PYZob{}}\PY{n}{slope\PYZus{}r}\PY{l+s+si}{:}\PY{l+s+s1}{.2f}\PY{l+s+si}{\PYZcb{}}\PY{l+s+s1}{\PYZsq{}}\PY{p}{,} \PY{n}{va}\PY{o}{=}\PY{l+s+s1}{\PYZsq{}}\PY{l+s+s1}{top}\PY{l+s+s1}{\PYZsq{}}\PY{p}{,} \PY{n}{color}\PY{o}{=}\PY{l+s+s1}{\PYZsq{}}\PY{l+s+s1}{green}\PY{l+s+s1}{\PYZsq{}}\PY{p}{)}


\PY{n}{plt}\PY{o}{.}\PY{n}{xlabel}\PY{p}{(}\PY{l+s+s1}{\PYZsq{}}\PY{l+s+s1}{步长 h (Step size h)}\PY{l+s+s1}{\PYZsq{}}\PY{p}{)}
\PY{n}{plt}\PY{o}{.}\PY{n}{ylabel}\PY{p}{(}\PY{l+s+s1}{\PYZsq{}}\PY{l+s+s1}{绝对误差 (Absolute Error)}\PY{l+s+s1}{\PYZsq{}}\PY{p}{)}
\PY{n}{plt}\PY{o}{.}\PY{n}{title}\PY{p}{(}\PY{l+s+s1}{\PYZsq{}}\PY{l+s+s1}{不同数值积分方法的误差与步长关系 (Error vs. Step Size)}\PY{l+s+s1}{\PYZsq{}}\PY{p}{)}
\PY{n}{plt}\PY{o}{.}\PY{n}{legend}\PY{p}{(}\PY{p}{)}
\PY{n}{plt}\PY{o}{.}\PY{n}{grid}\PY{p}{(}\PY{k+kc}{True}\PY{p}{,} \PY{n}{which}\PY{o}{=}\PY{l+s+s2}{\PYZdq{}}\PY{l+s+s2}{both}\PY{l+s+s2}{\PYZdq{}}\PY{p}{,} \PY{n}{ls}\PY{o}{=}\PY{l+s+s2}{\PYZdq{}}\PY{l+s+s2}{\PYZhy{}}\PY{l+s+s2}{\PYZdq{}}\PY{p}{)}
\PY{n}{plt}\PY{o}{.}\PY{n}{gca}\PY{p}{(}\PY{p}{)}\PY{o}{.}\PY{n}{invert\PYZus{}xaxis}\PY{p}{(}\PY{p}{)} 
\PY{n}{plt}\PY{o}{.}\PY{n}{show}\PY{p}{(}\PY{p}{)}

\PY{c+c1}{\PYZsh{} Print Romberg results (as before)}
\PY{n+nb}{print}\PY{p}{(}\PY{l+s+sa}{f}\PY{l+s+s2}{\PYZdq{}}\PY{l+s+se}{\PYZbs{}n}\PY{l+s+s2}{Romberg Result: }\PY{l+s+si}{\PYZob{}}\PY{n}{romberg\PYZus{}result}\PY{l+s+si}{:}\PY{l+s+s2}{.10f}\PY{l+s+si}{\PYZcb{}}\PY{l+s+s2}{\PYZdq{}}\PY{p}{)}
\PY{n+nb}{print}\PY{p}{(}\PY{l+s+sa}{f}\PY{l+s+s2}{\PYZdq{}}\PY{l+s+s2}{Exact Value:    }\PY{l+s+si}{\PYZob{}}\PY{n}{exact\PYZus{}value}\PY{l+s+si}{:}\PY{l+s+s2}{.10f}\PY{l+s+si}{\PYZcb{}}\PY{l+s+s2}{\PYZdq{}}\PY{p}{)}
\PY{n+nb}{print}\PY{p}{(}\PY{l+s+sa}{f}\PY{l+s+s2}{\PYZdq{}}\PY{l+s+s2}{Romberg Error:  }\PY{l+s+si}{\PYZob{}}\PY{n}{romberg\PYZus{}error}\PY{l+s+si}{:}\PY{l+s+s2}{.2e}\PY{l+s+si}{\PYZcb{}}\PY{l+s+s2}{\PYZdq{}}\PY{p}{)}

\PY{c+c1}{\PYZsh{} Comparison (as before)}
\PY{k}{if} \PY{l+m+mi}{128} \PY{o+ow}{in} \PY{n}{n\PYZus{}values}\PY{p}{:} \PY{c+c1}{\PYZsh{} n=128 corresponds to h = (1\PYZhy{}0)/128}
    \PY{n}{idx} \PY{o}{=} \PY{n}{n\PYZus{}values}\PY{o}{.}\PY{n}{index}\PY{p}{(}\PY{l+m+mi}{128}\PY{p}{)} \PY{c+c1}{\PYZsh{} For Trapezoidal/Simpson}
    \PY{c+c1}{\PYZsh{} For Romberg, R[7,7] uses h up to (b\PYZhy{}a)/2\PYZca{}7 = (b\PYZhy{}a)/128}
    \PY{n+nb}{print}\PY{p}{(}\PY{l+s+sa}{f}\PY{l+s+s2}{\PYZdq{}}\PY{l+s+se}{\PYZbs{}n}\PY{l+s+s2}{与 n=128 (h=}\PY{l+s+si}{\PYZob{}}\PY{p}{(}\PY{l+m+mf}{1.0}\PY{o}{/}\PY{l+m+mi}{128}\PY{p}{)}\PY{l+s+si}{:}\PY{l+s+s2}{.2e}\PY{l+s+si}{\PYZcb{}}\PY{l+s+s2}{) 附近结果比较:}\PY{l+s+s2}{\PYZdq{}}\PY{p}{)}
    \PY{n+nb}{print}\PY{p}{(}\PY{l+s+sa}{f}\PY{l+s+s2}{\PYZdq{}}\PY{l+s+s2}{梯形法误差 (n=128): }\PY{l+s+si}{\PYZob{}}\PY{n}{trapezoidal\PYZus{}errors}\PY{p}{[}\PY{n}{idx}\PY{p}{]}\PY{l+s+si}{:}\PY{l+s+s2}{.2e}\PY{l+s+si}{\PYZcb{}}\PY{l+s+s2}{\PYZdq{}}\PY{p}{)}
    \PY{n+nb}{print}\PY{p}{(}\PY{l+s+sa}{f}\PY{l+s+s2}{\PYZdq{}}\PY{l+s+s2}{辛普森法误差 (n=128): }\PY{l+s+si}{\PYZob{}}\PY{n}{simpson\PYZus{}errors}\PY{p}{[}\PY{n}{idx}\PY{p}{]}\PY{l+s+si}{:}\PY{l+s+s2}{.2e}\PY{l+s+si}{\PYZcb{}}\PY{l+s+s2}{\PYZdq{}}\PY{p}{)}
    \PY{c+c1}{\PYZsh{} Find Romberg error for h closest to 1/128 if available}
    \PY{n}{h\PYZus{}target\PYZus{}romberg} \PY{o}{=} \PY{p}{(}\PY{n}{b}\PY{o}{\PYZhy{}}\PY{n}{a}\PY{p}{)}\PY{o}{/}\PY{l+m+mi}{128}
    \PY{k}{if} \PY{n}{romberg\PYZus{}h\PYZus{}plot\PYZus{}vals}\PY{p}{:}
        \PY{n}{closest\PYZus{}h\PYZus{}idx\PYZus{}r} \PY{o}{=} \PY{n}{np}\PY{o}{.}\PY{n}{argmin}\PY{p}{(}\PY{n}{np}\PY{o}{.}\PY{n}{abs}\PY{p}{(}\PY{n}{np}\PY{o}{.}\PY{n}{array}\PY{p}{(}\PY{n}{romberg\PYZus{}h\PYZus{}plot\PYZus{}vals}\PY{p}{)} \PY{o}{\PYZhy{}} \PY{n}{h\PYZus{}target\PYZus{}romberg}\PY{p}{)}\PY{p}{)}
        \PY{k}{if} \PY{n}{np}\PY{o}{.}\PY{n}{isclose}\PY{p}{(}\PY{n}{romberg\PYZus{}h\PYZus{}plot\PYZus{}vals}\PY{p}{[}\PY{n}{closest\PYZus{}h\PYZus{}idx\PYZus{}r}\PY{p}{]}\PY{p}{,} \PY{n}{h\PYZus{}target\PYZus{}romberg}\PY{p}{)}\PY{p}{:}
             \PY{n+nb}{print}\PY{p}{(}\PY{l+s+sa}{f}\PY{l+s+s2}{\PYZdq{}}\PY{l+s+s2}{龙贝格法 R[}\PY{l+s+si}{\PYZob{}}\PY{n}{closest\PYZus{}h\PYZus{}idx\PYZus{}r}\PY{l+s+si}{\PYZcb{}}\PY{l+s+s2}{,}\PY{l+s+si}{\PYZob{}}\PY{n}{closest\PYZus{}h\PYZus{}idx\PYZus{}r}\PY{l+s+si}{\PYZcb{}}\PY{l+s+s2}{] 误差 (h=}\PY{l+s+si}{\PYZob{}}\PY{n}{romberg\PYZus{}h\PYZus{}plot\PYZus{}vals}\PY{p}{[}\PY{n}{closest\PYZus{}h\PYZus{}idx\PYZus{}r}\PY{p}{]}\PY{l+s+si}{:}\PY{l+s+s2}{.2e}\PY{l+s+si}{\PYZcb{}}\PY{l+s+s2}{): }\PY{l+s+si}{\PYZob{}}\PY{n}{romberg\PYZus{}diag\PYZus{}err\PYZus{}vals}\PY{p}{[}\PY{n}{closest\PYZus{}h\PYZus{}idx\PYZus{}r}\PY{p}{]}\PY{l+s+si}{:}\PY{l+s+s2}{.2e}\PY{l+s+si}{\PYZcb{}}\PY{l+s+s2}{\PYZdq{}}\PY{p}{)}
        \PY{k}{else}\PY{p}{:} \PY{c+c1}{\PYZsh{} Fallback to the R[7,7] error if max\PYZus{}iter was 8}
            \PY{k}{if} \PY{n}{romberg\PYZus{}max\PYZus{}iter} \PY{o}{\PYZhy{}}\PY{l+m+mi}{1} \PY{o}{==} \PY{l+m+mi}{7} \PY{o+ow}{and} \PY{n+nb}{len}\PY{p}{(}\PY{n}{romberg\PYZus{}diag\PYZus{}err\PYZus{}vals}\PY{p}{)} \PY{o}{==} \PY{n}{romberg\PYZus{}max\PYZus{}iter}\PY{p}{:}
                 \PY{n+nb}{print}\PY{p}{(}\PY{l+s+sa}{f}\PY{l+s+s2}{\PYZdq{}}\PY{l+s+s2}{龙贝格法 R[7,7] 误差 (h=}\PY{l+s+si}{\PYZob{}}\PY{p}{(}\PY{n}{b}\PY{o}{\PYZhy{}}\PY{n}{a}\PY{p}{)}\PY{o}{/}\PY{l+m+mi}{2}\PY{o}{*}\PY{o}{*}\PY{l+m+mi}{7}\PY{l+s+si}{:}\PY{l+s+s2}{.2e}\PY{l+s+si}{\PYZcb{}}\PY{l+s+s2}{): }\PY{l+s+si}{\PYZob{}}\PY{n}{romberg\PYZus{}diag\PYZus{}err\PYZus{}vals}\PY{p}{[}\PY{l+m+mi}{7}\PY{p}{]}\PY{l+s+si}{:}\PY{l+s+s2}{.2e}\PY{l+s+si}{\PYZcb{}}\PY{l+s+s2}{\PYZdq{}}\PY{p}{)}
            \PY{k}{else}\PY{p}{:}
                 \PY{n+nb}{print}\PY{p}{(}\PY{l+s+sa}{f}\PY{l+s+s2}{\PYZdq{}}\PY{l+s+s2}{龙贝格法最终误差: }\PY{l+s+si}{\PYZob{}}\PY{n}{romberg\PYZus{}error}\PY{l+s+si}{:}\PY{l+s+s2}{.2e}\PY{l+s+si}{\PYZcb{}}\PY{l+s+s2}{\PYZdq{}}\PY{p}{)} \PY{c+c1}{\PYZsh{} Overall final Romberg error}
    \PY{k}{else}\PY{p}{:}
        \PY{n+nb}{print}\PY{p}{(}\PY{l+s+sa}{f}\PY{l+s+s2}{\PYZdq{}}\PY{l+s+s2}{龙贝格法最终误差: }\PY{l+s+si}{\PYZob{}}\PY{n}{romberg\PYZus{}error}\PY{l+s+si}{:}\PY{l+s+s2}{.2e}\PY{l+s+si}{\PYZcb{}}\PY{l+s+s2}{\PYZdq{}}\PY{p}{)}
\end{Verbatim}
\end{tcolorbox}

    \begin{Verbatim}[commandchars=\\\{\}]
Exact value: -0.4444444444444444

Part (1): Composite Trapezoidal and Simpson's Rule Analysis
---------------------------------------------------------
       n            h        Trapezoidal      Error\_T            Simpson
Error\_S
       4     2.50e-01      -0.3581040588     8.63e-02      -0.3957838998
4.87e-02
       8     1.25e-01      -0.4080900395     3.64e-02      -0.4247520331
1.97e-02
      16     6.25e-02      -0.4294745848     1.50e-02      -0.4366027666
7.84e-03
      32     3.12e-02      -0.4383894861     6.05e-03      -0.4413611198
3.08e-03
 1048576     9.54e-07      -0.4444444414     3.02e-09      -0.4444444431
1.31e-09

Part (2): Romberg Integration
-----------------------------
Romberg Table (R[i,j]):
R[0,0] = 0.0000000000
R[1,0] = -0.2450645359  R[1,1] = -0.3267527145
R[2,0] = -0.3581040588  R[2,1] = -0.3957838998  R[2,2] = -0.4003859788
R[3,0] = -0.4080900395  R[3,1] = -0.4247520331  R[3,2] = -0.4266832420  R[3,3] =
-0.4271006589
R[4,0] = -0.4294745848  R[4,1] = -0.4366027666  R[4,2] = -0.4373928155  R[4,3] =
-0.4375628088  R[4,4] = -0.4376038368
R[5,0] = -0.4383894861  R[5,1] = -0.4413611198  R[5,2] = -0.4416783434  R[5,3] =
-0.4417463676  R[5,4] = -0.4417627737  R[5,5] = -0.4417668392
R[6,0] = -0.4420306837  R[6,1] = -0.4432444162  R[6,2] = -0.4433699693  R[6,3] =
-0.4433968205  R[6,4] = -0.4434032929  R[6,5] = -0.4434048965  R[6,6] =
-0.4434052965
R[7,0] = -0.4434936549  R[7,1] = -0.4439813120  R[7,2] = -0.4440304384  R[7,3] =
-0.4440409220  R[7,4] = -0.4440434479  R[7,5] = -0.4440440737  R[7,6] =
-0.4440442298  R[7,7] = -0.4440442688
R[8,0] = -0.4440736367  R[8,1] = -0.4442669639  R[8,2] = -0.4442860074  R[8,3] =
-0.4442900640  R[8,4] = -0.4442910410  R[8,5] = -0.4442912831  R[8,6] =
-0.4442913434  R[8,7] = -0.4442913585  R[8,8] = -0.4442913623
R[9,0] = -0.4443010379  R[9,1] = -0.4443768383  R[9,2] = -0.4443841633  R[9,3] =
-0.4443857213  R[9,4] = -0.4443860964  R[9,5] = -0.4443861893  R[9,6] =
-0.4443862125  R[9,7] = -0.4443862183  R[9,8] = -0.4443862198  R[9,9] =
-0.4443862201
R[10,0] = -0.4443893780  R[10,1] = -0.4444188248  R[10,2] = -0.4444216238
R[10,3] = -0.4444222185  R[10,4] = -0.4444223616  R[10,5] = -0.4444223970
R[10,6] = -0.4444224059  R[10,7] = -0.4444224081  R[10,8] = -0.4444224086
R[10,9] = -0.4444224088  R[10,10] = -0.4444224088
R[11,0] = -0.4444234290  R[11,1] = -0.4444347794  R[11,2] = -0.4444358430
R[11,3] = -0.4444360687  R[11,4] = -0.4444361230  R[11,5] = -0.4444361365
R[11,6] = -0.4444361398  R[11,7] = -0.4444361407  R[11,8] = -0.4444361409
R[11,9] = -0.4444361409  R[11,10] = -0.4444361409  R[11,11] = -0.4444361409
R[12,0] = -0.4444364667  R[12,1] = -0.4444408127  R[12,2] = -0.4444412149
R[12,3] = -0.4444413001  R[12,4] = -0.4444413207  R[12,5] = -0.4444413257
R[12,6] = -0.4444413270  R[12,7] = -0.4444413273  R[12,8] = -0.4444413274
R[12,9] = -0.4444413274  R[12,10] = -0.4444413274  R[12,11] = -0.4444413274
R[12,12] = -0.4444413274
R[13,0] = -0.4444414301  R[13,1] = -0.4444430845  R[13,2] = -0.4444432360
R[13,3] = -0.4444432681  R[13,4] = -0.4444432758  R[13,5] = -0.4444432777
R[13,6] = -0.4444432782  R[13,7] = -0.4444432783  R[13,8] = -0.4444432783
R[13,9] = -0.4444432783  R[13,10] = -0.4444432783  R[13,11] = -0.4444432783
R[13,12] = -0.4444432783  R[13,13] = -0.4444432783
R[14,0] = -0.4444433101  R[14,1] = -0.4444439368  R[14,2] = -0.4444439936
R[14,3] = -0.4444440057  R[14,4] = -0.4444440085  R[14,5] = -0.4444440093
R[14,6] = -0.4444440094  R[14,7] = -0.4444440095  R[14,8] = -0.4444440095
R[14,9] = -0.4444440095  R[14,10] = -0.4444440095  R[14,11] = -0.4444440095
R[14,12] = -0.4444440095  R[14,13] = -0.4444440095  R[14,14] = -0.4444440095
R[15,0] = -0.4444440191  R[15,1] = -0.4444442555  R[15,2] = -0.4444442767
R[15,3] = -0.4444442812  R[15,4] = -0.4444442823  R[15,5] = -0.4444442826
R[15,6] = -0.4444442826  R[15,7] = -0.4444442827  R[15,8] = -0.4444442827
R[15,9] = -0.4444442827  R[15,10] = -0.4444442827  R[15,11] = -0.4444442827
R[15,12] = -0.4444442827  R[15,13] = -0.4444442827  R[15,14] = -0.4444442827
R[15,15] = -0.4444442827
R[16,0] = -0.4444442855  R[16,1] = -0.4444443743  R[16,2] = -0.4444443822
R[16,3] = -0.4444443839  R[16,4] = -0.4444443843  R[16,5] = -0.4444443844
R[16,6] = -0.4444443844  R[16,7] = -0.4444443844  R[16,8] = -0.4444443844
R[16,9] = -0.4444443844  R[16,10] = -0.4444443844  R[16,11] = -0.4444443844
R[16,12] = -0.4444443844  R[16,13] = -0.4444443844  R[16,14] = -0.4444443844
R[16,15] = -0.4444443844  R[16,16] = -0.4444443844
R[17,0] = -0.4444443852  R[17,1] = -0.4444444185  R[17,2] = -0.4444444214
R[17,3] = -0.4444444220  R[17,4] = -0.4444444222  R[17,5] = -0.4444444222
R[17,6] = -0.4444444222  R[17,7] = -0.4444444222  R[17,8] = -0.4444444222
R[17,9] = -0.4444444222  R[17,10] = -0.4444444222  R[17,11] = -0.4444444222
R[17,12] = -0.4444444222  R[17,13] = -0.4444444222  R[17,14] = -0.4444444222
R[17,15] = -0.4444444222  R[17,16] = -0.4444444222  R[17,17] = -0.4444444222
R[18,0] = -0.4444444224  R[18,1] = -0.4444444348  R[18,2] = -0.4444444359
R[18,3] = -0.4444444362  R[18,4] = -0.4444444362  R[18,5] = -0.4444444362
R[18,6] = -0.4444444362  R[18,7] = -0.4444444362  R[18,8] = -0.4444444362
R[18,9] = -0.4444444362  R[18,10] = -0.4444444362  R[18,11] = -0.4444444362
R[18,12] = -0.4444444362  R[18,13] = -0.4444444362  R[18,14] = -0.4444444362
R[18,15] = -0.4444444362  R[18,16] = -0.4444444362  R[18,17] = -0.4444444362
R[18,18] = -0.4444444362
R[19,0] = -0.4444444363  R[19,1] = -0.4444444409  R[19,2] = -0.4444444413
R[19,3] = -0.4444444414  R[19,4] = -0.4444444414  R[19,5] = -0.4444444414
R[19,6] = -0.4444444414  R[19,7] = -0.4444444414  R[19,8] = -0.4444444414
R[19,9] = -0.4444444414  R[19,10] = -0.4444444414  R[19,11] = -0.4444444414
R[19,12] = -0.4444444414  R[19,13] = -0.4444444414  R[19,14] = -0.4444444414
R[19,15] = -0.4444444414  R[19,16] = -0.4444444414  R[19,17] = -0.4444444414
R[19,18] = -0.4444444414  R[19,19] = -0.4444444414
R[20,0] = -0.4444444414  R[20,1] = -0.4444444431  R[20,2] = -0.4444444433
R[20,3] = -0.4444444433  R[20,4] = -0.4444444433  R[20,5] = -0.4444444433
R[20,6] = -0.4444444433  R[20,7] = -0.4444444433  R[20,8] = -0.4444444433
R[20,9] = -0.4444444433  R[20,10] = -0.4444444433  R[20,11] = -0.4444444433
R[20,12] = -0.4444444433  R[20,13] = -0.4444444433  R[20,14] = -0.4444444433
R[20,15] = -0.4444444433  R[20,16] = -0.4444444433  R[20,17] = -0.4444444433
R[20,18] = -0.4444444433  R[20,19] = -0.4444444433  R[20,20] = -0.4444444433

Max iterations (21) reached.
    \end{Verbatim}

    \begin{center}
    \adjustimage{max size={0.9\linewidth}{0.9\paperheight}}{output_2_1.png}
    \end{center}
    { \hspace*{\fill} \\}
    
    \begin{Verbatim}[commandchars=\\\{\}]

Romberg Result: -0.4444444433
Exact Value:    -0.4444444444
Romberg Error:  1.12e-09

与 n=128 (h=7.81e-03) 附近结果比较:
梯形法误差 (n=128): 9.51e-04
辛普森法误差 (n=128): 4.63e-04
龙贝格法 R[7,7] 误差 (h=7.81e-03): 4.00e-04
    \end{Verbatim}

    \subsubsection{复合梯形及复合辛普森求积计算结果分析}\label{ux590dux5408ux68afux5f62ux53caux590dux5408ux8f9bux666eux68eeux6c42ux79efux8ba1ux7b97ux7ed3ux679cux5206ux6790}

从python计算结果中,我们可以看到:

\begin{longtable}[]{@{}
  >{\raggedright\arraybackslash}p{(\linewidth - 10\tabcolsep) * \real{0.0928}}
  >{\raggedright\arraybackslash}p{(\linewidth - 10\tabcolsep) * \real{0.1031}}
  >{\raggedright\arraybackslash}p{(\linewidth - 10\tabcolsep) * \real{0.1649}}
  >{\raggedright\arraybackslash}p{(\linewidth - 10\tabcolsep) * \real{0.2268}}
  >{\raggedright\arraybackslash}p{(\linewidth - 10\tabcolsep) * \real{0.1649}}
  >{\raggedright\arraybackslash}p{(\linewidth - 10\tabcolsep) * \real{0.2474}}@{}}
\toprule\noalign{}
\begin{minipage}[b]{\linewidth}\raggedright
n
\end{minipage} & \begin{minipage}[b]{\linewidth}\raggedright
h
\end{minipage} & \begin{minipage}[b]{\linewidth}\raggedright
梯形法近似值
\end{minipage} & \begin{minipage}[b]{\linewidth}\raggedright
梯形法误差 (Error\_T)
\end{minipage} & \begin{minipage}[b]{\linewidth}\raggedright
辛普森法近似值
\end{minipage} & \begin{minipage}[b]{\linewidth}\raggedright
辛普森法误差 (Error\_S)
\end{minipage} \\
\midrule\noalign{}
\endhead
\bottomrule\noalign{}
\endlastfoot
4 & 2.50e-01 & -0.3581040588 & 8.63e-02 & -0.3957838998 & 4.87e-02 \\
8 & 1.25e-01 & -0.4080900395 & 3.64e-02 & -0.4247520331 & 1.97e-02 \\
16 & 6.25e-02 & -0.4294745848 & 1.50e-02 & -0.4366027666 & 7.84e-03 \\
32 & 3.12e-02 & -0.4383894861 & 6.05e-03 & -0.4413611198 & 3.08e-03 \\
\ldots{} & \ldots{} & \ldots{} & \ldots{} & \ldots{} & \ldots{} \\
1048576 & 9.54e-07 & -0.4444444414 & 3.02e-09 & -0.4444444431 &
1.31e-09 \\
\end{longtable}

\textbf{误差分析:}

\begin{enumerate}
\def\labelenumi{\arabic{enumi}.}
\item
  \textbf{误差随 \(h\) 的减小而减小}:
  对于复合梯形法和复合辛普森法,当步长 \(h\) 减小(即子区间数 \(n\)
  增大)时,计算得到的积分近似值越来越接近精确值
  \(-4/9\),相应的绝对误差 (Error\_T 和 Error\_S) 也随之减小。
\item
  \textbf{精度比较}:

  \begin{itemize}
  \tightlist
  \item
    在相同的步长 \(h\) (或相同的 \(n\))
    下,复合辛普森法的误差通常小于复合梯形法的误差。例如,当 \(n=32\)
    时,辛普森法的误差 (\texttt{3.08e-03}) 约为梯形法误差
    (\texttt{6.05e-03}) 的一半。
  \end{itemize}
\item
  \textbf{是否存在一个最小的 \(h\),使得精度不能再被改善?}

  \begin{itemize}
  \tightlist
  \item
    当 \(n\) 增大到 \(1048576\) (\(h \approx 9.54 \times 10^{-7}\))
    时,梯形法和辛普森法的误差分别减小到了 \texttt{3.02e-09} 和
    \texttt{1.31e-09}。这表明在测试的 \(h\) 范围内,精度仍在持续改善。
  \item
    当 \(h\)
    过小时,舍入误差的累积可能会开始主导总误差,导致精度不再提高甚至下降。从计算结果看,对于
    \(n\) 高达 \(10^6\)
    的情况,截断误差的减小仍然是主要的,未达到机器运算的精度上限。在实际计算中,收到机器计算精度的限制,会存在一个最小的
    \(h\) 使得精度不再发山。
  \end{itemize}
\end{enumerate}

\subsubsection{龙贝格求积结果分析}\label{ux9f99ux8d1dux683cux6c42ux79efux7ed3ux679cux5206ux6790}

\textbf{龙贝格表 (R{[}i,j{]}) 分析:}

\begin{verbatim}
R[0,0] = 0.0000000000
R[1,0] = -0.2450645359  R[1,1] = -0.3267527145
R[2,0] = -0.3581040588  R[2,1] = -0.3957838998  R[2,2] = -0.4003859788
R[3,0] = -0.4080900395  R[3,1] = -0.4247520331  R[3,2] = -0.4266832420  R[3,3] = -0.4271006589
R[4,0] = -0.4294745848  R[4,1] = -0.4366027666  R[4,2] = -0.4373928155  R[4,3] = -0.4375628088  R[4,4] = -0.4376038368
R[5,0] = -0.4383894861  R[5,1] = -0.4413611198  R[5,2] = -0.4416783434  R[5,3] = -0.4417463676  R[5,4] = -0.4417627737  R[5,5] = -0.4417668392
R[6,0] = -0.4420306837  R[6,1] = -0.4432444162  R[6,2] = -0.4433699693  R[6,3] = -0.4433968205  R[6,4] = -0.4434032929  R[6,5] = -0.4434048965  R[6,6] = -0.4434052965
R[7,0] = -0.4434936549  R[7,1] = -0.4439813120  R[7,2] = -0.4440304384  R[7,3] = -0.4440409220  R[7,4] = -0.4440434479  R[7,5] = -0.4440440737  R[7,6] = -0.4440442298  R[7,7] = -0.4440442688
\end{verbatim}

\begin{enumerate}
\def\labelenumi{\arabic{enumi}.}
\tightlist
\item
  \textbf{第一列 \(R[i,0]\)}:

  \begin{itemize}
  \tightlist
  \item
    \(R[0,0] = 0.0\) 是因为 \(f(0)=0\) 且
    \(f(1)=\sqrt{1}\ln(1)=0\),所以初始梯形近似 \(h/2(f(a)+f(b))\) 为0。
  \item
    \(R[i,0]\) 是使用 \(n=2^i\) 个子区间的复合梯形公式计算得到的近似值。

    \begin{itemize}
    \tightlist
    \item
      \(R[2,0] = -0.3581040588\) (对应 \(n=2^2=4\)) 与第一部分中 \(n=4\)
      的梯形法结果一致。
    \item
      \(R[3,0] = -0.4080900395\) (对应 \(n=2^3=8\)) 与第一部分中 \(n=8\)
      的梯形法结果一致。
    \end{itemize}
  \item
    这一列的精度随着 \(i\) 的增加(即 \(h\) 减小)而提高。
  \end{itemize}
\item
  \textbf{外推列 \(R[i,j]\) (\(j>0\))}:

  \begin{itemize}
  \tightlist
  \item
    每一列 \(R[i,j]\) (固定 \(j\)) 的值通常比其左边一列 \(R[i,j-1]\)
    的值更精确。
  \item
    对角线元素 \(R[i,i]\) 通常是龙贝格表中给定 \(i\)(即最多使用 \(2^i\)
    个子区间的梯形规则信息)时最精确的估计。
  \item
    观察对角线元素: \(R[0,0] = 0.0\) \(R[1,1] = -0.3267...\)
    \(R[2,2] = -0.4003...\) \ldots{} \(R[7,7] = -0.4440442688\)
    这些值逐步趋向精确值 \(-4/9 \approx -0.4444444444\)。
  \end{itemize}
\end{enumerate}

\textbf{最终结果与比较:}

\begin{itemize}
\tightlist
\item
  \textbf{龙贝格结果}: \(R[7,7] = -0.4440442688\)
\item
  \textbf{精确值}: \(-0.4444444444\)
\item
  \textbf{龙贝格误差}:
  \(|-0.4440442688 - (-0.4444444444)| \approx 4.00 \times 10^{-4}\)
\end{itemize}

\textbf{与 \(n \approx 128\) 时的复合方法比较:} 龙贝格算法的 \(R[7,7]\)
是基于 \(R[7,0]\) (即 \(n=2^7=128\) 的梯形公式)
及之前的梯形值外推得到的。

\begin{itemize}
\tightlist
\item
  梯形法误差 (\(n=128\)): \texttt{9.51e-04}

  \begin{itemize}
  \tightlist
  \item
    这对应于 \(R[7,0]\) 的误差:
    \(|-0.4434936549 - (-0.4444444444)| \approx 9.5079 \times 10^{-4}\)。
  \end{itemize}
\item
  辛普森法误差 (\(n=128\)): \texttt{4.63e-04}
\item
  龙贝格 \(R[7,7]\) 误差: \texttt{4.00e-04}
\end{itemize}

\textbf{结论:} 1. 对于 \(n=128\) 的情况,龙贝格积分 (\(R[7,7]\))
提供了比复合梯形法和复合辛普森法更高的精度。它的误差 (\texttt{4.00e-04})
略小于辛普森法的误差 (\texttt{4.63e-04}),并且显著小于梯形法的误差
(\texttt{9.51e-04})。 2.
这表明即使对于在端点处具有奇异性的函数,龙贝格算法通过理查森外推仍然能够有效地提高数值积分的精度。

    \section{问题二}\label{ux95eeux9898ux4e8c}

解线性方程组 \(A x = b\),其中

\[
A = \begin{pmatrix}
  10 & -7 & 0 & 1\\
  -3 & 2.099999 & 6 & 2\\
  5 & -1 & 5 & -1\\
  2 & 1 & 0 & 2\\
\end{pmatrix}, \quad 
x = \begin{pmatrix}
  x_1\\ x_2\\ x_3\\ x_4
\end{pmatrix}, \quad
b = \begin{pmatrix}
  8\\ 5.900001\\ 5\\ 1
\end{pmatrix}
\]

输出 \(A x = b\) 中系数 \(A = L U\) 分解的矩阵 L 和 U,解向量 x 以及
\(det A\);列主元法的行交换次序,解向量 x 以及
\(\det A\);比较两种方法得到的结果。

    \subsection{Solution}\label{solution}

使用 python 进行数值计算。

    \begin{tcolorbox}[breakable, size=fbox, boxrule=1pt, pad at break*=1mm,colback=cellbackground, colframe=cellborder]
\prompt{In}{incolor}{17}{\boxspacing}
\begin{Verbatim}[commandchars=\\\{\}]
\PY{k+kn}{import} \PY{n+nn}{numpy} \PY{k}{as} \PY{n+nn}{np}
\PY{k+kn}{from} \PY{n+nn}{scipy}\PY{n+nn}{.}\PY{n+nn}{linalg} \PY{k+kn}{import} \PY{n}{lu\PYZus{}factor}\PY{p}{,} \PY{n}{lu\PYZus{}solve}\PY{p}{,} \PY{n}{lu}

\PY{c+c1}{\PYZsh{} 设置打印选项}
\PY{n}{np}\PY{o}{.}\PY{n}{set\PYZus{}printoptions}\PY{p}{(}\PY{n}{precision}\PY{o}{=}\PY{l+m+mi}{7}\PY{p}{,} \PY{n}{suppress}\PY{o}{=}\PY{k+kc}{True}\PY{p}{)}

\PY{c+c1}{\PYZsh{} 定义矩阵 A 和向量 b}
\PY{n}{A} \PY{o}{=} \PY{n}{np}\PY{o}{.}\PY{n}{array}\PY{p}{(}\PY{p}{[}
    \PY{p}{[}\PY{l+m+mi}{10}\PY{p}{,} \PY{o}{\PYZhy{}}\PY{l+m+mi}{7}\PY{p}{,} \PY{l+m+mi}{0}\PY{p}{,} \PY{l+m+mi}{1}\PY{p}{]}\PY{p}{,}
    \PY{p}{[}\PY{o}{\PYZhy{}}\PY{l+m+mi}{3}\PY{p}{,} \PY{l+m+mf}{2.099999}\PY{p}{,} \PY{l+m+mi}{6}\PY{p}{,} \PY{l+m+mi}{2}\PY{p}{]}\PY{p}{,}
    \PY{p}{[}\PY{l+m+mi}{5}\PY{p}{,} \PY{o}{\PYZhy{}}\PY{l+m+mi}{1}\PY{p}{,} \PY{l+m+mi}{5}\PY{p}{,} \PY{o}{\PYZhy{}}\PY{l+m+mi}{1}\PY{p}{]}\PY{p}{,}
    \PY{p}{[}\PY{l+m+mi}{2}\PY{p}{,} \PY{l+m+mi}{1}\PY{p}{,} \PY{l+m+mi}{0}\PY{p}{,} \PY{l+m+mi}{2}\PY{p}{]}
\PY{p}{]}\PY{p}{,} \PY{n}{dtype}\PY{o}{=}\PY{n+nb}{float}\PY{p}{)}

\PY{n}{b} \PY{o}{=} \PY{n}{np}\PY{o}{.}\PY{n}{array}\PY{p}{(}\PY{p}{[}\PY{l+m+mi}{8}\PY{p}{,} \PY{l+m+mf}{5.900001}\PY{p}{,} \PY{l+m+mi}{5}\PY{p}{,} \PY{l+m+mi}{1}\PY{p}{]}\PY{p}{,} \PY{n}{dtype}\PY{o}{=}\PY{n+nb}{float}\PY{p}{)}
\PY{n}{n} \PY{o}{=} \PY{n}{A}\PY{o}{.}\PY{n}{shape}\PY{p}{[}\PY{l+m+mi}{0}\PY{p}{]}

\PY{c+c1}{\PYZsh{} \PYZhy{}\PYZhy{}\PYZhy{} 方法一:标准 LU 分解 (A=LU, Doolittle 法) \PYZhy{}\PYZhy{}\PYZhy{}}
\PY{n+nb}{print}\PY{p}{(}\PY{l+s+s2}{\PYZdq{}}\PY{l+s+s2}{方法一:标准 LU 分解 (A=LU)}\PY{l+s+s2}{\PYZdq{}}\PY{p}{)}
\PY{n}{L\PYZus{}std} \PY{o}{=} \PY{n}{np}\PY{o}{.}\PY{n}{eye}\PY{p}{(}\PY{n}{n}\PY{p}{,} \PY{n}{dtype}\PY{o}{=}\PY{n+nb}{float}\PY{p}{)}
\PY{n}{U\PYZus{}std\PYZus{}calc} \PY{o}{=} \PY{n}{A}\PY{o}{.}\PY{n}{copy}\PY{p}{(}\PY{p}{)} \PY{c+c1}{\PYZsh{} U\PYZus{}std\PYZus{}calc 将被转换为 U 矩阵}

\PY{c+c1}{\PYZsh{} Doolittle 分解过程}
\PY{k}{for} \PY{n}{k\PYZus{}std} \PY{o+ow}{in} \PY{n+nb}{range}\PY{p}{(}\PY{n}{n} \PY{o}{\PYZhy{}} \PY{l+m+mi}{1}\PY{p}{)}\PY{p}{:}
    \PY{k}{if} \PY{n}{np}\PY{o}{.}\PY{n}{isclose}\PY{p}{(}\PY{n}{U\PYZus{}std\PYZus{}calc}\PY{p}{[}\PY{n}{k\PYZus{}std}\PY{p}{,} \PY{n}{k\PYZus{}std}\PY{p}{]}\PY{p}{,} \PY{l+m+mf}{0.0}\PY{p}{)}\PY{p}{:}
        \PY{n+nb}{print}\PY{p}{(}\PY{l+s+sa}{f}\PY{l+s+s2}{\PYZdq{}}\PY{l+s+s2}{警告: 标准LU分解中,主元 U\PYZus{}std\PYZus{}calc[}\PY{l+s+si}{\PYZob{}}\PY{n}{k\PYZus{}std}\PY{l+s+si}{\PYZcb{}}\PY{l+s+s2}{,}\PY{l+s+si}{\PYZob{}}\PY{n}{k\PYZus{}std}\PY{l+s+si}{\PYZcb{}}\PY{l+s+s2}{] (}\PY{l+s+si}{\PYZob{}}\PY{n}{U\PYZus{}std\PYZus{}calc}\PY{p}{[}\PY{n}{k\PYZus{}std}\PY{p}{,}\PY{+w}{ }\PY{n}{k\PYZus{}std}\PY{p}{]}\PY{l+s+si}{:}\PY{l+s+s2}{.3e}\PY{l+s+si}{\PYZcb{}}\PY{l+s+s2}{) 接近于零。}\PY{l+s+s2}{\PYZdq{}}\PY{p}{)}
    \PY{k}{for} \PY{n}{i\PYZus{}std} \PY{o+ow}{in} \PY{n+nb}{range}\PY{p}{(}\PY{n}{k\PYZus{}std} \PY{o}{+} \PY{l+m+mi}{1}\PY{p}{,} \PY{n}{n}\PY{p}{)}\PY{p}{:}
        \PY{n}{multiplier\PYZus{}std} \PY{o}{=} \PY{n}{U\PYZus{}std\PYZus{}calc}\PY{p}{[}\PY{n}{i\PYZus{}std}\PY{p}{,} \PY{n}{k\PYZus{}std}\PY{p}{]} \PY{o}{/} \PY{n}{U\PYZus{}std\PYZus{}calc}\PY{p}{[}\PY{n}{k\PYZus{}std}\PY{p}{,} \PY{n}{k\PYZus{}std}\PY{p}{]}
        \PY{n}{L\PYZus{}std}\PY{p}{[}\PY{n}{i\PYZus{}std}\PY{p}{,} \PY{n}{k\PYZus{}std}\PY{p}{]} \PY{o}{=} \PY{n}{multiplier\PYZus{}std}
        \PY{n}{U\PYZus{}std\PYZus{}calc}\PY{p}{[}\PY{n}{i\PYZus{}std}\PY{p}{,} \PY{n}{k\PYZus{}std}\PY{p}{:}\PY{p}{]} \PY{o}{=} \PY{n}{U\PYZus{}std\PYZus{}calc}\PY{p}{[}\PY{n}{i\PYZus{}std}\PY{p}{,} \PY{n}{k\PYZus{}std}\PY{p}{:}\PY{p}{]} \PY{o}{\PYZhy{}} \PY{n}{multiplier\PYZus{}std} \PY{o}{*} \PY{n}{U\PYZus{}std\PYZus{}calc}\PY{p}{[}\PY{n}{k\PYZus{}std}\PY{p}{,} \PY{n}{k\PYZus{}std}\PY{p}{:}\PY{p}{]}
\PY{n}{U\PYZus{}std} \PY{o}{=} \PY{n}{np}\PY{o}{.}\PY{n}{triu}\PY{p}{(}\PY{n}{U\PYZus{}std\PYZus{}calc}\PY{p}{)} \PY{c+c1}{\PYZsh{} 提取上三角部分}

\PY{n+nb}{print}\PY{p}{(}\PY{l+s+s2}{\PYZdq{}}\PY{l+s+se}{\PYZbs{}n}\PY{l+s+s2}{L\PYZus{}std (标准LU的L矩阵):}\PY{l+s+se}{\PYZbs{}n}\PY{l+s+s2}{\PYZdq{}}\PY{p}{,} \PY{n}{L\PYZus{}std}\PY{p}{)}
\PY{n+nb}{print}\PY{p}{(}\PY{l+s+s2}{\PYZdq{}}\PY{l+s+s2}{U\PYZus{}std (标准LU的U矩阵):}\PY{l+s+se}{\PYZbs{}n}\PY{l+s+s2}{\PYZdq{}}\PY{p}{,} \PY{n}{U\PYZus{}std}\PY{p}{)}

\PY{c+c1}{\PYZsh{} 解 Ly = b (前向替换)}
\PY{n}{y\PYZus{}std} \PY{o}{=} \PY{n}{np}\PY{o}{.}\PY{n}{zeros}\PY{p}{(}\PY{n}{n}\PY{p}{,} \PY{n}{dtype}\PY{o}{=}\PY{n+nb}{float}\PY{p}{)}
\PY{k}{for} \PY{n}{i\PYZus{}std} \PY{o+ow}{in} \PY{n+nb}{range}\PY{p}{(}\PY{n}{n}\PY{p}{)}\PY{p}{:}
    \PY{n}{y\PYZus{}std}\PY{p}{[}\PY{n}{i\PYZus{}std}\PY{p}{]} \PY{o}{=} \PY{n}{b}\PY{p}{[}\PY{n}{i\PYZus{}std}\PY{p}{]} \PY{o}{\PYZhy{}} \PY{n}{np}\PY{o}{.}\PY{n}{dot}\PY{p}{(}\PY{n}{L\PYZus{}std}\PY{p}{[}\PY{n}{i\PYZus{}std}\PY{p}{,} \PY{p}{:}\PY{n}{i\PYZus{}std}\PY{p}{]}\PY{p}{,} \PY{n}{y\PYZus{}std}\PY{p}{[}\PY{p}{:}\PY{n}{i\PYZus{}std}\PY{p}{]}\PY{p}{)}

\PY{c+c1}{\PYZsh{} 解 Ux = y (后向替换)}
\PY{n}{x\PYZus{}std} \PY{o}{=} \PY{n}{np}\PY{o}{.}\PY{n}{zeros}\PY{p}{(}\PY{n}{n}\PY{p}{,} \PY{n}{dtype}\PY{o}{=}\PY{n+nb}{float}\PY{p}{)}
\PY{k}{if} \PY{o+ow}{not} \PY{n}{np}\PY{o}{.}\PY{n}{isclose}\PY{p}{(}\PY{n}{U\PYZus{}std}\PY{p}{[}\PY{n}{n}\PY{o}{\PYZhy{}}\PY{l+m+mi}{1}\PY{p}{,} \PY{n}{n}\PY{o}{\PYZhy{}}\PY{l+m+mi}{1}\PY{p}{]}\PY{p}{,} \PY{l+m+mf}{0.0}\PY{p}{)}\PY{p}{:} \PY{c+c1}{\PYZsh{} 检查最后一个主元}
    \PY{k}{for} \PY{n}{i\PYZus{}std} \PY{o+ow}{in} \PY{n+nb}{range}\PY{p}{(}\PY{n}{n} \PY{o}{\PYZhy{}} \PY{l+m+mi}{1}\PY{p}{,} \PY{o}{\PYZhy{}}\PY{l+m+mi}{1}\PY{p}{,} \PY{o}{\PYZhy{}}\PY{l+m+mi}{1}\PY{p}{)}\PY{p}{:}
        \PY{k}{if} \PY{n}{np}\PY{o}{.}\PY{n}{isclose}\PY{p}{(}\PY{n}{U\PYZus{}std}\PY{p}{[}\PY{n}{i\PYZus{}std}\PY{p}{,} \PY{n}{i\PYZus{}std}\PY{p}{]}\PY{p}{,} \PY{l+m+mf}{0.0}\PY{p}{)}\PY{p}{:}
            \PY{n+nb}{print}\PY{p}{(}\PY{l+s+sa}{f}\PY{l+s+s2}{\PYZdq{}}\PY{l+s+s2}{错误: U\PYZus{}std[}\PY{l+s+si}{\PYZob{}}\PY{n}{i\PYZus{}std}\PY{l+s+si}{\PYZcb{}}\PY{l+s+s2}{,}\PY{l+s+si}{\PYZob{}}\PY{n}{i\PYZus{}std}\PY{l+s+si}{\PYZcb{}}\PY{l+s+s2}{] 主元为零,标准LU分解无法回代。}\PY{l+s+s2}{\PYZdq{}}\PY{p}{)}
            \PY{n}{x\PYZus{}std}\PY{o}{.}\PY{n}{fill}\PY{p}{(}\PY{n}{np}\PY{o}{.}\PY{n}{nan}\PY{p}{)}
            \PY{k}{break}
        \PY{n}{x\PYZus{}std}\PY{p}{[}\PY{n}{i\PYZus{}std}\PY{p}{]} \PY{o}{=} \PY{p}{(}\PY{n}{y\PYZus{}std}\PY{p}{[}\PY{n}{i\PYZus{}std}\PY{p}{]} \PY{o}{\PYZhy{}} \PY{n}{np}\PY{o}{.}\PY{n}{dot}\PY{p}{(}\PY{n}{U\PYZus{}std}\PY{p}{[}\PY{n}{i\PYZus{}std}\PY{p}{,} \PY{n}{i\PYZus{}std}\PY{o}{+}\PY{l+m+mi}{1}\PY{p}{:}\PY{p}{]}\PY{p}{,} \PY{n}{x\PYZus{}std}\PY{p}{[}\PY{n}{i\PYZus{}std}\PY{o}{+}\PY{l+m+mi}{1}\PY{p}{:}\PY{p}{]}\PY{p}{)}\PY{p}{)} \PY{o}{/} \PY{n}{U\PYZus{}std}\PY{p}{[}\PY{n}{i\PYZus{}std}\PY{p}{,} \PY{n}{i\PYZus{}std}\PY{p}{]}
\PY{k}{else}\PY{p}{:}
    \PY{n+nb}{print}\PY{p}{(}\PY{l+s+sa}{f}\PY{l+s+s2}{\PYZdq{}}\PY{l+s+s2}{错误: U\PYZus{}std[}\PY{l+s+si}{\PYZob{}}\PY{n}{n}\PY{o}{\PYZhy{}}\PY{l+m+mi}{1}\PY{l+s+si}{\PYZcb{}}\PY{l+s+s2}{,}\PY{l+s+si}{\PYZob{}}\PY{n}{n}\PY{o}{\PYZhy{}}\PY{l+m+mi}{1}\PY{l+s+si}{\PYZcb{}}\PY{l+s+s2}{] 主元为零,标准LU分解无法回代。}\PY{l+s+s2}{\PYZdq{}}\PY{p}{)}
    \PY{n}{x\PYZus{}std}\PY{o}{.}\PY{n}{fill}\PY{p}{(}\PY{n}{np}\PY{o}{.}\PY{n}{nan}\PY{p}{)}


\PY{n+nb}{print}\PY{p}{(}\PY{l+s+s2}{\PYZdq{}}\PY{l+s+se}{\PYZbs{}n}\PY{l+s+s2}{x\PYZus{}std (标准LU的解向量):}\PY{l+s+se}{\PYZbs{}n}\PY{l+s+s2}{\PYZdq{}}\PY{p}{,} \PY{n}{x\PYZus{}std}\PY{p}{)}

\PY{n}{det\PYZus{}A\PYZus{}std} \PY{o}{=} \PY{n}{np}\PY{o}{.}\PY{n}{nan}
\PY{k}{if} \PY{o+ow}{not} \PY{n}{np}\PY{o}{.}\PY{n}{any}\PY{p}{(}\PY{n}{np}\PY{o}{.}\PY{n}{isnan}\PY{p}{(}\PY{n}{x\PYZus{}std}\PY{p}{)}\PY{p}{)}\PY{p}{:}
    \PY{n}{det\PYZus{}A\PYZus{}std} \PY{o}{=} \PY{n}{np}\PY{o}{.}\PY{n}{prod}\PY{p}{(}\PY{n}{np}\PY{o}{.}\PY{n}{diag}\PY{p}{(}\PY{n}{U\PYZus{}std}\PY{p}{)}\PY{p}{)}
\PY{n+nb}{print}\PY{p}{(}\PY{l+s+s2}{\PYZdq{}}\PY{l+s+s2}{det(A)\PYZus{}std (标准LU计算的行列式):}\PY{l+s+s2}{\PYZdq{}}\PY{p}{,} \PY{n}{det\PYZus{}A\PYZus{}std}\PY{p}{)}
\PY{n+nb}{print}\PY{p}{(}\PY{l+s+s2}{\PYZdq{}}\PY{l+s+s2}{\PYZhy{}}\PY{l+s+s2}{\PYZdq{}} \PY{o}{*} \PY{l+m+mi}{50}\PY{p}{)}

\PY{n+nb}{print}\PY{p}{(}\PY{l+s+s2}{\PYZdq{}}\PY{l+s+se}{\PYZbs{}n}\PY{l+s+s2}{方法二:列主元 LU 分解 (PA=LU)}\PY{l+s+s2}{\PYZdq{}}\PY{p}{)}

\PY{n}{lu\PYZus{}arr\PYZus{}piv}\PY{p}{,} \PY{n}{piv\PYZus{}indices\PYZus{}piv} \PY{o}{=} \PY{n}{lu\PYZus{}factor}\PY{p}{(}\PY{n}{A}\PY{p}{)}

\PY{c+c1}{\PYZsh{} 从 lu\PYZus{}arr\PYZus{}piv 中提取 L 和 U (Doolittle 形式的 PA=LU, L对角线为1)}
\PY{n}{L\PYZus{}piv} \PY{o}{=} \PY{n}{np}\PY{o}{.}\PY{n}{tril}\PY{p}{(}\PY{n}{lu\PYZus{}arr\PYZus{}piv}\PY{p}{,} \PY{n}{k}\PY{o}{=}\PY{o}{\PYZhy{}}\PY{l+m+mi}{1}\PY{p}{)} \PY{o}{+} \PY{n}{np}\PY{o}{.}\PY{n}{eye}\PY{p}{(}\PY{n}{n}\PY{p}{)} \PY{c+c1}{\PYZsh{} L 对角线为1}
\PY{n}{U\PYZus{}piv} \PY{o}{=} \PY{n}{np}\PY{o}{.}\PY{n}{triu}\PY{p}{(}\PY{n}{lu\PYZus{}arr\PYZus{}piv}\PY{p}{)}                  \PY{c+c1}{\PYZsh{} U}

\PY{n}{P\PYZus{}matrix\PYZus{}piv} \PY{o}{=} \PY{n}{np}\PY{o}{.}\PY{n}{eye}\PY{p}{(}\PY{n}{n}\PY{p}{)}

\PY{n}{P\PYZus{}scipy}\PY{p}{,} \PY{n}{L\PYZus{}scipy\PYZus{}alt}\PY{p}{,} \PY{n}{U\PYZus{}scipy\PYZus{}alt} \PY{o}{=} \PY{n}{lu}\PY{p}{(}\PY{n}{A}\PY{p}{)}

\PY{n+nb}{print}\PY{p}{(}\PY{l+s+s2}{\PYZdq{}}\PY{l+s+se}{\PYZbs{}n}\PY{l+s+s2}{行交换次序 (piv\PYZus{}indices from lu\PYZus{}factor):}\PY{l+s+s2}{\PYZdq{}}\PY{p}{)}
\PY{n+nb}{print}\PY{p}{(}\PY{n}{piv\PYZus{}indices\PYZus{}piv}\PY{p}{)}

\PY{n+nb}{print}\PY{p}{(}\PY{l+s+s2}{\PYZdq{}}\PY{l+s+se}{\PYZbs{}n}\PY{l+s+s2}{L\PYZus{}piv (列主元LU的L矩阵):}\PY{l+s+se}{\PYZbs{}n}\PY{l+s+s2}{\PYZdq{}}\PY{p}{,} \PY{n}{L\PYZus{}piv}\PY{p}{)}
\PY{n+nb}{print}\PY{p}{(}\PY{l+s+s2}{\PYZdq{}}\PY{l+s+s2}{U\PYZus{}piv (列主元LU的U矩阵):}\PY{l+s+se}{\PYZbs{}n}\PY{l+s+s2}{\PYZdq{}}\PY{p}{,} \PY{n}{U\PYZus{}piv}\PY{p}{)}

\PY{c+c1}{\PYZsh{} 解 Ax = b (使用 lu\PYZus{}solve 更可靠)}
\PY{n}{x\PYZus{}piv} \PY{o}{=} \PY{n}{lu\PYZus{}solve}\PY{p}{(}\PY{p}{(}\PY{n}{lu\PYZus{}arr\PYZus{}piv}\PY{p}{,} \PY{n}{piv\PYZus{}indices\PYZus{}piv}\PY{p}{)}\PY{p}{,} \PY{n}{b}\PY{p}{)}
\PY{n+nb}{print}\PY{p}{(}\PY{l+s+s2}{\PYZdq{}}\PY{l+s+se}{\PYZbs{}n}\PY{l+s+s2}{x\PYZus{}piv (列主元LU的解向量):}\PY{l+s+se}{\PYZbs{}n}\PY{l+s+s2}{\PYZdq{}}\PY{p}{,} \PY{n}{x\PYZus{}piv}\PY{p}{)}

\PY{n}{sign\PYZus{}piv} \PY{o}{=} \PY{l+m+mf}{1.0}

\PY{n}{temp\PYZus{}piv\PYZus{}for\PYZus{}sign} \PY{o}{=} \PY{n+nb}{list}\PY{p}{(}\PY{n}{piv\PYZus{}indices\PYZus{}piv}\PY{p}{)}
\PY{n}{num\PYZus{}swaps\PYZus{}piv} \PY{o}{=} \PY{l+m+mi}{0}

\PY{n}{s\PYZus{}piv} \PY{o}{=} \PY{n}{np}\PY{o}{.}\PY{n}{sum}\PY{p}{(}\PY{n}{np}\PY{o}{.}\PY{n}{arange}\PY{p}{(}\PY{n}{n}\PY{p}{)} \PY{o}{!=} \PY{n}{piv\PYZus{}indices\PYZus{}piv}\PY{p}{)} \PY{c+c1}{\PYZsh{} Number of elements not in their original place}

\PY{n}{det\PYZus{}P\PYZus{}val} \PY{o}{=} \PY{n}{np}\PY{o}{.}\PY{n}{linalg}\PY{o}{.}\PY{n}{det}\PY{p}{(}\PY{n}{P\PYZus{}scipy}\PY{p}{)} \PY{c+c1}{\PYZsh{} P\PYZus{}scipy from A = P L U}

\PY{n}{det\PYZus{}A\PYZus{}piv} \PY{o}{=} \PY{n}{np}\PY{o}{.}\PY{n}{prod}\PY{p}{(}\PY{n}{np}\PY{o}{.}\PY{n}{diag}\PY{p}{(}\PY{n}{U\PYZus{}piv}\PY{p}{)}\PY{p}{)} \PY{o}{/} \PY{n}{det\PYZus{}P\PYZus{}val}

\PY{n+nb}{print}\PY{p}{(}\PY{l+s+s2}{\PYZdq{}}\PY{l+s+s2}{det(A)\PYZus{}piv (列主元LU计算的行列式):}\PY{l+s+s2}{\PYZdq{}}\PY{p}{,} \PY{n}{det\PYZus{}A\PYZus{}piv}\PY{p}{)}
\end{Verbatim}
\end{tcolorbox}

    \begin{Verbatim}[commandchars=\\\{\}]
方法一:标准 LU 分解 (A=LU)

L\_std (标准LU的L矩阵):
 [[       1.               0.               0.               0.       ]
 [      -0.3              1.               0.               0.       ]
 [       0.5       -2499999.9996506        1.               0.       ]
 [       0.2       -2399999.9996645        0.9599997        1.       ]]
U\_std (标准LU的U矩阵):
 [[      10.              -7.               0.               1.       ]
 [       0.              -0.000001         6.               2.3      ]
 [       0.               0.        15000004.9979033  5749998.4991963]
 [       0.               0.               0.               5.0799989]]

x\_std (标准LU的解向量):
 [-0. -1.  1.  1.]
det(A)\_std (标准LU计算的行列式): -762.0000901449143
--------------------------------------------------

方法二:列主元 LU 分解 (PA=LU)

行交换次序 (piv\_indices from lu\_factor):
[0 2 2 3]

L\_piv (列主元LU的L矩阵):
 [[ 1.         0.         0.         0.       ]
 [ 0.5        1.         0.         0.       ]
 [-0.3       -0.0000004  1.         0.       ]
 [ 0.2        0.96      -0.7999997  1.       ]]
U\_piv (列主元LU的U矩阵):
 [[10.        -7.         0.         1.       ]
 [ 0.         2.5        5.        -1.5      ]
 [ 0.         0.         6.000002   2.2999994]
 [ 0.         0.         0.         5.0799989]]

x\_piv (列主元LU的解向量):
 [-0. -1.  1.  1.]
det(A)\_piv (列主元LU计算的行列式): -762.0000900000001
    \end{Verbatim}

    \subsubsection{\texorpdfstring{方法一:标准 LU 分解
(\(A=LU\))}{方法一:标准 LU 分解 (A=LU)}}\label{ux65b9ux6cd5ux4e00ux6807ux51c6-lu-ux5206ux89e3-alu}

通过 Doolittle 方法进行标准 LU 分解,我们得到 \(L_{std}\)
(单位下三角矩阵) 和 \(U_{std}\) (上三角矩阵)。

\textbf{\(L_{std}\) 矩阵:} \[ L_{std} = \begin{pmatrix}
1.0 & 0.0 & 0.0 & 0.0 \\
-0.3 & 1.0 & 0.0 & 0.0 \\
0.5 & -2499999.9996506 & 1.0 & 0.0 \\
0.2 & -2399999.9996645 & 0.9599997 & 1.0
\end{pmatrix} \]

\textbf{\(U_{std}\) 矩阵:} \[ U_{std} = \begin{pmatrix}
10.0 & -7.0 & 0.0 & 1.0 \\
0.0 & -0.000001 & 6.0 & 2.3 \\
0.0 & 0.0 & 15000004.9979033 & 5749998.4991963 \\
0.0 & 0.0 & 0.0 & 5.0799989
\end{pmatrix} \] 我们注意到 \(U_{std}[1,1]\) 的值非常小 (为
\(-0.000001\)),这是由于 \(A[1,1]\) 原始值为
\(2.099999\),在第一步高斯消元后,该位置的元素变为
\(2.099999 - (\frac{-3}{10})(-7) = 2.099999 - 2.1 = -0.000001\)。这个小主元是数值不稳定的潜在来源,并导致
\(L_{std}\) 矩阵中出现绝对值非常大的元素 (例如 \(L_{std}[2,1]\) 和
\(L_{std}[3,1]\))。

\textbf{解向量 \(x_{std}\):} 通过求解 \(L_{std}y = b\) 和
\(U_{std}x_{std} = y\) 得到:
\[ x_{std} = \begin{pmatrix} -0.0 \\ -1.0 \\ 1.0 \\ 1.0 \end{pmatrix} \]

\textbf{行列式 \(det(A)_{std}\):}
\(det(A)_{std} = \prod_{i} U_{std}[i,i]\)
\[ det(A)_{std} \approx -762.0000901 \]

\subsubsection{\texorpdfstring{方法二:列主元 LU 分解
(\(PA=LU\))}{方法二:列主元 LU 分解 (PA=LU)}}\label{ux65b9ux6cd5ux4e8cux5217ux4e3bux5143-lu-ux5206ux89e3-palu}

使用列主元策略进行 LU 分解,得到 \(P\) (置换矩阵),\(L_{piv}\)
(单位下三角矩阵),和 \(U_{piv}\) (上三角矩阵),使得
\(PA = L_{piv}U_{piv}\)。

\textbf{行交换次序 (\texttt{piv\_indices\_piv} from
\texttt{lu\_factor}):} \texttt{scipy.linalg.lu\_factor} 返回的
\texttt{piv\_indices\_piv} 数组为 \texttt{{[}0\ 2\ 2\ 3{]}}。这个数组被
\texttt{lu\_solve} 用来正确地进行求解。它编码了行交换操作。在
\texttt{scipy} 的 \texttt{lu\_factor} 实现中,\texttt{piv{[}k{]}}
指示的是在分解的第 \texttt{k} 步与第 \texttt{k}
行发生交换的行的索引。对于行列式的符号,我们依赖于从
\texttt{scipy.linalg.lu} (返回 \(A=PLU\)) 中得到的置换矩阵 \(P_{scipy}\)
的行列式。

\textbf{\(L_{piv}\) 矩阵:} \[ L_{piv} = \begin{pmatrix}
1.0 & 0.0 & 0.0 & 0.0 \\
0.5 & 1.0 & 0.0 & 0.0 \\
-0.3 & -0.0000004 & 1.0 & 0.0 \\
0.2 & 0.96 & -0.7999997 & 1.0
\end{pmatrix} \]

\textbf{\(U_{piv}\) 矩阵:} \[ U_{piv} = \begin{pmatrix}
10.0 & -7.0 & 0.0 & 1.0 \\
0.0 & 2.5 & 5.0 & -1.5 \\
0.0 & 0.0 & 6.000002 & 2.2999994 \\
0.0 & 0.0 & 0.0 & 5.0799989
\end{pmatrix} \] 与 \(U_{std}\) 不同,\(U_{piv}\)
的对角线元素通过列主元选择,避免了过小的主元,因此 \(L_{piv}\)
中的元素数量级正常。

\textbf{解向量 \(x_{piv}\):} 通过 \texttt{lu\_solve} (内部使用 \(PA=LU\)
分解) 得到:
\[ x_{piv} = \begin{pmatrix} -0.0 \\ -1.0 \\ 1.0 \\ 1.0 \end{pmatrix} \]

\textbf{行列式 \(det(A)_{piv}\):}
\(det(A)_{piv} = \frac{\prod_{i} U_{piv}[i,i]}{det(P_{scipy})}\) (其中
\(P_{scipy}\) 是 \(A=P_{scipy}LU\) 中的置换矩阵)。
\[ det(A)_{piv} \approx -762.0000900 \]

\subsubsection{结果比较}\label{ux7ed3ux679cux6bd4ux8f83}

\begin{enumerate}
\def\labelenumi{\arabic{enumi}.}
\item
  \textbf{解向量的比较}:

  \(x_{std} = \begin{pmatrix} -0.0 \\ -1.0 \\ 1.0 \\ 1.0 \end{pmatrix}\)
  \(x_{piv} = \begin{pmatrix} -0.0 \\ -1.0 \\ 1.0 \\ 1.0 \end{pmatrix}\)
  在这个特定的例子和所使用的数值精度下,两种方法得到的解向量是相同的:\(x = [0, -1, 1, 1]^T\)。我们可以验证这确实是方程组的精确解:
  \(A x = \begin{pmatrix} 10 & -7 & 0 & 1\\ -3 & 2.099999 & 6 & 2\\ 5 & -1 & 5 & -1\\ 2 & 1 & 0 & 2 \end{pmatrix} \begin{pmatrix} 0 \\ -1 \\ 1 \\ 1 \end{pmatrix} = \begin{pmatrix} 0+7+0+1 \\ 0-2.099999+6+2 \\ 0+1+5-1 \\ 0-1+0+2 \end{pmatrix} = \begin{pmatrix} 8 \\ 5.900001 \\ 5 \\ 1 \end{pmatrix} = b\).
  尽管标准 LU 分解中遇到了数值不稳定的情况(\(L_{std}\)
  矩阵中的大元素),但最终的解向量 \(x_{std}\) 仍然是准确的。这是因为
  NumPy/SciPy 在内部运算时维持了足够的精度。
\item
  \textbf{行列式的比较}: \(det(A)_{std} \approx -762.0000901449143\)
  \(det(A)_{piv} \approx -762.0000900000001\) 两个行列式的值也非常接近。
\end{enumerate}

\textbf{结论:}

对于本问题中的矩阵 \(A\),标准 LU
分解(无行交换)在过程中遇到了一个绝对值非常小的主元
(\(U_{std}[1,1] \approx -10^{-6}\))。这直接导致了 \(L_{std}\)
矩阵中对应列的乘子变得非常大 (数量级达到
\(10^6\))。这种情况是数值不稳定性的典型表现,因为大的乘子会放大后续计算中的舍入误差。

尽管在这个特定的高精度计算环境下,标准 LU 分解得到的最终解向量
\(x_{std}\) 和行列式 \(det(A)_{std}\)
与列主元法的结果非常吻合(并且解是正确的),但这并不能说明标准 LU
分解总是可靠的。如果计算精度有限,或者矩阵的病态性更强,标准 LU
分解很可能给出错误的解。

列主元 LU
分解通过在每一步选择列中绝对值最大的元素作为主元,并进行必要的行交换,有效地避免了小主元问题。这使得
\(L_{piv}\) 和 \(U_{piv}\)
矩阵中的元素保持在合理的数量级,从而提高了计算的稳定性和结果的准确性。

因此,尽管两种方法在此例中得到了相同的解向量,但从数值稳定性的角度看,列主元法是更优越和更可靠的方法。\(L_{std}\)
矩阵中出现的大数值元素本身就是不采用主元策略可能导致问题的警示。在实际应用中,总是推荐使用带有主元选择的
LU 分解方法。

    \section{问题三}\label{ux95eeux9898ux4e09}

给出方程组

\[
\begin{cases}
  3x_1 - \cos(x_2 x_3) - 1/2 = 0, \\
  x_1^2 - 81(x_2 + 0.1)^2 + \sin x_3 + 1.06 = 0,\\
  e^{-x_1 x_2} + 20x_3 + (10 \pi)/3 - 1 = 0
\end{cases}
\]

用牛顿法求解方程,至少用三个不同初值计算,计算到
\(||x^{(k)} - x^{(k-1)}||<10^{-8}\) 停止。

    \subsection{Solution}\label{solution}

牛顿法的迭代公式为:
\[ \mathbf{x}^{(k+1)} = \mathbf{x}^{(k)} - [J(\mathbf{x}^{(k)})]^{-1} \mathbf{F}(\mathbf{x}^{(k)}) \]
其中 \(J(\mathbf{x})\) 是 \(\mathbf{F}(\mathbf{x})\) 的雅可比矩阵: \[
J(\mathbf{x}) = \begin{pmatrix}
3 & x_3 \sin(x_2 x_3) & x_2 \sin(x_2 x_3) \\
2x_1 & -162(x_2 + 0.1) & \cos x_3 \\
-x_2 e^{-x_1 x_2} & -x_1 e^{-x_1 x_2} & 20
\end{pmatrix}
\] 迭代停止条件为
\(||\mathbf{x}^{(k)} - \mathbf{x}^{(k-1)}||_2 < 10^{-8}\)。

选择了以下三个不同的初始猜测值 \(\mathbf{x}^{(0)}\) 进行计算:

\begin{enumerate}
\def\labelenumi{\arabic{enumi}.}
\tightlist
\item
  \(\mathbf{x}^{(0)} = [0.1, 0.1, -0.1]^T\)
\item
  \(\mathbf{x}^{(0)} = [0.5, 0.0, -0.5]^T\)
\item
  \(\mathbf{x}^{(0)} = [0.0, 0.0, 0.0]^T\)
\end{enumerate}

使用 python 进行计算。

    \begin{tcolorbox}[breakable, size=fbox, boxrule=1pt, pad at break*=1mm,colback=cellbackground, colframe=cellborder]
\prompt{In}{incolor}{19}{\boxspacing}
\begin{Verbatim}[commandchars=\\\{\}]
\PY{k+kn}{import} \PY{n+nn}{numpy} \PY{k}{as} \PY{n+nn}{np}
\PY{k+kn}{from} \PY{n+nn}{numpy}\PY{n+nn}{.}\PY{n+nn}{linalg} \PY{k+kn}{import} \PY{n}{solve}\PY{p}{,} \PY{n}{norm}

\PY{c+c1}{\PYZsh{} 定义方程组 F(x)}
\PY{k}{def} \PY{n+nf}{F}\PY{p}{(}\PY{n}{x\PYZus{}vec}\PY{p}{)}\PY{p}{:}
    \PY{n}{x1}\PY{p}{,} \PY{n}{x2}\PY{p}{,} \PY{n}{x3} \PY{o}{=} \PY{n}{x\PYZus{}vec}\PY{p}{[}\PY{l+m+mi}{0}\PY{p}{]}\PY{p}{,} \PY{n}{x\PYZus{}vec}\PY{p}{[}\PY{l+m+mi}{1}\PY{p}{]}\PY{p}{,} \PY{n}{x\PYZus{}vec}\PY{p}{[}\PY{l+m+mi}{2}\PY{p}{]}
    \PY{n}{f1} \PY{o}{=} \PY{l+m+mi}{3}\PY{o}{*}\PY{n}{x1} \PY{o}{\PYZhy{}} \PY{n}{np}\PY{o}{.}\PY{n}{cos}\PY{p}{(}\PY{n}{x2}\PY{o}{*}\PY{n}{x3}\PY{p}{)} \PY{o}{\PYZhy{}} \PY{l+m+mf}{0.5}
    \PY{n}{f2} \PY{o}{=} \PY{n}{x1}\PY{o}{*}\PY{o}{*}\PY{l+m+mi}{2} \PY{o}{\PYZhy{}} \PY{l+m+mi}{81}\PY{o}{*}\PY{p}{(}\PY{n}{x2} \PY{o}{+} \PY{l+m+mf}{0.1}\PY{p}{)}\PY{o}{*}\PY{o}{*}\PY{l+m+mi}{2} \PY{o}{+} \PY{n}{np}\PY{o}{.}\PY{n}{sin}\PY{p}{(}\PY{n}{x3}\PY{p}{)} \PY{o}{+} \PY{l+m+mf}{1.06}
    \PY{n}{f3} \PY{o}{=} \PY{n}{np}\PY{o}{.}\PY{n}{exp}\PY{p}{(}\PY{o}{\PYZhy{}}\PY{n}{x1}\PY{o}{*}\PY{n}{x2}\PY{p}{)} \PY{o}{+} \PY{l+m+mi}{20}\PY{o}{*}\PY{n}{x3} \PY{o}{+} \PY{p}{(}\PY{l+m+mi}{10}\PY{o}{*}\PY{n}{np}\PY{o}{.}\PY{n}{pi}\PY{p}{)}\PY{o}{/}\PY{l+m+mi}{3} \PY{o}{\PYZhy{}} \PY{l+m+mi}{1}
    \PY{k}{return} \PY{n}{np}\PY{o}{.}\PY{n}{array}\PY{p}{(}\PY{p}{[}\PY{n}{f1}\PY{p}{,} \PY{n}{f2}\PY{p}{,} \PY{n}{f3}\PY{p}{]}\PY{p}{)}

\PY{c+c1}{\PYZsh{} 定义雅可比矩阵 J(x)}
\PY{k}{def} \PY{n+nf}{J}\PY{p}{(}\PY{n}{x\PYZus{}vec}\PY{p}{)}\PY{p}{:}
    \PY{n}{x1}\PY{p}{,} \PY{n}{x2}\PY{p}{,} \PY{n}{x3} \PY{o}{=} \PY{n}{x\PYZus{}vec}\PY{p}{[}\PY{l+m+mi}{0}\PY{p}{]}\PY{p}{,} \PY{n}{x\PYZus{}vec}\PY{p}{[}\PY{l+m+mi}{1}\PY{p}{]}\PY{p}{,} \PY{n}{x\PYZus{}vec}\PY{p}{[}\PY{l+m+mi}{2}\PY{p}{]}
    \PY{c+c1}{\PYZsh{} df1/dx1, df1/dx2, df1/dx3}
    \PY{n}{j11} \PY{o}{=} \PY{l+m+mf}{3.0}
    \PY{n}{j12} \PY{o}{=} \PY{n}{x3} \PY{o}{*} \PY{n}{np}\PY{o}{.}\PY{n}{sin}\PY{p}{(}\PY{n}{x2}\PY{o}{*}\PY{n}{x3}\PY{p}{)}
    \PY{n}{j13} \PY{o}{=} \PY{n}{x2} \PY{o}{*} \PY{n}{np}\PY{o}{.}\PY{n}{sin}\PY{p}{(}\PY{n}{x2}\PY{o}{*}\PY{n}{x3}\PY{p}{)}
    \PY{c+c1}{\PYZsh{} df2/dx1, df2/dx2, df2/dx3}
    \PY{n}{j21} \PY{o}{=} \PY{l+m+mi}{2}\PY{o}{*}\PY{n}{x1}
    \PY{n}{j22} \PY{o}{=} \PY{o}{\PYZhy{}}\PY{l+m+mi}{162} \PY{o}{*} \PY{p}{(}\PY{n}{x2} \PY{o}{+} \PY{l+m+mf}{0.1}\PY{p}{)}
    \PY{n}{j23} \PY{o}{=} \PY{n}{np}\PY{o}{.}\PY{n}{cos}\PY{p}{(}\PY{n}{x3}\PY{p}{)}
    \PY{c+c1}{\PYZsh{} df3/dx1, df3/dx2, df3/dx3}
    \PY{n}{j31} \PY{o}{=} \PY{o}{\PYZhy{}}\PY{n}{x2} \PY{o}{*} \PY{n}{np}\PY{o}{.}\PY{n}{exp}\PY{p}{(}\PY{o}{\PYZhy{}}\PY{n}{x1}\PY{o}{*}\PY{n}{x2}\PY{p}{)}
    \PY{n}{j32} \PY{o}{=} \PY{o}{\PYZhy{}}\PY{n}{x1} \PY{o}{*} \PY{n}{np}\PY{o}{.}\PY{n}{exp}\PY{p}{(}\PY{o}{\PYZhy{}}\PY{n}{x1}\PY{o}{*}\PY{n}{x2}\PY{p}{)}
    \PY{n}{j33} \PY{o}{=} \PY{l+m+mf}{20.0}
    \PY{k}{return} \PY{n}{np}\PY{o}{.}\PY{n}{array}\PY{p}{(}\PY{p}{[}\PY{p}{[}\PY{n}{j11}\PY{p}{,} \PY{n}{j12}\PY{p}{,} \PY{n}{j13}\PY{p}{]}\PY{p}{,}
                     \PY{p}{[}\PY{n}{j21}\PY{p}{,} \PY{n}{j22}\PY{p}{,} \PY{n}{j23}\PY{p}{]}\PY{p}{,}
                     \PY{p}{[}\PY{n}{j31}\PY{p}{,} \PY{n}{j32}\PY{p}{,} \PY{n}{j33}\PY{p}{]}\PY{p}{]}\PY{p}{)}

\PY{c+c1}{\PYZsh{} 牛顿法实现}
\PY{k}{def} \PY{n+nf}{newton\PYZus{}method}\PY{p}{(}\PY{n}{F\PYZus{}func}\PY{p}{,} \PY{n}{J\PYZus{}func}\PY{p}{,} \PY{n}{x0}\PY{p}{,} \PY{n}{tol}\PY{o}{=}\PY{l+m+mf}{1e\PYZhy{}8}\PY{p}{,} \PY{n}{max\PYZus{}iter}\PY{o}{=}\PY{l+m+mi}{100}\PY{p}{)}\PY{p}{:}
    \PY{n}{x\PYZus{}k} \PY{o}{=} \PY{n}{np}\PY{o}{.}\PY{n}{array}\PY{p}{(}\PY{n}{x0}\PY{p}{,} \PY{n}{dtype}\PY{o}{=}\PY{n+nb}{float}\PY{p}{)}
    \PY{n+nb}{print}\PY{p}{(}\PY{l+s+sa}{f}\PY{l+s+s2}{\PYZdq{}}\PY{l+s+s2}{初始猜测值 x\PYZca{}(0): }\PY{l+s+si}{\PYZob{}}\PY{n}{x\PYZus{}k}\PY{l+s+si}{\PYZcb{}}\PY{l+s+s2}{\PYZdq{}}\PY{p}{)}
    
    \PY{n}{iterations\PYZus{}summary} \PY{o}{=} \PY{p}{[}\PY{p}{]} \PY{c+c1}{\PYZsh{} 用于存储每次迭代的关键信息}

    \PY{k}{for} \PY{n}{k} \PY{o+ow}{in} \PY{n+nb}{range}\PY{p}{(}\PY{n}{max\PYZus{}iter}\PY{p}{)}\PY{p}{:}
        \PY{n}{F\PYZus{}val} \PY{o}{=} \PY{n}{F\PYZus{}func}\PY{p}{(}\PY{n}{x\PYZus{}k}\PY{p}{)}
        \PY{n}{J\PYZus{}val} \PY{o}{=} \PY{n}{J\PYZus{}func}\PY{p}{(}\PY{n}{x\PYZus{}k}\PY{p}{)}

        \PY{k}{try}\PY{p}{:}
            \PY{c+c1}{\PYZsh{} 解 J(x\PYZus{}k) * delta\PYZus{}x = \PYZhy{}F(x\PYZus{}k)}
            \PY{n}{delta\PYZus{}x} \PY{o}{=} \PY{n}{solve}\PY{p}{(}\PY{n}{J\PYZus{}val}\PY{p}{,} \PY{o}{\PYZhy{}}\PY{n}{F\PYZus{}val}\PY{p}{)}
        \PY{k}{except} \PY{n}{np}\PY{o}{.}\PY{n}{linalg}\PY{o}{.}\PY{n}{LinAlgError}\PY{p}{:}
            \PY{n+nb}{print}\PY{p}{(}\PY{l+s+sa}{f}\PY{l+s+s2}{\PYZdq{}}\PY{l+s+s2}{第 }\PY{l+s+si}{\PYZob{}}\PY{n}{k}\PY{l+s+si}{\PYZcb{}}\PY{l+s+s2}{ 次迭代时雅可比矩阵奇异。停止。}\PY{l+s+s2}{\PYZdq{}}\PY{p}{)}
            \PY{n}{iterations\PYZus{}summary}\PY{o}{.}\PY{n}{append}\PY{p}{(}\PY{p}{\PYZob{}}
                \PY{l+s+s1}{\PYZsq{}}\PY{l+s+s1}{k}\PY{l+s+s1}{\PYZsq{}}\PY{p}{:} \PY{n}{k}\PY{p}{,} \PY{l+s+s1}{\PYZsq{}}\PY{l+s+s1}{x\PYZus{}k}\PY{l+s+s1}{\PYZsq{}}\PY{p}{:} \PY{n}{x\PYZus{}k}\PY{o}{.}\PY{n}{tolist}\PY{p}{(}\PY{p}{)}\PY{p}{,} \PY{l+s+s1}{\PYZsq{}}\PY{l+s+s1}{norm\PYZus{}F\PYZus{}xk}\PY{l+s+s1}{\PYZsq{}}\PY{p}{:} \PY{n}{norm}\PY{p}{(}\PY{n}{F\PYZus{}val}\PY{p}{)}\PY{p}{,} 
                \PY{l+s+s1}{\PYZsq{}}\PY{l+s+s1}{norm\PYZus{}delta\PYZus{}x}\PY{l+s+s1}{\PYZsq{}}\PY{p}{:} \PY{l+s+s1}{\PYZsq{}}\PY{l+s+s1}{N/A (Singular Jacobian)}\PY{l+s+s1}{\PYZsq{}}\PY{p}{,} \PY{l+s+s1}{\PYZsq{}}\PY{l+s+s1}{status}\PY{l+s+s1}{\PYZsq{}}\PY{p}{:} \PY{l+s+s1}{\PYZsq{}}\PY{l+s+s1}{Singular Jacobian}\PY{l+s+s1}{\PYZsq{}}
            \PY{p}{\PYZcb{}}\PY{p}{)}
            \PY{k}{return} \PY{k+kc}{None}\PY{p}{,} \PY{n}{iterations\PYZus{}summary}

        \PY{n}{x\PYZus{}k\PYZus{}plus\PYZus{}1} \PY{o}{=} \PY{n}{x\PYZus{}k} \PY{o}{+} \PY{n}{delta\PYZus{}x}
        \PY{c+c1}{\PYZsh{} 计算 ||x\PYZca{}(k+1) \PYZhy{} x\PYZca{}(k)||, 即 ||delta\PYZus{}x||}
        \PY{n}{diff\PYZus{}norm} \PY{o}{=} \PY{n}{norm}\PY{p}{(}\PY{n}{delta\PYZus{}x}\PY{p}{)}
        
        \PY{n}{current\PYZus{}F\PYZus{}norm} \PY{o}{=} \PY{n}{norm}\PY{p}{(}\PY{n}{F\PYZus{}func}\PY{p}{(}\PY{n}{x\PYZus{}k\PYZus{}plus\PYZus{}1}\PY{p}{)}\PY{p}{)}
        
        \PY{n}{iter\PYZus{}info} \PY{o}{=} \PY{p}{\PYZob{}}
            \PY{l+s+s1}{\PYZsq{}}\PY{l+s+s1}{k}\PY{l+s+s1}{\PYZsq{}}\PY{p}{:} \PY{n}{k}\PY{p}{,} 
            \PY{l+s+s1}{\PYZsq{}}\PY{l+s+s1}{x\PYZus{}k\PYZus{}plus\PYZus{}1}\PY{l+s+s1}{\PYZsq{}}\PY{p}{:} \PY{n}{x\PYZus{}k\PYZus{}plus\PYZus{}1}\PY{o}{.}\PY{n}{tolist}\PY{p}{(}\PY{p}{)}\PY{p}{,} 
            \PY{l+s+s1}{\PYZsq{}}\PY{l+s+s1}{norm\PYZus{}F\PYZus{}xk\PYZus{}plus\PYZus{}1}\PY{l+s+s1}{\PYZsq{}}\PY{p}{:} \PY{n}{current\PYZus{}F\PYZus{}norm}\PY{p}{,}
            \PY{l+s+s1}{\PYZsq{}}\PY{l+s+s1}{norm\PYZus{}delta\PYZus{}x}\PY{l+s+s1}{\PYZsq{}}\PY{p}{:} \PY{n}{diff\PYZus{}norm}\PY{p}{,}
            \PY{l+s+s1}{\PYZsq{}}\PY{l+s+s1}{status}\PY{l+s+s1}{\PYZsq{}}\PY{p}{:} \PY{l+s+s1}{\PYZsq{}}\PY{l+s+s1}{Iterating}\PY{l+s+s1}{\PYZsq{}}
        \PY{p}{\PYZcb{}}
        \PY{n}{iterations\PYZus{}summary}\PY{o}{.}\PY{n}{append}\PY{p}{(}\PY{n}{iter\PYZus{}info}\PY{p}{)}
        
        \PY{n+nb}{print}\PY{p}{(}\PY{l+s+sa}{f}\PY{l+s+s2}{\PYZdq{}}\PY{l+s+s2}{迭代 }\PY{l+s+si}{\PYZob{}}\PY{n}{k}\PY{l+s+si}{\PYZcb{}}\PY{l+s+s2}{: x\PYZca{}(}\PY{l+s+si}{\PYZob{}}\PY{n}{k}\PY{o}{+}\PY{l+m+mi}{1}\PY{l+s+si}{\PYZcb{}}\PY{l+s+s2}{) = }\PY{l+s+si}{\PYZob{}}\PY{n}{x\PYZus{}k\PYZus{}plus\PYZus{}1}\PY{l+s+si}{\PYZcb{}}\PY{l+s+s2}{, ||x\PYZca{}(}\PY{l+s+si}{\PYZob{}}\PY{n}{k}\PY{o}{+}\PY{l+m+mi}{1}\PY{l+s+si}{\PYZcb{}}\PY{l+s+s2}{)\PYZhy{}x\PYZca{}(}\PY{l+s+si}{\PYZob{}}\PY{n}{k}\PY{l+s+si}{\PYZcb{}}\PY{l+s+s2}{)|| = }\PY{l+s+si}{\PYZob{}}\PY{n}{diff\PYZus{}norm}\PY{l+s+si}{:}\PY{l+s+s2}{.2e}\PY{l+s+si}{\PYZcb{}}\PY{l+s+s2}{, ||F(x\PYZca{}(}\PY{l+s+si}{\PYZob{}}\PY{n}{k}\PY{o}{+}\PY{l+m+mi}{1}\PY{l+s+si}{\PYZcb{}}\PY{l+s+s2}{))|| = }\PY{l+s+si}{\PYZob{}}\PY{n}{current\PYZus{}F\PYZus{}norm}\PY{l+s+si}{:}\PY{l+s+s2}{.2e}\PY{l+s+si}{\PYZcb{}}\PY{l+s+s2}{\PYZdq{}}\PY{p}{)}
        
        \PY{n}{x\PYZus{}k} \PY{o}{=} \PY{n}{x\PYZus{}k\PYZus{}plus\PYZus{}1}

        \PY{k}{if} \PY{n}{diff\PYZus{}norm} \PY{o}{\PYZlt{}} \PY{n}{tol}\PY{p}{:}
            \PY{n+nb}{print}\PY{p}{(}\PY{l+s+sa}{f}\PY{l+s+s2}{\PYZdq{}}\PY{l+s+s2}{在 }\PY{l+s+si}{\PYZob{}}\PY{n}{k}\PY{o}{+}\PY{l+m+mi}{1}\PY{l+s+si}{\PYZcb{}}\PY{l+s+s2}{ 次迭代后达到收敛标准。}\PY{l+s+s2}{\PYZdq{}}\PY{p}{)}
            \PY{n}{iterations\PYZus{}summary}\PY{p}{[}\PY{o}{\PYZhy{}}\PY{l+m+mi}{1}\PY{p}{]}\PY{p}{[}\PY{l+s+s1}{\PYZsq{}}\PY{l+s+s1}{status}\PY{l+s+s1}{\PYZsq{}}\PY{p}{]} \PY{o}{=} \PY{l+s+s1}{\PYZsq{}}\PY{l+s+s1}{Converged}\PY{l+s+s1}{\PYZsq{}}
            \PY{k}{return} \PY{n}{x\PYZus{}k}\PY{p}{,} \PY{n}{iterations\PYZus{}summary}
            
    \PY{n+nb}{print}\PY{p}{(}\PY{l+s+sa}{f}\PY{l+s+s2}{\PYZdq{}}\PY{l+s+s2}{达到最大迭代次数 (}\PY{l+s+si}{\PYZob{}}\PY{n}{max\PYZus{}iter}\PY{l+s+si}{\PYZcb{}}\PY{l+s+s2}{) 未收敛。}\PY{l+s+s2}{\PYZdq{}}\PY{p}{)}
    \PY{n}{iterations\PYZus{}summary}\PY{p}{[}\PY{o}{\PYZhy{}}\PY{l+m+mi}{1}\PY{p}{]}\PY{p}{[}\PY{l+s+s1}{\PYZsq{}}\PY{l+s+s1}{status}\PY{l+s+s1}{\PYZsq{}}\PY{p}{]} \PY{o}{=} \PY{l+s+s1}{\PYZsq{}}\PY{l+s+s1}{Max iterations reached}\PY{l+s+s1}{\PYZsq{}}
    \PY{k}{return} \PY{n}{x\PYZus{}k}\PY{p}{,} \PY{n}{iterations\PYZus{}summary}

\PY{c+c1}{\PYZsh{} \PYZhy{}\PYZhy{}\PYZhy{} 使用至少三个不同的初值进行计算 \PYZhy{}\PYZhy{}\PYZhy{}}
\PY{n}{initial\PYZus{}guesses} \PY{o}{=} \PY{p}{[}
    \PY{p}{[}\PY{l+m+mf}{0.1}\PY{p}{,} \PY{l+m+mf}{0.1}\PY{p}{,} \PY{o}{\PYZhy{}}\PY{l+m+mf}{0.1}\PY{p}{]}\PY{p}{,}
    \PY{p}{[}\PY{l+m+mf}{0.5}\PY{p}{,} \PY{l+m+mf}{0.0}\PY{p}{,} \PY{o}{\PYZhy{}}\PY{l+m+mf}{0.5}\PY{p}{]}\PY{p}{,}
    \PY{p}{[}\PY{l+m+mf}{0.0}\PY{p}{,} \PY{l+m+mf}{0.0}\PY{p}{,} \PY{l+m+mf}{0.0}\PY{p}{]}
\PY{p}{]}

\PY{n}{all\PYZus{}results\PYZus{}newton} \PY{o}{=} \PY{p}{[}\PY{p}{]}

\PY{n+nb}{print}\PY{p}{(}\PY{l+s+s2}{\PYZdq{}}\PY{l+s+s2}{牛顿法求解非线性方程组:}\PY{l+s+se}{\PYZbs{}n}\PY{l+s+s2}{\PYZdq{}}\PY{p}{)}
\PY{k}{for} \PY{n}{i}\PY{p}{,} \PY{n}{x\PYZus{}init\PYZus{}guess} \PY{o+ow}{in} \PY{n+nb}{enumerate}\PY{p}{(}\PY{n}{initial\PYZus{}guesses}\PY{p}{)}\PY{p}{:}
    \PY{n+nb}{print}\PY{p}{(}\PY{l+s+sa}{f}\PY{l+s+s2}{\PYZdq{}}\PY{l+s+se}{\PYZbs{}n}\PY{l+s+s2}{\PYZhy{}\PYZhy{}\PYZhy{} 尝试初值 }\PY{l+s+si}{\PYZob{}}\PY{n}{i}\PY{o}{+}\PY{l+m+mi}{1}\PY{l+s+si}{\PYZcb{}}\PY{l+s+s2}{: }\PY{l+s+si}{\PYZob{}}\PY{n}{x\PYZus{}init\PYZus{}guess}\PY{l+s+si}{\PYZcb{}}\PY{l+s+s2}{ \PYZhy{}\PYZhy{}\PYZhy{}}\PY{l+s+s2}{\PYZdq{}}\PY{p}{)}
    \PY{n}{solution\PYZus{}vec}\PY{p}{,} \PY{n}{iter\PYZus{}data} \PY{o}{=} \PY{n}{newton\PYZus{}method}\PY{p}{(}\PY{n}{F}\PY{p}{,} \PY{n}{J}\PY{p}{,} \PY{n}{x\PYZus{}init\PYZus{}guess}\PY{p}{,} \PY{n}{tol}\PY{o}{=}\PY{l+m+mf}{1e\PYZhy{}8}\PY{p}{,} \PY{n}{max\PYZus{}iter}\PY{o}{=}\PY{l+m+mi}{50}\PY{p}{)}
    \PY{n}{all\PYZus{}results\PYZus{}newton}\PY{o}{.}\PY{n}{append}\PY{p}{(}\PY{p}{\PYZob{}}
        \PY{l+s+s1}{\PYZsq{}}\PY{l+s+s1}{initial\PYZus{}guess}\PY{l+s+s1}{\PYZsq{}}\PY{p}{:} \PY{n}{x\PYZus{}init\PYZus{}guess}\PY{p}{,}
        \PY{l+s+s1}{\PYZsq{}}\PY{l+s+s1}{solution}\PY{l+s+s1}{\PYZsq{}}\PY{p}{:} \PY{n}{solution\PYZus{}vec}\PY{o}{.}\PY{n}{tolist}\PY{p}{(}\PY{p}{)} \PY{k}{if} \PY{n}{solution\PYZus{}vec} \PY{o+ow}{is} \PY{o+ow}{not} \PY{k+kc}{None} \PY{k}{else} \PY{k+kc}{None}\PY{p}{,}
        \PY{l+s+s1}{\PYZsq{}}\PY{l+s+s1}{iterations\PYZus{}data}\PY{l+s+s1}{\PYZsq{}}\PY{p}{:} \PY{n}{iter\PYZus{}data}\PY{p}{,}
        \PY{l+s+s1}{\PYZsq{}}\PY{l+s+s1}{converged}\PY{l+s+s1}{\PYZsq{}}\PY{p}{:} \PY{n}{iter\PYZus{}data}\PY{p}{[}\PY{o}{\PYZhy{}}\PY{l+m+mi}{1}\PY{p}{]}\PY{p}{[}\PY{l+s+s1}{\PYZsq{}}\PY{l+s+s1}{status}\PY{l+s+s1}{\PYZsq{}}\PY{p}{]} \PY{o}{==} \PY{l+s+s1}{\PYZsq{}}\PY{l+s+s1}{Converged}\PY{l+s+s1}{\PYZsq{}} \PY{k}{if} \PY{n}{iter\PYZus{}data} \PY{k}{else} \PY{k+kc}{False}\PY{p}{,}
        \PY{l+s+s1}{\PYZsq{}}\PY{l+s+s1}{num\PYZus{}iterations}\PY{l+s+s1}{\PYZsq{}}\PY{p}{:} \PY{n+nb}{len}\PY{p}{(}\PY{n}{iter\PYZus{}data}\PY{p}{)} \PY{k}{if} \PY{n}{iter\PYZus{}data} \PY{k}{else} \PY{l+m+mi}{0}\PY{p}{,}
        \PY{l+s+s1}{\PYZsq{}}\PY{l+s+s1}{final\PYZus{}norm\PYZus{}delta\PYZus{}x}\PY{l+s+s1}{\PYZsq{}}\PY{p}{:} \PY{n}{iter\PYZus{}data}\PY{p}{[}\PY{o}{\PYZhy{}}\PY{l+m+mi}{1}\PY{p}{]}\PY{p}{[}\PY{l+s+s1}{\PYZsq{}}\PY{l+s+s1}{norm\PYZus{}delta\PYZus{}x}\PY{l+s+s1}{\PYZsq{}}\PY{p}{]} \PY{k}{if} \PY{n}{iter\PYZus{}data} \PY{o+ow}{and} \PY{n}{iter\PYZus{}data}\PY{p}{[}\PY{o}{\PYZhy{}}\PY{l+m+mi}{1}\PY{p}{]}\PY{p}{[}\PY{l+s+s1}{\PYZsq{}}\PY{l+s+s1}{norm\PYZus{}delta\PYZus{}x}\PY{l+s+s1}{\PYZsq{}}\PY{p}{]} \PY{o}{!=} \PY{l+s+s1}{\PYZsq{}}\PY{l+s+s1}{N/A (Singular Jacobian)}\PY{l+s+s1}{\PYZsq{}} \PY{k}{else} \PY{k+kc}{None}\PY{p}{,}
        \PY{l+s+s1}{\PYZsq{}}\PY{l+s+s1}{final\PYZus{}norm\PYZus{}F}\PY{l+s+s1}{\PYZsq{}}\PY{p}{:} \PY{n}{iter\PYZus{}data}\PY{p}{[}\PY{o}{\PYZhy{}}\PY{l+m+mi}{1}\PY{p}{]}\PY{p}{[}\PY{l+s+s1}{\PYZsq{}}\PY{l+s+s1}{norm\PYZus{}F\PYZus{}xk\PYZus{}plus\PYZus{}1}\PY{l+s+s1}{\PYZsq{}}\PY{p}{]} \PY{k}{if} \PY{n}{iter\PYZus{}data} \PY{o+ow}{and} \PY{l+s+s1}{\PYZsq{}}\PY{l+s+s1}{norm\PYZus{}F\PYZus{}xk\PYZus{}plus\PYZus{}1}\PY{l+s+s1}{\PYZsq{}} \PY{o+ow}{in} \PY{n}{iter\PYZus{}data}\PY{p}{[}\PY{o}{\PYZhy{}}\PY{l+m+mi}{1}\PY{p}{]} \PY{k}{else} \PY{p}{(}\PY{n}{norm}\PY{p}{(}\PY{n}{F}\PY{p}{(}\PY{n}{solution\PYZus{}vec}\PY{p}{)}\PY{p}{)} \PY{k}{if} \PY{n}{solution\PYZus{}vec} \PY{o+ow}{is} \PY{o+ow}{not} \PY{k+kc}{None} \PY{k}{else} \PY{k+kc}{None}\PY{p}{)}
    \PY{p}{\PYZcb{}}\PY{p}{)}
    \PY{k}{if} \PY{n}{solution\PYZus{}vec} \PY{o+ow}{is} \PY{o+ow}{not} \PY{k+kc}{None}\PY{p}{:}
        \PY{n+nb}{print}\PY{p}{(}\PY{l+s+sa}{f}\PY{l+s+s2}{\PYZdq{}}\PY{l+s+s2}{初值 }\PY{l+s+si}{\PYZob{}}\PY{n}{x\PYZus{}init\PYZus{}guess}\PY{l+s+si}{\PYZcb{}}\PY{l+s+s2}{ 的解: }\PY{l+s+si}{\PYZob{}}\PY{n}{solution\PYZus{}vec}\PY{l+s+si}{\PYZcb{}}\PY{l+s+s2}{\PYZdq{}}\PY{p}{)}
        \PY{n+nb}{print}\PY{p}{(}\PY{l+s+sa}{f}\PY{l+s+s2}{\PYZdq{}}\PY{l+s+s2}{解的函数范数 ||F(解)||: }\PY{l+s+si}{\PYZob{}}\PY{n}{norm}\PY{p}{(}\PY{n}{F}\PY{p}{(}\PY{n}{solution\PYZus{}vec}\PY{p}{)}\PY{p}{)}\PY{l+s+si}{:}\PY{l+s+s2}{.2e}\PY{l+s+si}{\PYZcb{}}\PY{l+s+s2}{\PYZdq{}}\PY{p}{)}
    \PY{k}{else}\PY{p}{:}
        \PY{n+nb}{print}\PY{p}{(}\PY{l+s+sa}{f}\PY{l+s+s2}{\PYZdq{}}\PY{l+s+s2}{初值 }\PY{l+s+si}{\PYZob{}}\PY{n}{x\PYZus{}init\PYZus{}guess}\PY{l+s+si}{\PYZcb{}}\PY{l+s+s2}{ 未找到解。}\PY{l+s+s2}{\PYZdq{}}\PY{p}{)}
    \PY{n+nb}{print}\PY{p}{(}\PY{l+s+s2}{\PYZdq{}}\PY{l+s+s2}{\PYZhy{}}\PY{l+s+s2}{\PYZdq{}} \PY{o}{*} \PY{l+m+mi}{50}\PY{p}{)}
\end{Verbatim}
\end{tcolorbox}

    \begin{Verbatim}[commandchars=\\\{\}]
牛顿法求解非线性方程组:


--- 尝试初值 1: [0.1, 0.1, -0.1] ---
初始猜测值 x\^{}(0): [ 0.1  0.1 -0.1]
迭代 0: x\^{}(1) = [ 0.4998697  0.0194668 -0.5215205], ||x\^{}(1)-x\^{}(0)|| = 5.87e-01,
||F(x\^{}(1))|| = 3.46e-01
迭代 1: x\^{}(2) = [ 0.5000142  0.0015886 -0.523557 ], ||x\^{}(2)-x\^{}(1)|| = 1.80e-02,
||F(x\^{}(2))|| = 2.59e-02
迭代 2: x\^{}(3) = [ 0.5000001  0.0000124 -0.5235985], ||x\^{}(3)-x\^{}(2)|| = 1.58e-03,
||F(x\^{}(3))|| = 2.01e-04
迭代 3: x\^{}(4) = [ 0.5        0.        -0.5235988], ||x\^{}(4)-x\^{}(3)|| = 1.24e-05,
||F(x\^{}(4))|| = 1.25e-08
迭代 4: x\^{}(5) = [ 0.5        0.        -0.5235988], ||x\^{}(5)-x\^{}(4)|| = 7.76e-10,
||F(x\^{}(5))|| = 1.78e-15
在 5 次迭代后达到收敛标准。
初值 [0.1, 0.1, -0.1] 的解: [ 0.5        0.        -0.5235988]
解的函数范数 ||F(解)||: 1.78e-15
--------------------------------------------------

--- 尝试初值 2: [0.5, 0.0, -0.5] ---
初始猜测值 x\^{}(0): [ 0.5  0.  -0.5]
迭代 0: x\^{}(1) = [ 0.5       -0.0000084 -0.523599 ], ||x\^{}(1)-x\^{}(0)|| = 2.36e-02,
||F(x\^{}(1))|| = 1.35e-04
迭代 1: x\^{}(2) = [ 0.5        0.        -0.5235988], ||x\^{}(2)-x\^{}(1)|| = 8.37e-06,
||F(x\^{}(2))|| = 5.68e-09
迭代 2: x\^{}(3) = [ 0.5       -0.        -0.5235988], ||x\^{}(3)-x\^{}(2)|| = 3.51e-10,
||F(x\^{}(3))|| = 1.78e-15
在 3 次迭代后达到收敛标准。
初值 [0.5, 0.0, -0.5] 的解: [ 0.5       -0.        -0.5235988]
解的函数范数 ||F(解)||: 1.78e-15
--------------------------------------------------

--- 尝试初值 3: [0.0, 0.0, 0.0] ---
初始猜测值 x\^{}(0): [0. 0. 0.]
迭代 0: x\^{}(1) = [ 0.5       -0.0168888 -0.5235988], ||x\^{}(1)-x\^{}(0)|| = 7.24e-01,
||F(x\^{}(1))|| = 2.51e-01
迭代 1: x\^{}(2) = [ 0.5000157  0.00172   -0.5235536], ||x\^{}(2)-x\^{}(1)|| = 1.86e-02,
||F(x\^{}(2))|| = 2.80e-02
迭代 2: x\^{}(3) = [ 0.5000001  0.0000146 -0.5235984], ||x\^{}(3)-x\^{}(2)|| = 1.71e-03,
||F(x\^{}(3))|| = 2.36e-04
迭代 3: x\^{}(4) = [ 0.5        0.        -0.5235988], ||x\^{}(4)-x\^{}(3)|| = 1.46e-05,
||F(x\^{}(4))|| = 1.72e-08
迭代 4: x\^{}(5) = [ 0.5       -0.        -0.5235988], ||x\^{}(5)-x\^{}(4)|| = 1.06e-09,
||F(x\^{}(5))|| = 1.80e-15
在 5 次迭代后达到收敛标准。
初值 [0.0, 0.0, 0.0] 的解: [ 0.5       -0.        -0.5235988]
解的函数范数 ||F(解)||: 1.80e-15
--------------------------------------------------
    \end{Verbatim}

    \subsubsection{计算结果}\label{ux8ba1ux7b97ux7ed3ux679c}

以下表格总结了使用不同初始值进行牛顿法迭代的结果:

\begin{longtable}[]{@{}
  >{\raggedright\arraybackslash}p{(\linewidth - 10\tabcolsep) * \real{0.1579}}
  >{\raggedright\arraybackslash}p{(\linewidth - 10\tabcolsep) * \real{0.0421}}
  >{\raggedright\arraybackslash}p{(\linewidth - 10\tabcolsep) * \real{0.0632}}
  >{\raggedright\arraybackslash}p{(\linewidth - 10\tabcolsep) * \real{0.1947}}
  >{\raggedright\arraybackslash}p{(\linewidth - 10\tabcolsep) * \real{0.2789}}
  >{\raggedright\arraybackslash}p{(\linewidth - 10\tabcolsep) * \real{0.2632}}@{}}
\toprule\noalign{}
\begin{minipage}[b]{\linewidth}\raggedright
初始猜测值 \(\mathbf{x}^{(0)}\)
\end{minipage} & \begin{minipage}[b]{\linewidth}\raggedright
是否收敛
\end{minipage} & \begin{minipage}[b]{\linewidth}\raggedright
迭代次数 \(k\)
\end{minipage} & \begin{minipage}[b]{\linewidth}\raggedright
最终解 \(\mathbf{x}^{(k)}\) (近似值)
\end{minipage} & \begin{minipage}[b]{\linewidth}\raggedright
最终 \(||\mathbf{x}^{(k)} - \mathbf{x}^{(k-1)}||_2\)
\end{minipage} & \begin{minipage}[b]{\linewidth}\raggedright
最终 \(||\mathbf{F}(\mathbf{x}^{(k)})||_2\)
\end{minipage} \\
\midrule\noalign{}
\endhead
\bottomrule\noalign{}
\endlastfoot
\texttt{{[}0.1,\ 0.1,\ -0.1{]}} & 是 & 5 &
\texttt{{[}0.5,\ 3.36e-18,\ -0.52359878{]}} & \texttt{7.76e-10} &
\texttt{1.78e-15} \\
\texttt{{[}0.5,\ 0.0,\ -0.5{]}} & 是 & 3 &
\texttt{{[}0.5,\ -1.81e-18,\ -0.52359878{]}} & \texttt{3.51e-10} &
\texttt{1.78e-15} \\
\texttt{{[}0.0,\ 0.0,\ 0.0{]}} & 是 & 5 &
\texttt{{[}0.5,\ -8.88e-19,\ -0.52359878{]}} & \texttt{1.06e-09} &
\texttt{1.80e-15} \\
\end{longtable}

\emph{注:最终解 \(\mathbf{x}^{(k)}\) 中的 \(x_2\)
值极小,接近于0。\(x_3 \approx -0.5235987755982989 \approx -\pi/6\)。}

\subsubsection{结果分析}\label{ux7ed3ux679cux5206ux6790}

\begin{itemize}
\tightlist
\item
  \textbf{收敛性}:

  \begin{itemize}
  \tightlist
  \item
    对于初始值 \texttt{{[}0.1,\ 0.1,\ -0.1{]}},牛顿法在 5
    次迭代后收敛。
  \item
    对于初始值 \texttt{{[}0.5,\ 0.0,\ -0.5{]}},牛顿法在 3
    次迭代后收敛。
  \item
    对于初始值 \texttt{{[}0.0,\ 0.0,\ 0.0{]}},牛顿法在 5 次迭代后收敛。
    所有测试的初始值都成功收敛。
  \end{itemize}
\item
  \textbf{解的唯一性}:

  \begin{itemize}
  \tightlist
  \item
    所有三个不同的初始猜测值都收敛到了相同的解(在数值精度范围内):
    \(\mathbf{x} \approx [0.5, 0.0, -0.5235987755982989]^T\)。 其中
    \(x_1 = 0.5\), \(x_2 \approx 0\), \(x_3 = -\frac{\pi}{6}\)。
  \item
    这强烈表明该非线性方程组在这些初始值所在的吸引域内有一个唯一的、稳定的实数解。
  \end{itemize}
\item
  \textbf{收敛速度}:

  \begin{itemize}
  \tightlist
  \item
    牛顿法在所有情况下都表现出快速的收敛特性。迭代次数(3到5次)非常少,这符合牛顿法在良好初值和非奇异雅可比矩阵条件下的二次收敛特性。
  \end{itemize}
\item
  \textbf{最终误差}:

  \begin{itemize}
  \tightlist
  \item
    所有收敛的解都满足停止准则
    \(||\mathbf{x}^{(k)} - \mathbf{x}^{(k-1)}||_2 < 10^{-8}\)。具体值分别为
    \texttt{7.76e-10}, \texttt{3.51e-10}, 和 \texttt{1.06e-09},均小于
    \(10^{-8}\)。
  \item
    同时,最终解代入原方程组后得到的范数
    \(||\mathbf{F}(\mathbf{x}^{(k)})||_2\) 也非常小(约为
    \texttt{1.8e-15}),这表明所找到的解非常精确地满足了原方程组。
  \end{itemize}
\end{itemize}

\textbf{结论}
对于给定的非线性方程组,牛顿法在测试的三个不同初始值下均表现出良好的收敛性,并且都收敛到了同一个解
\(\mathbf{x} \approx [0.5, 0.0, -\pi/6]^T\)。这表明该解是数值稳定的,并且牛顿法对于此问题和这些初值是有效的。迭代次数少,最终误差小,验证了牛顿法的有效性和快速收敛性。


    % Add a bibliography block to the postdoc
    
    
    
\end{document}
