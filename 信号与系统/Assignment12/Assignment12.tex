\documentclass[11pt]{article}

    \usepackage[breakable]{tcolorbox}
    \usepackage{parskip} % Stop auto-indenting (to mimic markdown behaviour)
    \usepackage{xeCJK}
    

    % Basic figure setup, for now with no caption control since it's done
    % automatically by Pandoc (which extracts ![](path) syntax from Markdown).
    \usepackage{graphicx}
    % Keep aspect ratio if custom image width or height is specified
    \setkeys{Gin}{keepaspectratio}
    % Maintain compatibility with old templates. Remove in nbconvert 6.0
    \let\Oldincludegraphics\includegraphics
    % Ensure that by default, figures have no caption (until we provide a
    % proper Figure object with a Caption API and a way to capture that
    % in the conversion process - todo).
    \usepackage{caption}
    \DeclareCaptionFormat{nocaption}{}
    \captionsetup{format=nocaption,aboveskip=0pt,belowskip=0pt}

    \usepackage{float}
    \floatplacement{figure}{H} % forces figures to be placed at the correct location
    \usepackage{xcolor} % Allow colors to be defined
    \usepackage{enumerate} % Needed for markdown enumerations to work
    \usepackage{geometry} % Used to adjust the document margins
    \usepackage{amsmath} % Equations
    \usepackage{amssymb} % Equations
    \usepackage{textcomp} % defines textquotesingle
    % Hack from http://tex.stackexchange.com/a/47451/13684:
    \AtBeginDocument{%
        \def\PYZsq{\textquotesingle}% Upright quotes in Pygmentized code
    }
    \usepackage{upquote} % Upright quotes for verbatim code
    \usepackage{eurosym} % defines \euro

    \usepackage{iftex}
    \ifPDFTeX
        \usepackage[T1]{fontenc}
        \IfFileExists{alphabeta.sty}{
              \usepackage{alphabeta}
          }{
              \usepackage[mathletters]{ucs}
              \usepackage[utf8x]{inputenc}
          }
    \else
        \usepackage{fontspec}
        \usepackage{unicode-math}
    \fi

    \usepackage{fancyvrb} % verbatim replacement that allows latex
    \usepackage{grffile} % extends the file name processing of package graphics
                         % to support a larger range
    \makeatletter % fix for old versions of grffile with XeLaTeX
    \@ifpackagelater{grffile}{2019/11/01}
    {
      % Do nothing on new versions
    }
    {
      \def\Gread@@xetex#1{%
        \IfFileExists{"\Gin@base".bb}%
        {\Gread@eps{\Gin@base.bb}}%
        {\Gread@@xetex@aux#1}%
      }
    }
    \makeatother
    \usepackage[Export]{adjustbox} % Used to constrain images to a maximum size
    \adjustboxset{max size={0.9\linewidth}{0.9\paperheight}}

    % The hyperref package gives us a pdf with properly built
    % internal navigation ('pdf bookmarks' for the table of contents,
    % internal cross-reference links, web links for URLs, etc.)
    \usepackage{hyperref}
    % The default LaTeX title has an obnoxious amount of whitespace. By default,
    % titling removes some of it. It also provides customization options.
    \usepackage{titling}
    \usepackage{longtable} % longtable support required by pandoc >1.10
    \usepackage{booktabs}  % table support for pandoc > 1.12.2
    \usepackage{array}     % table support for pandoc >= 2.11.3
    \usepackage{calc}      % table minipage width calculation for pandoc >= 2.11.1
    \usepackage[inline]{enumitem} % IRkernel/repr support (it uses the enumerate* environment)
    \usepackage[normalem]{ulem} % ulem is needed to support strikethroughs (\sout)
                                % normalem makes italics be italics, not underlines
    \usepackage{soul}      % strikethrough (\st) support for pandoc >= 3.0.0
    \usepackage{mathrsfs}
    

    
    % Colors for the hyperref package
    \definecolor{urlcolor}{rgb}{0,.145,.698}
    \definecolor{linkcolor}{rgb}{.71,0.21,0.01}
    \definecolor{citecolor}{rgb}{.12,.54,.11}

    % ANSI colors
    \definecolor{ansi-black}{HTML}{3E424D}
    \definecolor{ansi-black-intense}{HTML}{282C36}
    \definecolor{ansi-red}{HTML}{E75C58}
    \definecolor{ansi-red-intense}{HTML}{B22B31}
    \definecolor{ansi-green}{HTML}{00A250}
    \definecolor{ansi-green-intense}{HTML}{007427}
    \definecolor{ansi-yellow}{HTML}{DDB62B}
    \definecolor{ansi-yellow-intense}{HTML}{B27D12}
    \definecolor{ansi-blue}{HTML}{208FFB}
    \definecolor{ansi-blue-intense}{HTML}{0065CA}
    \definecolor{ansi-magenta}{HTML}{D160C4}
    \definecolor{ansi-magenta-intense}{HTML}{A03196}
    \definecolor{ansi-cyan}{HTML}{60C6C8}
    \definecolor{ansi-cyan-intense}{HTML}{258F8F}
    \definecolor{ansi-white}{HTML}{C5C1B4}
    \definecolor{ansi-white-intense}{HTML}{A1A6B2}
    \definecolor{ansi-default-inverse-fg}{HTML}{FFFFFF}
    \definecolor{ansi-default-inverse-bg}{HTML}{000000}

    % common color for the border for error outputs.
    \definecolor{outerrorbackground}{HTML}{FFDFDF}

    % commands and environments needed by pandoc snippets
    % extracted from the output of `pandoc -s`
    \providecommand{\tightlist}{%
      \setlength{\itemsep}{0pt}\setlength{\parskip}{0pt}}
    \DefineVerbatimEnvironment{Highlighting}{Verbatim}{commandchars=\\\{\}}
    % Add ',fontsize=\small' for more characters per line
    \newenvironment{Shaded}{}{}
    \newcommand{\KeywordTok}[1]{\textcolor[rgb]{0.00,0.44,0.13}{\textbf{{#1}}}}
    \newcommand{\DataTypeTok}[1]{\textcolor[rgb]{0.56,0.13,0.00}{{#1}}}
    \newcommand{\DecValTok}[1]{\textcolor[rgb]{0.25,0.63,0.44}{{#1}}}
    \newcommand{\BaseNTok}[1]{\textcolor[rgb]{0.25,0.63,0.44}{{#1}}}
    \newcommand{\FloatTok}[1]{\textcolor[rgb]{0.25,0.63,0.44}{{#1}}}
    \newcommand{\CharTok}[1]{\textcolor[rgb]{0.25,0.44,0.63}{{#1}}}
    \newcommand{\StringTok}[1]{\textcolor[rgb]{0.25,0.44,0.63}{{#1}}}
    \newcommand{\CommentTok}[1]{\textcolor[rgb]{0.38,0.63,0.69}{\textit{{#1}}}}
    \newcommand{\OtherTok}[1]{\textcolor[rgb]{0.00,0.44,0.13}{{#1}}}
    \newcommand{\AlertTok}[1]{\textcolor[rgb]{1.00,0.00,0.00}{\textbf{{#1}}}}
    \newcommand{\FunctionTok}[1]{\textcolor[rgb]{0.02,0.16,0.49}{{#1}}}
    \newcommand{\RegionMarkerTok}[1]{{#1}}
    \newcommand{\ErrorTok}[1]{\textcolor[rgb]{1.00,0.00,0.00}{\textbf{{#1}}}}
    \newcommand{\NormalTok}[1]{{#1}}

    % Additional commands for more recent versions of Pandoc
    \newcommand{\ConstantTok}[1]{\textcolor[rgb]{0.53,0.00,0.00}{{#1}}}
    \newcommand{\SpecialCharTok}[1]{\textcolor[rgb]{0.25,0.44,0.63}{{#1}}}
    \newcommand{\VerbatimStringTok}[1]{\textcolor[rgb]{0.25,0.44,0.63}{{#1}}}
    \newcommand{\SpecialStringTok}[1]{\textcolor[rgb]{0.73,0.40,0.53}{{#1}}}
    \newcommand{\ImportTok}[1]{{#1}}
    \newcommand{\DocumentationTok}[1]{\textcolor[rgb]{0.73,0.13,0.13}{\textit{{#1}}}}
    \newcommand{\AnnotationTok}[1]{\textcolor[rgb]{0.38,0.63,0.69}{\textbf{\textit{{#1}}}}}
    \newcommand{\CommentVarTok}[1]{\textcolor[rgb]{0.38,0.63,0.69}{\textbf{\textit{{#1}}}}}
    \newcommand{\VariableTok}[1]{\textcolor[rgb]{0.10,0.09,0.49}{{#1}}}
    \newcommand{\ControlFlowTok}[1]{\textcolor[rgb]{0.00,0.44,0.13}{\textbf{{#1}}}}
    \newcommand{\OperatorTok}[1]{\textcolor[rgb]{0.40,0.40,0.40}{{#1}}}
    \newcommand{\BuiltInTok}[1]{{#1}}
    \newcommand{\ExtensionTok}[1]{{#1}}
    \newcommand{\PreprocessorTok}[1]{\textcolor[rgb]{0.74,0.48,0.00}{{#1}}}
    \newcommand{\AttributeTok}[1]{\textcolor[rgb]{0.49,0.56,0.16}{{#1}}}
    \newcommand{\InformationTok}[1]{\textcolor[rgb]{0.38,0.63,0.69}{\textbf{\textit{{#1}}}}}
    \newcommand{\WarningTok}[1]{\textcolor[rgb]{0.38,0.63,0.69}{\textbf{\textit{{#1}}}}}


    % Define a nice break command that doesn't care if a line doesn't already
    % exist.
    \def\br{\hspace*{\fill} \\* }
    % Math Jax compatibility definitions
    \def\gt{>}
    \def\lt{<}
    \let\Oldtex\TeX
    \let\Oldlatex\LaTeX
    \renewcommand{\TeX}{\textrm{\Oldtex}}
    \renewcommand{\LaTeX}{\textrm{\Oldlatex}}
    % Document parameters
    % Document title
    \title{Assignment12}
    
    
    
    
    
    
    
% Pygments definitions
\makeatletter
\def\PY@reset{\let\PY@it=\relax \let\PY@bf=\relax%
    \let\PY@ul=\relax \let\PY@tc=\relax%
    \let\PY@bc=\relax \let\PY@ff=\relax}
\def\PY@tok#1{\csname PY@tok@#1\endcsname}
\def\PY@toks#1+{\ifx\relax#1\empty\else%
    \PY@tok{#1}\expandafter\PY@toks\fi}
\def\PY@do#1{\PY@bc{\PY@tc{\PY@ul{%
    \PY@it{\PY@bf{\PY@ff{#1}}}}}}}
\def\PY#1#2{\PY@reset\PY@toks#1+\relax+\PY@do{#2}}

\@namedef{PY@tok@w}{\def\PY@tc##1{\textcolor[rgb]{0.73,0.73,0.73}{##1}}}
\@namedef{PY@tok@c}{\let\PY@it=\textit\def\PY@tc##1{\textcolor[rgb]{0.24,0.48,0.48}{##1}}}
\@namedef{PY@tok@cp}{\def\PY@tc##1{\textcolor[rgb]{0.61,0.40,0.00}{##1}}}
\@namedef{PY@tok@k}{\let\PY@bf=\textbf\def\PY@tc##1{\textcolor[rgb]{0.00,0.50,0.00}{##1}}}
\@namedef{PY@tok@kp}{\def\PY@tc##1{\textcolor[rgb]{0.00,0.50,0.00}{##1}}}
\@namedef{PY@tok@kt}{\def\PY@tc##1{\textcolor[rgb]{0.69,0.00,0.25}{##1}}}
\@namedef{PY@tok@o}{\def\PY@tc##1{\textcolor[rgb]{0.40,0.40,0.40}{##1}}}
\@namedef{PY@tok@ow}{\let\PY@bf=\textbf\def\PY@tc##1{\textcolor[rgb]{0.67,0.13,1.00}{##1}}}
\@namedef{PY@tok@nb}{\def\PY@tc##1{\textcolor[rgb]{0.00,0.50,0.00}{##1}}}
\@namedef{PY@tok@nf}{\def\PY@tc##1{\textcolor[rgb]{0.00,0.00,1.00}{##1}}}
\@namedef{PY@tok@nc}{\let\PY@bf=\textbf\def\PY@tc##1{\textcolor[rgb]{0.00,0.00,1.00}{##1}}}
\@namedef{PY@tok@nn}{\let\PY@bf=\textbf\def\PY@tc##1{\textcolor[rgb]{0.00,0.00,1.00}{##1}}}
\@namedef{PY@tok@ne}{\let\PY@bf=\textbf\def\PY@tc##1{\textcolor[rgb]{0.80,0.25,0.22}{##1}}}
\@namedef{PY@tok@nv}{\def\PY@tc##1{\textcolor[rgb]{0.10,0.09,0.49}{##1}}}
\@namedef{PY@tok@no}{\def\PY@tc##1{\textcolor[rgb]{0.53,0.00,0.00}{##1}}}
\@namedef{PY@tok@nl}{\def\PY@tc##1{\textcolor[rgb]{0.46,0.46,0.00}{##1}}}
\@namedef{PY@tok@ni}{\let\PY@bf=\textbf\def\PY@tc##1{\textcolor[rgb]{0.44,0.44,0.44}{##1}}}
\@namedef{PY@tok@na}{\def\PY@tc##1{\textcolor[rgb]{0.41,0.47,0.13}{##1}}}
\@namedef{PY@tok@nt}{\let\PY@bf=\textbf\def\PY@tc##1{\textcolor[rgb]{0.00,0.50,0.00}{##1}}}
\@namedef{PY@tok@nd}{\def\PY@tc##1{\textcolor[rgb]{0.67,0.13,1.00}{##1}}}
\@namedef{PY@tok@s}{\def\PY@tc##1{\textcolor[rgb]{0.73,0.13,0.13}{##1}}}
\@namedef{PY@tok@sd}{\let\PY@it=\textit\def\PY@tc##1{\textcolor[rgb]{0.73,0.13,0.13}{##1}}}
\@namedef{PY@tok@si}{\let\PY@bf=\textbf\def\PY@tc##1{\textcolor[rgb]{0.64,0.35,0.47}{##1}}}
\@namedef{PY@tok@se}{\let\PY@bf=\textbf\def\PY@tc##1{\textcolor[rgb]{0.67,0.36,0.12}{##1}}}
\@namedef{PY@tok@sr}{\def\PY@tc##1{\textcolor[rgb]{0.64,0.35,0.47}{##1}}}
\@namedef{PY@tok@ss}{\def\PY@tc##1{\textcolor[rgb]{0.10,0.09,0.49}{##1}}}
\@namedef{PY@tok@sx}{\def\PY@tc##1{\textcolor[rgb]{0.00,0.50,0.00}{##1}}}
\@namedef{PY@tok@m}{\def\PY@tc##1{\textcolor[rgb]{0.40,0.40,0.40}{##1}}}
\@namedef{PY@tok@gh}{\let\PY@bf=\textbf\def\PY@tc##1{\textcolor[rgb]{0.00,0.00,0.50}{##1}}}
\@namedef{PY@tok@gu}{\let\PY@bf=\textbf\def\PY@tc##1{\textcolor[rgb]{0.50,0.00,0.50}{##1}}}
\@namedef{PY@tok@gd}{\def\PY@tc##1{\textcolor[rgb]{0.63,0.00,0.00}{##1}}}
\@namedef{PY@tok@gi}{\def\PY@tc##1{\textcolor[rgb]{0.00,0.52,0.00}{##1}}}
\@namedef{PY@tok@gr}{\def\PY@tc##1{\textcolor[rgb]{0.89,0.00,0.00}{##1}}}
\@namedef{PY@tok@ge}{\let\PY@it=\textit}
\@namedef{PY@tok@gs}{\let\PY@bf=\textbf}
\@namedef{PY@tok@gp}{\let\PY@bf=\textbf\def\PY@tc##1{\textcolor[rgb]{0.00,0.00,0.50}{##1}}}
\@namedef{PY@tok@go}{\def\PY@tc##1{\textcolor[rgb]{0.44,0.44,0.44}{##1}}}
\@namedef{PY@tok@gt}{\def\PY@tc##1{\textcolor[rgb]{0.00,0.27,0.87}{##1}}}
\@namedef{PY@tok@err}{\def\PY@bc##1{{\setlength{\fboxsep}{\string -\fboxrule}\fcolorbox[rgb]{1.00,0.00,0.00}{1,1,1}{\strut ##1}}}}
\@namedef{PY@tok@kc}{\let\PY@bf=\textbf\def\PY@tc##1{\textcolor[rgb]{0.00,0.50,0.00}{##1}}}
\@namedef{PY@tok@kd}{\let\PY@bf=\textbf\def\PY@tc##1{\textcolor[rgb]{0.00,0.50,0.00}{##1}}}
\@namedef{PY@tok@kn}{\let\PY@bf=\textbf\def\PY@tc##1{\textcolor[rgb]{0.00,0.50,0.00}{##1}}}
\@namedef{PY@tok@kr}{\let\PY@bf=\textbf\def\PY@tc##1{\textcolor[rgb]{0.00,0.50,0.00}{##1}}}
\@namedef{PY@tok@bp}{\def\PY@tc##1{\textcolor[rgb]{0.00,0.50,0.00}{##1}}}
\@namedef{PY@tok@fm}{\def\PY@tc##1{\textcolor[rgb]{0.00,0.00,1.00}{##1}}}
\@namedef{PY@tok@vc}{\def\PY@tc##1{\textcolor[rgb]{0.10,0.09,0.49}{##1}}}
\@namedef{PY@tok@vg}{\def\PY@tc##1{\textcolor[rgb]{0.10,0.09,0.49}{##1}}}
\@namedef{PY@tok@vi}{\def\PY@tc##1{\textcolor[rgb]{0.10,0.09,0.49}{##1}}}
\@namedef{PY@tok@vm}{\def\PY@tc##1{\textcolor[rgb]{0.10,0.09,0.49}{##1}}}
\@namedef{PY@tok@sa}{\def\PY@tc##1{\textcolor[rgb]{0.73,0.13,0.13}{##1}}}
\@namedef{PY@tok@sb}{\def\PY@tc##1{\textcolor[rgb]{0.73,0.13,0.13}{##1}}}
\@namedef{PY@tok@sc}{\def\PY@tc##1{\textcolor[rgb]{0.73,0.13,0.13}{##1}}}
\@namedef{PY@tok@dl}{\def\PY@tc##1{\textcolor[rgb]{0.73,0.13,0.13}{##1}}}
\@namedef{PY@tok@s2}{\def\PY@tc##1{\textcolor[rgb]{0.73,0.13,0.13}{##1}}}
\@namedef{PY@tok@sh}{\def\PY@tc##1{\textcolor[rgb]{0.73,0.13,0.13}{##1}}}
\@namedef{PY@tok@s1}{\def\PY@tc##1{\textcolor[rgb]{0.73,0.13,0.13}{##1}}}
\@namedef{PY@tok@mb}{\def\PY@tc##1{\textcolor[rgb]{0.40,0.40,0.40}{##1}}}
\@namedef{PY@tok@mf}{\def\PY@tc##1{\textcolor[rgb]{0.40,0.40,0.40}{##1}}}
\@namedef{PY@tok@mh}{\def\PY@tc##1{\textcolor[rgb]{0.40,0.40,0.40}{##1}}}
\@namedef{PY@tok@mi}{\def\PY@tc##1{\textcolor[rgb]{0.40,0.40,0.40}{##1}}}
\@namedef{PY@tok@il}{\def\PY@tc##1{\textcolor[rgb]{0.40,0.40,0.40}{##1}}}
\@namedef{PY@tok@mo}{\def\PY@tc##1{\textcolor[rgb]{0.40,0.40,0.40}{##1}}}
\@namedef{PY@tok@ch}{\let\PY@it=\textit\def\PY@tc##1{\textcolor[rgb]{0.24,0.48,0.48}{##1}}}
\@namedef{PY@tok@cm}{\let\PY@it=\textit\def\PY@tc##1{\textcolor[rgb]{0.24,0.48,0.48}{##1}}}
\@namedef{PY@tok@cpf}{\let\PY@it=\textit\def\PY@tc##1{\textcolor[rgb]{0.24,0.48,0.48}{##1}}}
\@namedef{PY@tok@c1}{\let\PY@it=\textit\def\PY@tc##1{\textcolor[rgb]{0.24,0.48,0.48}{##1}}}
\@namedef{PY@tok@cs}{\let\PY@it=\textit\def\PY@tc##1{\textcolor[rgb]{0.24,0.48,0.48}{##1}}}

\def\PYZbs{\char`\\}
\def\PYZus{\char`\_}
\def\PYZob{\char`\{}
\def\PYZcb{\char`\}}
\def\PYZca{\char`\^}
\def\PYZam{\char`\&}
\def\PYZlt{\char`\<}
\def\PYZgt{\char`\>}
\def\PYZsh{\char`\#}
\def\PYZpc{\char`\%}
\def\PYZdl{\char`\$}
\def\PYZhy{\char`\-}
\def\PYZsq{\char`\'}
\def\PYZdq{\char`\"}
\def\PYZti{\char`\~}
% for compatibility with earlier versions
\def\PYZat{@}
\def\PYZlb{[}
\def\PYZrb{]}
\makeatother


    % For linebreaks inside Verbatim environment from package fancyvrb.
    \makeatletter
        \newbox\Wrappedcontinuationbox
        \newbox\Wrappedvisiblespacebox
        \newcommand*\Wrappedvisiblespace {\textcolor{red}{\textvisiblespace}}
        \newcommand*\Wrappedcontinuationsymbol {\textcolor{red}{\llap{\tiny$\m@th\hookrightarrow$}}}
        \newcommand*\Wrappedcontinuationindent {3ex }
        \newcommand*\Wrappedafterbreak {\kern\Wrappedcontinuationindent\copy\Wrappedcontinuationbox}
        % Take advantage of the already applied Pygments mark-up to insert
        % potential linebreaks for TeX processing.
        %        {, <, #, %, $, ' and ": go to next line.
        %        _, }, ^, &, >, - and ~: stay at end of broken line.
        % Use of \textquotesingle for straight quote.
        \newcommand*\Wrappedbreaksatspecials {%
            \def\PYGZus{\discretionary{\char`\_}{\Wrappedafterbreak}{\char`\_}}%
            \def\PYGZob{\discretionary{}{\Wrappedafterbreak\char`\{}{\char`\{}}%
            \def\PYGZcb{\discretionary{\char`\}}{\Wrappedafterbreak}{\char`\}}}%
            \def\PYGZca{\discretionary{\char`\^}{\Wrappedafterbreak}{\char`\^}}%
            \def\PYGZam{\discretionary{\char`\&}{\Wrappedafterbreak}{\char`\&}}%
            \def\PYGZlt{\discretionary{}{\Wrappedafterbreak\char`\<}{\char`\<}}%
            \def\PYGZgt{\discretionary{\char`\>}{\Wrappedafterbreak}{\char`\>}}%
            \def\PYGZsh{\discretionary{}{\Wrappedafterbreak\char`\#}{\char`\#}}%
            \def\PYGZpc{\discretionary{}{\Wrappedafterbreak\char`\%}{\char`\%}}%
            \def\PYGZdl{\discretionary{}{\Wrappedafterbreak\char`\$}{\char`\$}}%
            \def\PYGZhy{\discretionary{\char`\-}{\Wrappedafterbreak}{\char`\-}}%
            \def\PYGZsq{\discretionary{}{\Wrappedafterbreak\textquotesingle}{\textquotesingle}}%
            \def\PYGZdq{\discretionary{}{\Wrappedafterbreak\char`\"}{\char`\"}}%
            \def\PYGZti{\discretionary{\char`\~}{\Wrappedafterbreak}{\char`\~}}%
        }
        % Some characters . , ; ? ! / are not pygmentized.
        % This macro makes them "active" and they will insert potential linebreaks
        \newcommand*\Wrappedbreaksatpunct {%
            \lccode`\~`\.\lowercase{\def~}{\discretionary{\hbox{\char`\.}}{\Wrappedafterbreak}{\hbox{\char`\.}}}%
            \lccode`\~`\,\lowercase{\def~}{\discretionary{\hbox{\char`\,}}{\Wrappedafterbreak}{\hbox{\char`\,}}}%
            \lccode`\~`\;\lowercase{\def~}{\discretionary{\hbox{\char`\;}}{\Wrappedafterbreak}{\hbox{\char`\;}}}%
            \lccode`\~`\:\lowercase{\def~}{\discretionary{\hbox{\char`\:}}{\Wrappedafterbreak}{\hbox{\char`\:}}}%
            \lccode`\~`\?\lowercase{\def~}{\discretionary{\hbox{\char`\?}}{\Wrappedafterbreak}{\hbox{\char`\?}}}%
            \lccode`\~`\!\lowercase{\def~}{\discretionary{\hbox{\char`\!}}{\Wrappedafterbreak}{\hbox{\char`\!}}}%
            \lccode`\~`\/\lowercase{\def~}{\discretionary{\hbox{\char`\/}}{\Wrappedafterbreak}{\hbox{\char`\/}}}%
            \catcode`\.\active
            \catcode`\,\active
            \catcode`\;\active
            \catcode`\:\active
            \catcode`\?\active
            \catcode`\!\active
            \catcode`\/\active
            \lccode`\~`\~
        }
    \makeatother

    \let\OriginalVerbatim=\Verbatim
    \makeatletter
    \renewcommand{\Verbatim}[1][1]{%
        %\parskip\z@skip
        \sbox\Wrappedcontinuationbox {\Wrappedcontinuationsymbol}%
        \sbox\Wrappedvisiblespacebox {\FV@SetupFont\Wrappedvisiblespace}%
        \def\FancyVerbFormatLine ##1{\hsize\linewidth
            \vtop{\raggedright\hyphenpenalty\z@\exhyphenpenalty\z@
                \doublehyphendemerits\z@\finalhyphendemerits\z@
                \strut ##1\strut}%
        }%
        % If the linebreak is at a space, the latter will be displayed as visible
        % space at end of first line, and a continuation symbol starts next line.
        % Stretch/shrink are however usually zero for typewriter font.
        \def\FV@Space {%
            \nobreak\hskip\z@ plus\fontdimen3\font minus\fontdimen4\font
            \discretionary{\copy\Wrappedvisiblespacebox}{\Wrappedafterbreak}
            {\kern\fontdimen2\font}%
        }%

        % Allow breaks at special characters using \PYG... macros.
        \Wrappedbreaksatspecials
        % Breaks at punctuation characters . , ; ? ! and / need catcode=\active
        \OriginalVerbatim[#1,codes*=\Wrappedbreaksatpunct]%
    }
    \makeatother

    % Exact colors from NB
    \definecolor{incolor}{HTML}{303F9F}
    \definecolor{outcolor}{HTML}{D84315}
    \definecolor{cellborder}{HTML}{CFCFCF}
    \definecolor{cellbackground}{HTML}{F7F7F7}

    % prompt
    \makeatletter
    \newcommand{\boxspacing}{\kern\kvtcb@left@rule\kern\kvtcb@boxsep}
    \makeatother
    \newcommand{\prompt}[4]{
        {\ttfamily\llap{{\color{#2}[#3]:\hspace{3pt}#4}}\vspace{-\baselineskip}}
    }
    

    
    % Prevent overflowing lines due to hard-to-break entities
    \sloppy
    % Setup hyperref package
    \hypersetup{
      breaklinks=true,  % so long urls are correctly broken across lines
      colorlinks=true,
      urlcolor=urlcolor,
      linkcolor=linkcolor,
      citecolor=citecolor,
      }
    % Slightly bigger margins than the latex defaults
    
    \geometry{verbose,tmargin=1in,bmargin=1in,lmargin=1in,rmargin=1in}
    
    

\begin{document}
    
    \maketitle
    
    

    
    \section{5.21(j)}\label{j}

计算下列信号的傅立叶变换

\[
x[n] = (n - 1)\left(\frac{1}{3} \right)^{|n|}
\]

\subsection{Answer}\label{answer}

将求和分为两部分:

\begin{itemize}
\tightlist
\item
  \(n\ge0\) 时, \[
  x[n]=(n-1)\Bigl(\frac{1}{3}\Bigr)^n;
  \]
\item
  \(n<0\) 时,由于 \(|n|=-n\),有 \[
  x[n]=(n-1)\Bigl(\frac{1}{3}\Bigr)^{-n}.
  \]
\end{itemize}

傅立叶变换定义为 \[
X(e^{j\omega})=\sum_{n=-\infty}^{\infty} x[n]e^{-j\omega n}.
\] 将求和分开,记 \[
X(e^{j\omega})=A+B,
\] 其中 \[
A=\sum_{n=0}^{\infty}(n-1)\Bigl(\frac{1}{3}\Bigr)^n e^{-j\omega n},
\] \[
B=\sum_{n=-\infty}^{-1}(n-1)\Bigl(\frac{1}{3}\Bigr)^{-n} e^{-j\omega n}.
\]

对于 \(n\ge0\)

令 \(r=\frac{e^{-j\omega}}{3}\),则 \[
A=\sum_{n=0}^\infty (n-1)r^n.
\] 利用已知级数 \[
\sum_{n=0}^\infty r^n=\frac{1}{1-r},\quad \sum_{n=0}^\infty n\,r^n=\frac{r}{(1-r)^2},
\] 可得 \[
A=\frac{r}{(1-r)^2}-\frac{1}{1-r}
=\frac{r-(1-r)}{(1-r)^2}=\frac{2r-1}{(1-r)^2}.
\] 即 \[
A=\frac{\displaystyle \frac{2e^{-j\omega}}{3}-1}{\Bigl(1-\frac{e^{-j\omega}}{3}\Bigr)^2}.
\]

对于 \(n<0\)

令 \(m=-n\)(当 \(n<0\) 时,\(m=1,2,\dots\)),

\[
B=\sum_{n=-\infty}^{-1}(n-1)\Bigl(\frac{1}{3}\Bigr)^{-n} e^{-j\omega n}
=\sum_{m=1}^{\infty} (-m-1)\Bigl(\frac{1}{3}\Bigr)^{m} e^{j\omega m}.
\] 令 \(s=\frac{e^{j\omega}}{3}\),则 \[
B=-\sum_{m=1}^\infty (m+1)s^m.
\] 同样利用 \[
\sum_{m=1}^\infty s^m=\frac{s}{1-s},\quad \sum_{m=1}^\infty m\,s^m=\frac{s}{(1-s)^2},
\] 有 \[
B=-\left[\frac{s}{(1-s)^2}+\frac{s}{1-s}\right]
=-\frac{s}{1-s}\Biggl[\frac{1}{1-s}+1\Biggr]
=-\frac{s(2-s)}{(1-s)^2}.
\] 即 \[
B=-\frac{\displaystyle \frac{e^{j\omega}}{3}\Bigl(2-\frac{e^{j\omega}}{3}\Bigr)}{\Bigl(1-\frac{e^{j\omega}}{3}\Bigr)^2}.
\]

将 A 和 B 合并, \[
\boxed{
X(e^{j\omega})=\frac{\displaystyle \frac{2e^{-j\omega}}{3}-1}{\Bigl(1-\frac{e^{-j\omega}}{3}\Bigr)^2}
-\frac{\displaystyle \frac{e^{j\omega}}{3}\Bigl(2-\frac{e^{j\omega}}{3}\Bigr)}{\Bigl(1-\frac{e^{j\omega}}{3}\Bigr)^2}.}
\]

    \section{5.26(a)}\label{a}

由图可知,

\[
X_2(e^{j\omega}) = \sum_{k = -\infty}^\infty\Re\{X_1(e^{j(\omega + \frac{2k\pi}{3})})\}
\]

故

\[
x_2[n] = {\rm Ev}\{ x_1[n] \} \cdot \sum_{k = -\infty}^\infty e^{j\frac{2k\pi}{3}n}
\]

我们注意到,这是一个经典的周期性求和公式,其闭合形式可以写为 \[
\sum_{k=-\infty}^{\infty} e^{j\frac{2\pi}{3}kn} =
\begin{cases}
3, & \text{if } n \mod 3 = 0, \\
0, & \text{otherwise.}
\end{cases}
\] 也就是说,当 \(n\) 是 3 的整数倍时,上式取值为 3,否则为 0。

因此,我们可以将 \[
x_2[n] = \operatorname{Ev}\{ x_1[n] \} \cdot \sum_{k=-\infty}^{\infty} e^{j\frac{2\pi k}{3}n}
\] 写为 \[
x_2[n] = 3\,\operatorname{Ev}\{ x_1[n] \}\cdot \sum_{m=-\infty}^{\infty} \delta[n-3m]
\]

    \section{5.26(d)}\label{d}

\[
x_4[n] = x_1[n] * h[n]
\]

其中,\(h[n] = \frac{\sin (\pi n /6)}{\pi n}\)

\subsection{Answer}\label{answer}

我们知道,离散时间系统中,若 \[
h[n] = \frac{\sin\left(\frac{\pi n}{6}\right)}{\pi n},
\] 其 DTFT 为 \[
H(e^{j\omega}) = \begin{cases}
1, & |\omega|\le \frac{\pi}{6},\\[1mm]
0, & |\omega|>\frac{\pi}{6}.
\end{cases}
\] \(H(e^{j\omega})\) 就是一个截止频率为 \(\pi/6\) 的方波函数。

因此 \[
X_4(e^{j\omega}) = X_1(e^{j\omega})\cdot H(e^{j\omega}).
\] 利用上面的方波函数,便有 \[
X_4(e^{j\omega}) = \begin{cases}
X_1(e^{j\omega}), & |\omega|\le \frac{\pi}{6},\\[1mm]
0, & |\omega|> \frac{\pi}{6}.
\end{cases}
\]

    \begin{tcolorbox}[breakable, size=fbox, boxrule=1pt, pad at break*=1mm,colback=cellbackground, colframe=cellborder]
\prompt{In}{incolor}{ }{\boxspacing}
\begin{Verbatim}[commandchars=\\\{\}]
\PY{k+kn}{import} \PY{n+nn}{numpy} \PY{k}{as} \PY{n+nn}{np}
\PY{k+kn}{import} \PY{n+nn}{matplotlib}\PY{n+nn}{.}\PY{n+nn}{pyplot} \PY{k}{as} \PY{n+nn}{plt}
\PY{n}{omega} \PY{o}{=} \PY{n}{np}\PY{o}{.}\PY{n}{linspace}\PY{p}{(}\PY{o}{\PYZhy{}}\PY{n}{np}\PY{o}{.}\PY{n}{pi}\PY{p}{,} \PY{n}{np}\PY{o}{.}\PY{n}{pi}\PY{p}{,} \PY{l+m+mi}{1000}\PY{p}{)}
\PY{n}{X1} \PY{o}{=} \PY{l+m+mi}{1} \PY{o}{\PYZhy{}} \PY{l+m+mi}{1}\PY{n}{j} \PY{o}{*} \PY{n}{omega} \PY{o}{*} \PY{l+m+mi}{6} \PY{o}{/} \PY{n}{np}\PY{o}{.}\PY{n}{pi}
\PY{c+c1}{\PYZsh{} 定义理想低通滤波器 H(e\PYZca{}(jω))}
\PY{c+c1}{\PYZsh{} 当 |ω| \PYZlt{}= π/6 时 H=1,否则 H=0}
\PY{n}{H} \PY{o}{=} \PY{n}{np}\PY{o}{.}\PY{n}{zeros\PYZus{}like}\PY{p}{(}\PY{n}{omega}\PY{p}{)}
\PY{n}{H}\PY{p}{[}\PY{n}{np}\PY{o}{.}\PY{n}{abs}\PY{p}{(}\PY{n}{omega}\PY{p}{)} \PY{o}{\PYZlt{}}\PY{o}{=} \PY{n}{np}\PY{o}{.}\PY{n}{pi}\PY{o}{/}\PY{l+m+mi}{6}\PY{p}{]} \PY{o}{=} \PY{l+m+mi}{1}
\PY{c+c1}{\PYZsh{} 计算 X4(e\PYZca{}(jω)) = X1(e\PYZca{}(jω)) * H(e\PYZca{}(jω))}
\PY{n}{X4} \PY{o}{=} \PY{n}{X1} \PY{o}{*} \PY{n}{H}
\PY{c+c1}{\PYZsh{} 绘制 X4(e\PYZca{}(jω)) 的实部和虚部}
\PY{n}{plt}\PY{o}{.}\PY{n}{figure}\PY{p}{(}\PY{n}{figsize}\PY{o}{=}\PY{p}{(}\PY{l+m+mi}{10}\PY{p}{,} \PY{l+m+mi}{6}\PY{p}{)}\PY{p}{)}
\PY{c+c1}{\PYZsh{} 实部}
\PY{n}{plt}\PY{o}{.}\PY{n}{subplot}\PY{p}{(}\PY{l+m+mi}{2}\PY{p}{,} \PY{l+m+mi}{1}\PY{p}{,} \PY{l+m+mi}{1}\PY{p}{)}
\PY{n}{plt}\PY{o}{.}\PY{n}{plot}\PY{p}{(}\PY{n}{omega}\PY{p}{,} \PY{n}{np}\PY{o}{.}\PY{n}{real}\PY{p}{(}\PY{n}{X4}\PY{p}{)}\PY{p}{,} \PY{l+s+s1}{\PYZsq{}}\PY{l+s+s1}{b\PYZhy{}}\PY{l+s+s1}{\PYZsq{}}\PY{p}{,} \PY{n}{lw}\PY{o}{=}\PY{l+m+mi}{2}\PY{p}{)}
\PY{n}{plt}\PY{o}{.}\PY{n}{title}\PY{p}{(}\PY{l+s+sa}{r}\PY{l+s+s1}{\PYZsq{}}\PY{l+s+s1}{\PYZdl{}}\PY{l+s+s1}{\PYZbs{}}\PY{l+s+s1}{Re}\PY{l+s+s1}{\PYZbs{}}\PY{l+s+s1}{\PYZob{}}\PY{l+s+s1}{X\PYZus{}4(e\PYZca{}}\PY{l+s+s1}{\PYZob{}}\PY{l+s+s1}{j}\PY{l+s+s1}{\PYZbs{}}\PY{l+s+s1}{omega\PYZcb{})}\PY{l+s+s1}{\PYZbs{}}\PY{l+s+s1}{\PYZcb{}\PYZdl{}}\PY{l+s+s1}{\PYZsq{}}\PY{p}{)}
\PY{n}{plt}\PY{o}{.}\PY{n}{ylabel}\PY{p}{(}\PY{l+s+s1}{\PYZsq{}}\PY{l+s+s1}{Re}\PY{l+s+s1}{\PYZsq{}}\PY{p}{)}
\PY{n}{plt}\PY{o}{.}\PY{n}{grid}\PY{p}{(}\PY{k+kc}{True}\PY{p}{)}
\PY{c+c1}{\PYZsh{} 虚部}
\PY{n}{plt}\PY{o}{.}\PY{n}{subplot}\PY{p}{(}\PY{l+m+mi}{2}\PY{p}{,} \PY{l+m+mi}{1}\PY{p}{,} \PY{l+m+mi}{2}\PY{p}{)}
\PY{n}{plt}\PY{o}{.}\PY{n}{plot}\PY{p}{(}\PY{n}{omega}\PY{p}{,} \PY{n}{np}\PY{o}{.}\PY{n}{imag}\PY{p}{(}\PY{n}{X4}\PY{p}{)}\PY{p}{,} \PY{l+s+s1}{\PYZsq{}}\PY{l+s+s1}{r\PYZhy{}}\PY{l+s+s1}{\PYZsq{}}\PY{p}{,} \PY{n}{lw}\PY{o}{=}\PY{l+m+mi}{2}\PY{p}{)}
\PY{n}{plt}\PY{o}{.}\PY{n}{title}\PY{p}{(}\PY{l+s+sa}{r}\PY{l+s+s1}{\PYZsq{}}\PY{l+s+s1}{\PYZdl{}}\PY{l+s+s1}{\PYZbs{}}\PY{l+s+s1}{Im}\PY{l+s+s1}{\PYZbs{}}\PY{l+s+s1}{\PYZob{}}\PY{l+s+s1}{X\PYZus{}4(e\PYZca{}}\PY{l+s+s1}{\PYZob{}}\PY{l+s+s1}{j}\PY{l+s+s1}{\PYZbs{}}\PY{l+s+s1}{omega\PYZcb{})}\PY{l+s+s1}{\PYZbs{}}\PY{l+s+s1}{\PYZcb{}\PYZdl{}}\PY{l+s+s1}{\PYZsq{}}\PY{p}{)}
\PY{n}{plt}\PY{o}{.}\PY{n}{xlabel}\PY{p}{(}\PY{l+s+sa}{r}\PY{l+s+s1}{\PYZsq{}}\PY{l+s+s1}{\PYZdl{}}\PY{l+s+s1}{\PYZbs{}}\PY{l+s+s1}{omega\PYZdl{} (rad/s)}\PY{l+s+s1}{\PYZsq{}}\PY{p}{)}
\PY{n}{plt}\PY{o}{.}\PY{n}{ylabel}\PY{p}{(}\PY{l+s+s1}{\PYZsq{}}\PY{l+s+s1}{Im}\PY{l+s+s1}{\PYZsq{}}\PY{p}{)}
\PY{n}{plt}\PY{o}{.}\PY{n}{grid}\PY{p}{(}\PY{k+kc}{True}\PY{p}{)}
\PY{n}{plt}\PY{o}{.}\PY{n}{tight\PYZus{}layout}\PY{p}{(}\PY{p}{)}
\PY{n}{plt}\PY{o}{.}\PY{n}{show}\PY{p}{(}\PY{p}{)}
\end{Verbatim}
\end{tcolorbox}

    \begin{center}
    \adjustimage{max size={0.9\linewidth}{0.9\paperheight}}{output_3_0.png}
    \end{center}
    { \hspace*{\fill} \\}
    
    \section{5.21(k)}\label{k}

求傅立叶变换

\[
x[n] = \left(\frac{\sin (\pi n/5)}{\pi n}\right)\cos \left( \frac{7\pi}{2}n \right)
\]

\subsection{Answer}\label{answer}

\begin{enumerate}
\def\labelenumi{\arabic{enumi}.}
\item
  \textbf{理想低通滤波器的 DTFT:}\\
  设 \[
  h[n] = \frac{\sin (\pi n/5)}{\pi n},
  \] 则已知其 DTFT 为 \[
  H(e^{j\omega}) =
  \begin{cases}
  1, & |\omega|\le \frac{\pi}{5},\\[1mm]
  0, & |\omega|> \frac{\pi}{5}.
  \end{cases}
  \] 也就是说,\(H(e^{j\omega})\)就是一个截止频率为 \(\pi/5\)
  的矩形函数。
\item
  \textbf{乘以余弦的调制性质:}\\
  利用欧拉公式,有 \[
  \cos\Bigl(\frac{7\pi}{2}n\Bigr) = \frac{1}{2}\left(e^{j\frac{7\pi}{2}n}+e^{-j\frac{7\pi}{2}n}\right).
  \] 根据傅立叶变换的调制性质,如果信号 \(h[n]\) 的傅立叶变换为
  \(H(e^{j\omega})\),则 \[
  h[n]\,e^{\pm j\frac{7\pi}{2}n}\quad \Longleftrightarrow \quad H\Bigl(e^{j(\omega\mp \frac{7\pi}{2})}\Bigr).
  \]
\end{enumerate}

因此,由线性性和调制定理,我们可将 \[
x[n] = h[n]\cos\Bigl(\frac{7\pi}{2}n\Bigr)
\] 写成 \[
x[n] = \frac{1}{2}\Bigl\{ h[n]e^{j\frac{7\pi}{2}n} + h[n]e^{-j\frac{7\pi}{2}n}\Bigr\}.
\] 对应的 DTFT 为 \[
X(e^{j\omega}) = \frac{1}{2}\left\{ H\Bigl(e^{j(\omega - \frac{7\pi}{2})}\Bigr) + H\Bigl(e^{j(\omega + \frac{7\pi}{2})}\Bigr) \right\}.
\]

由于 \(H(e^{j\omega})\) 是理想低通型,我们可写出 \[
X(e^{j\omega}) = \frac{1}{2}\Bigl\{
\mathbf{1}_{|\omega-\frac{7\pi}{2}|\le \pi/5} + \mathbf{1}_{|\omega+\frac{7\pi}{2}|\le \pi/5}
\Bigr\},
\] 其中 \(\mathbf{1}_A\) 表示当条件 \(A\) 成立时取 1,否则取 0。

接下来我们将这两个频带移动到主频带 \((-\pi,\pi]\)(因为 DTFT 具有
\(2\pi\) 周期性)。

有 \[
X(e^{j\omega}) = \frac{1}{2}\left\{
\begin{array}{ll}
1,  & \omega \in [-0.7\pi,\,-0.3\pi] \cup [0.3\pi,\,0.7\pi],\\[1mm]
0,  & \text{otherwise}.
\end{array}
\right.
\]

    \section{5.33}\label{section}

差分方程为

\[
y[n] + \frac{1}{2}y[n - 1] = x[n]
\]

求系统响应

\subsection{\texorpdfstring{(b. iv) 输入为
\(x[n] = \delta[n] - \frac{1}{2}\delta[n-1]\)}{(b. iv) 输入为 x{[}n{]} = \textbackslash delta{[}n{]} - \textbackslash frac\{1\}\{2\}\textbackslash delta{[}n-1{]}}}\label{b.-iv-ux8f93ux5165ux4e3a-xn-deltan---frac12deltan-1}

由差分方程的性质,有

\[
H(e^{j\omega}) = \frac{Y(e^{j\omega})}{X(e^{j\omega})} = \frac{\sum_{k = 0}^m b_ke^{-jk\omega}}{\sum_{k = 0}^na_ke^{-jk\omega}}
\]

故当 \(a_0 = 1,\quad a_1 = \frac{1}{2}, \quad b_0=1\) 时,有

\[
H(e^{j\omega}) = \frac{1}{1+\frac{1}{2}e^{j\omega}}
\]

故

\[
h[n] = \left(-\frac{1}{2}\right)^nu[n]
\]

根据线性时不变系统的卷积原理, \[
y[n] = x[n] * h[n] = \sum_{k=-\infty}^{\infty} x[k]\,h[n-k].
\]

由于 \(x[n]\) 只有在 \(n=0\) 与 \(n=1\) 处非零,上式可以写为 \[
y[n] = x[0]\,h[n] + x[1]\,h[n-1] = h[n] - \frac{1}{2}\,h[n-1].
\]

接下来代入 \(h[n] = \left(-\frac{1}{2}\right)^n u[n]\):

\begin{itemize}
\item
  当 \(n=0\) 时,由于 \(h[-1] = 0\)(因果性), \[
  y[0] = h[0] - \frac{1}{2}\,h[-1] = 1 - 0 = 1.
  \]
\item
  当 \(n \ge 1\) 时, \[
  y[n] = \left(-\frac{1}{2}\right)^n - \frac{1}{2}\left(-\frac{1}{2}\right)^{n-1}.
  \]
\end{itemize}

因此, \[
y[n] = \left(-\frac{1}{2}\right)^n - \Bigl[-\left(-\frac{1}{2}\right)^n\Bigr]
= \left(-\frac{1}{2}\right)^n + \left(-\frac{1}{2}\right)^n
= 2\left(-\frac{1}{2}\right)^n,\quad n\ge1.
\]

综上所述, \[
y[n] =
\begin{cases}
1, & n=0,\\[1mm]
2\left(-\frac{1}{2}\right)^n, & n\ge1,\\[1mm]
0, & n<0.
\end{cases}
\]

\subsection{\texorpdfstring{(c.~i)
\(X(e^{j\omega}) = \frac{1 - \frac{1}{4}e^{-j\omega}}{1 + \frac{1}{2} e^{-j\omega}}\)}{(c.~i) X(e\^{}\{j\textbackslash omega\}) = \textbackslash frac\{1 - \textbackslash frac\{1\}\{4\}e\^{}\{-j\textbackslash omega\}\}\{1 + \textbackslash frac\{1\}\{2\} e\^{}\{-j\textbackslash omega\}\}}}\label{c.-i-xejomega-frac1---frac14e-jomega1-frac12-e-jomega}

\[
H(e^{j\omega}) = \frac{1}{1+\frac{1}{2}e^{j\omega}}
\]

\[
Y(e^{j\omega}) = H(e^{j\omega})X(e^{j\omega}) = \frac{-1/2}{1 + \frac{1}{2}e^{j\omega}} + \frac{3/2}{(1 + \frac{1}{2}e^{-j\omega})^2}
\]

利用常用变换对,有

\[
y[n] = \left(-\frac{1}{2}\right)^nu[n] - \frac{3}{4}n\left(-\frac{1}{2}\right)^{n-1}u[n-1]
\]


    % Add a bibliography block to the postdoc
    
    
    
\end{document}
