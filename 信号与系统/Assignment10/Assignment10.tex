\documentclass[11pt]{article}

    \usepackage[breakable]{tcolorbox}
    \usepackage{parskip} % Stop auto-indenting (to mimic markdown behaviour)
    \usepackage{xeCJK}

    % Basic figure setup, for now with no caption control since it's done
    % automatically by Pandoc (which extracts ![](path) syntax from Markdown).
    \usepackage{graphicx}
    % Keep aspect ratio if custom image width or height is specified
    \setkeys{Gin}{keepaspectratio}
    % Maintain compatibility with old templates. Remove in nbconvert 6.0
    \let\Oldincludegraphics\includegraphics
    % Ensure that by default, figures have no caption (until we provide a
    % proper Figure object with a Caption API and a way to capture that
    % in the conversion process - todo).
    \usepackage{caption}
    \DeclareCaptionFormat{nocaption}{}
    \captionsetup{format=nocaption,aboveskip=0pt,belowskip=0pt}

    \usepackage{float}
    \floatplacement{figure}{H} % forces figures to be placed at the correct location
    \usepackage{xcolor} % Allow colors to be defined
    \usepackage{enumerate} % Needed for markdown enumerations to work
    \usepackage{geometry} % Used to adjust the document margins
    \usepackage{amsmath} % Equations
    \usepackage{amssymb} % Equations
    \usepackage{textcomp} % defines textquotesingle
    % Hack from http://tex.stackexchange.com/a/47451/13684:
    \AtBeginDocument{%
        \def\PYZsq{\textquotesingle}% Upright quotes in Pygmentized code
    }
    \usepackage{upquote} % Upright quotes for verbatim code
    \usepackage{eurosym} % defines \euro

    \usepackage{iftex}
    \ifPDFTeX
        \usepackage[T1]{fontenc}
        \IfFileExists{alphabeta.sty}{
              \usepackage{alphabeta}
          }{
              \usepackage[mathletters]{ucs}
              \usepackage[utf8x]{inputenc}
          }
    \else
        \usepackage{fontspec}
        \usepackage{unicode-math}
    \fi

    \usepackage{fancyvrb} % verbatim replacement that allows latex
    \usepackage{grffile} % extends the file name processing of package graphics
                         % to support a larger range
    \makeatletter % fix for old versions of grffile with XeLaTeX
    \@ifpackagelater{grffile}{2019/11/01}
    {
      % Do nothing on new versions
    }
    {
      \def\Gread@@xetex#1{%
        \IfFileExists{"\Gin@base".bb}%
        {\Gread@eps{\Gin@base.bb}}%
        {\Gread@@xetex@aux#1}%
      }
    }
    \makeatother
    \usepackage[Export]{adjustbox} % Used to constrain images to a maximum size
    \adjustboxset{max size={0.9\linewidth}{0.9\paperheight}}

    % The hyperref package gives us a pdf with properly built
    % internal navigation ('pdf bookmarks' for the table of contents,
    % internal cross-reference links, web links for URLs, etc.)
    \usepackage{hyperref}
    % The default LaTeX title has an obnoxious amount of whitespace. By default,
    % titling removes some of it. It also provides customization options.
    \usepackage{titling}
    \usepackage{longtable} % longtable support required by pandoc >1.10
    \usepackage{booktabs}  % table support for pandoc > 1.12.2
    \usepackage{array}     % table support for pandoc >= 2.11.3
    \usepackage{calc}      % table minipage width calculation for pandoc >= 2.11.1
    \usepackage[inline]{enumitem} % IRkernel/repr support (it uses the enumerate* environment)
    \usepackage[normalem]{ulem} % ulem is needed to support strikethroughs (\sout)
                                % normalem makes italics be italics, not underlines
    \usepackage{soul}      % strikethrough (\st) support for pandoc >= 3.0.0
    \usepackage{mathrsfs}
    

    
    % Colors for the hyperref package
    \definecolor{urlcolor}{rgb}{0,.145,.698}
    \definecolor{linkcolor}{rgb}{.71,0.21,0.01}
    \definecolor{citecolor}{rgb}{.12,.54,.11}

    % ANSI colors
    \definecolor{ansi-black}{HTML}{3E424D}
    \definecolor{ansi-black-intense}{HTML}{282C36}
    \definecolor{ansi-red}{HTML}{E75C58}
    \definecolor{ansi-red-intense}{HTML}{B22B31}
    \definecolor{ansi-green}{HTML}{00A250}
    \definecolor{ansi-green-intense}{HTML}{007427}
    \definecolor{ansi-yellow}{HTML}{DDB62B}
    \definecolor{ansi-yellow-intense}{HTML}{B27D12}
    \definecolor{ansi-blue}{HTML}{208FFB}
    \definecolor{ansi-blue-intense}{HTML}{0065CA}
    \definecolor{ansi-magenta}{HTML}{D160C4}
    \definecolor{ansi-magenta-intense}{HTML}{A03196}
    \definecolor{ansi-cyan}{HTML}{60C6C8}
    \definecolor{ansi-cyan-intense}{HTML}{258F8F}
    \definecolor{ansi-white}{HTML}{C5C1B4}
    \definecolor{ansi-white-intense}{HTML}{A1A6B2}
    \definecolor{ansi-default-inverse-fg}{HTML}{FFFFFF}
    \definecolor{ansi-default-inverse-bg}{HTML}{000000}

    % common color for the border for error outputs.
    \definecolor{outerrorbackground}{HTML}{FFDFDF}

    % commands and environments needed by pandoc snippets
    % extracted from the output of `pandoc -s`
    \providecommand{\tightlist}{%
      \setlength{\itemsep}{0pt}\setlength{\parskip}{0pt}}
    \DefineVerbatimEnvironment{Highlighting}{Verbatim}{commandchars=\\\{\}}
    % Add ',fontsize=\small' for more characters per line
    \newenvironment{Shaded}{}{}
    \newcommand{\KeywordTok}[1]{\textcolor[rgb]{0.00,0.44,0.13}{\textbf{{#1}}}}
    \newcommand{\DataTypeTok}[1]{\textcolor[rgb]{0.56,0.13,0.00}{{#1}}}
    \newcommand{\DecValTok}[1]{\textcolor[rgb]{0.25,0.63,0.44}{{#1}}}
    \newcommand{\BaseNTok}[1]{\textcolor[rgb]{0.25,0.63,0.44}{{#1}}}
    \newcommand{\FloatTok}[1]{\textcolor[rgb]{0.25,0.63,0.44}{{#1}}}
    \newcommand{\CharTok}[1]{\textcolor[rgb]{0.25,0.44,0.63}{{#1}}}
    \newcommand{\StringTok}[1]{\textcolor[rgb]{0.25,0.44,0.63}{{#1}}}
    \newcommand{\CommentTok}[1]{\textcolor[rgb]{0.38,0.63,0.69}{\textit{{#1}}}}
    \newcommand{\OtherTok}[1]{\textcolor[rgb]{0.00,0.44,0.13}{{#1}}}
    \newcommand{\AlertTok}[1]{\textcolor[rgb]{1.00,0.00,0.00}{\textbf{{#1}}}}
    \newcommand{\FunctionTok}[1]{\textcolor[rgb]{0.02,0.16,0.49}{{#1}}}
    \newcommand{\RegionMarkerTok}[1]{{#1}}
    \newcommand{\ErrorTok}[1]{\textcolor[rgb]{1.00,0.00,0.00}{\textbf{{#1}}}}
    \newcommand{\NormalTok}[1]{{#1}}

    % Additional commands for more recent versions of Pandoc
    \newcommand{\ConstantTok}[1]{\textcolor[rgb]{0.53,0.00,0.00}{{#1}}}
    \newcommand{\SpecialCharTok}[1]{\textcolor[rgb]{0.25,0.44,0.63}{{#1}}}
    \newcommand{\VerbatimStringTok}[1]{\textcolor[rgb]{0.25,0.44,0.63}{{#1}}}
    \newcommand{\SpecialStringTok}[1]{\textcolor[rgb]{0.73,0.40,0.53}{{#1}}}
    \newcommand{\ImportTok}[1]{{#1}}
    \newcommand{\DocumentationTok}[1]{\textcolor[rgb]{0.73,0.13,0.13}{\textit{{#1}}}}
    \newcommand{\AnnotationTok}[1]{\textcolor[rgb]{0.38,0.63,0.69}{\textbf{\textit{{#1}}}}}
    \newcommand{\CommentVarTok}[1]{\textcolor[rgb]{0.38,0.63,0.69}{\textbf{\textit{{#1}}}}}
    \newcommand{\VariableTok}[1]{\textcolor[rgb]{0.10,0.09,0.49}{{#1}}}
    \newcommand{\ControlFlowTok}[1]{\textcolor[rgb]{0.00,0.44,0.13}{\textbf{{#1}}}}
    \newcommand{\OperatorTok}[1]{\textcolor[rgb]{0.40,0.40,0.40}{{#1}}}
    \newcommand{\BuiltInTok}[1]{{#1}}
    \newcommand{\ExtensionTok}[1]{{#1}}
    \newcommand{\PreprocessorTok}[1]{\textcolor[rgb]{0.74,0.48,0.00}{{#1}}}
    \newcommand{\AttributeTok}[1]{\textcolor[rgb]{0.49,0.56,0.16}{{#1}}}
    \newcommand{\InformationTok}[1]{\textcolor[rgb]{0.38,0.63,0.69}{\textbf{\textit{{#1}}}}}
    \newcommand{\WarningTok}[1]{\textcolor[rgb]{0.38,0.63,0.69}{\textbf{\textit{{#1}}}}}


    % Define a nice break command that doesn't care if a line doesn't already
    % exist.
    \def\br{\hspace*{\fill} \\* }
    % Math Jax compatibility definitions
    \def\gt{>}
    \def\lt{<}
    \let\Oldtex\TeX
    \let\Oldlatex\LaTeX
    \renewcommand{\TeX}{\textrm{\Oldtex}}
    \renewcommand{\LaTeX}{\textrm{\Oldlatex}}
    % Document parameters
    % Document title
    \title{Assignment10}
    
    
    
    
    
    
    
% Pygments definitions
\makeatletter
\def\PY@reset{\let\PY@it=\relax \let\PY@bf=\relax%
    \let\PY@ul=\relax \let\PY@tc=\relax%
    \let\PY@bc=\relax \let\PY@ff=\relax}
\def\PY@tok#1{\csname PY@tok@#1\endcsname}
\def\PY@toks#1+{\ifx\relax#1\empty\else%
    \PY@tok{#1}\expandafter\PY@toks\fi}
\def\PY@do#1{\PY@bc{\PY@tc{\PY@ul{%
    \PY@it{\PY@bf{\PY@ff{#1}}}}}}}
\def\PY#1#2{\PY@reset\PY@toks#1+\relax+\PY@do{#2}}

\@namedef{PY@tok@w}{\def\PY@tc##1{\textcolor[rgb]{0.73,0.73,0.73}{##1}}}
\@namedef{PY@tok@c}{\let\PY@it=\textit\def\PY@tc##1{\textcolor[rgb]{0.24,0.48,0.48}{##1}}}
\@namedef{PY@tok@cp}{\def\PY@tc##1{\textcolor[rgb]{0.61,0.40,0.00}{##1}}}
\@namedef{PY@tok@k}{\let\PY@bf=\textbf\def\PY@tc##1{\textcolor[rgb]{0.00,0.50,0.00}{##1}}}
\@namedef{PY@tok@kp}{\def\PY@tc##1{\textcolor[rgb]{0.00,0.50,0.00}{##1}}}
\@namedef{PY@tok@kt}{\def\PY@tc##1{\textcolor[rgb]{0.69,0.00,0.25}{##1}}}
\@namedef{PY@tok@o}{\def\PY@tc##1{\textcolor[rgb]{0.40,0.40,0.40}{##1}}}
\@namedef{PY@tok@ow}{\let\PY@bf=\textbf\def\PY@tc##1{\textcolor[rgb]{0.67,0.13,1.00}{##1}}}
\@namedef{PY@tok@nb}{\def\PY@tc##1{\textcolor[rgb]{0.00,0.50,0.00}{##1}}}
\@namedef{PY@tok@nf}{\def\PY@tc##1{\textcolor[rgb]{0.00,0.00,1.00}{##1}}}
\@namedef{PY@tok@nc}{\let\PY@bf=\textbf\def\PY@tc##1{\textcolor[rgb]{0.00,0.00,1.00}{##1}}}
\@namedef{PY@tok@nn}{\let\PY@bf=\textbf\def\PY@tc##1{\textcolor[rgb]{0.00,0.00,1.00}{##1}}}
\@namedef{PY@tok@ne}{\let\PY@bf=\textbf\def\PY@tc##1{\textcolor[rgb]{0.80,0.25,0.22}{##1}}}
\@namedef{PY@tok@nv}{\def\PY@tc##1{\textcolor[rgb]{0.10,0.09,0.49}{##1}}}
\@namedef{PY@tok@no}{\def\PY@tc##1{\textcolor[rgb]{0.53,0.00,0.00}{##1}}}
\@namedef{PY@tok@nl}{\def\PY@tc##1{\textcolor[rgb]{0.46,0.46,0.00}{##1}}}
\@namedef{PY@tok@ni}{\let\PY@bf=\textbf\def\PY@tc##1{\textcolor[rgb]{0.44,0.44,0.44}{##1}}}
\@namedef{PY@tok@na}{\def\PY@tc##1{\textcolor[rgb]{0.41,0.47,0.13}{##1}}}
\@namedef{PY@tok@nt}{\let\PY@bf=\textbf\def\PY@tc##1{\textcolor[rgb]{0.00,0.50,0.00}{##1}}}
\@namedef{PY@tok@nd}{\def\PY@tc##1{\textcolor[rgb]{0.67,0.13,1.00}{##1}}}
\@namedef{PY@tok@s}{\def\PY@tc##1{\textcolor[rgb]{0.73,0.13,0.13}{##1}}}
\@namedef{PY@tok@sd}{\let\PY@it=\textit\def\PY@tc##1{\textcolor[rgb]{0.73,0.13,0.13}{##1}}}
\@namedef{PY@tok@si}{\let\PY@bf=\textbf\def\PY@tc##1{\textcolor[rgb]{0.64,0.35,0.47}{##1}}}
\@namedef{PY@tok@se}{\let\PY@bf=\textbf\def\PY@tc##1{\textcolor[rgb]{0.67,0.36,0.12}{##1}}}
\@namedef{PY@tok@sr}{\def\PY@tc##1{\textcolor[rgb]{0.64,0.35,0.47}{##1}}}
\@namedef{PY@tok@ss}{\def\PY@tc##1{\textcolor[rgb]{0.10,0.09,0.49}{##1}}}
\@namedef{PY@tok@sx}{\def\PY@tc##1{\textcolor[rgb]{0.00,0.50,0.00}{##1}}}
\@namedef{PY@tok@m}{\def\PY@tc##1{\textcolor[rgb]{0.40,0.40,0.40}{##1}}}
\@namedef{PY@tok@gh}{\let\PY@bf=\textbf\def\PY@tc##1{\textcolor[rgb]{0.00,0.00,0.50}{##1}}}
\@namedef{PY@tok@gu}{\let\PY@bf=\textbf\def\PY@tc##1{\textcolor[rgb]{0.50,0.00,0.50}{##1}}}
\@namedef{PY@tok@gd}{\def\PY@tc##1{\textcolor[rgb]{0.63,0.00,0.00}{##1}}}
\@namedef{PY@tok@gi}{\def\PY@tc##1{\textcolor[rgb]{0.00,0.52,0.00}{##1}}}
\@namedef{PY@tok@gr}{\def\PY@tc##1{\textcolor[rgb]{0.89,0.00,0.00}{##1}}}
\@namedef{PY@tok@ge}{\let\PY@it=\textit}
\@namedef{PY@tok@gs}{\let\PY@bf=\textbf}
\@namedef{PY@tok@gp}{\let\PY@bf=\textbf\def\PY@tc##1{\textcolor[rgb]{0.00,0.00,0.50}{##1}}}
\@namedef{PY@tok@go}{\def\PY@tc##1{\textcolor[rgb]{0.44,0.44,0.44}{##1}}}
\@namedef{PY@tok@gt}{\def\PY@tc##1{\textcolor[rgb]{0.00,0.27,0.87}{##1}}}
\@namedef{PY@tok@err}{\def\PY@bc##1{{\setlength{\fboxsep}{\string -\fboxrule}\fcolorbox[rgb]{1.00,0.00,0.00}{1,1,1}{\strut ##1}}}}
\@namedef{PY@tok@kc}{\let\PY@bf=\textbf\def\PY@tc##1{\textcolor[rgb]{0.00,0.50,0.00}{##1}}}
\@namedef{PY@tok@kd}{\let\PY@bf=\textbf\def\PY@tc##1{\textcolor[rgb]{0.00,0.50,0.00}{##1}}}
\@namedef{PY@tok@kn}{\let\PY@bf=\textbf\def\PY@tc##1{\textcolor[rgb]{0.00,0.50,0.00}{##1}}}
\@namedef{PY@tok@kr}{\let\PY@bf=\textbf\def\PY@tc##1{\textcolor[rgb]{0.00,0.50,0.00}{##1}}}
\@namedef{PY@tok@bp}{\def\PY@tc##1{\textcolor[rgb]{0.00,0.50,0.00}{##1}}}
\@namedef{PY@tok@fm}{\def\PY@tc##1{\textcolor[rgb]{0.00,0.00,1.00}{##1}}}
\@namedef{PY@tok@vc}{\def\PY@tc##1{\textcolor[rgb]{0.10,0.09,0.49}{##1}}}
\@namedef{PY@tok@vg}{\def\PY@tc##1{\textcolor[rgb]{0.10,0.09,0.49}{##1}}}
\@namedef{PY@tok@vi}{\def\PY@tc##1{\textcolor[rgb]{0.10,0.09,0.49}{##1}}}
\@namedef{PY@tok@vm}{\def\PY@tc##1{\textcolor[rgb]{0.10,0.09,0.49}{##1}}}
\@namedef{PY@tok@sa}{\def\PY@tc##1{\textcolor[rgb]{0.73,0.13,0.13}{##1}}}
\@namedef{PY@tok@sb}{\def\PY@tc##1{\textcolor[rgb]{0.73,0.13,0.13}{##1}}}
\@namedef{PY@tok@sc}{\def\PY@tc##1{\textcolor[rgb]{0.73,0.13,0.13}{##1}}}
\@namedef{PY@tok@dl}{\def\PY@tc##1{\textcolor[rgb]{0.73,0.13,0.13}{##1}}}
\@namedef{PY@tok@s2}{\def\PY@tc##1{\textcolor[rgb]{0.73,0.13,0.13}{##1}}}
\@namedef{PY@tok@sh}{\def\PY@tc##1{\textcolor[rgb]{0.73,0.13,0.13}{##1}}}
\@namedef{PY@tok@s1}{\def\PY@tc##1{\textcolor[rgb]{0.73,0.13,0.13}{##1}}}
\@namedef{PY@tok@mb}{\def\PY@tc##1{\textcolor[rgb]{0.40,0.40,0.40}{##1}}}
\@namedef{PY@tok@mf}{\def\PY@tc##1{\textcolor[rgb]{0.40,0.40,0.40}{##1}}}
\@namedef{PY@tok@mh}{\def\PY@tc##1{\textcolor[rgb]{0.40,0.40,0.40}{##1}}}
\@namedef{PY@tok@mi}{\def\PY@tc##1{\textcolor[rgb]{0.40,0.40,0.40}{##1}}}
\@namedef{PY@tok@il}{\def\PY@tc##1{\textcolor[rgb]{0.40,0.40,0.40}{##1}}}
\@namedef{PY@tok@mo}{\def\PY@tc##1{\textcolor[rgb]{0.40,0.40,0.40}{##1}}}
\@namedef{PY@tok@ch}{\let\PY@it=\textit\def\PY@tc##1{\textcolor[rgb]{0.24,0.48,0.48}{##1}}}
\@namedef{PY@tok@cm}{\let\PY@it=\textit\def\PY@tc##1{\textcolor[rgb]{0.24,0.48,0.48}{##1}}}
\@namedef{PY@tok@cpf}{\let\PY@it=\textit\def\PY@tc##1{\textcolor[rgb]{0.24,0.48,0.48}{##1}}}
\@namedef{PY@tok@c1}{\let\PY@it=\textit\def\PY@tc##1{\textcolor[rgb]{0.24,0.48,0.48}{##1}}}
\@namedef{PY@tok@cs}{\let\PY@it=\textit\def\PY@tc##1{\textcolor[rgb]{0.24,0.48,0.48}{##1}}}

\def\PYZbs{\char`\\}
\def\PYZus{\char`\_}
\def\PYZob{\char`\{}
\def\PYZcb{\char`\}}
\def\PYZca{\char`\^}
\def\PYZam{\char`\&}
\def\PYZlt{\char`\<}
\def\PYZgt{\char`\>}
\def\PYZsh{\char`\#}
\def\PYZpc{\char`\%}
\def\PYZdl{\char`\$}
\def\PYZhy{\char`\-}
\def\PYZsq{\char`\'}
\def\PYZdq{\char`\"}
\def\PYZti{\char`\~}
% for compatibility with earlier versions
\def\PYZat{@}
\def\PYZlb{[}
\def\PYZrb{]}
\makeatother


    % For linebreaks inside Verbatim environment from package fancyvrb.
    \makeatletter
        \newbox\Wrappedcontinuationbox
        \newbox\Wrappedvisiblespacebox
        \newcommand*\Wrappedvisiblespace {\textcolor{red}{\textvisiblespace}}
        \newcommand*\Wrappedcontinuationsymbol {\textcolor{red}{\llap{\tiny$\m@th\hookrightarrow$}}}
        \newcommand*\Wrappedcontinuationindent {3ex }
        \newcommand*\Wrappedafterbreak {\kern\Wrappedcontinuationindent\copy\Wrappedcontinuationbox}
        % Take advantage of the already applied Pygments mark-up to insert
        % potential linebreaks for TeX processing.
        %        {, <, #, %, $, ' and ": go to next line.
        %        _, }, ^, &, >, - and ~: stay at end of broken line.
        % Use of \textquotesingle for straight quote.
        \newcommand*\Wrappedbreaksatspecials {%
            \def\PYGZus{\discretionary{\char`\_}{\Wrappedafterbreak}{\char`\_}}%
            \def\PYGZob{\discretionary{}{\Wrappedafterbreak\char`\{}{\char`\{}}%
            \def\PYGZcb{\discretionary{\char`\}}{\Wrappedafterbreak}{\char`\}}}%
            \def\PYGZca{\discretionary{\char`\^}{\Wrappedafterbreak}{\char`\^}}%
            \def\PYGZam{\discretionary{\char`\&}{\Wrappedafterbreak}{\char`\&}}%
            \def\PYGZlt{\discretionary{}{\Wrappedafterbreak\char`\<}{\char`\<}}%
            \def\PYGZgt{\discretionary{\char`\>}{\Wrappedafterbreak}{\char`\>}}%
            \def\PYGZsh{\discretionary{}{\Wrappedafterbreak\char`\#}{\char`\#}}%
            \def\PYGZpc{\discretionary{}{\Wrappedafterbreak\char`\%}{\char`\%}}%
            \def\PYGZdl{\discretionary{}{\Wrappedafterbreak\char`\$}{\char`\$}}%
            \def\PYGZhy{\discretionary{\char`\-}{\Wrappedafterbreak}{\char`\-}}%
            \def\PYGZsq{\discretionary{}{\Wrappedafterbreak\textquotesingle}{\textquotesingle}}%
            \def\PYGZdq{\discretionary{}{\Wrappedafterbreak\char`\"}{\char`\"}}%
            \def\PYGZti{\discretionary{\char`\~}{\Wrappedafterbreak}{\char`\~}}%
        }
        % Some characters . , ; ? ! / are not pygmentized.
        % This macro makes them "active" and they will insert potential linebreaks
        \newcommand*\Wrappedbreaksatpunct {%
            \lccode`\~`\.\lowercase{\def~}{\discretionary{\hbox{\char`\.}}{\Wrappedafterbreak}{\hbox{\char`\.}}}%
            \lccode`\~`\,\lowercase{\def~}{\discretionary{\hbox{\char`\,}}{\Wrappedafterbreak}{\hbox{\char`\,}}}%
            \lccode`\~`\;\lowercase{\def~}{\discretionary{\hbox{\char`\;}}{\Wrappedafterbreak}{\hbox{\char`\;}}}%
            \lccode`\~`\:\lowercase{\def~}{\discretionary{\hbox{\char`\:}}{\Wrappedafterbreak}{\hbox{\char`\:}}}%
            \lccode`\~`\?\lowercase{\def~}{\discretionary{\hbox{\char`\?}}{\Wrappedafterbreak}{\hbox{\char`\?}}}%
            \lccode`\~`\!\lowercase{\def~}{\discretionary{\hbox{\char`\!}}{\Wrappedafterbreak}{\hbox{\char`\!}}}%
            \lccode`\~`\/\lowercase{\def~}{\discretionary{\hbox{\char`\/}}{\Wrappedafterbreak}{\hbox{\char`\/}}}%
            \catcode`\.\active
            \catcode`\,\active
            \catcode`\;\active
            \catcode`\:\active
            \catcode`\?\active
            \catcode`\!\active
            \catcode`\/\active
            \lccode`\~`\~
        }
    \makeatother

    \let\OriginalVerbatim=\Verbatim
    \makeatletter
    \renewcommand{\Verbatim}[1][1]{%
        %\parskip\z@skip
        \sbox\Wrappedcontinuationbox {\Wrappedcontinuationsymbol}%
        \sbox\Wrappedvisiblespacebox {\FV@SetupFont\Wrappedvisiblespace}%
        \def\FancyVerbFormatLine ##1{\hsize\linewidth
            \vtop{\raggedright\hyphenpenalty\z@\exhyphenpenalty\z@
                \doublehyphendemerits\z@\finalhyphendemerits\z@
                \strut ##1\strut}%
        }%
        % If the linebreak is at a space, the latter will be displayed as visible
        % space at end of first line, and a continuation symbol starts next line.
        % Stretch/shrink are however usually zero for typewriter font.
        \def\FV@Space {%
            \nobreak\hskip\z@ plus\fontdimen3\font minus\fontdimen4\font
            \discretionary{\copy\Wrappedvisiblespacebox}{\Wrappedafterbreak}
            {\kern\fontdimen2\font}%
        }%

        % Allow breaks at special characters using \PYG... macros.
        \Wrappedbreaksatspecials
        % Breaks at punctuation characters . , ; ? ! and / need catcode=\active
        \OriginalVerbatim[#1,codes*=\Wrappedbreaksatpunct]%
    }
    \makeatother

    % Exact colors from NB
    \definecolor{incolor}{HTML}{303F9F}
    \definecolor{outcolor}{HTML}{D84315}
    \definecolor{cellborder}{HTML}{CFCFCF}
    \definecolor{cellbackground}{HTML}{F7F7F7}

    % prompt
    \makeatletter
    \newcommand{\boxspacing}{\kern\kvtcb@left@rule\kern\kvtcb@boxsep}
    \makeatother
    \newcommand{\prompt}[4]{
        {\ttfamily\llap{{\color{#2}[#3]:\hspace{3pt}#4}}\vspace{-\baselineskip}}
    }
    

    
    % Prevent overflowing lines due to hard-to-break entities
    \sloppy
    % Setup hyperref package
    \hypersetup{
      breaklinks=true,  % so long urls are correctly broken across lines
      colorlinks=true,
      urlcolor=urlcolor,
      linkcolor=linkcolor,
      citecolor=citecolor,
      }
    % Slightly bigger margins than the latex defaults
    
    \geometry{verbose,tmargin=1in,bmargin=1in,lmargin=1in,rmargin=1in}
    
    

\begin{document}
    
    \maketitle
    
    

    
    \section{4.25}\label{section}

已知

\[
x(t) = \begin{cases}
2, & -1 \le t \le 0 \\
2 - t, & 0 < t \le 1 \\
t, & 1 < t \le 2 \\
2, & 2 < t \le 3 \\
0, & \rm otherwise
\end{cases}
\]

\(X(j\omega)\) 为 \(x(t)\) 的傅立叶变换,求:

\subsection{\texorpdfstring{(a)
\(\sphericalangle X(j\omega)\)}{(a) \textbackslash sphericalangle X(j\textbackslash omega)}}\label{a-sphericalangle-xjomega}

令 \(y(t) = x(t + 1)\),则 \(y(t)\) 为实偶信号。故 \(Y(j\omega)\)
也为实偶函数,有

\[
\arg Y(j\omega) = 0
\]

又由傅立叶变换的时移性质,有

\[
x(t) \stackrel{\mathcal{F}}{\longleftrightarrow} e^{-j\omega}Y(j\omega)
\]

即

\[
X(j\omega) = e^{-j\omega}Y(j\omega)
\]

故 \(\arg (X(j\omega)e^{j\omega})=0\),得到

\[\boxed{
\sphericalangle X(j\omega) = -\omega.}
\]

\subsection{\texorpdfstring{(b) \(X(j0)\)}{(b) X(j0)}}\label{b-xj0}

\[\boxed{
X(j0) = \int_{-\infty}^{+\infty}x(t)e^{-j0t}dt = \int_{-\infty}^{+\infty}x(t)dt = 7.}
\]

\subsection{\texorpdfstring{(c)
\(\int_{-\infty}^{+\infty}X(j\omega)d\omega\)}{(c) \textbackslash int\_\{-\textbackslash infty\}\^{}\{+\textbackslash infty\}X(j\textbackslash omega)d\textbackslash omega}}\label{c-int_-inftyinftyxjomegadomega}

\[\boxed{
\int_{-\infty}^{+\infty}X(j\omega) d\omega = \left[\int_{-\infty}^{+\infty}X(j\omega)e^{j\omega t}dt\right]_{t = 0} = 2\pi x(0) = 4\pi.}
\]

\subsection{\texorpdfstring{(d)
\(\int_{-\infty}^{+\infty}X(j\omega) \frac{2\sin \omega}{\omega}e^{j2\omega} d\omega\)}{(d) \textbackslash int\_\{-\textbackslash infty\}\^{}\{+\textbackslash infty\}X(j\textbackslash omega) \textbackslash frac\{2\textbackslash sin \textbackslash omega\}\{\textbackslash omega\}e\^{}\{j2\textbackslash omega\} d\textbackslash omega}}\label{d-int_-inftyinftyxjomega-frac2sin-omegaomegaej2omega-domega}

令
\(z(t) = \begin{cases} 1, & -1\le t\le 1 \\ 0, & \rm otherwise\end{cases}\)
为方波信号,则

\[
z(t) \stackrel{\mathcal{F}}{\longleftrightarrow } 2\frac{\sin \omega}{\omega}
\]

故

\[
z(t+2) \stackrel{\mathcal{F}} {\longleftrightarrow} \frac{2\sin \omega}{\omega}e^{j2\omega}
\]

由傅立叶变换的卷积性质,

\[\boxed{
\int_{-\infty}^{+\infty} X(j\omega)\cdot \frac{2\sin\omega}{\omega}e^{j2\omega} d\omega = 2\pi [x(t)*z(t+2)]_{t = 0} = 7\pi.}
\]

\subsection{\texorpdfstring{(f) 画出 \(\mathfrak{R}\{X(j\omega)\}\)
的逆变换}{(f) 画出 \textbackslash mathfrak\{R\}\textbackslash\{X(j\textbackslash omega)\textbackslash\} 的逆变换}}\label{f-ux753bux51fa-mathfrakrxjomega-ux7684ux9006ux53d8ux6362}

\[\boxed{
\mathcal{F}^{-1} \{\mathfrak{R}\{X(j\omega)\}\} = {\rm Ev}\{x(t)\} = \frac{1}{2}[x(t) + x(-t)].}
\]

    \begin{tcolorbox}[breakable, size=fbox, boxrule=1pt, pad at break*=1mm,colback=cellbackground, colframe=cellborder]
\prompt{In}{incolor}{2}{\boxspacing}
\begin{Verbatim}[commandchars=\\\{\}]
\PY{k+kn}{import} \PY{n+nn}{numpy} \PY{k}{as} \PY{n+nn}{np}
\PY{k+kn}{import} \PY{n+nn}{matplotlib}\PY{n+nn}{.}\PY{n+nn}{pyplot} \PY{k}{as} \PY{n+nn}{plt}
\PY{k}{def} \PY{n+nf}{x}\PY{p}{(}\PY{n}{t}\PY{p}{)}\PY{p}{:}
\PY{+w}{    }\PY{l+s+sd}{\PYZdq{}\PYZdq{}\PYZdq{}}
\PY{l+s+sd}{    定义输入信号 x(t):}
\PY{l+s+sd}{      x(t) = 2               , \PYZhy{}1 \PYZlt{}= t \PYZlt{}= 0}
\PY{l+s+sd}{           = 2 \PYZhy{} t           , 0 \PYZlt{} t \PYZlt{}= 1}
\PY{l+s+sd}{           = t               , 1 \PYZlt{} t \PYZlt{}= 2}
\PY{l+s+sd}{           = 2               , 2 \PYZlt{} t \PYZlt{}= 3}
\PY{l+s+sd}{           = 0               , otherwise}
\PY{l+s+sd}{    \PYZdq{}\PYZdq{}\PYZdq{}}
    \PY{n}{t} \PY{o}{=} \PY{n}{np}\PY{o}{.}\PY{n}{array}\PY{p}{(}\PY{n}{t}\PY{p}{)}
    \PY{n}{res} \PY{o}{=} \PY{n}{np}\PY{o}{.}\PY{n}{zeros\PYZus{}like}\PY{p}{(}\PY{n}{t}\PY{p}{)}
    \PY{c+c1}{\PYZsh{} 区间 \PYZhy{}1 \PYZlt{}= t \PYZlt{}= 0}
    \PY{n}{mask} \PY{o}{=} \PY{p}{(}\PY{n}{t} \PY{o}{\PYZgt{}}\PY{o}{=} \PY{o}{\PYZhy{}}\PY{l+m+mi}{1}\PY{p}{)} \PY{o}{\PYZam{}} \PY{p}{(}\PY{n}{t} \PY{o}{\PYZlt{}}\PY{o}{=} \PY{l+m+mi}{0}\PY{p}{)}
    \PY{n}{res}\PY{p}{[}\PY{n}{mask}\PY{p}{]} \PY{o}{=} \PY{l+m+mi}{2}
    \PY{c+c1}{\PYZsh{} 区间 0 \PYZlt{} t \PYZlt{}= 1}
    \PY{n}{mask} \PY{o}{=} \PY{p}{(}\PY{n}{t} \PY{o}{\PYZgt{}} \PY{l+m+mi}{0}\PY{p}{)} \PY{o}{\PYZam{}} \PY{p}{(}\PY{n}{t} \PY{o}{\PYZlt{}}\PY{o}{=} \PY{l+m+mi}{1}\PY{p}{)}
    \PY{n}{res}\PY{p}{[}\PY{n}{mask}\PY{p}{]} \PY{o}{=} \PY{l+m+mi}{2} \PY{o}{\PYZhy{}} \PY{n}{t}\PY{p}{[}\PY{n}{mask}\PY{p}{]}
    \PY{c+c1}{\PYZsh{} 区间 1 \PYZlt{} t \PYZlt{}= 2}
    \PY{n}{mask} \PY{o}{=} \PY{p}{(}\PY{n}{t} \PY{o}{\PYZgt{}} \PY{l+m+mi}{1}\PY{p}{)} \PY{o}{\PYZam{}} \PY{p}{(}\PY{n}{t} \PY{o}{\PYZlt{}}\PY{o}{=} \PY{l+m+mi}{2}\PY{p}{)}
    \PY{n}{res}\PY{p}{[}\PY{n}{mask}\PY{p}{]} \PY{o}{=} \PY{n}{t}\PY{p}{[}\PY{n}{mask}\PY{p}{]}
    \PY{c+c1}{\PYZsh{} 区间 2 \PYZlt{} t \PYZlt{}= 3}
    \PY{n}{mask} \PY{o}{=} \PY{p}{(}\PY{n}{t} \PY{o}{\PYZgt{}} \PY{l+m+mi}{2}\PY{p}{)} \PY{o}{\PYZam{}} \PY{p}{(}\PY{n}{t} \PY{o}{\PYZlt{}}\PY{o}{=} \PY{l+m+mi}{3}\PY{p}{)}
    \PY{n}{res}\PY{p}{[}\PY{n}{mask}\PY{p}{]} \PY{o}{=} \PY{l+m+mi}{2}
    \PY{k}{return} \PY{n}{res}

\PY{k}{def} \PY{n+nf}{ev\PYZus{}x}\PY{p}{(}\PY{n}{t}\PY{p}{)}\PY{p}{:}
\PY{+w}{    }\PY{l+s+sd}{\PYZdq{}\PYZdq{}\PYZdq{}}
\PY{l+s+sd}{    求 x(t) 的偶部,即:}
\PY{l+s+sd}{      Ev\PYZob{}x(t)\PYZcb{} = (x(t) + x(\PYZhy{}t)) / 2}
\PY{l+s+sd}{    \PYZdq{}\PYZdq{}\PYZdq{}}
    \PY{k}{return} \PY{l+m+mf}{0.5} \PY{o}{*} \PY{p}{(}\PY{n}{x}\PY{p}{(}\PY{n}{t}\PY{p}{)} \PY{o}{+} \PY{n}{x}\PY{p}{(}\PY{o}{\PYZhy{}}\PY{n}{t}\PY{p}{)}\PY{p}{)}

\PY{c+c1}{\PYZsh{} 选取足够覆盖非零部分的 t 范围}
\PY{c+c1}{\PYZsh{} x(t) 非零区间为 [\PYZhy{}1, 3],x(\PYZhy{}t)非零区间为 [\PYZhy{}3, 1],故 ev\PYZob{}x(t)\PYZcb{} 非零区间为 [\PYZhy{}3, 3]}
\PY{n}{t\PYZus{}vals} \PY{o}{=} \PY{n}{np}\PY{o}{.}\PY{n}{linspace}\PY{p}{(}\PY{o}{\PYZhy{}}\PY{l+m+mi}{4}\PY{p}{,} \PY{l+m+mi}{4}\PY{p}{,} \PY{l+m+mi}{600}\PY{p}{)}
\PY{n}{y\PYZus{}vals} \PY{o}{=} \PY{n}{ev\PYZus{}x}\PY{p}{(}\PY{n}{t\PYZus{}vals}\PY{p}{)}

\PY{c+c1}{\PYZsh{} 绘图}
\PY{n}{plt}\PY{o}{.}\PY{n}{figure}\PY{p}{(}\PY{n}{figsize}\PY{o}{=}\PY{p}{(}\PY{l+m+mi}{8}\PY{p}{,} \PY{l+m+mi}{4}\PY{p}{)}\PY{p}{)}
\PY{n}{plt}\PY{o}{.}\PY{n}{plot}\PY{p}{(}\PY{n}{t\PYZus{}vals}\PY{p}{,} \PY{n}{y\PYZus{}vals}\PY{p}{,} \PY{l+s+s1}{\PYZsq{}}\PY{l+s+s1}{b\PYZhy{}}\PY{l+s+s1}{\PYZsq{}}\PY{p}{,} \PY{n}{linewidth}\PY{o}{=}\PY{l+m+mi}{2}\PY{p}{,} \PY{n}{label}\PY{o}{=}\PY{l+s+sa}{r}\PY{l+s+s1}{\PYZsq{}}\PY{l+s+s1}{\PYZdl{}Ev}\PY{l+s+s1}{\PYZbs{}}\PY{l+s+s1}{\PYZob{}}\PY{l+s+s1}{x(t)}\PY{l+s+s1}{\PYZbs{}}\PY{l+s+s1}{\PYZcb{}=}\PY{l+s+s1}{\PYZbs{}}\PY{l+s+s1}{frac}\PY{l+s+si}{\PYZob{}1\PYZcb{}}\PY{l+s+si}{\PYZob{}2\PYZcb{}}\PY{l+s+s1}{[x(t)+x(\PYZhy{}t)]\PYZdl{}}\PY{l+s+s1}{\PYZsq{}}\PY{p}{)}
\PY{n}{plt}\PY{o}{.}\PY{n}{xlabel}\PY{p}{(}\PY{l+s+s1}{\PYZsq{}}\PY{l+s+s1}{t}\PY{l+s+s1}{\PYZsq{}}\PY{p}{)}
\PY{n}{plt}\PY{o}{.}\PY{n}{ylabel}\PY{p}{(}\PY{l+s+s1}{\PYZsq{}}\PY{l+s+s1}{Amplitude}\PY{l+s+s1}{\PYZsq{}}\PY{p}{)}
\PY{n}{plt}\PY{o}{.}\PY{n}{title}\PY{p}{(}\PY{l+s+s1}{\PYZsq{}}\PY{l+s+s1}{Even Part of x(t) (Inverse Transform of \PYZdl{}}\PY{l+s+s1}{\PYZbs{}}\PY{l+s+s1}{Re}\PY{l+s+s1}{\PYZbs{}}\PY{l+s+s1}{\PYZob{}}\PY{l+s+s1}{X(j}\PY{l+s+s1}{\PYZbs{}}\PY{l+s+s1}{omega)}\PY{l+s+s1}{\PYZbs{}}\PY{l+s+s1}{\PYZcb{}\PYZdl{})}\PY{l+s+s1}{\PYZsq{}}\PY{p}{)}
\PY{n}{plt}\PY{o}{.}\PY{n}{legend}\PY{p}{(}\PY{p}{)}
\PY{n}{plt}\PY{o}{.}\PY{n}{grid}\PY{p}{(}\PY{k+kc}{True}\PY{p}{)}
\PY{n}{plt}\PY{o}{.}\PY{n}{show}\PY{p}{(}\PY{p}{)}
\end{Verbatim}
\end{tcolorbox}

    \begin{Verbatim}[commandchars=\\\{\}]
<>:45: SyntaxWarning: invalid escape sequence '\textbackslash{}R'
<>:45: SyntaxWarning: invalid escape sequence '\textbackslash{}R'
/var/folders/c8/c991tv453sl7r0wy6k3h9zjr0000gn/T/ipykernel\_41050/3953701225.py:4
5: SyntaxWarning: invalid escape sequence '\textbackslash{}R'
  plt.title('Even Part of x(t) (Inverse Transform of \$\textbackslash{}Re\textbackslash{}\{X(j\textbackslash{}omega)\textbackslash{}\}\$)')
    \end{Verbatim}

    \begin{center}
    \adjustimage{max size={0.9\linewidth}{0.9\paperheight}}{output_1_1.png}
    \end{center}
    { \hspace*{\fill} \\}
    
    \section{4.35}\label{section}

有一个连续时间线性时不变系统,其频率响应为

\[
H(j\omega) = \frac{a - j\omega}{a + j\omega}
\]

其中,\(a > 0\)。

\subsection{\texorpdfstring{(a) \(|H(j\omega)|\)
是多少?\(\angle H(j\omega)\)
是多少?该系统的单位冲激响应是什么?}{(a) \textbar H(j\textbackslash omega)\textbar{} 是多少?\textbackslash angle H(j\textbackslash omega) 是多少?该系统的单位冲激响应是什么?}}\label{a-hjomega-ux662fux591aux5c11angle-hjomega-ux662fux591aux5c11ux8be5ux7cfbux7edfux7684ux5355ux4f4dux51b2ux6fc0ux54cdux5e94ux662fux4ec0ux4e48}

计算分子分母的模: \(|a - j\omega|=\sqrt{a^2+\omega^2}\)
\(|a + j\omega|=\sqrt{a^2+\omega^2}\)

因此, \[\boxed{
|H(j\omega)|=\frac{|a-j\omega|}{|a+j\omega|}=\frac{\sqrt{a^2+\omega^2}}{\sqrt{a^2+\omega^2}}=1.}
\]

\begin{center}\rule{0.5\linewidth}{0.5pt}\end{center}

利用复数的极角运算法则: \[
\angle H(j\omega)= \angle (a - j\omega)- \angle (a+j\omega).
\]

\begin{itemize}
\tightlist
\item
  对于 \(a-j\omega\)(实部正、虚部负),其相角为 \[
  \angle (a-j\omega)= -\arctan\left(\frac{\omega}{a}\right).
  \]
\item
  对于 \(a+j\omega\)(实部正、虚部正),其相角为 \[
  \angle (a+j\omega)= \arctan\left(\frac{\omega}{a}\right).
  \]
\end{itemize}

因此, \[
\boxed{
\angle H(j\omega)= -\arctan\left(\frac{\omega}{a}\right) - \arctan\left(\frac{\omega}{a}\right)
= - 2\arctan\left(\frac{\omega}{a}\right).}
\]

\begin{center}\rule{0.5\linewidth}{0.5pt}\end{center}

因为

\[
H(j\omega) = \frac{a - j\omega}{a + j\omega} = \frac{2a}{a+j\omega} - 1
\]

已知当 \(a > 0\) 时,有

\[
\mathcal{F}\{e^{-at}u(t)\} = \frac{1}{a + j\omega}, \quad \mathcal{F}\{\delta(t)\} =1
\]

故单位冲激响应为

\[\boxed{
h(t) = 2ae^{-at}u(t) - \delta(t)}
\]

\subsection{\texorpdfstring{(b) 当
\(a=1\),\(x(t) = \cos(t/\sqrt 3) + \cos t + \cos \sqrt 3 t\)
时,求系统的输出
\(y(t)\)}{(b) 当 a=1,x(t) = \textbackslash cos(t/\textbackslash sqrt 3) + \textbackslash cos t + \textbackslash cos \textbackslash sqrt 3 t 时,求系统的输出 y(t)}}\label{b-ux5f53-a1xt-costsqrt-3-cos-t-cos-sqrt-3-t-ux65f6ux6c42ux7cfbux7edfux7684ux8f93ux51fa-yt}

对于线性时不变系统,频率响应为 \(H(j\omega)\),若输入为一个余弦信号 \[
\cos(\omega_0t),
\] 则对应的输出为 \[
y(t)= |H(j\omega_0)|\cos\Bigl(\omega_0 t + \angle H(j\omega_0)\Bigr).
\] 由于 \(H(j\omega)\) 是全通系统(幅度为
1),所以各频率分量的幅度不变,仅增加相位偏移。

令 \(a=1\) 后, \[
H(j\omega)= \frac{1-j\omega}{1+j\omega},
\] 各频率分量的相位为: \[
\angle H(j\omega)= -2\arctan(\omega).
\]

\begin{enumerate}
\def\labelenumi{\arabic{enumi}.}
\item
  \textbf{对于} \(\cos\Bigl(\frac{t}{\sqrt3}\Bigr)\):

  设 \(\omega_1=\frac{1}{\sqrt3}\),则 \[
  \angle H\Bigl(j\frac{1}{\sqrt3}\Bigr) = -2\arctan\Bigl(\frac{1}{\sqrt3}\Bigr).
  \] 我们知道 \(\arctan(1/\sqrt3)=\pi/6\),故相位为 \[
  -2\cdot\frac{\pi}{6} = -\frac{\pi}{3}.
  \] 输出分量为 \[
  y_1(t)= \cos\Bigl(\frac{t}{\sqrt3} - \frac{\pi}{3}\Bigr).
  \]
\item
  \textbf{对于} \(\cos t\):

  设 \(\omega_2=1\),则 \[
  \angle H(j1)= -2\arctan(1)= -2\cdot \frac{\pi}{4} = -\frac{\pi}{2}.
  \] 输出分量为 \[
  y_2(t)= \cos\Bigl(t - \frac{\pi}{2}\Bigr).
  \]
\item
  \textbf{对于} \(\cos\Bigl(\sqrt3\,t\Bigr)\):

  设 \(\omega_3=\sqrt3\),则 \[
  \angle H(j\sqrt3)= -2\arctan(\sqrt3) = -2\cdot \frac{\pi}{3} = -\frac{2\pi}{3}.
  \] 输出分量为 \[
  y_3(t)= \cos\Bigl(\sqrt3\,t - \frac{2\pi}{3}\Bigr).
  \]
\end{enumerate}

由于系统是线性时不变的,各分量独立经过系统后叠加,所以系统输出为

\[\boxed{
y(t)= y_1(t) + y_2(t) + y_3(t)
=\cos\Bigl(\frac{t}{\sqrt3} - \frac{\pi}{3}\Bigr) + \cos\Bigl(t - \frac{\pi}{2}\Bigr) + \cos\Bigl(\sqrt3\,t - \frac{2\pi}{3}\Bigr).}
\]

这就是当 \(a=1\) 时系统的输出。

    \begin{tcolorbox}[breakable, size=fbox, boxrule=1pt, pad at break*=1mm,colback=cellbackground, colframe=cellborder]
\prompt{In}{incolor}{9}{\boxspacing}
\begin{Verbatim}[commandchars=\\\{\}]
\PY{k+kn}{import} \PY{n+nn}{numpy} \PY{k}{as} \PY{n+nn}{np}
\PY{k+kn}{import} \PY{n+nn}{matplotlib}\PY{n+nn}{.}\PY{n+nn}{pyplot} \PY{k}{as} \PY{n+nn}{plt}
\PY{k+kn}{import} \PY{n+nn}{matplotlib}\PY{n+nn}{.}\PY{n+nn}{pyplot} \PY{k}{as} \PY{n+nn}{plt}

\PY{n}{plt}\PY{o}{.}\PY{n}{rcParams}\PY{p}{[}\PY{l+s+s1}{\PYZsq{}}\PY{l+s+s1}{font.family}\PY{l+s+s1}{\PYZsq{}}\PY{p}{]} \PY{o}{=} \PY{p}{[}\PY{l+s+s1}{\PYZsq{}}\PY{l+s+s1}{Cochin}\PY{l+s+s1}{\PYZsq{}}\PY{p}{]}
\PY{c+c1}{\PYZsh{} 参数设置}
\PY{n}{a} \PY{o}{=} \PY{l+m+mi}{1}  \PY{c+c1}{\PYZsh{} 题目中令 a = 1}
\PY{n}{sqrt3} \PY{o}{=} \PY{n}{np}\PY{o}{.}\PY{n}{sqrt}\PY{p}{(}\PY{l+m+mi}{3}\PY{p}{)}

\PY{c+c1}{\PYZsh{} 定义时间区间}
\PY{n}{t} \PY{o}{=} \PY{n}{np}\PY{o}{.}\PY{n}{linspace}\PY{p}{(}\PY{l+m+mi}{0}\PY{p}{,} \PY{l+m+mi}{10}\PY{p}{,} \PY{l+m+mi}{1000}\PY{p}{)}  \PY{c+c1}{\PYZsh{} 选取 0 到 10 秒,足够显示波形}

\PY{c+c1}{\PYZsh{} 输入信号 x(t)}
\PY{c+c1}{\PYZsh{} x(t) = cos(t/√3) + cos(t) + cos(√3 t)}
\PY{n}{x\PYZus{}t} \PY{o}{=} \PY{n}{np}\PY{o}{.}\PY{n}{cos}\PY{p}{(}\PY{n}{t}\PY{o}{/}\PY{n}{sqrt3}\PY{p}{)} \PY{o}{+} \PY{n}{np}\PY{o}{.}\PY{n}{cos}\PY{p}{(}\PY{n}{t}\PY{p}{)} \PY{o}{+} \PY{n}{np}\PY{o}{.}\PY{n}{cos}\PY{p}{(}\PY{n}{sqrt3} \PY{o}{*} \PY{n}{t}\PY{p}{)}

\PY{c+c1}{\PYZsh{} 系统频率响应为 H(jω) = (1 \PYZhy{} jω)/(1 + jω)}
\PY{c+c1}{\PYZsh{} 对于余弦输入,LTI 系统的输出只改变相位:}
\PY{c+c1}{\PYZsh{} 当输入为 cos(ω0 t) 时,输出 y(t)= cos(ω0 t + angle(H(jω0)))}
\PY{c+c1}{\PYZsh{} 其中,angle(H(jω0)) = \PYZhy{}2 arctan(ω0)}
\PY{c+c1}{\PYZsh{} 分别对分量进行处理:}
\PY{c+c1}{\PYZsh{} 分量1: ω1 = 1/√3, 相位偏移 = \PYZhy{}2*arctan(1/√3) = \PYZhy{}2*(π/6) = \PYZhy{}π/3}
\PY{c+c1}{\PYZsh{} 分量2: ω2 = 1,     相位偏移 = \PYZhy{}2*arctan(1)   = \PYZhy{}π/2}
\PY{c+c1}{\PYZsh{} 分量3: ω3 = √3,    相位偏移 = \PYZhy{}2*arctan(√3)  = \PYZhy{}2*(π/3) = \PYZhy{}2π/3}

\PY{n}{y1} \PY{o}{=} \PY{n}{np}\PY{o}{.}\PY{n}{cos}\PY{p}{(}\PY{n}{t}\PY{o}{/}\PY{n}{sqrt3} \PY{o}{\PYZhy{}} \PY{n}{np}\PY{o}{.}\PY{n}{pi}\PY{o}{/}\PY{l+m+mi}{3}\PY{p}{)}
\PY{n}{y2} \PY{o}{=} \PY{n}{np}\PY{o}{.}\PY{n}{cos}\PY{p}{(}\PY{n}{t} \PY{o}{\PYZhy{}} \PY{n}{np}\PY{o}{.}\PY{n}{pi}\PY{o}{/}\PY{l+m+mi}{2}\PY{p}{)}
\PY{n}{y3} \PY{o}{=} \PY{n}{np}\PY{o}{.}\PY{n}{cos}\PY{p}{(}\PY{n}{sqrt3} \PY{o}{*} \PY{n}{t} \PY{o}{\PYZhy{}} \PY{l+m+mi}{2}\PY{o}{*}\PY{n}{np}\PY{o}{.}\PY{n}{pi}\PY{o}{/}\PY{l+m+mi}{3}\PY{p}{)}

\PY{c+c1}{\PYZsh{} 输出信号为各分量的叠加}
\PY{n}{y\PYZus{}t} \PY{o}{=} \PY{n}{y1} \PY{o}{+} \PY{n}{y2} \PY{o}{+} \PY{n}{y3}

\PY{c+c1}{\PYZsh{} 绘图}
\PY{n}{plt}\PY{o}{.}\PY{n}{figure}\PY{p}{(}\PY{n}{figsize}\PY{o}{=}\PY{p}{(}\PY{l+m+mi}{10}\PY{p}{,} \PY{l+m+mi}{6}\PY{p}{)}\PY{p}{)}

\PY{c+c1}{\PYZsh{} 绘制输入信号}
\PY{n}{plt}\PY{o}{.}\PY{n}{subplot}\PY{p}{(}\PY{l+m+mi}{2}\PY{p}{,} \PY{l+m+mi}{1}\PY{p}{,} \PY{l+m+mi}{1}\PY{p}{)}
\PY{n}{plt}\PY{o}{.}\PY{n}{plot}\PY{p}{(}\PY{n}{t}\PY{p}{,} \PY{n}{x\PYZus{}t}\PY{p}{,} \PY{l+s+s1}{\PYZsq{}}\PY{l+s+s1}{b}\PY{l+s+s1}{\PYZsq{}}\PY{p}{,} \PY{n}{label}\PY{o}{=}\PY{l+s+sa}{r}\PY{l+s+s1}{\PYZsq{}}\PY{l+s+s1}{\PYZdl{}x(t)=}\PY{l+s+s1}{\PYZbs{}}\PY{l+s+s1}{cos(t/}\PY{l+s+s1}{\PYZbs{}}\PY{l+s+s1}{sqrt}\PY{l+s+si}{\PYZob{}3\PYZcb{}}\PY{l+s+s1}{)+}\PY{l+s+s1}{\PYZbs{}}\PY{l+s+s1}{cos(t)+}\PY{l+s+s1}{\PYZbs{}}\PY{l+s+s1}{cos(}\PY{l+s+s1}{\PYZbs{}}\PY{l+s+s1}{sqrt}\PY{l+s+si}{\PYZob{}3\PYZcb{}}\PY{l+s+s1}{t)\PYZdl{}}\PY{l+s+s1}{\PYZsq{}}\PY{p}{)}
\PY{n}{plt}\PY{o}{.}\PY{n}{title}\PY{p}{(}\PY{l+s+s1}{\PYZsq{}}\PY{l+s+s1}{\PYZdl{}x(t)\PYZdl{}}\PY{l+s+s1}{\PYZsq{}}\PY{p}{)}
\PY{n}{plt}\PY{o}{.}\PY{n}{xlabel}\PY{p}{(}\PY{l+s+s1}{\PYZsq{}}\PY{l+s+s1}{t (s)}\PY{l+s+s1}{\PYZsq{}}\PY{p}{)}
\PY{n}{plt}\PY{o}{.}\PY{n}{ylabel}\PY{p}{(}\PY{l+s+s1}{\PYZsq{}}\PY{l+s+s1}{Amplitude}\PY{l+s+s1}{\PYZsq{}}\PY{p}{)}
\PY{n}{plt}\PY{o}{.}\PY{n}{legend}\PY{p}{(}\PY{p}{)}
\PY{n}{plt}\PY{o}{.}\PY{n}{grid}\PY{p}{(}\PY{k+kc}{True}\PY{p}{)}

\PY{c+c1}{\PYZsh{} 绘制输出信号}
\PY{n}{plt}\PY{o}{.}\PY{n}{subplot}\PY{p}{(}\PY{l+m+mi}{2}\PY{p}{,} \PY{l+m+mi}{1}\PY{p}{,} \PY{l+m+mi}{2}\PY{p}{)}
\PY{n}{plt}\PY{o}{.}\PY{n}{plot}\PY{p}{(}\PY{n}{t}\PY{p}{,} \PY{n}{y\PYZus{}t}\PY{p}{,} \PY{l+s+s1}{\PYZsq{}}\PY{l+s+s1}{r}\PY{l+s+s1}{\PYZsq{}}\PY{p}{,} \PY{n}{label}\PY{o}{=}\PY{l+s+sa}{r}\PY{l+s+s1}{\PYZsq{}}\PY{l+s+s1}{\PYZdl{}y(t)=}\PY{l+s+s1}{\PYZbs{}}\PY{l+s+s1}{cos(t/}\PY{l+s+s1}{\PYZbs{}}\PY{l+s+s1}{sqrt}\PY{l+s+si}{\PYZob{}3\PYZcb{}}\PY{l+s+s1}{\PYZhy{}}\PY{l+s+s1}{\PYZbs{}}\PY{l+s+s1}{pi/3)+}\PY{l+s+s1}{\PYZbs{}}\PY{l+s+s1}{cos(t\PYZhy{}}\PY{l+s+s1}{\PYZbs{}}\PY{l+s+s1}{pi/2)+}\PY{l+s+s1}{\PYZbs{}}\PY{l+s+s1}{cos(}\PY{l+s+s1}{\PYZbs{}}\PY{l+s+s1}{sqrt}\PY{l+s+si}{\PYZob{}3\PYZcb{}}\PY{l+s+s1}{t\PYZhy{}2}\PY{l+s+s1}{\PYZbs{}}\PY{l+s+s1}{pi/3)\PYZdl{}}\PY{l+s+s1}{\PYZsq{}}\PY{p}{)}
\PY{n}{plt}\PY{o}{.}\PY{n}{title}\PY{p}{(}\PY{l+s+s1}{\PYZsq{}}\PY{l+s+s1}{\PYZdl{}y(t)\PYZdl{} (a=1)}\PY{l+s+s1}{\PYZsq{}}\PY{p}{)}
\PY{n}{plt}\PY{o}{.}\PY{n}{xlabel}\PY{p}{(}\PY{l+s+s1}{\PYZsq{}}\PY{l+s+s1}{t (s)}\PY{l+s+s1}{\PYZsq{}}\PY{p}{)}
\PY{n}{plt}\PY{o}{.}\PY{n}{ylabel}\PY{p}{(}\PY{l+s+s1}{\PYZsq{}}\PY{l+s+s1}{Amplitude}\PY{l+s+s1}{\PYZsq{}}\PY{p}{)}
\PY{n}{plt}\PY{o}{.}\PY{n}{legend}\PY{p}{(}\PY{p}{)}
\PY{n}{plt}\PY{o}{.}\PY{n}{grid}\PY{p}{(}\PY{k+kc}{True}\PY{p}{)}

\PY{n}{plt}\PY{o}{.}\PY{n}{tight\PYZus{}layout}\PY{p}{(}\PY{p}{)}
\PY{n}{plt}\PY{o}{.}\PY{n}{show}\PY{p}{(}\PY{p}{)}
\end{Verbatim}
\end{tcolorbox}

    \begin{center}
    \adjustimage{max size={0.9\linewidth}{0.9\paperheight}}{output_3_0.png}
    \end{center}
    { \hspace*{\fill} \\}
    
    \section{4.22(a)}\label{a}

对下列变换求对应的连续时间信号

\[
X(j\omega) = \frac{2\sin(3(\omega - 2\pi))}{\omega - 2\pi}
\]

\subsection{Answer}\label{answer}

已知
\(x_1(t) = \begin{cases} 1, & -T_1 < t < T_1 \\ 0, & \rm other wise\end{cases}\),则

\[
X_1(j\omega) = \frac{2\sin \omega T_1}{\omega}
\]

由傅立叶变化的频移性质,有

\[
e^{j2\pi t}x_1(t) \stackrel{\mathcal{F}}{\longleftrightarrow} X_1(j(\omega-2\pi))
\]

即

\[
\mathcal{F}\{ e^{j2\pi t}x_1(t) \} = \frac{2\sin((\omega - 2\pi) T_1)}{\omega - 2\pi}
\]

令 \(T_1 = 3\),则得到该变换对应的连续时间信号

\[
x(t) = \begin{cases}
e^{j2\pi t}, & -3 < t < 3 \\
0, & \rm otherwise
\end{cases}
\]

    \section{4.28(a)}\label{a}

设 \(x(t)\) 有傅立叶变换 \(X(j\omega)\),令 \(p(t)\) 为基波频率为
\(\omega_0\) 的周期信号,其傅立叶级数表示是

\[
p(t) = \sum_{n = -\infty}^{+\infty}a_ne^{jn\omega_0t}
\]

求

\[
y(t) = x(t)p(t)
\]

的傅立叶变换表达式。

\subsection{Answer}\label{answer}

由傅立叶变换的相乘性质,有

\[
y(t)=x(t)p(t) \stackrel{\mathcal{F}}{\longleftrightarrow} Y(j\omega)=\frac{1}{2\pi}(X*P)(j\omega)
\]

由周期信号的傅立叶变换性质可知,

\[
P(j\omega) = \sum_{n = -\infty}^{+\infty}2\pi a_n\delta(\omega - n\omega_0)
\]

故

\[
\boxed{
\begin{aligned}
Y(j\omega) &= \frac{1}{2\pi}(X*P)(j\omega) \\
&= \sum_{n = -\infty}^{+\infty}a_nX(j(\omega - n\omega_0))
\end{aligned}}
\]

    \section{4.36}\label{section}

考虑一个线性时不变系统,输入 \(x(t)\) 为

\[
x(t) = \left[e^{-t} + e^{-3t}\right]u(t)
\]

响应 \(y(t)\) 为

\[
y(t) = \left[2e^{-t}- 2e^{-4t}\right]u(t)
\]

\subsection{(a)
求系统的频率响应}\label{a-ux6c42ux7cfbux7edfux7684ux9891ux7387ux54cdux5e94}

设 \(y(t) = (h*x)(t)\), 则有

\[
\mathcal{F}\{y(t)\} = Y(j\omega) = H(j\omega)X(j\omega)
\]

又因为,

\[
\begin{aligned}
X(j\omega) &= \int_{-\infty}^{+\infty}x(t)e^{-j\omega t}dt \\
&= \frac{1}{1+j\omega} + \frac{1}{3+j\omega}\\
&=\frac{2(2+j\omega)}{(1+j\omega)(3+j\omega)}.
\\
Y(j\omega)
&= \int_{-\infty}^{+\infty}y(t)e^{-j\omega t}dt \\
&= \frac{2}{1+j\omega} - \frac{2}{4+j\omega}\\
&=\frac{6}{(1+j\omega)(4+j\omega)}.
\end{aligned}
\]

因此,由 \(H(j\omega)= \dfrac{Y(j\omega)}{X(j\omega)}\) 得

\[\boxed{
H(j\omega)=\frac{\frac{6}{(1+j\omega)(4+j\omega)}}
{\frac{2(2+j\omega)}{(1+j\omega)(3+j\omega)}}
=\frac{6}{2(2+j\omega)}\cdot\frac{(3+j\omega)}{(4+j\omega)}
=\frac{3(3+j\omega)}{(2+j\omega)(4+j\omega)}\,.}
\]

\subsection{(b)
确定系统的单位冲激响应}\label{b-ux786eux5b9aux7cfbux7edfux7684ux5355ux4f4dux51b2ux6fc0ux54cdux5e94}

注意到根据傅立叶变换对 \[
\mathcal{F}\{e^{-at}u(t)\}=\frac{1}{a+j\omega},
\] 我们可以逆变换简单分式。先对下面的分式作部分分式展开: \[
\frac{3(j\omega+3)}{(j\omega+2)(j\omega+4)}.
\] 设 \[
\frac{j\omega+3}{(j\omega+2)(j\omega+4)}
=\frac{A}{j\omega+2} + \frac{B}{j\omega+4}\,.
\] 两边通分后比较: \[
j\omega+3 = A(j\omega+4)+B(j\omega+2) = (A+B)j\omega + (4A+2B).
\] 由此,必须有 \[
A+B=1,\quad 4A+2B=3.
\] 解得: \[
A = \frac{1}{2}, \quad B=\frac{1}{2}.
\]

所以 \[
\frac{j\omega+3}{(j\omega+2)(j\omega+4)}
=\frac{1}{2}\left[\frac{1}{j\omega+2}+\frac{1}{j\omega+4}\right].
\] 再乘上系数 3 得 \[\boxed{
H(j\omega)=\frac{3}{2}\left[\frac{1}{j\omega+2}+\frac{1}{j\omega+4}\right].}
\]

\subsection{(c)求关联该系统输入和输出的微分方程}\label{cux6c42ux5173ux8054ux8be5ux7cfbux7edfux8f93ux5165ux548cux8f93ux51faux7684ux5faeux5206ux65b9ux7a0b}

连续时间线性时不变系统输入输出满足如下微分方程形式,

\[
\sum_{k = 0}^Na_k\frac{d^ky(t)}{dt^k} = \sum_{k = 0}^Mb_k\frac{d^kx(t)}{dt^k}
\]

同时,根据

\[
H(j\omega)=\frac{Y(j\omega)}{X(j\omega)} = \frac{\sum_{k =0}^Mb_k(j\omega)^k}{\sum_{k = 0}^na_k(j\omega)^k} = \frac{3(j\omega+ 3)}{(2+j\omega)(4+j\omega)}
\]

有,

\[
b_0 = 9, \quad b_1 = 3, \quad a_0 = 8, \quad a_1 = 6, \quad a_2 = 1
\]

故微分方程形式为

\[
\boxed{
\frac{d^2y(t)}{dt^2}+6\frac{dy(t)}{dt}+8=3\frac{dx(t)}{dt}+9.}
\]


    % Add a bibliography block to the postdoc
    
    
    
\end{document}
